\section{LAT Description and Observations}

The LAT is a $\gamma$-ray telescope that detects photons by conversion into electron-positron pairs and operates in the energy range between 20 MeV and more than 300 GeV. Details of the instrument and data processing are given in \cite{2009ApJ...697.1071A}. The on-orbit calibration is described in \cite{2009ApJ...696.1084A}.

The following analysis was performed using 45 months of data collected from August 4, 2008 to Mai 18, 2012 (MET : 239557440--356439741) within a $7$\degr $\times 7\degr$ square (see section 4) around the position of the TeV source aligned with Galactic coordinates. We excluded $\gamma$-rays coming from a zenith angle larger than 100$\degr$ because of possible contamination from secondary $\gamma$-rays from the Earth's atmosphere \citep{2009PhRvD..80l2004A}. We used the Pass 7 clean event class that has a substantial reduction in instrumental background above 10 GeV with only a small loss in effective area (compared to the standard event class).

This energy of 10 GeV is a good compromise between photon statistics, angular resolution necessary to study the shape of the TeV sources and proximity to the energy range covered by the $\breve{C}$erenkov telescopes. This also greatly reduces the contribution of the Galactic diffuse background.

The counts map in Figure~\ref{fig:PlanGal} summarizes the regions of the Galactic Plane analyzed here. The bright pulsars Vela and Geminga clearly stand out, as well as the SNR IC443. In addition to these famous objects, a large number of sources clearly appears along the Galactic Plane, several of them being coincident with TeV detected sources, such as HESS~J1614$-$518 and HESS~J1616$-$508. They will be discussed in Section~\ref{res}. The large number of sources visible in this counts map highlight perfectly the capabilities now offered at high energy by the LAT.


%Source shape analysis requires the best possible angular resolution. Since the sources are expected to have an hard spectrum we made a compromise between statistics and resolution by selecting photons in the 10--300~GeV. This drastically reduces the contribution of the Galactic diffuse background and improve the single-photon angular resolution.


