\section{Discussion}
\label{discussion}

%Since \cite{2011ApJ...726...35A} the number of PWNe detected by \emph{Fermi}-LAT increased to X (Crab Nebula, Vela-X, MSH 15-52, HESS J1640, HESS J1825-137, ). This number must be considered with the growing number of PWNe candidates (HESS J1023, HESS J1857, K3 nebula, HESS J1119, HESS J1356, HESS J1303, HESS J1841 and HESS J1848). Table \ref{tab:table_luminosity} shows that these detected sources are located close to young pulsars of characteristic age between 1 and 30 kyr and of spin-down power between $10^{36}$ and $10^{39}$.

%Assuming that the TeV sources were actually associated to the pulsars of Table \ref{tab:table_luminosity}, we used to the pulsars distances to represent on Figure \ref{fig:dotelpwn} the PWN luminosity as a function of the pulsar spin-down power.  

%In this section, we will study the behaviour of the population with physical parameters of the system as the age or the pulsar spin-down power. This analysis follows the work of \cite{2009ApJ...694...12M} where the authors used TeV and X-ray informations.

In this section we investigate the correlations between the pulsar/PWN system parameters (age, spin-down power) and the flux in GeV, TeV and X-ray energy ranges. Tables \ref{tab:table_luminosity} \& \ref{tab:table_luminosity2} summarize the pulsar characteristics and the multi-wavelength informations on each source. To allow the comparison between \cite{2009ApJ...694...12M} and our work, we studied the expected differences between the TeV energy range and the GeV energy range. We assumed that the TeV and the GeV emission coming from these sources were the two sides of the IC emission produced by the same population of electrons and we studied the spectral shape of this IC peak.

Figure \ref{fig:gammagamma} shows the TeV spectral index as a function of the GeV spectral index. This figure perfectly show that sources having a hard index in the GeV energy range ($\Gamma < 2$) have a harder index in the TeV energy range as well, confirming that we may be detecting the two sides of the same $\gamma$-ray peak.

To quantify the expected differences seen in GeV and TeV, we estimated the average energy of the IC peak. For that, we assumed that the $\gamma$-ray energy spectra can be described by a log-normal representation (Equation \ref{logp}) as done in  \cite{2008ApJ...674.1037A}.

\begin{equation}
\label{logp}
\frac{dN}{dE}=N_0 \times \left(\frac{E}{E_0}\right)^{-\left[ \alpha + \beta \times \log\left(\frac{E}{E_b}\right) \right]} 
\end{equation}

We fixed $E_0$ at 300 GeV and $\beta$ at 0.2 and fitted the prefactor $N_0$, the index $\alpha$ and the energy break $E_b$ using our \emph{Fermi}-LAT results and the TeV spectra. Then, we defined the energy of the peak ($E_{peak}$) as the energy at which the energy flux of the modeled peak is maximal. This also corresponds to Equation \ref{pos_epeak}.

\begin{equation}
\label{pos_epeak}
\alpha + \beta \times \log\left(\frac{E}{E_b}\right) = 2.0
\end{equation}

The fit results as well as the peak position are presented in Table \ref{tab:Epeak}. Figure \ref{fig:EpeakETeV} presents the TeV spectral index as a function of the energy of the maximum of the IC peak. This figure highlights a correlation between the TeV spectral index and the peak energy. The TeV index increases with decreasing peak energy except for HESS J1632$-$472. This means that the closer the energy peak is from the GeV energy range, the softer the index will be in the TeV energy range, which is consistent with our log parabola model. The exception of HESS J1632$-$472 could be due to the uncertainties due to the presence of neighbouring sources. As discussed in Section \ref{morph_res}, three sources are detected at lower energies \citep{2012ApJS..199...31N}. These sources are not significantly detected above 10 GeV. It means that these sources can contaminate the low energy of the spectrum derived in this analysis and therefore artificially decrease the fitted energy of the maximum of the IC peak.

Figure \ref{fig:Epeakage} shows the peak energy as a function of the pulsar characteristic age. This Figure presents no obvious correlation between the energy of the peak and the pulsar characteristic age as it would be expected from evolution models and presented in \cite{2012arXiv1202.1455M}. However, it is important to note that this Figure suffers two main biases. First, the characteristic age may not be a good age estimator for the PWN. For instance, MSH 15$-$52 is a known case where two ages are proposed, either the characteristic age of the pulsar 1.7 kyr or an age between 20 and 40 kyr as suggested by the size and general appearance of the SNR \citep{2001AA...374..259G}. Second, our sample is relatively restricted to PWNe with a characteristic age close to 10 kyr, except for MSH 15$-$52 and HESS J1119$-$614. In this context it is not surprising that no correlation is found between $E_{peak}$ and the characteristic age. 

From Table \ref{tab:Epeak} we derived the mean parameters corresponding to our sources of interest : $\bar{\alpha} = 2.1 \pm 0.2$ and $\log_{10}\left(\bar{E_{peak}}\right)= 5.7 \pm 0.6$. Using these parameters we computed the mean expected ratio between the GeV and the TeV fluxes : $\bar{R}=1.9 \pm xx$. Figure \ref{fig:rapportTeV} presents the ratio of the luminosity in the GeV energy range over the luminosity found in the TeV energy range summarized in Table \ref{tab:table_luminosity2}. We represented the mean ratio found between the GeV and TeV energy ranges as a dashed line. This figure shows that no sources are located at more than 2$\sigma$ from this mean ratio except HESS J1804$-$216. \cite{2012ApJ...744...80A} studied the link between the TeV and the GeV emission and concluded that a theory assuming the energy-dependent diffusion of particles accelerated in the SNR is more likely than a PWN scenario. Therefore, the emission is not clearly associated to an IC peak and could have an hadronic origin, which would explain why this source is an outlier.  

As described in \cite{2009ApJ...694...12M}, the $\gamma$-ray and the X-ray luminosities are expected to decrease with time, but the evolution should be different following the age and the pulsar spin-down power. From Table \ref{tab:table_luminosity2} we represent the ratio of the $\gamma$-ray luminosity over the X-ray luminosity as a function of the age and as a function of the pulsar spin-down power. We also represented respectively in solid and dashed line the relations derived in \cite{2009ApJ...694...12M} multiplied by $\bar{R}$ for the whole sample of sources and for the sources clearly identify to PWNe.

This figure show two correlations : one between the fluxes ratio and the spin-down power and the other between the same flux ratio and the characteristic age. These correlations are expected since the magnetic field depends on these two parameters. However, for each correlation, a sample of four upper limits consistent with fluxes measured at TeV energies are well below the correlation relations derived by \cite{2009ApJ...694...12M}. The overall agreement with the relations proposed by \cite{2009ApJ...694...12M} is relatively good but it can already be predicted that the GeV points will have a larger dispersion once these sources will be detected. 

It seems clear from these figures that \emph{Fermi}-LAT mainly detect young and middle-aged PWNe (1-30 kyr) around energetic pulsars with a spin-down power between 10$^{36}$ and 10$^{37}$ erg s$^{-1}$. Figure \ref{fig:dotelpwn} shows the $\gamma$-ray luminosity of the PWN as a function of the pulsar spin-down power assuming the distances summarized in Table \ref{tab:table_luminosity}. The solid, dashed and dot dashed lines respectively show the lines where the luminosity corresponds to 100\%, 10\% and 1\% of the spin-down power. Full and hollow markers respectively represent sources for which the TeV emission is clearly associated to a PWNe and sources for which the association is less clear. Sources with a pulsar-like spectrum are represented in green pentagons. For all sources in Table \ref{tab:pulsars} the pulsar are included in the models. Among the detected sources, eight show a $\gamma$-ray efficiency below 1\% and five are consistent within uncertainties with  an efficiency between 1 an 10 \%. HESS J1303$-$631 is also consistent with an efficiency of 100\% assuming a distance of 15 kpc. This can be due to the large uncertainty on the distance measured which can lead to a factor 3 between the estimated distance and the true distance and by the contamination by Kes 17 as discussed in Section \ref{morph_res}. Six upper limits on the luminosity of TeV sources clearly associated to PWNe are well below an efficiency of 1\%.

\section{Conclusion}

45 months of \emph{Fermi}-LAT observations have been used to look for counterparts to the TeV sources potentially associated to PWNe. Among the 58 sources studied, 31 have been detected. Among these 31, 23 were also detected in \cite{1FHL} and 15 in \cite{2012PhRvD..85h3008N}. Interestingly five new sources are detected in our analysis (HESS J1119$-$614, HESS J1303$-$631, HESS J1356$-$645, HESS J1841$-055$ and HESS J1848$-$018). We analyzed the morphology of the 31 sources detected and found that for 19 of them the GeV shape did not significantly improve the fit in comparison to the TeV shape. For five sources, we found a significant extension larger than the TeV morphology. Six sources are well fitted by a point-like source; all of them have a high index and are probably associated to pulsars emitting above 10 GeV.

Since \cite{2011ApJ...726...35A} the number of PWNe detected by \emph{Fermi}-LAT increased to 5 (Crab Nebula, Vela-X, MSH 15$-$52, HESS J1640$-$465, HESS J1825$-$137 ). This number must be considered with the growing number of PWNe candidates (HESS J1023$-$577, HESS J1119$-$614, HESS J1303$-$631, HESS J1356$-$645, HESS J1420$-$607, HESS J1841$-$055, HESS J1848$-$018 and HESS J1857+026). These PWNe and PWNe candidates are powered by young (between 1 and 30 kyr) and powerfull pulsars (spin down power between 10$^{36}$ and 10$^{39}$ erg s$^{-1}$ with an efficiency below 10\%.
