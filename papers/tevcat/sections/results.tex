\section{Results}
\label{res}

Using the procedure described above, we detected 30 sources among the 56 PWNe candidates selected. For each of these detected sources, as explained in Section~\ref{signi}, we determined the best morphology by comparing the likelihood of our fit obtained under three different hypotheses: TeV shape, point-like source and extended source. The results of the spatial analysis are shown in Table~\ref{tab:Morphology_results}, the last column summarizing the best shape found for each source. Once the best morphology found, we performed a spectral analysis whose results are reported in Tables~\ref{tab:det_sources} and \ref{tab:det_sources2}, while upper limits on non-detections are presented in Tables~\ref{tab:nondet_sources} and \ref{tab:nondet_sources2}.

\subsection{Extended sources detected above 10~GeV}
\label{morph_res}

Most sources are better described using the TeV morphology as a template. This is partly due to the 2 (3 for an extended source) additional degrees of freedom to take into account but also to the low statistics above 10~GeV. Indeed, these sources have on average a low TS value or, are at the limit of our extension threshold, such as HESS J1514-591.

Interestingly, 11 sources are better described using the morphology obtained at GeV energies using the LAT data, 5 of them being extended. Table \ref{tab:GeVmorph} summarizes the spatial fits in such a case. The results presented here are consistent with Tables 3 and 4 of \cite{2012arXiv1207.0027L}. One can note that the agreement for HESS~J1632$-$478 is not excellent. The difference in extension (0.35 $\pm 0.06$ vs 0.45 $\pm$ 0.04 in our analysis) certainly comes from the 3 additional 2FGL sources (2FGL~J1631.7$-$4720c, 2FGL~J1630.2$-$4752 and 2FGL~J1632.4$-$4820c) in the model used by \cite{2012arXiv1207.0027L}. These sources were below our $\text{TS}>$25 threshold to add a source, and being unidentified, we found no physical reason to add them as we did in the case of PSR~J1838-0536. 

HESS~J1303$-$631 is a new extended source detected at GeV energies. However, Figure~\ref{1303} shows that the \emph{Fermi}-LAT excess observed for this source is likely due to 2 point-like sources (one associated with HESS~J1303$-$631 and one associated with Kes 17 \citep{2011ApJ...740L..12W}) but the limited statistics at high energy does not allow us to separate them. When the region is fitted assuming the TeV morphology for HESS~J1303$-$631 and a separate source for Kes 17, the latter hardly reaches $\text{TS}>$ 20 and is thus too faint to be added to our model. Nevertheless, the difference between the true spectrum and the spectrum derived in this work will be included in the systematics on the flux taking into account the fact that we do not know the true morphology of the source.

\subsection{Pulsars detected above 10~GeV}
\label{pulsarsect}
Only six sources are better described by a point-source model above 10 GeV than with a uniform disk or by the TeV template reported in Table~\ref{tab:TeV_sources}: HESS~J1708$-$443, MGRO J0632+17, MGRO J1908$+$06, MGRO J2019$+$37, MGRO J2228+61 and VER J0006+727. It is not a surprise to see in Figure~\ref{fig:indexvssize} and Table~\ref{tab:det_sources} that these sources have a rather soft spectrum in comparison to the average index of TeV detected sources and that they are all coincident with bright gamma-ray pulsars. As can be seen on Figures \ref{fig:sedsourcespuls}, the spectrum obtained above 10 GeV for these sources is in good agreement with the spectrum derived in the 2FGL Catalog released by the Fermi collaboration \citep{2012ApJS..199...31N}. The $\gamma$-ray emission detected by the LAT above 10 GeV is therefore very likely due to the pulsar itself than to its associated PWN. 

Three other sources are coincident with bright Fermi pulsars and present a very soft spectrum in agreement with the 2FGL Catalog as well: HESS J1418-609, MGRO J1958$+$2848 and MGRO~J2031+41. Again, the gamma-ray signal is very likely due to magnetospheric emission and the fact that these sources are rather faint above 10 GeV can explain why the improvement obtained using a point-source model is not significant in comparison to the simple TeV morphology.

To study the contamination of these pulsars to our sources of interest, we used the procedure described in Section 4 but including the pulsars in our models of the regions. The spectra of \emph{Fermi}-LAT pulsars are well characterized by exponentially cutoff power laws with photon indices near 1.5 and cutoff energies between 0.5 and 6 GeV~\textbf{cite 2PC}. As we selected the data above 10 GeV, we cannot fit the spectral parameters of theses pulsars. Thus we included them in our model assuming the fixed photon index, cutoff energy and normalization extracted from the 2FGL Catalog. It should be noted that some pulsars may deviate from the simple exponential cutoff power law above 10 GeV. This has been proposed for instance for the famous case of the Crab pulsar (citer les papiers Veritas et Magic). In such cases, our fit could still be contaminated by the pulsar, especially in the first energy interval (between 10 and 31.6 GeV).

Table \ref{tab:pulsarfit} and Figures \ref{fig:sedsourcespuls}, \ref{fig:sedsourcespuls2} \& \ref{fig:sedsourcespuls3} show the results of this new fit using the conventions presented in Table \ref{tab:pulsars}. As expected for the six point-like sources presented in Figure~\ref{fig:sedsourcespuls}, the low energy part of the spectrum tends to disappear confirming that we only detected a pulsar emission. In the case of MGRO J1908+06, the detection is not significant anymore. To present conservative results, we derived the upper limits assuming the TeV shape of the source.

Figures \ref{fig:sedsourcespuls2} \& \ref{fig:sedsourcespuls3} show the same behavior for HESS~J1418-609 confirming that the signal observed was dominated by the pulsar emission. In the case of HESS~J1420-607 and HESS~J1119-614, the spectra are now slightly harder but still in very good agreement with the previous ones, which is a good indication that we are seeing the  emission from the PWNe and not from their associated pulsars. 

\subsection{Detections of PWNe candidates}
%{\bf {\color{red} My own choice looking at your previous results: Kooka, HESSJ1303, HESS J1356, G292.2, HESSJ1841 and HESSJ1848. Do a quick subsection for each of them (and modeling if needed) like in the PWN Cat paper}} 

In this section we will describe the new PWNe candidates found in this analysis. We chose them by looking for signal connecting to the TeV spectrum and showing an hard spectrum. HESS~J1420-607, HESS~J1303-631, HESS~J1356-645 \& HESS~J1119-614 were already proposed as PWNe by analyses at other wave-length. The detection of these sources by the \emph{Fermi}-LAT tends to confirm this hypothesis.

HESS~J1848-018, is classified as UNID. The detection presented in Table \ref{tab:det_sources} show a faint source with a soft spectrum. We will discuss this source in a PWN scenario. 

\subsubsection*{HESS J1420$-$607}

The complex of compact and extended radio/X-ray sources, called Kookaburra~\citep{1999ApJ...515..712R}, spans over about one square degree along the Galactic plane. It has been extensively studied to explain the EGRET source: 3EG J1420$-$6038/GeV J1417$-$6100 \citep{1999ApJS..123...79H}. Within the North-East excess of this complex, labeled 'K3',  was discovered the pulsar PSR J1420$-$6048, a young and energetic pulsar with period 68 ms, characteristic of 13 kyr, and spin down energy of $10^{37}$ erg s$^{-1}$ \citep{2001ApJ...552L..45D}. Following X-ray observations by ASCA and later by \emph{Chandra} and \emph{XMM-Newton} revealed an extended X-ray emission around this pulsar identified as a potential PWN \citep{2001ApJ...561L.187R,2005ApJ...627..904N}. In the South-West side of the large Kookaburra complex lies a bright nebula exhibiting an extended hard X-ray emission, G313.1+0.1, called the 'Rabbit'~\citep{1999ApJ...515..712R}. This X-ray excess was also proposed as a plausible PWN contributing also to the gamma-ray emission detected by EGRET.

In the TeV energy range, the survey of the Galactic plane by H.E.S.S. revealed two very high energy sources in this region: HESS J1420$-$607 and HESS J1418$-$609 \citep{2006AA...456..245A}. HESS J1420-607 is centered North of PSR~J1420$-$6048 (nearby the K3 nebula), while HESS J1418-609 is coincident with the Rabbit nebula. More recently, \emph{Fermi}-LAT detected  pulsed gamma-ray emission from PSR J1420$-$6048 and PSR J1418$-$6058, a new gamma-ray pulsar found through blind frequency searches. This new pulsar is coincident with an X-ray source in the Rabbit PWN and has a spin-down power high enough to power the TeV PWN candidate HESS J1418$-$609. 

In our analysis, HESS J1420$-$607 is detected with a TS of 41 which corresponds to a significance of $\sim$ 6$\sigma$. With a TS of 41.2, the point source hypothesis does not significantly improve the likelihood of our fit in comparison to the TeV morphology (Gaussian of 0.06$\degr$) presented in \citep{2006AA...456..245A}. Thus assuming the TeV shape, we found an integrated flux F(10-316 GeV)=$(3.2 \pm 0.9_{stat} \pm 1.2_{syst}) \times 10^{-10}$ ph cm$^{-2}$ s$^{-1}$, a spectral index of $\Gamma = 1.91 \pm 0.27_{stat} \pm 0.30_{syst}$ and an energy flux of E(10-316 GeV) = $(23.1 \pm 6.7_{stat} \pm 9.5_{syst}) \times 10^{-12}$ erg cm$^{-2}$ s$^{-1}$.

In a second step, following the procedure presented in Section~\ref{pulsarsect}, we refitted the spectrum of HESS J1420$-$607 including PSR J1420$-$6048 in our model and fixing its spectral parameters found in the 2FGL Catalog. The fit of HESS~J1420-607 leads to a lower significance of $~5.6 \sigma$ (TS=35.8, 2 d.o.f) with an integrated flux of F(10-316 GeV)= $(3.4 \pm 0.9_{stat} \pm xx_{syst}) \times 10^{-10}$ ph cm$^{-2}$ s$^{-1}$ and an index of $\Gamma = 1.80 \pm 0.29_{stat} \pm xx_{syst}$.  

The connection between the GeV flux as observed by Fermi and the TeV flux as seen by H.E.S.S., visible in Figure \ref{fig:hessj1420}, supports a common origin for the gamma-ray emission. The two black curves present the leptonic and hadronic models proposed by \cite{2010ApJ...711.1168V}. As said above, the lower energy part of our analysis could still be contaminated by the pulsar even when the 2FGL source associated with the pulsar is added. Therefore, the first energy bin should be taken as an upper limit. This implies that, with the current statistics, all models reproduce reasonably well the GeV and TeV data. A future \emph{Fermi}-LAT off-pulse analysis of this pulsar performed with more statistics could help discriminate between both models.


\subsubsection*{HESS J1356$-$645}

HESS~J1356$-$645 is a source detected in the TeV energy range by H.E.S.S. during the continuation of the Galactic Plane Survey~\citep{2011AA...533A.103H}. This extended source lies close to the pulsar PSR~J1357$-$6429 which was discovered during the Parkes multibeam survey of the Galactic Plane \citep{2004ApJ...611L..25C}. Its high spin-down power of $\dot{E} = 3.1 \times 10^{36}$ erg s$^{-1}$ makes it a good candidate to power a PWN. Archival radio and analysis of X-ray data from \emph{ROSAT/PSPC} and \emph{XMM/Newton} have revealed a faint extended structure coincident with the VHE emission~\citep{2011AA...533A.103H} thus providing another argument in favor of the PWN scenario. In parallel, \cite{2011AA...533A.102L} announced the detection of a pulsed signal from PSR~J1357$-$6429 in the $\gamma$-ray and X-ray energy ranges using \emph{Fermi}-LAT and \emph{XMM-Newton} data. However, using 29 months of LAT data between 0.1 and 100 GeV, no counterpart to the TeV emission was found in the off pulse window of the pulsar.

The 16 additional months of observations by \emph{Fermi}-LAT between 10 and 316 GeV included in our dataset now enables the detection of a faint counterpart to the TeV emission with a TS = 25.8 (4.7 $\sigma$ assuming 2 d.o.f.). Since the best GeV morphology does not improve the fit significantly, we used the TeV Gaussian of 0.17$\degr$ \citep{2011AA...533A.103H} for the spectral analysis and derived an integrated flux of F(10-316 GeV)=$(1.2 \pm 0.5_{stat} \pm 0.7_{syst}) \times 10^{-10}$ ph cm$^{-2}$ s$^{-1}$, an energy flux of E(10-316 GeV)=$(16.8 \pm 6.9 \pm 8.1) \times 10^{-12}$ erg cm$^{-2}$ s$^{-1}$ and a hard spectral index of $\Gamma = 1.01 \pm 0.38 \pm 0.25$. 

It should also be noted that, with its low energy cutoff at around 800 MeV in the \emph{Fermi}-LAT energy range \citep{2011AA...533A.102L}, PSR~J1357$-$6429 is not significant anymore in the 10 to 316 GeV energy range. Therefore, we do not expect to see any changes in the spectral parameters when adding PSR~J1357$-$6429 to the model of the region. This is verified in Table~\ref{tab:pulsarfit} as well as in Figure \ref{fig:hess1356}. 

The combined GeV-TeV data as seen in Figure \ref{fig:hess1356} provide new constraints concerning the spectral shape of the gamma-ray emission. It is clearly visible in this Figure that the \emph{Fermi}-LAT spectral points nicely match the H.E.S.S. ones, proving that the GeV and the TeV emission have a common origin. Assuming that the gamma-ray signal is coming from the PWN powered by PSR~J1357$-$6429, \cite{2011AA...533A.103H} proposed a leptonic scenario (black curve) which provides an excellent fit of the new multi-wavelength data. This model assumes a rather low magnetic field of $\sim 3 \mu$G similar to the value observed in other relic PWNe. 

\subsubsection*{HESS~J1119$-$614}

During the Parkes multibeam pulsar survey, \cite{2000ApJ...541..367C} discovered PSR~J1119$-$6127, a young ($\tau = 1.6$ kyr) pulsar with a high $\dot{E} = 2.3 \times 10^{36}$ erg s$^{-1}$ within the supernova remnant G292.2$-$0.5. Using \emph{Chandra} observations, \cite{2003ApJ...591L.143G} and \cite{2008ApJ...684..532S} revealed the presence of a faint and compact PWN close to this pulsar. More recently, a TeV gamma-ray source coincident with PSR J1119$-$6127 and G292.2$-$0.5 was announced\footnote{http://cxc.harvard.edu/cdo/snr09/pres/DjannatiAtai\_Arache\_v2.pdf}.

Using the method described above, a faint signal consistent with the location of the composite SNR G292.2$-$0.5 is detected with a TS of 27.3 (4.9 $\sigma$ with 2 d.o.f.). Since the best GeV morphology does not improve the fit significantly, we used the TeV Gaussian of 0.05$\degr$ for the spectral analysis and derived an integrated flux of F(10 - 316 GeV) = $(2.1 \pm 0.6_{stat} \pm 1.1_{syst}) \times 10^{-10}$ ph cm$^{-2}$ s$^{-1}$, an energy flux of E(10-316 GeV) = $(10.4 \pm 3.9_{stat} \pm 4.9_{syst}) \times 10^{-10}$ erg cm$^{-2}$ s$^{-1}$ and a soft index of $\Gamma = 2.15 \pm 0.35_{stat} \pm 0.20_{syst}$.

Nevertheless, as can be seen on Figure \ref{fig:hessj1119} and in Table \ref{tab:pulsarfit}, these parameters are contaminated by a low energy component associated to PSR~J1119$-$6127. Once the source 2FGL~J1118.8$-$6128, associated with PSR~J1119$-$6127, included in our model of the region, this contamination decreases and the significance of our GeV source is now just above the detection threshold that we fixed in Section XX, with TS = 16 (3.6 $\sigma$ with 2 d.o.f.). As can be seen in Figure \ref{fig:hessj1119}, the low energy point of the SED is now an upper limit. The best fit parameters in this hypothesis are an integrated flux of F(10-316 GeV) = $(1.5 \pm 0.6_{stat} \pm XX_{syst})\times 10^{-10}$ ph cm$^{-2}$ s$^{-1}$ and an harder index of $\Gamma = 1.80 \pm 0.48_{stat} \pm XX_{syst}$. To be conservative, we used this fit to derive the physical properties of HESS~J1119$-$614. 

\textbf{TO BE DONE}

\subsubsection*{HESS J1303$-$631}

HESS J1303$-$631 was serendipitously discovered in 2004 \citep{2005AA...439.1013A} during an observation campaign for the pulsar binary system PSR B1259$-$63. It was originally classified as a "dark" accelerator due to the lack of detected counterparts in radio and X-rays with \emph{Chandra} \citep{2005ApJ...629.1017M}. At this time, a search in the field of view only yielded one plausible counterpart located in the North-Western edge of HESS J1303$-$631, the high spin-down power pulsar PSR J1301$-$6305 ($\dot{E} = 1.70 \times 10^36$ erg s$^{-1}$). The recent detection of a very weak X-ray PWN using \emph{XMM-Newton} observations now allows the solid identification of this source as a VHE PWN associated to the pulsar PSR J1301$-$6305, leading \cite{2011arXiv1104.1680D} to classify the TeV excess as an older PWN with a complex morphology and an important offset between the pulsar and nebula.

Figures 1 \& 3 obtained by \cite{2011ApJ...740L..12W} show no significant emission coming from the location of HESS~J1303$-$631 using XX months of \emph{Fermi}-LAT data between 1 and 20 GeV. With 16 months of additional data and a higher energy threshold, our analysis now provides a first detection of GeV emission coincident with the TeV source. Nevertheless as discussed in Section \ref{morph_res}, Figure \ref{1303} shows that the detected emission might be contaminated by the source associated to Kes 17. When added to the model as an independent source, Kes 17 hardly reaches a TS of 20 which is lower than our threshold of 25 to add a source. Furthermore, Kes 17 is not included in the Hard Source List. Since we cannot separate these two sources with the current statistics, we decided to take into account this effect of source confusion in our systematics on HESS J1303$-$631.

Assuming our best GeV morphology represented by a Disk shape of 0.5$\degr$ (see Table \ref{tab:Morphology_results}), we obtained an integrated flux of F(10-316 GeV)=$(5.9 \pm 1.1_{stat} \pm 2.8_{syst}) \times 10^{-10}$ ph cm$^{-2}$ s$^{-1}$, an energy flux of E(10-316 GeV)= $(43.5 \pm 10.0_{stat} \pm 14.2_{syst}) \times 10^{-12}$ erg cm$^{-2}$ s$^{-1}$ and an index of $\Gamma = 1.71 \pm 0.19_{stat} \pm 0.28_{syst}$. This hard index is in the range of values obtained for PWNe as seen with \emph{Fermi} and is inconsistent with the spectral index of $\sim 2.4$ derived by \cite{2011ApJ...740L..12W} for Kes 17. This is a good evidence that the GeV emission is dominated by the PWN candidate. As can be seen in Fig XX (\textbf{modeling}), even though the connection between the GeV and the TeV energy range is not perfect, there is little to doubt that we have detected a counterpart to the TeV signal. A future analysis with more statistics would greatly help to separate the different components and conclude on the exact spectrum of HESS~J1303$-$631.

\subsubsection*{HESS J1841$-$055}

HESS~J1841$-$055 was discovered during the H.E.S.S. Galactic Plane Survey \citep{2008AA...477..353A} and remained an UNID source since then. The emission is highly extended and show possibly three peaks suggesting that the TeV emission is composed of more than one source. Using \emph{INTEGRAL} data, \cite{2009ApJ...697.1194S} proposed the  high-mass X-ray binary HMXB system AX J1841.0$-$0536 as a potential counterpart, at least for a part of the emission. \cite{2011ICRC....6..197T} proposed the association of HESS~J1841$-$055 to an ancient PWN powered by PSR~J1841$-$0524, PSR~J1838$-$0549 or both as each pulsar taken independently would need an efficiency greater than 100\% to power a potential PWN associated to the TeV source. More recently, the blind search detection of the new gamma-ray pulsar PSR~J1838$-$0537 with \emph{Fermi}-LAT provided another good counterpart of the TeV source. Indeed, assuming a distance of 2 kpc, \cite{2012ApJ...755L..20P} estimated that PSR J1838$-$0537 is sufficiently energetic to power the whole TeV source with a conversion efficiency of 0.5\%, similar to other suggested pulsar/PWN associations \citep{2008ApJ...682L..41H}.

HESS~J1841$-$055 is detected as a significantly extended source with our analysis (TS$_{ext}$= 38.3) at a position consistent with the TeV source but with a slightly larger extension (0.57 $\degr$ with respect to 0.41 $\degr$ $\times$ 0.25 $\degr$ for the TeV source). However, the GeV best morphology does not significantly improve the fit compared to the TeV morphology (TS$_{GeV/TeV}$= 13.0, 2.8$\sigma$ with 3 d.o.f.). Thus assuming the TeV shape, our best fit yielded an integrated flux of F(10-316 GeV) = $(9.3 \pm 1.9_{stat} \pm 3.9_{syst}) \times 10^{-10}$ for an energy flux of G(10-316 GeV) = $(79.6 \pm 16.0_{stat} \pm 19.1_{syst}) \times 10^{-12}$ erg cm$^{-2}$ s$^{-1}$ and a hard index of $\Gamma = 1.56 \pm 0.20_{stat} \pm 0.32_{syst}$ consistent with the average value for PWNe detected by the \emph{Fermi}-LAT.

As can be seen in Figure \ref{fig:1841}, the \emph{Fermi}-LAT spectral points nicely match the H.E.S.S. ones, suggesting a common origin. The hard \emph{Fermi}-LAT spectrum detected imply that a curvature must arise between the TeV energy range and the GeV energy range. This is typical of most PWNe detected by \emph{Fermi} and HESS which present an inverse-Compton emission peaking at few hundreds of GeV and would favor the PWN scenario. However, as said above, this source is extremely extended in both wavelengths and could be composed of several gamma-ray sources. Follow-up observations with HESS and \emph{Fermi} would be needed to unveil the real nature of HESS J1841$-$055.


\subsubsection*{HESS J1848-018}

HESS~J1848-018 \emph{(WR121a,W43,TeV J1848-017)}.

\subsection{Constraints obtained from non-detections}
%{\bf {\color{red} My own choice looking at your previous results: HESSJ1026, HESSJ1458, HESSJ1626-490, HESSJ1813 and so on. Do a quick subsection for each of them (and modeling if needed) like in the PWN Cat paper}}

This section will present the sources for which the emission was not significant but for which the upper limit show an interesting behavior to bring new constraints on the models.

\subsubsection*{HESS J1026$-$582}

HESS~J1026$-$582 was discovered during a new analysis of the region of HESS~J1023$-$577 \citep{2011AA...525A..46H}. The presence of PSR~J1028$-$5819, detected by the Parkes Radio telescope \cite{2008MNRAS.389.1881K}, close to the TeV emission suggested a PWN scenario to explain this VHE source. This hypothesis is consolidated by the spin-down power energy of PSR~J1028$-$5819, $\dot{E} = 8.43 \times 10^{35}$ erg s$^{-1}$ \citep{2009ApJ...695L..72A}, which is high enough to power a PWN.

No significant GeV emission coming from the location of the TeV excess was detected in our analysis. The very low TS value derived (see Tables \ref{tab:nondet_sources} \& \ref{tab:nondet_sources2}) give few hope for a future detection of this source in this energy range on a short time scale. The upper limits presented on Figure \ref show that a curvature is needed between the TeV and the GeV energy range. This suggests an inverse-Compton peak consistent with a PWN hypothesis.

\subsubsection*{HESS J1458$-$608}

PSR~J1459$-$60 \citep{2010ApJS..187..460A} is an energetic pulsar with an $\dot{E} = 9.2 \times 10^{35}$ erg s$^{-1}$ sufficient to power a PWN. A potential PWN, HESS~J1458$-$608, was discovered 9.6$\prime$ from PSR~J1459$-$60 using H.E.S.S. data after a dedicated observation \cite{2012arXiv1205.0719D} based on a marginal detection in the 2004 Galactic Plane Survey.

The source was not significantly detected in our analysis above 10 GeV (TS = 12.3). It can also be noted from Table \ref{tab:nondet_sources2} that the marginal emission detected comes from the low energy range studied and is more likely associated to PSR~J1459$-$60. Figure \ref{fig:hessj1458} shows that even taking into account the pulsar in our model of the region, the SED is still contaminated at low energy. Nevertheless, the upper limits of the two high energy bins show that a change in the slope of the spectrum is needed between the TeV and the GeV component.

\textbf{TO BE DONE}

   
\subsubsection*{HESS J1626$-$490}

HESS~J1626$-$490 is another unidentified source detected during the H.E.S.S. Galactic Plane Survey \citep{2008AA...477..353A}. \cite{2011ICRC....7...44E} found no X-Ray source fulfilling the energetic requirement to explain the TeV emission using \emph{XMM-Newton} observations. However, the author suggested that a hadronic scenario based on the interaction of the SNR~G335.2+00.1 with a $^{12}$CO molecular cloud could explain the TeV emission.

Figure XX shows that the upper limits derived following the procedure described in Section 4 require a change in the slope of the spectrum, consistent with the model presented in Figure 4 of \cite{2011ICRC....7...44E}. 

\subsubsection*{HESS J1813$-$178}

HESS J1813-178 was discovered during the H.E.S.S. Survey of the Inner Galaxy \citep{2005Sci...307.1938A}. It remained unidentified until the discovery of the SNR G12.8-0.0 \cite{2005ApJ...629L.105B}. X-ray studies resolved the emission into a point-like source and an extended nebula \cite{2007AA...470..249F}. The discovery of PSR J1813$-$1749 \cite{2009ApJ...700L.158G} confirmed a PWN/SNR scenario. With its $\dot{E} = 5.6 \times 10^{37}$ erg s$^{-1}$ \citep{2012ApJ...753L..14H}, this pulsar is the third most energetic pulsar of the Galaxy. Thus it could produce a PWN observable in TeV/GeV range.

\cite{2007AA...470..249F} \& \cite{2010ApJ...718..467F} preferred a leptonic scenario to explain the GeV/TeV emission to a potential hadronic scenario. The upper limits derived using the procedure described in sect. 4. show that the spectrum of HESS~J1813$-$178 cannot be flat between the TeV and the GeV energy ranges and suggest peak with an energy cutoff located between the two energy ranges. Therefore, this suggests a PWN scenario as determined in previous works.

\textbf{TO BE DONE}


\subsection{Conclusion}
To be done later....