\begin{deluxetable}{l*{5}l}
\tabletypesize{\scriptsize}
\tablecaption{List of TeV sources analyzed
\label{tab:TeV_sources}}
\tablehead{\colhead{TeV source} & \colhead{Type} & \colhead{(l,b) ($\degr$)} & \colhead{TeV morphology} & \colhead{Reference}}
\startdata
HESS J1018-589 & UNID &(284.23,-1.72) & pointsource& \citep{2012arXiv1203.3215H} \\
HESS J1023-577 & Massive Star Cluster &(284.22,-0.40) &gaussian(0.18)& \citep{2011AA...525A..46H}\\
HESS J1026-582 & PWN &(284.80,-0.52) &gaussian(0.14)& \citep{2011AA...525A..46H} \\
HESS J1119-614 & PWN &(292.10,-0.49) &gaussian(0.05)& \footnote{http://cxc.harvard.edu/cdo/snr09/pres/DjannatiAtai\_Arache\_v2.pdf}\\
HESS J1303-631 & PWN &(304.24,-0.36) &gaussian(0.16)& \citep{2005AA...439.1013A}\\
HESS J1356-645 & PWN &(309.81,-2.49) &gaussian(0.20)& \citep{2011AA...533A.103H}\\
HESS J1418-609 & PWN &(313.25,0.15) &Ell. gaussian(0.08,0.06)&\citep{2006AA...456..245A}\\
HESS J1420-607 & PWN &(313.56,0.27) &gaussian(0.06)& \citep{2006AA...456..245A}\\
HESS J1427-608 & UNID &(314.41,-0.14) &Ell. gaussian(0.04,0.08)& \citep{2008AA...477..353A}\\
HESS J1458-608 & PWN & (317.75, -1.7) & gaussian(0.17) & \citep{2012arXiv1205.0719D}\\
HESS J1503-582 & UNID &(319.62,0.29) &gaussian(0.26)& \citep{2008AIPC.1085..281R}\\
HESS J1507-622 & UNID &(317.95,-3.49) &gaussian(0.15)& \citep{2011AA...525A..45H}\\
HESS J1514-591 & PWN &(320.33,-1.19) & Ell. gaussian(0.11,0.04)& \citep{2005AA...435L..17A}\\
HESS J1554-550 & PWN &(327.16,-1.07) & pointsource & \citep{2012arXiv1201.0481A}\\
HESS J1614-518 & massive star cluster &(331.52,-0.58) &Ell. gaussian(0.23,0.15)& \citep{2006ApJ...636..777A}\\
HESS J1616-508 & PWN &(332.39,-0.14) &gaussian(0.14)& \citep{2006ApJ...636..777A}\\
HESS J1626-490 & UNID &(334.77,0.05) &Ell. gaussian(0.07,0.10)& \citep{2008AA...477..353A}\\
HESS J1632-478 & PWN &(336.38,0.19) &Ell. gaussian(0.21,0.06)&\citep{2006ApJ...636..777A}\\
HESS J1634-472 & UNID &(337.11,0.22) &gaussian(0.11)&\citep{2006ApJ...636..777A}\\
HESS J1640-465 & PWN &(338.32,-0.02) &gaussian(0.04)&\citep{2006ApJ...636..777A}\\
HESS J1646-458 & Massive Star Cluster & (339.57, -0.02) & gaussian(0.35) & \citep{2012AA...537A.114A}\\
HESS J1646-458 B & Massive Star Cluster & (339.01, -0.79) & gaussian(0.25) & \citep{2012AA...537A.114A}\\
HESS J1702-420 & UNID &(344.30,-0.18) &Ell. gaussian(0.30,0.15)&\citep{2006ApJ...636..777A}\\
HESS J1708-443 & PWN &(343.06,-2.38) &gaussian(0.29)& \citep{2011AA...528A.143H}\\
HESS J1718-385 & PWN &(348.83,-0.49) &Ell. gaussian(0.15,0.07)& \citep{2007AA...472..489A}\\
HESS J1729-345 & UNID &(353.44,-0.13) &gaussian(0.14)& \citep{2011AA...531A..81H}\\
HESS J1804-216 & UNID &(8.40,-0.03) &Ell. gaussian(0.16,0.27)& \citep{2006ApJ...636..777A} \\
HESS J1809-193 & PWN &(11.18,-0.09) &Ell. gaussian(0.53,0.25)&\citep{2007AA...472..489A}\\
HESS J1813-178 & PWN &(12.81,-0.03) &gaussian(0.04)& \citep{2006ApJ...636..777A}\\
HESS J1818-154 & PWN &(15.41,0.17) &gaussian(0.14)& \citep{2011arXiv1112.2901H} \\
HESS J1825-137 & PWN &(17.71,-0.70) &Ell. gaussian(0.13,0.12)&\citep{2006AA...460..365A}\\
HESS J1831-098 & PWN &(21.85,-0.11) &gaussian(0.15)& \citep{2011arXiv1110.6837S}\\
HESS J1833-105 & PWN &(21.51,-0.88) & pointsource& \citep{2008ICRC....2..823D}\\
HESS J1834-087 & UNID &(23.24,-0.31) &gaussian(0.09)& \citep{2006ApJ...636..777A}\\
HESS J1837-069 & UNID &(25.18,-0.12) &Ell. gaussian(0.12,0.05)&\citep{2006ApJ...636..777A}\\
HESS J1841-055 & UNID &(26.80,-0.20) &Ell. gaussian(0.41,0.25)& \citep{2008AA...477..353A}\\
HESS J1843-033 & UNID &(29.30,0.51) & pointsource& \citep{2008ICRC....2..579H}\\
HESS J1846-029 & PWN &(29.70,-0.24) & pointsource& \citep{2008ICRC....2..823D}\\
HESS J1848-018 & Massive Star Cluster &(31.00,-0.16) &gaussian(0.32)& \citep{2008AIPC.1085..372C}\\
HESS J1849-000 & PWN &(32.64,0.53) & pointsource& \citep{2008AIPC.1085..312T}\\
HESS J1857+026 & UNID &(35.96,-0.06) &Ell. gaussian(0.11,0.08)&\citep{2008AA...477..353A}\\
HESS J1858+020 & UNID &(35.58,-0.58) &Ell. gaussian(0.08,0.02)&\citep{2008AA...477..353A}\\
HESS J1912+101 & PWN &(44.39,-0.07) &gaussian(0.26)& \citep{2008AA...484..435A}\\
MGRO J0631+105 & PWN &(201.30,0.51) & pointsource & \citep{2009ApJ...700L.127A}\\
MGRO J0632+17 & PWN &(195.34,3.78) &gaussian(1.30)& \citep{2009ApJ...700L.127A} \\
MGRO J1844-035 & UNID &(28.91,-0.02) & pointsource& \citep{2009ApJ...700L.127A}\\
MGRO J1900+039 & UNID &(37.42,-0.11) & pointsource& \citep{2009ApJ...700L.127A}\\
MGRO J1908+06 & UNID &(40.39,-0.79) &gaussian(0.34)& \citep{2009AA...499..723A}\\
MGRO J1958+2848 & PWN &(65.85,-0.23) & pointsource& \citep{2009ApJ...700L.127A}\\
MGRO J2019+37 & PWN &(75.00,0.39) &gaussian(0.55)& \citep{2007ApJ...664L..91A}\\
MGRO J2031+41 A& UNID &(79.53,0.64) &gaussian(1.50)&\citep{2007ApJ...664L..91A}\\
MGRO J2031+41 B& UNID &(80.25,1.07) &gaussian(0.10)& \citep{2012ApJ...745L..22B}\\
MGRO J2228+611 & PWN &(106.57,2.91) & pointsource& \citep{2009ApJ...700L.127A}\\
VER J0006+727 & PWN &(119.58,10.20) & pointsource& \citep{2011arXiv1111.2591M}\\
VER J1930+188 & PWN &(54.10,0.26) & pointsource& \citep{2010ApJ...719L..69A} \\
VER J1959+208 & PSR &(59.20,-4.70) & pointsource& \citep{2003ApJ...583..853H}\\
VER J2016+372 & UNID &(74.94,1.15) & pointsource& \citep{2011arXiv1110.4656A}\\
W49A & Star Forming Region &(43.27,-0.00) & pointsource& \citep{2011arXiv1104.5003B}\\

%Vela Region & UNID &(263.23,-2.92) & pointsource& \\
\enddata
%\\
%a : P72Y0963 is associated with the pulsar emission of PSR~J0633+1746 located inside of the TeV template use for Geminga


%j'en suis la

\tablecomments{ The first two columns are TeV source names and types as defined in the TeV catalog provided by the university of Chicago: PWN for Pulsar Wind Nebula, PSR for Pulsars, UNID for Unidentified sources, Massive Star Cluster and Star Forming Regions. The 3$^{rd}$ column reports the Galactic coordinates for each source. The 4$^{th}$ column presents the shape that best describes the source at TeV energies, together with a reference in the  5$^{th}$ column.}

\end{deluxetable}


\clearpage
\begin{deluxetable}{llllc}
\tabletypesize{\scriptsize}
\tablecaption{List of sources with a pulsar within 0.5$\degr$
\label{tab:pulsars}}
\tablehead{\colhead{Source name} & \colhead{Pulsar Name} & \colhead{distance ($\degr$)} & \colhead{pulsar 2FGL name} & \colhead{Included in the model}}
\startdata
HESS J1018-589 & PSR J1016-5857 & 0.22 & 2FGL J1016.5-5858 & N \tablenotemark{a}\\
HESS J1023-577 & PSR J1023-5746 & 0.05 & 2FGL J1022.7-5741 & N \tablenotemark{b}\\
HESS J1026-5819 & PSR J1028-5819 & 0.27 & 2FGL J1028.5-5819 & Y\\
HESS J1119-614 & PSR J1119-6127 & 0.07 & 2FGL J1118.8-6128 & N \tablenotemark{a}\\
HESS J1356-645 & PSR J1357-6429 & 0.12 & 2FGL J1356.0-6436 & N \tablenotemark{a}\\
HESS J1418-609 & PSR J1418-6058 & 0.05 & 2FGL J1418.7-6058 & N \tablenotemark{b}\\
HESS J1420-607 & PSR J1420-6048 & 0.05 & 2FGL J1420.1-6047 & N \tablenotemark{b}\\
HESS J1458-608 & PSR J1459-6053 & 0.17 & 2FGL J1459.4-6054 & N \tablenotemark{a}\\
HESS J1514-591 & PSR J1513-5908 & 0.03 & - & N \tablenotemark{b}\\
HESS J1702-420 & PSR J1702-4128 & 0.53 & -- & N \tablenotemark{c}\\
HESS J1708-443 & PSR J1709-4429 & 0.25 & 2FGL J1709.7-4429 & N \tablenotemark{b}\\
HESS J1718-385 & PSR J1718-3825 & 0.13 & 2FGL J1718.3-3827 & N  \tablenotemark{a}\\
HESS J1804-216 & PSR J1803-2149 & 0.27 & 2FGL J1803.3-2148 & N \tablenotemark{b}\\
HESS J1833-105 & PSR J1833-1034 & 0.01 & 2FGL J1833.6-1032 & N  \tablenotemark{a}\\
MGRO J0631+105 & PSR J0631+1036 & 0.10 & 2FGL J0631.5+ 1035 & N  \tablenotemark{a}\\
MGRO J0632+17 & PSR J0633+1746 & 0.00 & 2FGL J0633.9+1746 & N  \tablenotemark{a}\\
MGRO J1908+06 & PSR J1907+0602 & 0.23 & 2FGL J1907.9+0602 & N  \tablenotemark{a}\\
MGRO J1958+2848 & PSR J1958+2846 & 0.12 & 2FGL J1958.6+2845 & N  \tablenotemark{a}\\
MGRO J2019+37 & PSR J2021+3651 & 0.36 & 2FGL J2021.0+3651 & N \tablenotemark{b}\\
MGRO J2228+61 & PSR J2229+6114 & 0.09 & 2FGL J2229+6114 & N  \tablenotemark{a}\\
VER J0006+727 & PSR J0007+7303 & 0.26 & 2FGL J0007.0+7303 & N  \tablenotemark{a}\\
\enddata\\
\begin{flushleft}
a- The distance between the pulsar and the source is lower than 0.27$\degr$.\\
b- The pulsar is located inside the edge of the TeV shape.\\
c- The pulsar shown no significant excess.\\
\end{flushleft}
\tablecomments{ $_{ }$List of sources having a known $\gamma$-ray pulsar within 0.5$\degr$. The two first column shows the TeV source and the pulsar name. The third column show the distance between the TeV source and the GeV pulsar. The pulsar position comes from \textbf{Citer 2PC}. In the fourth colum Y means that we added the pulsar to the model and N that we didn't added it.}

\end{deluxetable}

\clearpage
\tabletypesize{\scriptsize}
\begin{deluxetable}{l*{5}l}
\tablewidth{0pt}
\tablecaption{Locations and spectral parameters of additional background sources above 10 GeV 
\label{tab:newsources}}
\tablehead{\colhead{Source name} & \colhead{Galactic coordinates ($\degr$)} & \colhead{$\text{TS}$} & \colhead{Prefactor} & \colhead{Index}}
\startdata
2FGL J1504.5--6121 & (311.81,0.30) & 30.7 & ( 1.2 $\pm$ 0.4 ) $\times 10^{-15}$ & 1.8 $\pm$ 0.3 \\
2FGL J1836.8--0623c & (25.41,0.42) & 25.1 & ( 9.4 $\pm$ 1.9 ) $\times 10^{-16}$ & 2.0 $\pm$ 0.4 \\
2FGL J1823.1--1338c & (17.51,-0.12) & 30.4 & (4.9 $\pm$ 0.5) $\times 10^{-15}$  & 2.9 $\pm$ 0.7\\
PSR J1838--0536 & (26.28,0.62) & 16.1 & (5.0 $\pm$ 1.8) $\times 10^{-17}$ & 4.1 $\pm$ 1.0 \\
Background Source 1 & (333.59,-0.31) & 29 & (6.5 $\pm$ 2.5) $\times 10^{-17}$ & 4.3 $\pm$ 0.9\\
Background Source 2 & (336.96, -0.07) & 25.0 & (1.2 $\pm$ 0.4) $\times 10^{-15}$ & 1.9 $\pm$ 0.4\\
Background Source 3 & (339.37,-1.18) & 29.9 & (1.3 $\pm$ 0.4) $\times 10^{-15}$ & 1.5 $\pm$ 0.3\\
\enddata

\tablecomments{The first two columns describe the source names and their corresponding Galactic coordinates. The test statistic (TS) for the source significance is provided in the 3$^{rd}$ column. The spectral results are presented in columns 4 and 5 for a power-law model (Equation~\ref{PL}) with a scale parameter $E_0 = 56234$ MeV which corresponds to the middle of the 10--316 GeV interval in log scale. PSR~J1838$-$0536 has been added to help the morphology fit of HESS~J1841$-$055 which is a diffuse source. The spectral fit is consistent with the pulsar component (index of$\sim$ 5).}

\end{deluxetable}
\normalsize
\noindent

\clearpage
\tabletypesize{\scriptsize}
\begin{deluxetable}{l*{7}l}
\tablewidth{0pt}
\tablecaption{Morphological results for LAT-detected TeV sources above 10 GeV
\label{tab:Morphology_results}}
\tablehead{\colhead{TeV source} & \colhead{$\text{TS}_{TeV}$} & \colhead{$\text{TS}_{GeV}$} & \colhead{$\text{TS}_{ext}$} & \colhead{$\text{TS}_{GeV/TeV}$} & \colhead{Significance}  & \colhead{Morphology}}
\startdata
HESS J1018-589 & 29.1 & 29.1 & 0 & 0 & 0 & TeV\\
HESS J1023-575 & 52.0 & 49.4 & 1.5 & - & - & TeV\\
HESS J1119-614 & 27.3 & 27.3 & 12.2 & 0 & 0 & TeV \\
HESS J1303-631 & 37.2 & 52.5 & 24.9 & 15.3 & 3.1 & Disk\\
HESS J1356-645 & 25.8 & 28.8 & 2.5 & 3.0 & 1.2 & TeV\\
HESS J1418-609 & 30.4 & 30.9 & 0.1 & - & - & TeV\\
HESS J1420-607 & 41 & 41.2 & 1.8 & - & - & TeV\\
HESS J1507-622 & 20.9 & 22.1 & 7.2 & - & - & TeV\\
HESS J1514-591 & 169.5 & 147.3 & 10.4 & - & - & TeV\\
HESS J1614-518 & 119.6 & 146.3 & 59.6 & 26.7 & 4.5 & Disk\\
HESS J1616-508 & 76.2 & 79.6 & 19.9 & 3.4 & 1.0 & TeV\\
HESS J1632-478 & 120.8 & 143 & 52.4 & 22.2 & 4.3 & Disk\\
HESS J1634-472 & 28.7 & 29.6 & 5.2 & 0.9 & 0.5 & TeV\\
HESS J1640-465 & 49 & 52.9 & 0 & 3.9 & 1.5 & TeV\\
HESS J1708-443 & 721.8 & 1153.3 & 0 & 431.5 & & point--like\\
HESS J1804-216 & 131 & 137.8 & 41.1 & 6.8 & 1.8 & TeV\\
HESS J1825-137 & 55.7 & 78.1 & 27.9 & 22.4 & 4.0 & Disk\\
HESS J1834-087 & 31 & 28.7 & 2.6 & - & - & TeV\\
HESS J1837-069 & 74 & 106.7 & 45.8 & 32.7 & 5.1 & Disk\\
HESS J1841-055 & 52 & 65 & 38.3 & 13.0 & 2.8 & TeV\\
HESS J1848-018 & 18.7 & 18.6 & 0 & - & - & TeV\\
HESS J1857+026 & 52.5 & 53.1 & 12.1 & - & - & TeV\\
MGRO J0632+17 & 698.8 & 2056.3 & 2.5 & 1357.5 & &point--like\\
MGRO J1908+06 & 16.4 & 37.2 & 0.2 & 20.8 & 4.2 & point--like\\
MGRO J1958+2848 & 20.8 & 23.5 & 0.2 & 2.7 & 1.1 & TeV \\
MGRO J2019+37 & 31.1 & 98.5 & 0 & 61.3 & 8.0 & point--like\\
MGRO J2031+41 & 72.1 & 66.4 & 2.9 & - & - & TeV\\
MGRO J2228+61 & 94.5 & 114 & 0 & 19.5 & 4.0 & point--like\\
VER J0006+727 & 654.5 & 1205.6 & 1.3 & 551.1 &  & point--like\\
VER J2016+372 & 31.4 & 32.9 & 0.3 & 1.5 & 0.7 & TeV\\
\enddata

\tablecomments{Results of the morphological analysis for all LAT-detected TeV sources. The fits assumed either the TeV template defined in Table~\ref{tab:TeV_sources}, a point-source model or a uniform disk model (see Equation~\ref{eq:Disk}). The definition of $\text{TS}_{TeV}$, $\text{TS}_{GeV}$, $\text{TS}_{ext}$ and $\text{TS}_{GeV/TeV}$ is reported in Section~\ref{signi}.
The significance of the extension is measured using the $\text{TS}_{ext}$ criterium defined in Section~\ref{signi}.}

\end{deluxetable}
\normalsize
\noindent

\clearpage
\tabletypesize{\scriptsize}
\begin{deluxetable}{l*{5}l}
\tablewidth{0pt}
\tablecaption{Morphological results for LAT-detected TeV sources for which $\text{TS}_{GeV}$ is significantly better than $\text{TS}_{TeV}$ above 10 GeV
\label{tab:GeVmorph}}
\tablehead{\colhead{TeV source} & \colhead{Morphology} & \colhead{$\text{TS}_{ext}$} & \colhead{(l,b) $\degr$}  & \colhead{extension($\degr$)}}
\startdata
HESS J1303-631 & Disk & 24.9 & (304.44, -0.18) & 0.50 $\pm$ 0.05$_{stat}$ \\
HESS J1614-518 & Disk & 59.8 & (331.66, -0.66) & 0.42 $\pm$ 0.06$_{stat}$\\
HESS J1632-478 & Disk & 52.4 & (336.38, 0.19)& 0.45 $\pm$ 0.04$_{stat}$\\
HESS J1708-443 & point--like & 0 & (343.11,-2.70) & -\\
HESS J1825-137 & Disk & 27.9 & (17.56,-0.47) & 0.65 $\pm$ 0.04$_{stat}$\\
HESS J1837-069 & Disk & 45.8 & (25.08,-0.13) & 0.32 $\pm$ 0.05$_{stat}$\\
MGRO J0632+17 & point--like & 2.5 & (195.13, 4.27) &-\\
MGRO J1908+06 & point--like & 0.2 & (40.39, -0.79) & -\\
MGRO J2019+37 & point--like & 0 & (75.27, 0.14) & - \\
MGRO J2228+61 & point--like & 0 & (106.67, 2.93) & - \\
VER J0006+727 & point--like & 1.3 & (119.69, 10.47) &-\\
\enddata


\tablecomments{Results of the morphological analysis for LAT-detected TeV sources better described using the shape fitted at GeV energies. The fits assumed either a point-source model or a uniform disk model (see Equation~\ref{eq:Disk}) whose centroid and extension are provided in columns 4 and 5 respectively. The significance of the extension is measured using the $\text{TS}_{ext}$ criterium defined in Section~\ref{signi}.}

\end{deluxetable}
\normalsize
\noindent

\clearpage
\tabletypesize{\scriptsize}
\begin{deluxetable}{l*{5}l}
\tablewidth{0pt}
\tablecaption{Spectral fitting results for LAT-detected TeV sources above 10 GeV 
\label{tab:det_sources}}
\tablehead{
\colhead{TeV source} & \colhead{$\text{TS}$} & \colhead{F(10--316 GeV)} & \colhead{E(10--316 GeV)} & \colhead{$\Gamma$}\\
\colhead{} & \colhead{} &\colhead{($10^{-10}$ ph cm$^{-2}$ s$^{-1}$)} &\colhead{($10^{-12}$ erg cm$^{-2}$ s$^{-1}$)} &\colhead{}}
\startdata
HESS J1018-589 &$29.1$ & $1.7 \pm 0.5_{stat} \pm 0.6_{syst}$ & $7.1 \pm 3.1_{stat} \pm 3.4_{syst}$ & $2.41 \pm 0.49_{stat} \pm 0.49_{syst}$\\
HESS J1023-575 &$52.0$ & $4.5 \pm 0.9_{stat} \pm 2.1_{syst}$ & $24.6 \pm 6.8_{stat} \pm 9.5_{syst}$ & $2.04 \pm 0.26_{stat} \pm 0.32_{syst}$\\
HESS J1119-614 &$27.3$ & $2.1 \pm 0.6_{stat} \pm 0.7_{syst}$ & $10.4 \pm 3.9_{stat} \pm 4.0_{syst}$ & $2.15 \pm 0.37_{stat} \pm 0.38_{syst}$\\
HESS J1303-631 &$52.5$ & $5.9 \pm 1.1_{stat} \pm 4.0_{syst}$ & $43.5 \pm 10.0_{stat} \pm 23.4_{syst}$ & $1.71 \pm 0.19_{stat} \pm 0.39_{syst}$\\
HESS J1356-645 &$25.3$ & $1.2 \pm 0.4_{stat} \pm 0.5_{syst}$ & $16.8 \pm 6.9_{stat} \pm 6.9_{syst}$ & $0.99 \pm 0.39_{stat} \pm 0.40_{syst}$\\
HESS J1418-609 &$30.4$ & $4.1 \pm 1.0_{stat} \pm 1.3_{syst}$ & $10.7 \pm 3.8_{stat} \pm 4.3_{syst}$ & $3.54 \pm 0.81_{stat} \pm 0.57_{syst}$\\
HESS J1420-607 &$41.0$ & $3.2 \pm 0.9_{stat} \pm 1.0_{syst}$ & $23.1 \pm 5.7_{stat} \pm 6.3_{syst}$ & $1.91 \pm 0.27_{stat} \pm 0.31_{syst}$\\
HESS J1507-622 &$20.9$ & $1.5 \pm 0.5_{stat} \pm 0.5_{syst}$ & $6.7 \pm 1.8_{stat} \pm 3.0_{syst}$ & $2.33 \pm 0.43_{stat} \pm 0.48_{syst}$\\
HESS J1514-591 &$169.5$ & $6.4 \pm 0.9_{stat} \pm 1.3_{syst}$ & $46.8 \pm 8.7_{stat} \pm 9.4_{syst}$ & $1.72 \pm 0.15_{stat} \pm 0.17_{syst}$\\
HESS J1614-518 &$146.3$ & $12.1 \pm 1.4_{stat} \pm 3.1_{syst}$ & $78.7 \pm 12.2_{stat} \pm 19.7_{syst}$ & $1.85 \pm 0.14_{stat} \pm 0.18_{syst}$\\
HESS J1616-508 &$76.2$ & $9.4 \pm 1.4_{stat} \pm 2.3_{syst}$ & $47.1 \pm 9.3_{stat} \pm 10.5_{syst}$ & $2.16 \pm 0.19_{stat} \pm 0.20_{syst}$\\
HESS J1632-478 &$144.0$ & $15.3 \pm 1.7_{stat} \pm 5.3_{syst}$ & $94.8 \pm 13.8_{stat} \pm 14.2_{syst}$ & $1.91 \pm 0.14_{stat} \pm 0.19_{syst}$\\
HESS J1634-472 &$28.7$ & $5.1 \pm 1.2_{stat} \pm 2.5_{syst}$ & $30.1 \pm 8.7_{stat} \pm 12.5_{syst}$ & $1.97 \pm 0.26_{stat} \pm 0.29_{syst}$\\
HESS J1640-465 &$45.4$ & $4.9 \pm 1.0_{stat} \pm 1.7_{syst}$ & $29.4 \pm 7.6_{stat} \pm 8.2_{syst}$ & $1.95 \pm 0.23_{stat} \pm 0.20_{syst}$\\
HESS J1708-443 &$1153.2$ & $21.8 \pm 1.5_{stat} \pm 3.7_{syst}$ & $51.5 \pm 4.2_{stat} \pm 11.0_{syst}$ & $4.09 \pm 0.23_{stat} \pm 0.35_{syst}$\\
HESS J1804-216 &$131.2$ & $13.7 \pm 1.6_{stat} \pm 2.9_{syst}$ & $75.2 \pm 12.5_{stat} \pm 16.3_{syst}$ & $2.05 \pm 0.16_{stat} \pm 0.21_{syst}$\\
HESS J1825-137 &$78.1$ & $13.4 \pm 2.1_{stat} \pm 8.7_{syst}$ & $10.9 \pm 2.0_{stat} \pm 2.1_{syst}$ & $1.61 \pm 0.15_{stat} \pm 0.39_{syst}$\\
HESS J1834-087 &$30.3$ & $5.3 \pm 1.2_{stat} \pm 2.3_{syst}$ & $25.2 \pm 8.1_{stat} \pm 11.9_{syst}$ & $2.22 \pm 0.35_{stat} \pm 0.41_{syst}$\\
HESS J1837-069 &$106.7$ & $10.8 \pm 1.6_{stat} \pm 4.4_{syst}$ & $92.6 \pm 16.8_{stat} \pm 17.6_{syst}$ & $1.74 \pm 0.15_{stat} \pm 0.32_{syst}$\\
HESS J1841-055 &$52.0$ & $9.3 \pm 1.9_{stat} \pm 3.9_{syst}$ & $79.6 \pm 16.0_{stat} \pm 19.1_{syst}$ & $1.56 \pm 0.20_{stat} \pm 0.32_{syst}$\\
HESS J1848-018 &$18.7$ & $7.4 \pm 1.9_{stat} \pm 3.2_{syst}$ & $30.0 \pm 10.1_{stat} \pm 16.5_{syst}$ & $2.46 \pm 0.80_{stat} \pm 0.52_{syst}$\\
HESS J1857+026 &$52.5$ & $3.7 \pm 0.5_{stat} \pm 1.6_{syst}$ & $54.0 \pm 6.9_{stat} \pm 9.1_{syst}$ & $0.99 \pm 0.25_{stat} \pm 0.27_{syst}$\\
MGRO J0632+17 &$2056.8$ & $25.6 \pm 1.5_{stat} \pm 9.8_{syst}$ & $54.4 \pm 3.7_{stat} \pm 14.4_{syst}$ & $5.06 \pm 0.26_{stat} \pm 0.56_{syst}$\\
MGRO J1908+06 &$37.2$ & $2.3 \pm 0.7_{stat} \pm 1.2_{syst}$ & $26.2 \pm 2.3_{stat} \pm 3.4_{syst}$ & $6.17 \pm 1.17_{stat} \pm 1.50_{syst}$\\
MGRO J1958+2848 &$20.8$ & $1.3 \pm 0.4_{stat} \pm 0.6_{syst}$ & $3.0 \pm 1.1_{stat} \pm 1.4_{syst}$ & $4.36 \pm 1.09_{stat} \pm 1.17_{syst}$\\
MGRO J2019+37 &$98.5$ & $3.4 \pm 0.7_{stat} \pm 1.1_{syst}$ & $7.1 \pm 1.6_{stat} \pm 1.9_{syst}$ & $5.33 \pm 1.01_{stat} \pm 1.12_{syst}$\\
MGRO J2031+41 B&$72.1$ & $4.8 \pm 0.9_{stat} \pm 1.3_{syst}$ & $14.1 \pm 3.2_{stat} \pm 4.2_{syst}$ & $3.17 \pm 0.38_{stat} \pm 0.39_{syst}$\\
MGRO J2228+61 &$114.0$ & $2.9 \pm 0.5_{stat} \pm 0.5_{syst}$ & $8.5 \pm 2.0_{stat} \pm 2.5_{syst}$ & $3.21 \pm 0.43_{stat} \pm 0.45_{syst}$\\
VER J0006+727 &$1205.6$ & $12.4 \pm 0.9_{stat} \pm 1.3_{syst}$ & $31.6 \pm 3.2_{stat} \pm 4.2_{syst}$ & $3.85 \pm 0.38_{stat} \pm 0.39_{syst}$\\
VER J1959+208 &$111.5$ & $2.9 \pm 0.5_{stat} \pm 0.5_{syst}$ & $8.5 \pm 2.0_{stat} \pm 2.2_{syst}$ & $3.22 \pm 0.43_{stat} \pm 0.45_{syst}$\\
VER J2016+372 &$1204.6$ & $12.4 \pm 0.9_{stat} \pm 1.9_{syst}$ & $30.5 \pm 2.7_{stat} \pm 5.3_{syst}$ & $3.85 \pm 0.22_{stat} \pm 0.31_{syst}$\\
\enddata

\tablecomments{Results of the maximum likelihood spectral fits  
for LAT-detected TeV sources. The test statistic (TS) for the source significance is provided in column 2. 
Columns 3 and 4 list the photon flux F(10--316 GeV) and the 
energy flux G(10--316~GeV). The fits used the best morphology reported in Table~\ref{tab:GeVmorph} and a power-law spectral model (see Equation~\ref{PL}) with photon index $\Gamma$ given in column 5. 
The uncertainties on F(10--316 GeV), G(10--316~GeV), and $\Gamma$ correspond respectively to the statistical and systematics uncertainties (see Section~\ref{syst}).}

\end{deluxetable}
\normalsize
\noindent

\clearpage
\begin{landscape}
\begin{deluxetable}{l*{7}l}
\tabletypesize{\scriptsize}
\tablecaption{Spectral fitting results for LAT-detected TeV sources in three energy bands logarithmically-spaced
\label{tab:det_sources2}}
\tablewidth{0pt}
\tablehead{
\colhead{TeV source} & \colhead{$\text{TS (10--31 GeV)}$} & \colhead{F(10--31 GeV)} & \colhead{$\text{TS (31--100 GeV)}$} & \colhead{F(31--100 GeV)} & \colhead{$\text{TS (100--316 GeV)}$} & \colhead{F(100--316 GeV)}\\
\colhead{} &  \colhead{} & \colhead{($10^{-10}$ ph cm$^{-2}$ s$^{-1}$)} & \colhead{} & \colhead{($10^{-11}$ ph cm$^{-2}$ s$^{-1}$)} & \colhead{} & \colhead{($10^{-11}$ ph cm$^{-2}$ s$^{-1}$)}}
\startdata
HESS J1018-589 & 24.7 & 1.3 $\pm$ 0.5 $\pm$ 0.5 & 0.5 & $<$ 5.9 & 5.7 & $<$ 6.2 \\
HESS J1023-575 & 42.8 & 3.7 $\pm$ 0.8 $\pm$ 1.4 & 1.6 & $<$ 9.0 & 18.3 & 1.8 $\pm$ 0.5 $\pm$ 0.9\\
HESS J1119-614 & 17.1 & 1.6 $\pm$ 0.5 $\pm$ 0.5 & 1.0 & $<$ 7.5 & 11.0  & 1.3 $\pm$ 0.6 $\pm$ 0.6\\
HESS J1303-631 & 20.9 & 3.6 $\pm$ 0.9 $\pm$ 1.4 & 26.6 & 17.1 $\pm$ 0.5 $\pm$ 0.6& 7.8 & $<$ 10.7 \\
HESS J1356-645 & 0.2 & $<$ 9.4 & 14.7 & 1.4 $\pm$ 0.7 $\pm$ 0.4 & 10.7 & 6.5 $\pm$ 1.7 $\pm$ 1.7 \\
HESS J1418-609 & 28.7 & 3.7 $\pm$ 0.9 $\pm$ 1.2 & 1.7 & $<$ 10.5 & 0.2 & $<$ 5.4 \\
HESS J1420-607 & 18.1 & 2.4 $\pm$ 0.7 $\pm$ 0.7 & 12.0 & 6.8 $\pm$ 2.6 $\pm$ 3.0 & 12.0 & 4.0 $\pm$ 1.7 $\pm$ 2.1 \\
HESS J1507-622 & 18.2 & 1.4 $\pm$ 0.4 $\pm$ 0.4 & 2.8 & $<$ 7.0 & 0.5 & $<$ 3.9 &\\
HESS J1514-591 & 69.6 & 3.9 $\pm$ 0.7 $\pm$ 1.0 & 65.7 & 15.8 $\pm$ 3.9 $\pm$ 4.8 & 36.8 & 6.7 $\pm$ 2.5 $\pm$ 2.7\\
HESS J1614-518 & 73.4 & 7.9 $\pm$ 1.2 $\pm$ 2.6 & 52.4 & 27.6 $\pm$ 5.9 $\pm$ 11.6 & 27.1 & 9.1 $\pm$ 3.2 $\pm$ 4.6\\
HESS J1616-508 & 46.8 & 6.5 $\pm$ 1.2 $\pm$ 2.0 & 28.4 & 19.9 $\pm$ 5.4 $\pm$ 9.5 & 4.0 & $<$ 10.1 \\
HESS J1632-478 & 71.2 & 10.3 $\pm$ 1.5 $\pm$ 4.2 & 38.1 & 28.2 $\pm$ 6.4 $\pm$ 11.4 & 37.6 & 15.4 $\pm$ 4.4 $\pm$ 6.3 \\
HESS J1634-472 & 15.5 & 3.6 $\pm$ 1.0 $\pm$ 1.9 & 11.7  & 10.8 $\pm$ 3.7 $\pm$ 5.7 & 2.1 & $<$ 8.3 \\
HESS J1640-465 & 20.1 & 3.4 $\pm$ 0.9 $\pm$ 1.5 & 32.2 & 15.1 $\pm$ 4.5 $\pm$ 6.9 & 0 & $<$ 4.5 \\
HESS J1708-443 & 1131.3 & 22.0 $\pm$ 1.5 $\pm$ 3.2 & 22.0 & 5.0 $\pm$ 1.3 $\pm$ 2.0 & 0.0 & $<$ 6.3 \\
HESS J1804-216 & 83.7 & 9.6 $\pm$ 1.4 $\pm$ 2.5 & 37.1 & 24.6 $\pm$ 6.1 $\pm$ 10.7 & 20.8 & 10.4 $\pm$ 3.9 $\pm$ 5.1\\
HESS J1825-137 & 18.2 & 6.4 $\pm$ 1.7 $\pm$ 3.4 & 46.7 & 50.5 $\pm$ 9.4 $\pm$ 20.4 & 19.4 & 11.9 $\pm$ 4.4 $\pm$ 5.5 &\\
HESS J1834-087 & 21.9 & 4.4 $\pm$ 1.6 $\pm$ 1.9 & 7.0 & $<$ 18.7 & 2.5 & $<$ 9.7\\
HESS J1837-069 & 30.6 & 6.9 $\pm$ 1.4 $\pm$ 3.1 & 24.3 & 25.6 $\pm$ 6.7 $\pm$ 10.5 & 28.3 & 15.2 $\pm$ 4.6 $\pm$ 5.9 \\
HESS J1841-055 & 22.8 & 6.4 $\pm$ 1.6 $\pm$ 3.0 & 11.0 & 19.6 $\pm$ 7.1 $\pm$ 5.7 & 22.1 &14.5 $\pm$ 4.5 $\pm$ 7.0 \\
HESS J1848-018 & 16.0 & 5.8 $\pm$ 1.6 $\pm$ 3.1 & 4.2 & $<$ 26.0 & 0.4 & $<$ 10.1\\
HESS J1857+026 & 1.9 & $<$ 2.5 & 12.9 & 13.8 $\pm$ 4.5 $\pm$ 6.8 & 39.3 & 13.0 $\pm$ 4.5 $\pm$ 4.6\\
MGRO J0632+17 & 2144.4 & 27.6 $\pm$ 1.6 $\pm$ 9.8 & 13.0 & 3.0 $\pm$ 1.0 $\pm$ 1.7 & 0 & $<$ 3.8 \\
MGRO J1908+06 & 32.1  & 2.6 $\pm$ 0.5 $\pm$ 1.2 & 0.0  & $<$ 5.0 & 2.8 & 8.6 \\
MGRO J1958+2848 & 18.9 & 1.3 $\pm$ 0.5 $\pm$ 0.4 & 0 & $<$ 3.3 & 0 & $<$3.0\\
MGRO J2019+37 & 100.1 & 3.7 $\pm$ 0.7 $\pm$ 1.5 & 0 & $<$ 3.2 & 2.4 & $<$ 4.9 \\
MGRO J2031+41 & 66.7 & 4.6 $\pm$ 0.8 $\pm$ 1.3 & 5.2 & $<$ 9.4 & 0 & $<$ 3.6\\
MGRO J2228+61 & 108.8 & 2.7 $\pm$ 0.5 $\pm$ 0.5 & 8.0 & $<$ 7.0 & 0 & $<$2.3 \\
VER J0006+727 & 1181.0 & 12.3 $\pm$ 0.9 $\pm$ 2.0 & 38.4 &  3.9 $\pm$ 1.6 $\pm$ 1.6 & 1.5 & $<$ 3.2\\
VER J2016+372 & 25.9 & 1.6 $\pm$ 0.4 $\pm$ 0.5 & 3.4 & $<$ 5.9  & 3.2& $<$ 5.0 \\
\enddata

\tablecomments{Results of the maximum likelihood spectral fits  
for LAT-detected TeV sources in three different energy bands: 10--31 GeV, 31--100 GeV, 100--316 GeV using the same convention as in Table~\ref{tab:det_sources}. A 99\% c.l. upper limit is computed when the TS in the band is lower than 10. }

\end{deluxetable}
\normalsize
\noindent
\end{landscape}

\clearpage
\tabletypesize{\scriptsize}
\begin{deluxetable}{lccc}
\tablewidth{0pt}
\tablecaption{Upper limits for non-detected TeV sources above 10 GeV
\tabletypesize{\small}
\label{tab:nondet_sources}}
\tablehead{
\colhead{TeV source} & \colhead{TS} &  \colhead{F(10--316~GeV) } & \colhead{G(10--316~GeV)}\\
\colhead{} & \colhead{} & \colhead{($10^{-10}$ ph cm$^{-2}$ s$^{-1}$)} & \colhead{($10^{-12}$ erg cm$^{-2}$ s$^{-1}$)}}
\startdata
HESS J1026-582 & 1.0 & $<$ 1.6 & $<$ 9.3 \\
HESS J1427-608 & 4.4 & $<$ 2.0 & $<$ 11.4\\
HESS J1458-608 & 12.6 & $<$ 2.5 & $<$ 14.5\\
HESS J1503-582 & 9.8 & $<$ 3.9 & $<$ 22.3\\
HESS J1554-550 & 0 & $<$ 0.5 & $<$ 2.8\\
HESS J1626-490 & 1.5 & $<$ 2.7 & $<$ 15.4\\
HESS J1702-420 & 8.5 & $<$ 6.8 & $<$ 38.9\\
HESS J1718-385 & 2.9 & $<$ 2.5 & $<$ 14.6\\
HESS J1729-345 & 0 & $<$ 1.3 & $<$ 7.6\\
HESS J1809-193 & 10.4 & $<$ 11.0 & $<$ 63.4\\
HESS J1813-178 & 2.5 & $<$ 2.4 & $<$ 14.2\\
HESS J1818-154 & 0 & $<$ 1.4 & $<$ 8.1\\
HESS J1831-098 & 0 & $<$ 1.8 & $<$ 10.6\\
HESS J1833-105 & 4.1 & $<$ 2.1 & $<$ 11.7\\
HESS J1843-033 & 0 & $<$ 1.0 & $<$ 5.4\\
HESS J1846-029 & 2 & $<$ 2.0 & $<$ 11.2\\
HESS J1849-000 & 0.1 & $<$ 1.3 & $<$ 7.3\\
HESS J1858+020 & 0 & $<$ 1.2 & $<$ 6.6\\
HESS J1912+101 & 9.5 & $<$ 4.5 & $<$ 25.2\\
MGRO J0631+105 & 5.9 & $<$ 1.4 & $<$ 7.7 \\
MGRO J1844-035 & 0 & $<$ 1.4 & $<$ 8.0\\
MGRO J1900+039 & 0 & $<$ 1.2 & $<$ 6.9\\
MGRO J2031+41 & 14.7 & $<$ 30.0 & $<$ 168.9\\
VER J1930+188 & 0 & $<$ 1.0 & $<$ 5.5\\
VER J1959+208 & 0 & $<$ 0.3 & $<$ 1.9\\
W49 A & 3.2 & $<$ 2.4 & $<$ 37.6\\
\enddata


%j'en suis la

\tablecomments{ Results of the maximum likelihood spectral fits  
for non-detected TeV sources. The test statistic (TS) for the source significance is provided in column 2. 
Columns 3 and 4 list the 99\% c.l. upper limits on the photon flux F(10--316 GeV) and on the 
energy flux G(10--316~GeV). The fits used the TeV template defined in Table~\ref{tab:TeV_sources} and a power-law spectral model (see Equation~\ref{PL}) with photon index fixed at 2. }

\end{deluxetable}
\normalsize
\noindent

\clearpage
\begin{landscape}
\begin{deluxetable}{lcccccc}
\tabletypesize{\scriptsize}
\tablecaption{Upper limits for non-detected TeV sources in three energy bands logarithmically-spaced between 10 and 316 GeV
\tabletypesize{\scriptsize}
\label{tab:nondet_sources2}}
\tablewidth{0pt}
\tablehead{
\colhead{TeV source} & \colhead{$\text{TS (10--31 GeV)}$} & \colhead{F(10--31 GeV)} & \colhead{$\text{TS (31--100 GeV)}$} & \colhead{F(31--100 GeV)} & \colhead{$\text{TS (100--316 GeV)}$} & \colhead{F(100--316 GeV)}\\
\colhead{} &  \colhead{} & \colhead{($10^{-10}$ ph cm$^{-2}$ s$^{-1}$)} & \colhead{} & \colhead{($10^{-11}$ ph cm$^{-2}$ s$^{-1}$)} & \colhead{} & \colhead{($10^{-11}$ ph cm$^{-2}$ s$^{-1}$)}}
\startdata
HESS J1026-582 & 0.5 & $<$ 1.6  & 0.0 & $<$ 3.6 & 0.6& $<$ 7.4\\
HESS J1427-608 & 0.3 & $<$ 1.3 & 2.6 & $<$ 9.0 & 3.3 & $<$ 6.2\\
HESS J1458-608 & 12.3 & 1.7 $\pm$ 0.5 $\pm$ 0.7 & 0.4 & $<$ 5.0 & 0.0 & $<$ 3.4\\
HESS J1503-582 & 0.9 & $<$ 2.2& 4.3 & $<$ 14.8 & 5.2 & $<$ 9.9\\
HESS J1554-550 & 0.0 & $<$ 0.6 & 0.0 & $<$ 3.3 & 0.0 & $<$ 2.8\\
HESS J1626-490 & 0.7 & $<$ 2.2 & 0.9 & $<$ 10.7 & 0.0 & $<$ 7.2\\
HESS J1702-420 & 0.3 & $<$ 3.3 & 7.9 & $<$ 36.2 & 1.0 & $<$ 10.7\\
HESS J1718-385 & 0.2 & $<$ 1.7 & 0.3 & $<$ 8.9 & 3.0 & $<$ 10.2\\
HESS J1729-345 & 0.0 & $<$ 1.4 & 0.0 & $<$ 7.5  & 0.0 & $<$ 4.5 \\
HESS J1809-193 & 8.4 & $<$ 9.3 & 3.3 & $<$ 32.4 & 0.0 & $<$ 11.5\\
HESS J1813-178 & 0.0 & $<$ 1.4 & 3.0 & $<$ 16.3 & 0.5 & $<$ 7.1\\
HESS J1818-154 & 0.0 & $<$ 1.5 & 0.0 & $<$ 7.8 & 0.0 & $<$ 4.4\\
HESS J1831-098 & 0.0 & $<$ 1.5 & 0.1 & $<$ 10.1 & 0.2 & $<$ 6.2\\
HESS J1833-105 & 3.1 & $<$ 1.9 & 1.2 & $<$ 7.1 & 0.0 & $<$ 6.5\\
HESS J1843-033 & 0.0 & $<$ 1.0 & 0.0 & $<$ 4.9 & 0.0 & $<$ 5.3\\
HESS J1846-029 & 2.1 & $<$ 2.3 & 0.0 & $<$ 5.2 & 0.0 & $<$ 4.1\\
HESS J1849-000 & 0.2 & $<$ 1.4 & 0.0 & $<$ 4.8 & 0.0 & $<$ 4.7\\
HESS J1858+020 & 0.0 & $<$ 1.2 & 0.0 & $<$ 6.4 & 0.0 & $<$ 4.0\\
HESS J1912+101 & 0.9 & $<$ 2.4 & 6.9 & $<$ 19.5 & 3.2 & $<$ 10.7\\
MGRO J0631+105 & 3.5 & $<$ 1.2 & 3.2 & $<$ 6.1 & 0.0 & $<$ 3.6\\
MGRO J1844-035 & 0.0 & $<$ 1.1 & 0.4 & $<$ 2.5 & 0.0 & $<$ 3.5\\
MGRO J1900+039 & 0.0 & $<$ 1.4 & 0.0 & $<$ 6.3 & 0.0 & $<$ 4.4\\
MGRO J2031+41 & 8.7 & $<$ 22.6 & 6.1 & $<$ 83.6 & 0.9 & $<$ 22.8\\
VER J1930+188 & 0.7 & $<$ 1.1 & 0.0 & $<$ 4.4 & 0.0 & $<$ 3.1\\
VER J1959+208 & 0.0 & $<$ 0.3 & 0.0 & $<$ 2.9 & 0.0 & $<$ 3.9\\
W49 A & 3.0 & $<$ 5.5 & 0.5 & $<$ 16.4 & 0.0 & $<$ 8.0\\
\enddata


\tablecomments{Results of the maximum likelihood spectral fits  
for non-detected TeV sources in three different energy bands: 10--31 GeV, 31--100 GeV, 100--316 GeV using the same convention as in Table~\ref{tab:nondet_sources}. A 99\% c.l. upper limit is computed when the TS in the band is lower than 10.}

\end{deluxetable}
\normalsize
\noindent
\end{landscape}

%\begin{deluxetable}{lccc}
%\tabletypesize{\scriptsize}
%\tablecaption{Spectral results for sources with a $\gamma$-ray detected pulsar within 0.5$\degr$. 
%\tabletypesize{\scriptsize}
%\label{tab:pulsarfit}}
%\tablewidth{0pt}
%\tablehead{\colhead{Source name} & \colhead{TS} & \colhead{Flux} & \colhead{Index }\\
\colhead{} & \colhead{} &\colhead{($10^{-10}$ ph cm$^{-2}$ s$^{-1}$)} &\colhead{}}
\startdata
HESS J1018-589 & & &\\
HESS J1023-577 & 51.2 & 4.5 $\pm$ 0.9$_{stat}$ & 2.04 $\pm$ 0.26$_{stat}$\\
HESS J1119-614 & 16.0 & 1.5 $\pm$ 0.6$_{stat}$ & 1.80 $\pm$ 0.48$_{stat}$\\
HESS J1356-645 & 25.4 & 1.2 $\pm$ 0.6$_{stat}$ & 1.00 $\pm$ 0.38$_{stat}$\\
HESS J1418-609 & 14.9 & $<$ 4.4  & -\\
HESS J1420-607 & 35.8 & 3.4 $\pm$ 0.9$_{stat}$ & 1.80 $\pm$ 0.29$_{stat}$ \\
HESS J1458-608 & 11.8 & $<$2.5 & - \\
HESS J1708-443 & 32.4 & 5.4 $\pm$ 1.3$_{stat}$ & 2.11 $\pm$ 0.32$_{stat}$\\
HESS J1718-385 & 2.9 & $<$ 2.5 e-10\\
HESS J1804-216 & 121.0 & 13.1 $\pm$ 1.6$_{stat}$ & 2.00 $\pm$ 0.16$_{stat}$\\
HESS J1833-105 & 4.1 & $<$2.1 e-10 & - \\
MGRO J0631+105 & 2.5 & $<$ 1.0e-10 & - \\
MGRO J0632+17 & 9.2 & $<$ 5.5 & - \\
MGRO J1908+06 & 9.0 & $<$ 5.5 & -\\
MGRO J1958+2848 & 16.6 & 1.2 $\pm$ 0.4$_{stat}$ & 4.3 $\pm$ 1.08$_{stat}$\\
MGRO J2228+61 & 14.7 & $<$ 2.0 & - \\
VER J0006+727 & 2.2 & $<$ 1.2 & -\\
\enddata

%\tablecomments{{\bf {\color{red} Results of the maximum likelihood spectral fits  
%for sources with a known $\gamma$-ray pulsar within 0.5 $\degr$. The pulsar is included in the model with the spectral parameters derived in the 2FGL catalog \cite{2012ApJS..199...31N}. The source of interest (column 1) is fitted assuming the TeV morphology. A 99\% c.l. upper limit is computed when the TS is lower than 16.}}}

%\end{deluxetable}
%\normalsize
%\noindent

\begin{deluxetable}{lcccccc}
\tabletypesize{\scriptsize}
\tablecaption{Spectral results for sources with a $\gamma$-ray detected pulsar within 0.5$\degr$ with the pulsar included in the model
\tabletypesize{\scriptsize}
\label{tab:pulsarfit}}
\tablewidth{0pt}
\tablehead{\colhead{Source name} & \colhead{TS} & \colhead{F(10-316 GeV)} & \colhead{Index} & \colhead{shape}
\\\colhead{} & \colhead{} &\colhead{($10^{-10}$ ph cm$^{-2}$ s$^{-1}$)} &\colhead{} &\colhead{} &\colhead{}}
\startdata
HESS J1018-589 & 25.0 & 1.5 $\pm$ 0.5$_{stat}$ & 2.31 $\pm$ 0.5$_{stat}$ & TeV\\
HESS J1023-577 & 51.2 & 4.5 $\pm$ 0.9$_{stat}$ & 2.04 $\pm$ 0.26 & TeV\\
HESS J1119-614 & 16.0 & 1.5 $\pm$ 0.6$_{stat}$ & 1.80 $\pm$ 0.48$_{stat}$ & TeV\\
HESS J1356-645 & 25.4 & 1.2 $\pm$ 0.4$_{stat}$ & 1.00 $\pm$ 0.38$_{stat}$ & TeV\\
HESS J1418-609 & 12.6 & $<$ 4.4 & - & TeV\\
HESS J1420-607 & 35.8 & 3.4 $\pm$ 0.9$_{stat}$ & 1.80 $\pm$ 0.29$_{stat}$ & TeV\\
HESS J1458-608 & 11.0 & $<$2.5e-10  & - & TeV\\
HESS J1708-443 & 62.7 & 8.3 $\pm$ 1.5$_{stat}$ & 3.16 $\pm$ 0.34$_{stat}$ & point-like\\
HESS J1718-385 & 2.9 & $<$ 2.5 & - & TeV\\
HESS J1804-216 & 121.0 & 13.1 $\pm$ 1.6$_{stat}$ & 2.00 $\pm$ 0.16$_{stat}$ & TeV\\
HESS J1833-105 & 4.1 & $<$2.1 & - & TeV\\
MGRO J0631+105 & 2.5 & $<$ 1.0 & - & TeV\\
MGRO J0632+17 & 128.4 & 12.5 $\pm$ 1.5$_{stat}$& 4.09 $\pm$ 0.33$_{stat}$& point-like\\
MGRO J1908+06 & 9.0 & $<$ 5.5  &  - & TeV\\
MGRO J1958+2848 & 18.6 & 1.2 $\pm$ 0.4$_{stat}$ & 3.9 $\pm$ 1.0$_{stat}$ & point-like\\
MGRO J2019+37 & 16.8 & 2.1 $\pm$ 0.8$_{stat}$& 4.61 $\pm$ 0.76$_{stat}$ & point-like &\\
MGRO J2228+61 & 18.4 & 1.5 $\pm$ 0.5$_{stat}$ & 2.58 $\pm$ 0.49$_{stat}$ & point-like\\
VER J0006+727 & 27.9 & 3.1 $\pm$ 0.9$_{stat}$ & 2.79 $\pm$ 0.36 & point-like\\
\enddata

\tablecomments{Results of the maximum likelihood spectral fits  
for sources with a known $\gamma$-ray pulsar within 0.5 $\degr$. The pulsar is included in the model with the spectral parameters derived in the 2FGL catalog \cite{2012ApJS..199...31N}. A 99\% c.l. upper limit is computed when the TS is lower than 16. As PSR~J1028-5819 is already included and fitted in our model of the region of HESS J1026-5819 we did not report it in this table. PSR J1513-5908 and PSR J1702-4128 were detected after the 2FGL catalog analysis \citep{2012ApJS..199...31N}. Thus no spectral parameters were available.}

\end{deluxetable}
\normalsize
\noindent

