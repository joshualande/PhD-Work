\section{Results}
\label{res}

Using the procedure described above, we detected 30 sources among the 58 PWNe candidates selected. For each of these detected sources, as explained in Section~\ref{signi}, we determined the best morphology by comparing the likelihood of our fit obtained under three different hypotheses: TeV shape, point-like source and extended source. The results of the spatial analysis are shown in Table~\ref{tab:Morphology_results}, the last column summarizing the best shape found for each source. Once the best morphology found, we performed a spectral analysis whose results are reported in Tables~\ref{tab:det_sources} \& \ref{tab:det_sources2}, while upper limits on non-detections are presented in Tables ~\ref{tab:nondet_sources} \& \ref{tab:nondet_sources2}.

\subsection{Extended sources detected above 10~GeV}
\label{morph_res}

Most sources are better described using the TeV morphology as a template. This is partly due to the 2 (3 for an extended source) additional degrees of freedom to take into account but also to the low statistics above 10~GeV. Indeed, these sources have on average a low TS value or are at the limit of our extension threshold, such as HESS J1514-591.

Interestingly, 11 sources are better described using the morphology obtained at GeV energies using the LAT data, 5 of them being extended. Table \ref{tab:GeVmorph} summarizes the spatial fits in such a case. The results presented here are consistent with Tables 3 and 4 of \cite{2012arXiv1207.0027L}. One can note that the agreement for HESS~J1632$-$478 is not excellent. The difference in extension (0.35 $\pm 0.06$ vs 0.45 $\pm$ 0.04 in our analysis) certainly comes from the 3 additional 2FGL sources (2FGL~J1631.7$-$4720c, 2FGL~J1630.2$-$4752 and 2FGL~J1632.4$-$4820c) in the model used by \cite{2012arXiv1207.0027L}. These sources were below our $\text{TS}>$25 threshold to add a source, and being unidentified, we found no physical reason to add them in our model.

HESS~J1303$-$631 is a new extended source detected at GeV energies. However, Figure~\ref{1303} shows that the \emph{Fermi}-LAT excess observed for this source is likely due to 2 point-like sources (one associated with HESS~J1303$-$631 and one associated with Kes 17 \citep{2011ApJ...740L..12W}) but the limited statistics at high energy does not allow us to separate them. When the region is fitted assuming the TeV morphology for HESS~J1303$-$631 and a separate source for Kes 17, the latter hardly reaches $\text{TS} \sim$ 20 and is therefore too faint to be added to our model. Nevertheless, the difference between the true spectrum and the spectrum derived in this work will be included in the systematics on the flux taking into account the fact that we do not know the true morphology of the source.

\subsection{Pulsars detected above 10~GeV}
\label{pulsarsect}
Only six sources are better described by a point-source model above 10 GeV than by a uniform disk or by the TeV template reported in Table~\ref{tab:TeV_sources}: HESS~J1708$-$443, MGRO J0632+17, MGRO J1908$+$06, MGRO J2019$+$37, MGRO J2228+61 and VER J0006+727. It is not a surprise to see in Figure~\ref{fig:indexvssize} and Table~\ref{tab:det_sources} that these sources have a rather soft spectrum in comparison to the average index of TeV detected sources and that they are all coincident with bright $\gamma$-ray pulsars. As can be seen on Figure \ref{fig:sedsourcespuls}, the spectrum obtained above 10 GeV for these sources is in good agreement with the spectrum derived in the 2FGL Catalog released by the Fermi collaboration \citep{2012ApJS..199...31N}. The $\gamma$-ray emission detected by the LAT above 10 GeV is therefore very likely due to the pulsar itself than to its associated PWN. 

Three other sources are coincident with bright Fermi pulsars and present a very soft spectrum in agreement with the 2FGL Catalog as well: HESS J1418-609, MGRO J1958$+$2848 and MGRO~J2031+41. Again, the $\gamma$-ray signal is very likely due to magnetospheric emission. The faintness of these sources above 10 GeV and their almost point-like shape in the TeV energy range can explain why the improvement obtained using a point source model is not significant in comparison to the simple TeV morphology.

To study the contamination of these pulsars to our sources of interest, we used the procedure described in Section 4 but including the pulsars in our models of the regions. The spectra of \emph{Fermi}-LAT pulsars are well characterized by exponentially cutoff power laws with photon indices near 1.5 and cutoff energies between 0.5 and 6 GeV \citep{2PC}. As we selected the data above 10 GeV, we cannot fit the spectral parameters of theses pulsars. Therefore we included them in our model assuming the fixed photon index, cutoff energy and normalization extracted from the 2FGL Catalog. It should be noted that some pulsars may deviate from the simple exponential cutoff power law above 10 GeV. This has been proposed for instance for the famous case of the Crab pulsar \citep{2011Sci...334...69V, 2008Sci...322.1221A}. In such cases, our fit could still be contaminated by the pulsar, especially in the first energy interval (between 10 and 31.6 GeV).

Table~\ref{tab:pulsarfit} and Figures~\ref{fig:sedsourcespuls}, \ref{fig:sedsourcespuls2} \& \ref{fig:sedsourcespuls3} show the results of this new fit using the conventions presented in Table \ref{tab:pulsars}. As expected for the six point-like sources presented in Figure~\ref{fig:sedsourcespuls}, the low energy part of the spectrum tends to disappear confirming that we only detected a pulsar emission. In the case of MGRO J1908+06, the detection is not significant anymore. To present conservative results, we derived the upper limits assuming the TeV shape of this source.

Figures~\ref{fig:sedsourcespuls2} \& \ref{fig:sedsourcespuls3} show the same behaviour for HESS~J1418-609 confirming that the signal observed was dominated by the pulsar emission. In the case of HESS~J1420-607 and HESS~J1119-614, the spectra are now slightly harder but still in very good agreement with the previous ones, which is a good indication that we are mainly seeing the emission from the PWNe and not from their associated pulsars. 

\subsection{Detections of PWNe candidates}
%{\bf {\color{red} My own choice looking at your previous results: Kooka, HESSJ1303, HESS J1356, G292.2, HESSJ1841 and HESSJ1848. Do a quick subsection for each of them (and modeling if needed) like in the PWN Cat paper}} 

In this section we will describe the new PWNe candidates found in this analysis. We chose them by looking for signal connecting to the TeV spectrum and showing a hard spectrum. HESS~J1420-607, HESS~J1303-631, HESS~J1356-645 \& HESS~J1119-614 were already proposed as PWNe by analyses at other wavelength. The detection of these sources by the \emph{Fermi}-LAT tends to confirm this hypothesis.

HESS~J1848-018, is classified as UNID. The detection presented in Table~\ref{tab:det_sources} shows a faint source with a soft spectrum. We will discuss this source in a PWN scenario. 

\subsubsection*{HESS J1420$-$607}

The complex of compact and extended radio/X-ray sources, called Kookaburra~\citep{1999ApJ...515..712R}, spans over about one square degree along the Galactic plane. It has been extensively studied to explain the EGRET source: 3EG J1420$-$6038/GeV J1417$-$6100 \citep{1999ApJS..123...79H}. Within the North-East excess of this complex, labeled 'K3',  was discovered the pulsar PSR J1420$-$6048, a young and energetic pulsar with period 68 ms, characteristic age of 13 kyr, and spin down energy of $10^{37}$ erg s$^{-1}$ \citep{2001ApJ...552L..45D}. Following X-ray observations by ASCA and later by \emph{Chandra} and \emph{XMM-Newton} revealed an extended X-ray emission around this pulsar identified as a potential PWN \citep{2001ApJ...561L.187R,2005ApJ...627..904N}. In the South-West side of the large Kookaburra complex lies a bright nebula exhibiting an extended hard X-ray emission, G313.1+0.1, called the 'Rabbit'~\citep{1999ApJ...515..712R}. This X-ray excess was also proposed as a plausible PWN contributing to the $\gamma$-ray emission detected by EGRET.

In the TeV energy range, the survey of the Galactic plane by H.E.S.S. revealed two very high energy sources in this region: HESS J1420$-$607 and HESS J1418$-$609 \citep{2006AA...456..245A}. HESS J1420-607 is centered North of PSR~J1420$-$6048 (nearby the K3 nebula), while HESS J1418-609 is coincident with the Rabbit nebula. More recently, \emph{Fermi}-LAT detected pulsed $\gamma$-ray emission from PSR J1420$-$6048 and PSR J1418$-$6058, the latter being a new $\gamma$-ray pulsar found through blind frequency searches \citep{2010ApJS..187..460A, 2009Sci...325..840A}. This new pulsar is coincident with an X-ray source in the Rabbit PWN and has a spin-down power high enough to power the TeV PWN candidate HESS J1418$-$609. 

In our analysis, HESS J1420$-$607 is detected with a TS of 41 which corresponds to a significance of $\sim$ 6$\sigma$. Figure~\ref{fig:K3countsmap} shows two smoothed counts map centered on the location of the K3 nebula represented by the crossed circle. The Galactic and the isotropic diffuses emission were subtracted in order to show the excesses coming from HESS~J1420$-$607 (represented by the crossed circle) and HESS J1418$-$609 (represented by the crossed ellipse). The first counts map (Top) represents the excesses of the two sources above 10 GeV and the second (Bottom) represents the excesses above 30 GeV. One can see an apparent emission of the region of HESS~J1418$-$609 close to HESS~J1420$-$607 above 10 GeV. Therefore, above 10 GeV, the observed spectrum of HESS~J1420$-$607 is potentially contaminated by the residual emission from HESS J1418$-$607. However, on the counts map above 30 GeV, the residual emission from HESS J1418$-$609 disappears. This confirms the soft spectral index and the pulsar behaviour of the emission coming from the region of HESS~J1418$-$609. 

With a TS of 41.2, the point source hypothesis does not significantly improve the likelihood of our fit in comparison to the TeV morphology (Gaussian of 0.06$\degr$) presented in \cite{2006AA...456..245A}. Therefore, assuming the TeV shape, we found an integrated flux F(10-316 GeV)=$(3.2 \pm 0.9_{stat} \pm 1.0_{syst}) \times 10^{-10}$ ph cm$^{-2}$ s$^{-1}$, a spectral index of $\Gamma = 1.91 \pm 0.27_{stat} \pm 0.31_{syst}$ and an energy flux of G(10-316 GeV) = $(23.1 \pm 5.7_{stat} \pm 6.3_{syst}) \times 10^{-12}$ erg cm$^{-2}$ s$^{-1}$. In a second step, following the procedure presented in Section~\ref{pulsarsect}, we refitted the spectrum of HESS J1420$-$607 including PSR J1420$-$6048 in our model and fixing its spectral parameters found in the 2FGL Catalog. The fit of HESS~J1420$-$607 leads a lower significance of $~5.6 \sigma$ (TS=35.8, 2 d.o.f) with an integrated flux of F(10-316 GeV)= $(3.4 \pm 0.9_{stat} \pm 1.0_{syst}) \times 10^{-10}$ ph cm$^{-2}$ s$^{-1}$ and an index of $\Gamma = 1.80 \pm 0.29_{stat} \pm 0.32_{syst}$.  



%Due to the deviation from the simple exponential cut-off observed for several $\gamma$-ray pulsars, and specially for the Crab, the lower energy part of our analysis could still be contaminated by the pulsar even when the 2FGL source associated with the pulsar is added to the model. Therefore, the SED of HESS~J1420$-$607 is potentially contaminated by pulsar emission from PSR~J1420$-$6048 and PSR~J1418$-$6058 at low energy (10-30 GeV). However, it is unlikely contaminated by these components at high energy ($>$ 30 GeV) and the connection between the GeV flux as observed by \emph{Fermi}-LAT and the TeV flux as seen by H.E.S.S., visible in Figure \ref{fig:hessj1420}, supports a common origin for the $\gamma$-ray emission.  

%As discussed in Section \ref{pulsarsect}, another contamination could come from PSR~J1420$-$6048 and PSR~J1418$-$6058 (respectively 0.05$\degr$ and 0.27$\degr$ from the centroid of HESS~J1420$-$607) both detected by the LAT. However, the spectrum derived in this analysis is unlikely contaminated by this component at high energy ($>$ 30 GeV) and the connection between the GeV flux as observed by \emph{Fermi}-LAT and the TeV flux as seen by H.E.S.S., visible in Figure \ref{fig:hessj1420}, supports a common origin for the $\gamma$-ray emission.  


\cite{2010ApJ...711.1168V} used a two-zone time dependent numerical model with constant injection luminosity to investigate the physical properties of HESS J1420$-$607. The authors injected relativistic particles following a power-law spectrum into the inner nebula zone, they evolved this spectrum over time and injected the resultant spectrum into the outer nebula zone. On Figure \ref{fig:hessj1420} we reported in solid line the results obtained for a hadronic + leptonic model assuming a low density environment (n$\sim$1 cm$^{-3}$) and a magnetic field of $\sim$ 12$\mu$G and $\sim$ 9$\mu$G respectively in the inner and outer nebula. We also reported in dashed line the model proposed for a leptonic scenario assuming a magnetic field of $\sim$ 12$\mu$G and $\sim$ 8$\mu$G respectively in the inner and outer nebula. This magnetic field implied a lepton spectral break at $\sim$100 TeV after evolution in the inner nebula. More recently \cite{2012ApJ...750..162K} proposed a one-zone leptonic model assuming a power-law injection spectrum with an index of 2.3 with a cut-off around 40 TeV and a magnetic field around $\sim$ 3$\mu$G. We represented this model in dotted line on the same figure. All these models show a good agreement with the observed spectrum.




%Figure \ref{fig:hessj1420} shows the results obtained for a leptonic model in solid line and a leptonic + hadronic model in dashed line. We also reported in dotted line the leptonic model proposed by \cite{2012ApJ...750..162K} for a magnetic field of 3$\mu$G. All these models show a good agreement with the observed spectrum. 
However, as noted above, the low energy part of the $\emph{Fermi}$-LAT spectrum may be contaminated by the pulsed emission from PSR J1420$-$6048. This implies that, with the current statistics, all models reproduce reasonably well the GeV and TeV data. A future \emph{Fermi}-LAT off-pulse analysis of this pulsar performed with more statistics could help discriminate between the models.

\subsubsection*{HESS J1356$-$645}

HESS~J1356$-$645 is a source detected in the TeV energy range by H.E.S.S. during the Galactic Plane Survey~\citep{2011AA...533A.103H}. This extended source lies close to the pulsar PSR~J1357$-$6429 which was discovered during the Parkes multibeam survey of the Galactic Plane \citep{2004ApJ...611L..25C}. Its high spin-down power of $\dot{E} = 3.1 \times 10^{36}$ erg s$^{-1}$ makes it a good candidate to power a PWN. Archival radio and analysis of X-ray data from \emph{ROSAT/PSPC} and \emph{XMM/Newton} have revealed a faint extended structure coincident with the VHE emission~\citep{2011AA...533A.103H} providing another argument in favor of the PWN scenario. In parallel, \cite{2011AA...533A.102L} announced the detection of a pulsed signal from PSR~J1357$-$6429 in the $\gamma$-ray and X-ray energy ranges using \emph{Fermi}-LAT and \emph{XMM-Newton} data. However, using 29 months of LAT data between 0.1 and 100 GeV, no counterpart to the TeV emission was found in the off pulse window of the pulsar.

The 16 additional months of observations by \emph{Fermi}-LAT and the higher maximal energy used (316 GeV instead of 100 GeV) in our dataset now enables the detection of a faint counterpart to the TeV emission with a TS = 25.8 (4.7 $\sigma$ assuming 2 d.o.f.). Since the best GeV morphology does not improve the fit significantly, we used the TeV Gaussian of 0.17$\degr$ \citep{2011AA...533A.103H} for the spectral analysis and derived an integrated flux of F(10-316 GeV)=$(1.2 \pm 0.4_{stat} \pm 0.5_{syst}) \times 10^{-10}$ ph cm$^{-2}$ s$^{-1}$, an energy flux of G(10-316 GeV)=$(16.8 \pm 6.9 \pm 6.8) \times 10^{-12}$ erg cm$^{-2}$ s$^{-1}$ and a hard spectral index of $\Gamma = 0.99 \pm 0.39 \pm 0.40$. 

It should also be noted that, with its low energy cutoff at around 800 MeV in the \emph{Fermi}-LAT energy range \citep{2011AA...533A.102L}, PSR~J1357$-$6429 is not significant anymore in the 10 to 316 GeV energy range. Therefore, we do not expect to see any changes in the spectral parameters when adding PSR~J1357$-$6429 to the model of the region. This is verified in Table~\ref{tab:pulsarfit} as well as in Figure \ref{fig:hess1356}. 

The combined GeV-TeV data as seen in Figure \ref{fig:hess1356} provide new information concerning the spectral shape of the $\gamma$-ray emission. It is clearly visible in this figure that the \emph{Fermi}-LAT spectral points nicely match the H.E.S.S. ones, proving that the GeV and the TeV emission have a common origin. Assuming that the $\gamma$-ray signal is coming from the PWN powered by PSR~J1357$-$6429, \cite{2011AA...533A.103H} proposed a leptonic scenario (black curve) which provides an excellent fit of the new multi-wavelength data. This one-zone model is based on the evolution of an electron population injected following an exponentially cutoff power-law spectrum of index 2.5 and cut-off energy of 350 TeV. These electrons cool radiatively through of IC scattering on the Cosmic Microwave Background (CMB), Galactic infrared (T$\sim$ 35K and 350 K), optical (T$\sim$ 4600K) photons and through synchrotron emission in a magnetic field $\sim$ 3.5$\mu$G



%This one-zone model assumes a rather low magnetic field of $\sim 3.5 \mu$G similar to the value observed in other relic PWNe \citep{2008ApJ...689L.125D}. It comprises the Galactic infrared emission from dust at T $\sim$ 35K and 350 K) and optical emission from stars at T$\sim$ 4600 K. The electron spectrum was fitted as a simple exponentially cutoff power-law of index p=2.5 and E$_{cut}$= 350 TeV. These parameters remind the parameters found for others relic PWNe candidates as HESS J1857+026 \citep{Rousseau1857}.

The similarities between PSR J1357$-$6429 and the Vela pulsar and between their PWNe lead \cite{2011AA...533A.103H} to discuss a potential two leptonic-components emission. However, in the case of Vela-X, the "halo" is seen in the GeV and radio energy ranges and the "cocoon" in the TeV and X-ray energy ranges, while in the case of HESS J1356$-$645, a single lepton population explain the broad-band spectrum with reasonable parameters. However, unlike Vela-X, the current radio and X-ray data are very faint and do not provide any morphological constraints. Future multi-wavelength data are highly needed to better describe this source.

%For now, the faintness of the radio and X-ray emission prevent any clear association with the TeV emission and the poor informations on the morphology prevents one from distinguishing between shell-type and plerionic SNR as the origin of HESS~J1356-645 in these energy ranges. New radio and X-ray observations would be needed to confirm the association of the extended emission to HESS J1356-645 and better constrain the model. 

%However, radio observations are needed to confirm the association of the extended emission to HESS J1356-645 and better constrain the model. Furthermore, future VHE observations will help constraining the IC spectrum leading to a better knowledge of the magnetic field strength. In conclusion, we classify here HESS J1356-645 as another potential  PWN and others observations are needed to constrains the PWN model.

\subsubsection*{HESS~J1119$-$614}

During the Parkes multibeam pulsar survey, \cite{2000ApJ...541..367C} discovered PSR~J1119$-$6127, a young ($\tau = 1.6$ kyr) pulsar with a high spin-down power $\dot{E} = 2.3 \times 10^{36}$ erg s$^{-1}$ within the supernova remnant G292.2$-$0.5. Using \emph{Chandra} observations, \cite{2003ApJ...591L.143G} and \cite{2008ApJ...684..532S} revealed the presence of a faint and compact PWN close to this pulsar. More recently, a TeV $\gamma$-ray source coincident with PSR J1119$-$6127 and G292.2$-$0.5 was announced\footnote{http://cxc.harvard.edu/cdo/snr09/pres/DjannatiAtai\_Arache\_v2.pdf}.

Using the method described above, a faint signal consistent with the location of the composite SNR G292.2$-$0.5 is detected with a TS of 27.3 (4.9 $\sigma$ with 2 d.o.f.). Since the best GeV morphology does not improve the fit significantly, we used the TeV Gaussian of 0.05$\degr$ for the spectral analysis and derived an integrated flux of F(10 - 316 GeV) = $(2.1 \pm 0.6_{stat} \pm 0.7_{syst}) \times 10^{-10}$ ph cm$^{-2}$ s$^{-1}$, an energy flux of G(10-316 GeV) = $(10.4 \pm 3.9_{stat} \pm 4.0_{syst}) \times 10^{-10}$ erg cm$^{-2}$ s$^{-1}$ and a soft index of $\Gamma = 2.15 \pm 0.37_{stat} \pm 0.38_{syst}$.

Nevertheless, as can be seen on Figure \ref{fig:hessj1119} and in Table \ref{tab:pulsarfit}, these parameters are contaminated by a low energy component associated to PSR~J1119$-$6127. Once the source 2FGL~J1118.8$-$6128, associated with PSR~J1119$-$6127, included in our model of the region, this contamination decreases and the significance of our GeV source is now just above the detection threshold that we fixed in Section~\ref{signi}, with TS = 16 (3.6 $\sigma$ with 2 d.o.f.). As can be seen in Figure \ref{fig:hessj1119}, the low energy point of the SED is now an upper limit. The best fit parameters in this hypothesis are an integrated flux of F(10-316 GeV) = $(1.5 \pm 0.6_{stat} \pm 0.6_{syst})\times 10^{-10}$ ph cm$^{-2}$ s$^{-1}$ and an harder index of $\Gamma = 1.80 \pm 0.48_{stat} \pm 0.50_{syst}$. 

Figure \ref{fig:hessj1119} shows the multi-wavelength SED of HESS~J1119$-$614 along with the H.E.S.S. points from \cite{Mayerdiploma}. The leptonic model proposed by \cite{Mayerdiploma} (solid line) is a one-zone model where accelerated electrons cool radiatively by IC scattering on CMB photons, starlight photons radiated in the vicinity and photons radiated by dust and by synchrotron losses. It implies an initial period of the pulsar $P_0 =21.4$ ms, an initial magnetic field inside the PWN $B_0 = 406.1 \mu$G (leading to a current magnetic field of B$ \sim 32 \mu$G), a braking index of n=2.91 \citep{2000ApJ...541..367C} and a conversion efficiency of spin-down power into relativistic particles of $\eta = 0.3$. The model well matches the LAT and H.E.S.S. points.

The presence of PSR J1119-6127, the detection of a compact PWN in X-ray and the leptonic model proposed by \cite{Mayerdiploma} point toward the hypothesis in which the GeV-TeV emission comes from the PWN inside G292.2$-$0.5. Furthermore, the parameters derived in \cite{Mayerdiploma} and the presence of jets in the  X-ray data remind the case of MSH 15$-$52 \citep{2010ApJ...714..927A,1996PASJ...48L..33T}.

\subsubsection*{HESS J1303$-$631}

HESS J1303$-$631 was serendipitously discovered in 2004 \citep{2005AA...439.1013A} during an observation campaign of the pulsar binary system PSR B1259$-$63. It was originally classified as a "dark" accelerator due to the lack of detected counterparts in radio and X-rays with \emph{Chandra} \citep{2005ApJ...629.1017M}. \cite{dalton1303} found only one plausible counterpart in the vicinity of HESS~J1303-631 : PSR J1301$-$6305  with a spin-down power of $\dot{E} = 1.70 \times 10^{36}$ erg s$^{-1}$. The authors also presented the detection of a very weak X-ray PWN using \emph{XMM-Newton} observations and the energy dependent morphology at TeV energies leading to the conclusion that HESS J1303-631 is an old PWNe offset from the pulsar powering it.  


%At this time, a search in the field of view only yielded one plausible counterpart located in the North-Western edge of HESS J1303$-$631, the high spin-down power pulsar PSR J1301$-$6305 ($\dot{E} = 1.70 \times 10^{36}$ erg s$^{-1}$). The recent detection of a very weak X-ray PWN using \emph{XMM-Newton} observations now allows the solid identification of this source as a VHE PWN associated to the pulsar PSR J1301$-$6305, leading \cite{2011arXiv1104.1680D} to classify the TeV excess as an older PWN with a complex morphology and an important offset between the pulsar and nebula. This conclusion is supported by the energy dependant morphology presented in \cite{dalton1303}.

Figures 1 \& 3 in \cite{2011ApJ...740L..12W} show no significant emission coming from the location of HESS~J1303$-$631 using 30 months of \emph{Fermi}-LAT data between 1 and 20 GeV. With 15 months of additional data and a higher energy threshold, our analysis now provides a first detection of GeV emission coincident with the TeV source. Nevertheless as discussed in Section \ref{morph_res}, Figure \ref{1303} shows that the detected emission might be contaminated by a source associated to Kes 17. Since we cannot separate these two sources using our strategy with the current statistics, we decided to take into account this effect of source confusion in our systematics on HESS J1303$-$631. Therefore, we ran the analysis again, adding a source spatially consistent with Kes 17 and we performed a second fit. The maximal variation is seen for the lower energy bin of the SED.

Assuming our best GeV morphology represented by a Disk shape of 0.5$\degr$ (see Table \ref{tab:Morphology_results}), we obtained an integrated flux of F(10-316 GeV)=$(5.9 \pm 1.1_{stat} \pm 4.0_{syst}) \times 10^{-10}$ ph cm$^{-2}$ s$^{-1}$, an energy flux of G(10-316 GeV)= $(43.5 \pm 10.0_{stat} \pm 23.4_{syst}) \times 10^{-12}$ erg cm$^{-2}$ s$^{-1}$ and an index of $\Gamma = 1.71 \pm 0.19_{stat} \pm 0.39_{syst}$. This hard index is in the range of values obtained for PWNe as seen with \emph{Fermi}-LAT and is inconsistent with the spectral index of $\sim 2.4$ derived by \cite{2011ApJ...740L..12W} for Kes 17. This is a good evidence that the $\gamma$-ray emission above 10 GeV is dominated by the PWN candidate. As can be seen in Figure \ref{fig:hessj1303}, even though the connection between the GeV and the TeV energy range is not perfect, there is little doubt that we have detected a counterpart to the TeV signal.

Figure \ref{fig:hessj1303} shows the SED of HESS~J1303-631 together with the one-zone leptonic model proposed by \cite{dalton1303}. In this model, VHE $\gamma$-rays are created via IC scattering of electrons on the CMB photons, infrared and optical target photons being neglected. The model reproduces the radio, X-ray and TeV data with an electron spectral index of  $1.8^{+0.1}_{-0.1}$, a cut-off energy of $31^{+5}_{-4}$\,TeV, and an average magnetic field of $1.4^{+0.2}_{-0.2}\,\mu$G. However, the flux predicted in the GeV energy range is well below the flux detected by the \emph{Fermi}-LAT. This may be due to the absence of infrared and optical photon field or to contamination produced by Kes 17. A specific analysis is needed to conclude on the constraints that the \emph{Fermi}-LAT could bring on the $\gamma$-ray emission of this source. 


\subsubsection*{HESS J1841$-$055}

HESS~J1841$-$055 was discovered during the H.E.S.S. Galactic Plane Survey \citep{2008AA...477..353A} and remained unidentified since then. The emission is highly extended and shows possibly three peaks suggesting that the TeV emission is composed of more than one source. Using \emph{INTEGRAL} data, \cite{2009ApJ...697.1194S} proposed the  high-mass X-ray binary HMXB system AX J1841.0$-$0536 as a potential counterpart, at least for a part of the emission. \cite{2011ICRC....6..197T} proposed the association of HESS~J1841$-$055 to an ancient PWN powered by PSR~J1841$-$0524, PSR~J1838$-$0549 or both as each pulsar taken independently would need an efficiency greater than 100\% to power alone a potential PWN associated to the TeV source. More recently, the blind search detection of the new $\gamma$-ray pulsar PSR~J1838$-$0537 with \emph{Fermi}-LAT provided another good counterpart of the TeV source. Indeed, assuming a distance of 2 kpc, \cite{2012ApJ...755L..20P} estimated that PSR J1838$-$0537 is sufficiently energetic to power the whole TeV source with a conversion efficiency of 0.5\%, similar to other suggested pulsar/PWN associations \citep{2008ApJ...682L..41H}.

In this work, HESS~J1841$-$055 is detected as a significantly extended source (TS$_{ext}$= 38.3) at a position consistent with the TeV source but with a larger extension (0.57 $\degr$ with respect to 0.41 $\degr$ $\times$ 0.25 $\degr$ for the TeV source). However, the GeV best morphology does not significantly improve the fit compared to the TeV morphology (TS$_{GeV/TeV}$= 13.0, 2.8$\sigma$ with 3 d.o.f.). Therefore assuming the TeV shape, our best fit yielded an integrated flux of F(10-316 GeV) = $(9.3 \pm 1.9_{stat} \pm 3.9_{syst}) \times 10^{-10}$ for an energy flux of G(10-316 GeV) = $(79.6 \pm 16.0_{stat} \pm 19.1_{syst}) \times 10^{-12}$ erg cm$^{-2}$ s$^{-1}$ and a hard index of $\Gamma = 1.56 \pm 0.20_{stat} \pm 0.32_{syst}$ consistent with the average value for PWNe detected by the \emph{Fermi}-LAT.

As can be seen in Figure \ref{fig:1841}, the \emph{Fermi}-LAT spectral points nicely match the H.E.S.S. ones, suggesting a common origin. The hard \emph{Fermi}-LAT spectrum detected imply that a curvature must arise between the TeV energy range and the GeV energy range. This is typical of most PWNe detected by \emph{Fermi}-LAT and H.E.S.S. which present IC emission peaking at few hundreds of GeV and would favor the PWN scenario. However, this source is extremely extended in both wavelengths and could be composed of several $\gamma$-ray sources. Follow-up observations with IACTs and \emph{Fermi}-LAT would be needed to unveil the real nature of HESS J1841$-$055.


\subsubsection*{HESS J1848$-$018}

HESS~J1848$-$018 \emph{(TeV J1848$-$017)} was discovered during the extended H.E.S.S. Galactic Plane Survey \citep{chavesthesis}, in the direction of, but slightly offset from, the star forming region W43 (=G30.8-0.2). The TeV emission is characterized by a significant extension (0.32$\degr$) and a power-law spectrum of index $\sim 2.8$ and an integrated flux above 1 TeV $\sim 2 \times 10^{-12}$ cm$^{-2}$ s$^{-1}$. The absence of energetic pulsar or SNR within 0.5$\degr$ from HESS~J1848$-$018 favor an association with the star forming region W43. No obvious counterparts were found in radio and X-ray in the region of the TeV source except the Wolf-Rayet star WR 121a.



%The absence of pulsar or SNR able to explain the TeV emission favoured the association of HESS J1848-018 to the star forming region itself. 

%In a first step, a search for a radio or X-Ray counterpart led to no obvious counterpart for the TeV emission. However, the \emph{VII$^{th}$ Catalogue of Galactic Wolf-Rayet Stars} \citep{2001yCat.3215....0V} provided a new candidate \emph{WR 121a} (\emph{W43\#1}). 


Located 0.2$\degr$ from the centroid of HESS J1848$-$028, WR 121a is a WN7 subtype star, in a binary system \citep{2011AA...532A..92L}, associated with W43 and characterized by extreme mass loss rates. \cite{chavesthesis} also proposed an association with the molecular clouds contained in W43. These molecular clouds could lead to the production of high energy $\gamma$-rays from the $\pi^0$-meson decays following \emph{p-p} collisions in the ambient gas.

In the GeV energy ranges, \cite{2010AA...518A...8T} proposed an association with the spatially coincident source 0FGL J1848.6$-$0138. \cite{2011MmSAI..82..739L} analyzed the \emph{Fermi}-LAT data around HESS J1848$-$018 and found a 3.7$\sigma$ evidence for an extension ($\sigma \sim 0.3\degr$). This disfavours models in which the GeV emission would be produced by a pulsar alone. However, the statistic was not large enough to discriminate between one extended source and several point sources. Moreover, the spectrum was well described by a log-parabola and the SED was very similar to those otained for most pulsars detected by \emph{Fermi}-LAT. Therefore, the emission observed by the \emph{Fermi}-LAT could be an extended source contaminated by a radio-faint pulsar not yet detected.  



%However, studying the GeV spectrum, \cite{2011MmSAI..82..739L} shown that the GeV emission is more likely due to a pulsar component, for which the pulsation is not yet detected, instead of a counterpart to the TeV emission. \cite{2011AA...532A..92L} excluded a spatial association between 0FGL J1848.6-0138 and W43 using the position of 1FGL J1848.1-0145c \citep{2010ApJS..188..405A}. 

%Therefore, further analyses are needed to conclude on the association of HESS J1848-0138 either to W43/W121a or to 0FGL J1848.6-0138.

In our analysis, HESS~J1848$-$018 is detected as a faint source and the GeV best morphology does not significantly improve the fit in comparison to the TeV morphology. Therefore assuming the TeV shape, our best fit yielded an integrated flux of F(10-316 GeV) = $(7.4 \pm 1.9_{stat} \pm 3.2_{syst}) \times 10^{-10}$ cm$^{-2}$ s$^{-1}$ for an energy flux of G(10-316 GeV) = $(30.0 \pm 10.1_{stat} \pm 16.5_{syst}) \times 10^{-12}$ erg cm$^{-2}$ s$^{-1}$ and an index of $\Gamma = 2.46 \pm 0.80_{stat} \pm 0.52_{syst}$ consistent with the average value for PWNe detected by the \emph{Fermi}-LAT.

On Figure \ref{fig:1848}, we reported the H.E.S.S. spectral points from \cite{chavesthesis} and the radio points corresponding to the W43 central cluster from \cite{2011AA...532A..92L}. It is important to note that the point obtained in our analysis, is consistent with the red dashed curve which represent the spectrum derived by \cite{2011MmSAI..82..739L}. It is not absolutely clear from this Figure if the GeV and TeV spectra have a common or a distinct origin and future multi-wavelength data would be needed to discriminate between the Pulsar/PWN and massive star formation region and conclude on the nature of HESS J1848$-$018. 


%This lead to the conclusion that the emission detected in this work is not linked to the TeV emission but is consistent with the pulsar-like emission observed in \cite{2011MmSAI..82..739L}. Therefore, the discrimination between the Pulsar/PWN scenario and the massive star formation region needs help of others wavelengths to conclude on the true nature of HESS J1848$-$018.


\subsection{Constraints obtained from non-detections}
%{\bf {\color{red} My own choice looking at your previous results: HESSJ1026, HESSJ1458, HESSJ1626-490, HESSJ1813 and so on. Do a quick subsection for each of them (and modeling if needed) like in the PWN Cat paper}}

This section will present the sources for which the emission was not significant but for which the upper limits show an interesting behaviour to bring new constraints on the models.

\subsubsection*{HESS J1026$-$582}

HESS~J1026$-$582 is a source discovered by H.E.S.S. during a new analysis of the region of HESS~J1023$-$577 \citep{2011AA...525A..46H}. \cite{2011AA...525A..46H} have shown that the emission coming from this region could be splitted into two independent sources. While HESS~J1023$-$577 is located close to PSR~J1022$-$5746, the authors proposed an association between HESS~J1026$-$582 and PSR~J1028$-$5819, a pulsar detected by the Parkes Radio telescope \cite{2008MNRAS.389.1881K}. The proximity of the pulsar suggested a PWN scenario to explain the VHE emission. This hypothesis is consolidated by the spin-down power energy of PSR~J1028$-$5819, $\dot{E} = 8.43 \times 10^{35}$ erg s$^{-1}$ \citep{2009ApJ...695L..72A}, which is high enough to power a PWN. \cite{2012AA...543A.130M} compared the X-ray marginal emission observed in Suzaku data to a PWN scenario and conclude on a possible future detection. However, follow-up observations with \emph{XMM-Newton} and \emph{Chandra} are needed to confirm this detection. 

No significant GeV emission coming from the location of the TeV excess was detected in our analysis. The very low TS value of 1.0 with an integrated flux below $1.6\times10^{-10}$ ph cm$^{-2}$ s$^{-1}$ (see Tables \ref{tab:nondet_sources} \& \ref{tab:nondet_sources2}) give few hope for a future detection of this source by the LAT. The upper limits presented on Figure \ref{fig:1026} show that a curvature is needed between the TeV and the GeV energy range. This suggests an IC peak at energies higher than 100 GeV consistent with \emph{Fermi}-LAT observations of other PWNe. However, the lack of multi-wavelength data (especially in radio and X-ray), prevent any conclusion on this source.

\subsubsection*{HESS J1458$-$608}

PSR~J1459$-$60 \citep{2010ApJS..187..460A} is an energetic and old pulsar with a spin-down power of $\dot{E} = 9.2 \times 10^{35}$ erg s$^{-1}$, high enough to power a PWN, and a characteristic age of 64 kyr. X-ray observations with \emph{Swift} \citep{2011ApJS..194...17R} and \emph{Suzaku} \citep{Kanaiphd} yielded the discovery of an X-ray counterpart to PSR~J1459$-$60. HESS~J1458$-$608, was discovered 9.6$^{\prime}$ from PSR~J1459$-$60 using H.E.S.S. data after a dedicated observation \citep{2012arXiv1205.0719D} triggered by marginal detection in the 2004 Galactic Plane Survey. The proximity to the pulsar and the extension of HESS~J1458$-$608 suggested that both object could be related in a PSR/PWN scenario. In this case, the lack of observable radio and X-ray emission could be explained by the age of the system. 

In our work HESS~J1458$-$608 was not significantly detected above 10 GeV (TS = 12.3) with an integrated flux below $2.5 \times 10^{-10}$ ph cm$^{-2}$ s$^{-1}$. Furthermore, Table \ref{tab:nondet_sources2} shows that the observed marginal emission comes from the energy bin between 10 and 31 GeV. This means that the emission is more likely due to the pulsar than linked to the TeV source. Figure \ref{fig:1458} shows that even taking into account the pulsar in our model of the region, the SED is still contaminated at low energy. Nevertheless, the upper limits computed in the energy bins between 31 and 316 GeV, where no pulsar emission is expected anymore, show that a change in the slope of the spectrum is needed between the TeV and the GeV component. This could be consistent with an IC peak above 100 GeV which is the range observed for the PWN observed with the \emph{Fermi}-LAT.

\subsubsection*{HESS J1626$-$490}

HESS~J1626$-$490 is another unidentified source detected during the H.E.S.S. Galactic Plane Survey \citep{2008AA...477..353A}. In their search for counterparts, \cite{2011ICRC....7...44E} found no X-ray source fulfilling the energetic requirement to explain the TeV emission using \emph{XMM-Newton} observations. However, the author suggested that a hadronic scenario based on the interaction of the SNR~G335.2+00.1 with a $^{12}$CO molecular cloud could explain the TeV emission. This hypothesis is supported by a density depression in H I which could be explained by a recent event such as a SNR.

With a TS of 1.5, HESS~J1626$-$490 is not significantly detected in our analysis. The model presented in \cite{2011ICRC....7...44E} and reported on Figure \ref{fig:1626} reproduces the H.E.S.S. SED and is clearly below the upper limits obtained using \emph{Fermi}-LAT data. Only a radio or/and X-ray detection of synchrotron emission from a PWN or the detection of a pulsar could call this model into question.

\subsubsection*{HESS J1813$-$178}

HESS J1813-178 was discovered during the H.E.S.S. Survey of the Inner Galaxy \citep{2005Sci...307.1938A} and remained unidentified until the discovery of the SNR G12.8-0.0 \citep{2005ApJ...629L.105B} in the radio band. Using \emph{XMM-Newton} observations, \cite{2007AA...470..249F} detected a complex morphology composed of a point-like source and an extended nebula. The morphological and spectral ressemblance of the central object with a PWN, lead \cite{2007AA...470..249F} to propose a PWN/SNR scenario to describe the X-ray sources. This hypothesis was strengthened by the discovery of PSR J1813$-$1749 \citep{2009ApJ...700L.158G}, one of the most energetics pulsar in our Galaxy with a spin-down power of $\dot{E} = 5.6 \times 10^{37}$ erg s$^{-1}$. However, the nature of the TeV emission remains unclear as both the SNR or the PWN could produce emission at these energies.

%Related to the two regions observed in radio and X-Ray, two hypotheses are competing to explain the TeV emission of HESS J1813$-$178 : either the emission comes from the SNR or it comes from the PWN. Figure \ref{fig:hessj1813} show the SED with the Radio and X-Ray spectral points derived using the PWN region. We overlayed 

Our analysis yielded a TS of 2.5 with an upper limit on the integrated flux of $2.4 \times 10^{-10}$ ph cm$^{-2}$ s$^{-1}$ assuming the TeV morphology as spatial shape. Figure \ref{fig:hessj1813} shows the multi-wavelength SED of HESS J1813$-$178. The red and blue points corresponds to the \emph{Fermi}-LAT and H.E.S.S. spectral points. The green and magenta show respectively the X-ray and radio points derived using the PWN region in Figure \ref{fig:hessj1813} (Top) and the SNR region in Figure \ref{fig:hessj1813} (Bottom). The upper limits derived using the procedure described in Section 4 show that the spectrum of HESS~J1813$-$178 cannot be flat between the TeV and the GeV energy ranges and suggest a peak with an energy cutoff located between the two energy ranges. 

\cite{2007AA...470..249F} and \cite{2010ApJ...718..467F} investigated a leptonic model, in which the core X-ray and VHE $\gamma$-ray emission are associated. Both take into account IC scattering on CMB, infrared and near infrared photon fields and synchrotron emission produced with a rather low magnetic field (B $\sim 7\,\mu$G). The main difference between these two models lies in the injected electron population which follows power-law spectrum with an index of 2.0 in \cite{2007AA...470..249F} and a maxwellian + power-law tail spectrum \citep{2008ApJ...682L...5S} with an index of 2.4 in \cite{2010ApJ...718..467F}. 

%\cite{2007AA...470..249F} proposed a one-zone time dependent model with a constant injection power-law spectrum of index 2.0 reported in solid line on Figure \ref{fig:hessj1813} (Top). The authors considered the IC cooling of accelerated electrons over the CMB, infrared (1 eV cm$^{-3}) $ and near infrared (with a strong component $\sim$ 1000 eV cm$^{-3}$) photon fields, a magnetic field of 7.5 $\mu$G and electron maximum and minimum energies of 1MeV and 1.5 PeV. This maximum energy suggests that HESS J1813$�$178 is a highly effective accelerator. \cite{2010ApJ...718..467F} presented a leptonic model assuming a maxwellian + power-law tail injection spectrum proposed by \cite{2008ApJ...682L...5S} with an index of 2.4 represented as a dashed line on Figure \ref{fig:hessj1813} (Top). The interstellar soft photons scattered in the IC process contain the CMB, infrared (T= 35K, U=1.5 eV cm$^{-3}$) and starlight (T$_{st}$ = 3000 K, U$_{st}$ = 1.5 eV cm$^{-3}$). They considered a distance of 4kpc with a density of 1.4 cm$^{-3}$, an age of 1.2 kyr, an ambient magnetic field of 5 $\mu$G and a maximum electron energy of 1 PeV.

Both also investigated the possibility for the TeV energy to be created by the SNR shell. We overlaid in solid line on Figure \ref{fig:hessj1813} (Top) the model proposed by \cite{2007AA...470..249F} assuming a hadronic scenario. The main differences between these two models lies in the injected protons and electron populations which follows power-law spectrum with an index of 2.1 in \cite{2007AA...470..249F} and is computed in a semi-anytical nonlinear model in \cite{2010ApJ...718..467F}.

%a maxwellian + power-law tail spectrum \citep{2008ApJ...682L...5S} with an index of 2.4 in \cite{2010ApJ...718..467F}.

%The accelerated particles were injected assuming a power-law spectrum of index 2.1 with a minimum and maximum energy of 1GeV and 100 TeV. The authors assumed a ambient density of 1 cm$^{-3}$ while the model proposed by \cite{2010ApJ...718..467F} assumed a density of 2.8 cm$^{-3}$, a electron to proton ration of 4.5$\times 10^{-4}$ and the same parameters than those used for the leptonic model. However, even assuming a denser medium to enhance the p-p collision in the SNR shell, the IC scattering of electrons and positrons in the PWNe is found prominent enough to contribute significantly to the VHE emission.

The upper limits derived in our analysis rejected one of the two model proposed in each hypothesis. Therefore, no conclusion can be reached on the nature of the TeV emission, being a PWN or a SNR. However, at the light of the different model discussed on can exclude a model assuming a maxwellian + power-law tail injection spectrum with an index of 2.4 assuming the same parameters than those derived in \cite{2010ApJ...718..467F} in the case where the TeV emission comes from the PWN core. These upper limits also constrain the hadronic model by rejecting a power-law injection spectrum of 2.1 with the parameters proposed by \cite{2007AA...470..249F}. Therefore, whatever the origin of the $\gamma$-ray emission, the injected spectrum of the primary electron and protons need to be relatively hard to reproduce the \emph{Fermi}-LAT upper limits ($\Gamma \leq 2.1$).

%Therefore, the model matching the data show that VHE emission comes either from the PWN core of the system or from p-p collision in the SNR shell with a non negligible leptonic IC component due to the PWN core.

%Two models are represented on each plot. The solid line correspond to the model proposed by \cite{2007AA...470..249F} while the dashed line show the model proposed by \cite{2010ApJ...718..467F}. On Figure \ref{fig:hessj1813} (Top), both models are obtained for a leptonic scenario, while Figure \ref{fig:hessj1813} (Bottom) show the models obtained assuming an hadronic scenario.

%In a leptonic scenario, the upper limit we obtained are not consistent with the model proposed by \cite{2010ApJ...718..467F}, while one of the models proposed by \cite{2007AA...470..249F} for a PWN of 1 kyr and a magnetic field of 7.5 $\mu$G nicely  matches the spectral points. 

%Assuming now that the TeV emission comes from the SNR shell, Figure \ref{fig:hessj1813} (Bottom) we overlayed to the hadronic model proposed by \cite{2007AA...470..249F} and \cite{2010ApJ...718..467F}. While the hadronic hypothesis proposed by \cite{2007AA...470..249F} predict a flux over the detection threshold of the LAT in the 10 to 31 GeV energy bin, the hadronic model proposed by \cite{2010ApJ...718..467F} for an ISM number sensity of n$_0$ = 2.8 cm$^{-3}$ and electron to proton ratio of K$_{ep} = 4.5 \times 10^{-4}$ matches the radio and X-ray points as well as the GeV-TeV SED. It is important to note that the model proposed by \cite{2007AA...470..249F} in which the GeV-TeV peak comes from the IC emission from the electron population responsible for the radio emission (the SNR Shell), also matches the GeV-TeV SED. To conclude on this source \textbf{I HAVE NO IDEA HOW TO END THIS SENTENCE}.



%\subsection{Conclusion}
%To be done later....