\section{Introduction}

Since 2003, the continuous observation of the Galactic Plane by $\breve{C}$erenkov telescopes have yielded the detection of more than 80 Galactic sources. Among them, pulsar wind nebulae (PWNe) is the dominant class with more than 30 firm identifications. However, a similar number of Galactic sources cannot be associated to a counterpart in any other wavelength. They are ranked in the unidentified (UNID) class.

Multi-wavelength observations are essential to constrain the emission mechanisms occurring in these sources and identify their nature. The Large Area Telescope (LAT) aboard the \emph{Fermi} Gamma-ray Space Telescope (\emph{Fermi}) is especially useful in this context. Covering the energy range from $\sim$ 100 MeV to more than 300 GeV, the LAT provides a precise view of the $\gamma$-ray sky. With 2 years of observations, the \emph{Fermi}--LAT Second Catalog \citep{2012ApJS..199...31N} already contains 1873 sources, 1397 being identified and 576 without counterparts.

Two different scenarii are proposed to explain the observed $\gamma$-ray radiation : the accelerated particles are either electrons (leptonic scenario) or protons (hadronic scenario). In the hadronic scenario, $\gamma$-ray photons are created by $\pi^0$ decay from the interaction of accelerated hadrons with nuclei of the interstellar medium. In the leptonic scenario, $\gamma$-ray photons are created by inverse Compton (IC) scattering of the accelerated leptons on the ambient photon fields (CMB, Stellar Radiation, IR, ...).

Leptonic sources such as relic PWNe \citep{2009ASSL..357..451D,2009arXiv0906.2644D} and hadronic sources such as old supernova remnants interacting with molecular clouds \citep{2011arXiv1104.1197U} have been proposed to explain the population of unidentified sources.

The distinction between these two scenarii is done by studying the link between the TeV and the GeV energy range. \cite{2010ApJ...720..266S,2011ApJ...738...42G,Rousseau1857} have demonstrated the potential that can be provided by the use of LAT observations to study PWNe candidates. Up to now, except Vela--X \citep{2010ApJ...713..146A}, the 7 PWNe firmly identified by Fermi are all associated to a TeV counterpart and show a hard spectrum consistent with an IC peak above 100 GeV (\cite{2010ApJ...708.1254A},\cite{2011ApJ...738...42G}).


The motivation of this work is to bring new constraints on already known PWNe and to look for new candidates among the unidentified TeV sources. A search for $\gamma$-ray emission in the off-pulse window of the $\gamma$-ray detected pulsars, updating the method proposed in \cite{2011ApJ...726...35A}, will be presented in \cite{2PC}. The strategy presented in our work (this paper) is closer to \cite{2010AA...518A...8T} in the sense that we analyzed the \emph{Fermi}-LAT data around already known potential PWNe.

The sources included in our search were detected by H.E.S.S., VERITAS, MAGIC and MILAGRO. H.E.S.S. \citep{2006AA...457..899A} is composed of four telescopes observing the very high energy (VHE) sky from 0.1 to 100 TeV with a mean point spread function (PSF) of $\sim$0.1$\degr$ for a point--source sensitivity around 1$\%$ of the Crab flux above 200 GeV for 25 hours of observations. The four telescopes of VERITAS \citep{2002APh....17..221W} detect the $\sim$ 0.1 to 30 TeV $\gamma$-rays with a sensitivity of 1$\%$ of the Crab in less than 30 hours and an angular resolution lower than 0.1$\degr$ at 1 TeV. MAGIC \citep{2012APh....35..435A} consists of two telescopes observing from $\sim$ 50 GeV to several tens of TeV with a sensitivity of $\sim$ 0.8$\%$ of the Crab nebula flux above 300 GeV with a $\sim$ 0.1$\degr$ PSF. MILAGRO \citep{2004ApJ...608..680A} is a water $\breve{C}$erenkov detector with a large field of view (FoV) of $\sim$2sr and a $\sim$ 0.45$\degr$ angular resolution observing $\gamma$-rays from $\sim$ 1 to 100 TeV. 
%{\bf {\color{red} Manque les references}}

%references utiles :
%HESS
%http://arxiv.org/pdf/1203.3215v2.pdf
%Sensitivity : http://arxiv.org/pdf/1204.5690v1.pdf
%VERITAS
%http://arxiv.org/pdf/1205.5287v1.pdf
%sensitivity : http://veritas.sao.arizona.edu/about-veritas-mainmenu-81/veritas-specifications-mainmenu-111
%MAGIC
%http://arxiv.org/pdf/1201.4074v2.pdf
%PSF : http://magic.mppmu.mpg.de/magic/factsheet/
%MGRO
%http://scipp.ucsc.edu/milagro/papers/lansdell1205.pdf
%

Starting from the TeV catalog website \footnote{http://tevcat.uchicago.edu/}, which summarizes all sources detected at TeV energies, we established a list of sources possibly associated to PWNe (Section 3), that we studied using the method and the tools described in Section 4. The most interesting cases as well as the spatial and spectral results will be discussed (Section 5) before being presented in the context of a complete population study (Section \ref{discussion}).

%will be presented in the context of a population study (Section 6) before discussing the case one by one (Section 7). \textbf{a remoduler une fois le reste fait.} 