Since its launch, the \emph{Fermi} satellite has firmly identified 7 pulsar wind nebulae (PWNe) plus a large number of candidates, all powered by young and energetic pulsars. Furthermore, PWNe are the most populous class in the TeV energy range followed by the unidentified sources (UNID). Using 45 months of Fermi--LAT data, we looked around the position of 58 TeV sources to bring new constraints on the models and new clues on the nature of sources without counterparts. For each of them we derived a $\gamma$--ray flux or an upper limit (when the TeV source is not detected at GeV energies by the \emph{Fermi}--LAT) above 10~GeV.

The wealth of multi-wavelength data available and the new results provided by \emph{Fermi}--LAT is an extraordinary opportunity to constrain the origin of the $\gamma$-ray emission of the large sample of UNID and the radiative processes taking place in known PWNe. 