\subsection{Off-peak Analysis Method}
\label{subsec:off_peak_analysis}

We developed a procedure for characterizing emission found
in off-peak phase region for all pulsars in this catalog.  
This procedure used both the spectral and spatial characteristics
of any observed emission to determine its physical origin.

PWNe are often expected to be spatially
resolvable at GeV energies. For example, \velax and HESS\;J1825-137 are PWN
that have been spatially resolved by the \lat 
\citep{LAT_collaboration_Vela_X_2010,LAT_collaboration_HESS_J1825_2011}.
On the other hand, not all PWN are expected to be
significantly spatially resolved due to the finite
instrument resolution of the LAT. For example, the Crab nebula 
cannot be resolved by that LAT LAT but is distinguished from the Crab
pulsar in the off-peak region by its hard spectrum for E$\gtrsim$1 \gev.
\citep{LAT_collaboration_crab_2010}.

On the other hand, pulsars can have DC emission in the off-peak region
due to the geometry of the pulsar magnetospheres.  A previous analysis
by the LAT collaboration found five off-peak regions to have significant
emission which is point-like in nature and characterized by a
pulsar-like cutoff spectrum \citep{LAT_collaboration_PWNCAT_2011}.
We can therefore use either spatial extension or a hard spectrum to
distinguish PWN emission and point-like emission with a cutoff spectrum
to distinguish magnetospheric emission.

To perform this analysis, we used the likelihood fitting package \pointlike
to study the spatial character of emission in the off-peak regions
and \gtlike in binned model to study the spectral character of
the emission. These tools provide complementary features and this
method is very similar to the approach used in the second \lat catalog
\citep{LAT_Collaboration_2FGL_2012} and a follow up search for spatially
extended sources \citep{LAT_collaboration_extended_search_2012}.

For this analysis, we build a model of the sky consistent with the
second \lat catalog. We included as background sources all nearby
sources from the second catalog and we also used the same background models 
\citep{LAT_Collaboration_2FGL_2012}.  For out analysis, we used
an energy range from 100 \mev to 316 \gev ($10^{5.5}$ \mev).  For this analysis, 
we removed all photons from the on-peak emission and scaled the exposure so as to fit
the all-phase flux assuming the emission was constant with phase.

First, we assumed any potential emission in the off-peak region 
to have a point-like
spatial model and (unless otherwise noted) a power-law spectral model.
We used \pointlike to fit the position of the off-peak region following
the procedure described in \cite{LAT_Collaboration_2FGL_2012}.  We used
the best fit positions obtained by \pointlike and performed a
spectral analysis using \gtlike.

With the best-fit position, we used \gtlike to 
test the significance of the detection of the source.
We define the 
likelihood-ratio test for the detection of the source as
\begin{equation}
  \ts = 2 \log(\likelihood_\text{pt}/\likelihood_\text{bg})
\end{equation}
where $\likelihood_\text{pt}$ is the Poisson likelihood for a model
including the source and $\likelihood_\text{bg}$ the likelihood for a
model not including the source.  We set the threshold for detection of
significant emission at $\ts>25$, corresponding to a significance just
over $4\sigma$ \citep{LAT_Collaboration_1FGL_2010}.

For the significantly-detected source, we tested to see if the spectrum of the
source is significantly cutoff following the prescription in
\cite{LAT_collaboration_PWNCAT_2011}. 
We fit the source in \gtlike 
with both a power-law and exponentially-cutoff spectral model
and define the likelihood-ratio test for a cutoff spectrum as
\begin{equation}
  \tscut= 2 \log(\likelihood_\text{cutoff}/\likelihood_\text{pt})
\end{equation}
where $\likelihood_\text{cutoff}$ is the Poisson likelihood for a model
including the cutoff spectrum.  We set the threshold
for detecting a significant cutoff at $\tscut>16$, corresponding to a
$4\sigma$ detection \citep{LAT_collaboration_PWNCAT_2011} .

We used \pointlike to simultaneously fit the position and the
extension of the assumed radially-symmetric Gaussian source, following the description in
\citep{LAT_collaboration_extended_search_2012}.  
After the extension fit,
we refit the spectrum of the spatially extended source using \gtlike
We tested the significance of the spatial extension
and computed the likelihood-ratio test for the significance of the extension:
\begin{equation}
  \tsext = 2 \log(\likelihood_\text{ext}/\likelihood_\text{pt}).
\end{equation}
Here, $\likelihood_\text{ext}$ is the Poisson likelihood assuming the
source is spatially extended. We set the threshold for detecting the
significance of a spatially extended source at $\tsext>16$, corresponding
to a $4\sigma$ detection \citep{LAT_collaboration_extended_search_2012}.

The spectrum of the Crab nebula was uniquely challenging because the
GeV spectrum contains both a falling synchrotron spectrum and an
inverse component component \citep{LAT_collaboration_crab_2010}.
To avoid unnecessary complications in our pipeline, for this
particular source we used the best fit two-component spectral model from
\cite{LAT_collaboration_crab_2012} and fit only the overall normalization
of the source.

For sources that are not significantly detected, we compute flux upper
limits assuming a point-like spatial model and a fixed spectral index
of 2.0.  We also compute pulsed upper limit assuming a canonical pulsar
spectrum with an index of -1.7 and a cutoff energy of 3 \gev.

To better assess the spectral charter of any emission in the off-peak
region, we performed a spectral analysis in three independent energy bins
(100 \mev to 1 \gev, 1 \gev to 10 \gev, and 10 \gev to 316 \gev). In each
bin, we independently fit the flux and the spectral index of the source.
We also computed flux upper limit assuming a fixed spectral index of 2
when the source was not significantly detected in the energy range.

Following the discovery of time-variable emission from the Crab nebula
by the LAT, it is interesting to search for other variable PWN
\citep{LAT_Collaboration_Crab_Flare_2011}.  Therefore, we tested all 
off-peak regions for variability.  We divided the 3-year time range
into 36 month-long intervals and fit the flux of the source independently
in each time range. We computed the significance of variability
with a likelihood-ratio test by computing \tsvar following the procedure of 2FGL
\citep{LAT_Collaboration_2FGL_2012}.  Since we divided our time range into 
36 months-long bins, the null distribution for $\tsvar$ follows a $\chi^2$
distribution with 35 degrees of freedom. We set the detection criteria
for significant variability at $\tsvar>91.7$, corresponding to a $4\sigma$
significance detection threshold.

In many situations, our pipeline was significantly biased due to systematic
errors associated with not modeling nearby sources.
We are more sensitive to nearby
sources than the second \lat catalog both because of our expanded data
set (3 years of observations instead of 2) and also because, for
very bright pulsars, we are more sensitive to nearby sources once
the strong contribution from the pulsar has been removed.

The large and energy dependent point-spread function of the \lat causes the analysis
of any one source to be sensitively affected by the modeling of nearby
sources. Therefore, we had to, in many situations, iteratively improve
the model of a region by including new background sources. 
We modeled these sources by
generating maps of residual test statistic assuming the presence of a
new source (of a fixed spectral index of 2) and looking for residuals with
$\ts\ge25$. In these situations, we would include a new source into our
model, fit the position and spectrum of this source, and iterate until
there was no remaining significant residual.

Even so, there are still some lingering regions for where we were unable
to obtain an unbiased fit.  the emission for.  These issues are most
likely due to systematics associated with the model of the galactic
diffuse emission and issues associated with modeling nearby sources.
The most common symptom of a failed fit is a diverging localization
or an unphysically large extension which causes the source model
to incorporate multiple background sources. Systematics associated
with modeling extended sources are discussed more thoroughly in
\citep{LAT_collaboration_extended_search_2012}.  For the currently
analysis, we flagged these problematic regions 
and do not attempt a complete understanding of the region.
