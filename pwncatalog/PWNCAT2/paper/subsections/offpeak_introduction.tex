
In this section, we will search for emission in the phase range between
the peaks of the pulsar's light curve. This potential DC emission could originate
in the pulsar winds, from inside a pulsar's magnetosphere, or 
the emission could be physically unrealted to the pulsar.

\gev emission in the off-peak regions around LAT-detected
pulsars has been studied in several previous publications. In
particular, the spatially-extended \velax pulsar wind nebula has
been detected by the LAT in the off-peak region of the Vela pulsar
\citep{LAT_collaboration_Vela_X_2010} and the Crab nebula has been
detected by the LAT \citep{LAT_collaboration_crab_2010}.  Surprisingly,
this \gev emission from the Crab nebula was found to be variable in
time \citep{LAT_Collaboration_Crab_Flare_2011}.

Most prominently, a dedicated analysis was performed using LAT data of
the off-peak emission of 54 LAT-detected pulsars using 16 months of survey
observations \citep{LAT_collaboration_PWNCAT_2011}.  The search discovered
ten pulsars with significant off-peak emission.  Along with \velax and the
Crab nebula, the search discovered a source coincident with the TeV source
HESS\;J1023-575 in the off-peak window of PSR\;J1023-5746. In addtion,
five of the other regions showed a significantly cutoff pulsar-like
spectrum and are suspected to be of magnetospheric origin.

We expand upon this previous work by searching in the off-peak region
of all 117 pulsars in this catalog. In addition to the larger list of
pulsars, we use an expanded observation time, a larger energy range,
and an improved analysis method.
