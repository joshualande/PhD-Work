\subsection{Off-peak Phase Selection}
\label{subsec:off_peak_defintion}

To study the off-peak emission of \lat-detected pulsars, we first
developed a systematic method to define the off-peak region of a pulsar.
The primary constraint for this method was for it to be systematic,
model independent, and computationally efficient.

The method we developed proceeds by deconstructing the pulsar's
phaseogram into simple Bayesian Blocks using the 
algorithm described in \cite{Jackson_Bayesian_Blocks_2003}.
To produce Bayesian Blocks on a periodic phaseogram, we applied the blocks
three sets of the data from a phase of -1 to 2 and selected the blocks from a
phase of 0 to 1.

TO improve the off-peak region definition,
we first optimized the pulsar phaseogram
by varying the minimum energy and radius of the included photons to optimize
the H-test. We then selected the lowest block to be the off-peak
region, but removed 10\% of the emission from either side of the block
to avoid potential contamination.

There is one free parameter in the Baysian Block algorithm called
$\text{ncp}_\text{prior}$ which modifies the probability that the
algorithm will divide a block into smaller intervals. For our situation,
we found that setting $\text{ncp}_\text{prior}=8$ protected against
the Bayesian Block decomposition containing unphysically small blocks.
For a few very marginally-detected pulsars, the algorithm failed
decomposed the phaseogram into multiple blocks
and in these situations we decreased $\text{ncp}_\text{prior}$ until the
algorithm succeeded.

In some situations, there can be two well defined off-peak regions between two
pulsed peaks. The method we used to select a second phase range was
to take the second lowest Bayesian block
but only when the emission in this block is consistent with the emission in
the lowest block (at the 99\% confidence level) and when the
second block contained at least half as much phase as the first block.

Figure \figref{off_peak_select} shows the energy-and-radius optimized
light curve and the off-peak selection for a representative sample of pulsars.
The off-peak definition for all pulsars is contained in the auxiliary material.

\begin{figure}
  \ifdefined\bwfigures
   % plots from $pwndata/off_peak/off_peak_bb/pwncat2/v6/plots/
  \plotone{off_peak_select_bw.eps}
  \else
  \plotone{off_peak_select_color.eps}
  \fi
  \caption{The phaseogram and off-peak selection for 
  (a) PSRJ0007+7303, (b) PSRJ0534+2200, (c) PSRJ0633+1746, (d) PSRJ0835-4510,
  (e) PSRJ1702-4128, (f) PSRJ1747-4036, (g) PSRJ1801-2451, and (h) PSRJ2021+4026.
  The black histogram represents the energy-and-radius optimized phaseogram.
  The gray lines (colored red in the electronic version)
  represent the Bayesian block decomposition of the pulsar light curves.
  The hatched areas represent the off-peak regions selected by this method.}
  \label{fig:off_peak_select}
\end{figure}

