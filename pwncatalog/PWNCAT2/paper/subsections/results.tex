\section{Results}

First, we tested the sources to see if they
were spatially extended. The localization results are in \tabref{localization}.

\begin{deluxetable}{l*{7}c}
\tabletypesize{\scriptsize}
\tablecaption{Localization and extension fitting results
\label{tab:localization}
}
\tablehead{\colhead{PSR} & \colhead{$\ts_\text{point}$} & \colhead{GLON} & \colhead{GLAT} & \colhead{Pos Err} & \colhead{Offset} & \colhead{\tsext} & \colhead{Extension}\\ \colhead{ } & \colhead{ } & \colhead{(deg)} & \colhead{(deg)} & \colhead{ } & \colhead{(deg)} & \colhead{ } & \colhead{(deg)}}
\startdata
J0007+7303 & 84.0 & 119.64 & 10.35 & 0.11 & 0.11 & 9.5 & $<0.40$ \\
J0034$-$0534 & 42.4 & 111.53 & -68.03 & 0.06 & 0.04 & 0.0 & $<0.12$ \\
J0218+4232 & 34.7 & 139.56 & -17.53 & 0.08 & 0.05 & 0.0 & $<0.15$ \\
J0340+4130 & 25.1 & 153.81 & -11.00 & 0.06 & 0.04 & 0.0 & $<0.13$ \\
J0534+2200 & 4959.1 & 184.55 & -5.79 & 0.01 & 0.01 & 0.0 & $<0.02$ \\
J0633+1746 & 2842.4 & 195.12 & 4.22 & 0.02 & 0.05 & 3.3 & $<0.09$ \\
J0835$-$4510 & 304.7 & 263.46 & -3.15 & 0.08 & 0.37 & 295.3 & $0.73 \pm 0.06$ \\
J1023$-$5746 & 83.0 & 285.52 & -0.08 & 0.14 & 1.39 & 190.4 & $1.40 \pm 0.10$ \\
J1119$-$6127 & 123.2 & 291.97 & -0.61 & 0.06 & 0.20 & 41.0 & $0.29 \pm 0.06$ \\
J1513$-$5908 & 122.6 & 320.34 & -1.20 & 0.02 & 0.04 & 0.0 & $<0.12$ \\
J1620$-$4927 & 39.1 & 333.87 & 0.25 & 0.05 & 0.16 & 0.0 & $<0.34$ \\
J1709$-$4429 & 30.7 & 342.50 & -3.70 & 0.52 & 1.18 & 29.2 & $1.18 \pm 0.24$ \\
J1744$-$1134 & 74.4 & 14.79 & 9.18 & None & 0.00 & 0.0 & $<2.83$ \\
J1746$-$3239 & 47.6 & 357.60 & -1.30 & None & 1.08 & 139.2 & $2.07 \pm 0.14$ \\
J1747$-$2958 & 30.3 & 358.66 & 0.29 & 0.14 & 1.31 & 43.8 & $1.94 \pm 0.13$ \\
J1809$-$2332 & 29.0 & 7.33 & -2.26 & 0.14 & 0.27 & 14.6 & $<0.39$ \\
J1813$-$1246 & 53.3 & 17.32 & 2.46 & 0.05 & 0.07 & 0.2 & $<0.25$ \\
J1836+5925 & 5019.4 & 88.87 & 25.00 & 0.01 & 0.00 & 0.0 & $<0.06$ \\
J2021+4026 & 920.6 & 78.24 & 2.10 & 0.02 & 0.02 & 16.1 & $0.11 \pm 0.03$ \\
J2032+4127 & 28.5 & 79.78 & 0.78 & None & 0.50 & 62.8 & $1.25 \pm 0.18$ \\
J2055+2539 & 109.0 & 70.68 & -12.45 & 0.06 & 0.07 & 0.0 & $<0.14$ \\
J2124$-$3358 & 106.5 & 10.83 & -45.40 & 0.04 & 0.07 & 0.0 & $<0.09$ \\
J2302+4442 & 115.0 & 103.36 & -14.04 & 0.05 & 0.05 & 0.9 & $<0.22$ \\
\enddata
\tablecomments{\todo[inline]{Put table comments}}
\end{deluxetable}


Next, we performed a spectral analysis over all energy using the best
fit morphology. \tabref{all_energy} shows the results of the all energy
analysis of the off-peak emission for each pulsar.

Next, we fit a powerlaw independently in each energy
bin. \tabref{each_energy} shows the results of the analysis in separate
energy bins of each pulsar.

Finally, we tested sources to see which were
variable. \tabref{cutoff_test} shows the results of the cutoff test for
pulsars with significant low-energy emission.


\begin{deluxetable}{l*{5}c}
\tabletypesize{\scriptsize}
\tablecaption{All Energy spectral fit for the \todo[inline]{How many pulsars?} \lat-detected Pulsars
\label{tab:all_energy}
}
\tablehead{\colhead{PSR} & \colhead{\ts} & \colhead{$F_{0.1-316}$} & \colhead{$G_{0.1-316}$} & \colhead{$\Gamma$} & \colhead{Luminosity}\\ \colhead{ } & \colhead{ } & \colhead{($10^{-9}\ \ph\,\cm^{-2}\,\s^{-1}$)} & \colhead{($10^{-12}\ \erg\,\cm^{-2}\s^{-1}$)} & \colhead{ } & \colhead{($10^{33}\ \erg\,\s^{-1}$)}}
\startdata
J0007+7303 & 84.0 & $53.36 \pm 9.81$ & $20.08 \pm 2.37$ & $2.74 \pm 0.19$ & None \\
J0030+0451 & 14.1 & $<8.22$ & $<10.61$ & \nodata & None \\
J0034$-$0534 & 42.4 & $16.05 \pm 4.75$ & $8.52 \pm 1.65$ & $2.41 \pm 0.19$ & None \\
J0106+4855 & 0.0 & $<6.80$ & $<8.78$ & \nodata & None \\
J0218+4232 & 34.7 & $50.61 \pm 20.56$ & $18.40 \pm 3.35$ & $2.78 \pm 0.48$ & None \\
J0248+6021 & 18.8 & $<13.60$ & $<17.56$ & \nodata & None \\
J0340+4130 & 25.1 & $10.28 \pm 3.62$ & $9.32 \pm 2.46$ & $2.13 \pm 0.15$ & None \\
J0357+3205 & 0.0 & $<2.97$ & $<3.83$ & \nodata & None \\
J0437$-$4715 & 0.0 & $<1.85$ & $<2.39$ & \nodata & None \\
J0534+2200 & 4959.1 & $559.71 \pm 19.47$ & $397.02 \pm 12.21$ & $2.24 \pm 0.02$ & None \\
J0610$-$2100 & 0.0 & $<3.23$ & $<4.17$ & \nodata & None \\
J0613$-$0200 & 0.0 & $<3.37$ & $<4.35$ & \nodata & None \\
J0614$-$3329 & 15.6 & $<15.81$ & $<20.41$ & \nodata & None \\
J0622+3749 & 1.0 & $<7.81$ & $<10.08$ & \nodata & None \\
J0631+1036 & 14.5 & $<13.79$ & $<17.80$ & \nodata & None \\
J0633+0632 & 4.1 & $<10.19$ & $<13.16$ & \nodata & None \\
J0633+1746 & 2842.4 & $882.74 \pm 30.65$ & $579.06 \pm 23.61$ & $2.28 \pm 0.03$ & None \\
J0659+1414 & 0.0 & $<1.77$ & $<2.29$ & \nodata & None \\
J0729$-$1448 & 0.0 & $<4.85$ & $<6.25$ & \nodata & None \\
J0734$-$1559 & 24.5 & $<12.39$ & $<16.00$ & \nodata & None \\
J0742$-$2822 & 4.3 & $<6.84$ & $<8.83$ & \nodata & None \\
J0751+1807 & 8.1 & $<5.70$ & $<7.36$ & \nodata & None \\
J0835$-$4510 & 600.0 & $389.91 \pm 22.62$ & $327.74 \pm 20.41$ & $2.16 \pm 0.03$ & None \\
J0908$-$4913 & 15.1 & $<24.71$ & $<31.89$ & \nodata & None \\
J0940$-$5428 & 0.0 & $<1.73$ & $<2.24$ & \nodata & None \\
J1016$-$5857 & 0.0 & $<12.09$ & $<15.61$ & \nodata & None \\
J1019$-$5749 & 2.4 & $<12.59$ & $<16.25$ & \nodata & None \\
J1023$-$5746 & 273.4 & $399.13 \pm 37.06$ & $472.93 \pm 35.48$ & $2.03 \pm 0.04$ & None \\
J1024$-$0719 & 0.0 & $<2.30$ & $<2.97$ & \nodata & None \\
J1028$-$5819 & 8.0 & $<26.93$ & $<34.77$ & \nodata & None \\
J1044$-$5737 & 0.0 & $<17.76$ & $<22.92$ & \nodata & None \\
J1048$-$5832 & 0.0 & $<16.77$ & $<21.65$ & \nodata & None \\
J1057$-$5226 & 0.8 & $<5.03$ & $<6.49$ & \nodata & None \\
J1105$-$6107 & 11.0 & $<31.71$ & $<40.93$ & \nodata & None \\
J1119$-$6127 & 164.2 & $112.84 \pm 3.58$ & $92.50 \pm 2.17$ & $2.17 \pm 0.01$ & None \\
J1135$-$6055 & 4.2 & $<6.89$ & $<8.89$ & \nodata & None \\
J1231$-$1411 & 0.0 & $<3.21$ & $<4.14$ & \nodata & None \\
J1357$-$6429 & 0.0 & $<5.72$ & $<7.38$ & \nodata & None \\
J1410$-$6132 & 18.4 & $<42.29$ & $<54.59$ & \nodata & None \\
J1413$-$6205 & 0.0 & $<11.99$ & $<15.48$ & \nodata & None \\
J1418$-$6058 & 0.0 & $<34.10$ & $<44.02$ & \nodata & None \\
J1420$-$6048 & 12.1 & $<31.86$ & $<41.13$ & \nodata & None \\
J1429$-$5911 & 0.0 & $<12.66$ & $<16.34$ & \nodata & None \\
J1459$-$6053 & 0.0 & $<9.08$ & $<11.72$ & \nodata & None \\
J1509$-$5850 & 0.0 & $<9.66$ & $<12.47$ & \nodata & None \\
J1513$-$5908 & 122.6 & $19.15 \pm 7.39$ & $51.40 \pm 8.43$ & $1.79 \pm 0.12$ & None \\
J1531$-$5610 & 0.5 & $<3.52$ & $<4.54$ & \nodata & None \\
J1600$-$3053 & 0.0 & $<1.85$ & $<2.39$ & \nodata & None \\
J1614$-$2230 & 0.9 & $<5.93$ & $<7.65$ & \nodata & None \\
J1620$-$4927 & 39.1 & $79.65 \pm 20.62$ & $64.27 \pm 11.41$ & $2.18 \pm 0.10$ & None \\
J1702$-$4128 & 0.0 & $<5.75$ & $<7.42$ & \nodata & None \\
J1709$-$4429 & 69.0 & $181.80 \pm 36.37$ & $90.11 \pm 34.30$ & $2.46 \pm 0.41$ & None \\
J1713+0747 & 0.0 & $<4.79$ & $<6.18$ & \nodata & None \\
J1718$-$3825 & 0.0 & $<14.54$ & $<18.78$ & \nodata & None \\
J1732$-$3131 & 0.0 & $<8.65$ & $<11.16$ & \nodata & None \\
J1741$-$2054 & 0.0 & $<13.38$ & $<17.27$ & \nodata & None \\
J1744$-$1134 & 74.4 & $47.10 \pm 8.73$ & $27.61 \pm 3.67$ & $2.34 \pm 0.08$ & None \\
J1746$-$3239 & 186.8 & $461.05 \pm 37.05$ & $624.30 \pm 40.49$ & $1.98 \pm 0.03$ & None \\
J1747$-$2958 & 74.0 & $260.88 \pm 40.86$ & $512.22 \pm 68.60$ & $1.87 \pm 0.04$ & None \\
J1803$-$2149 & 6.1 & $<27.06$ & $<34.93$ & \nodata & None \\
J1809$-$2332 & 29.0 & $85.89 \pm 68.64$ & $43.14 \pm 9.05$ & $2.45 \pm 0.62$ & None \\
J1813$-$1246 & 53.3 & $191.30 \pm 40.97$ & $83.32 \pm 11.83$ & $2.57 \pm 0.14$ & None \\
J1823$-$3021A & 2.7 & $<5.16$ & $<6.66$ & \nodata & None \\
J1826$-$1256 & 18.4 & $<66.21$ & $<85.47$ & \nodata & None \\
J1836+5925 & 5019.4 & $561.39 \pm 17.71$ & $538.66 \pm 25.37$ & $2.11 \pm 0.02$ & None \\
J1846+0919 & 0.0 & $<3.35$ & $<4.32$ & \nodata & None \\
J1907+0602 & 0.0 & $<7.27$ & $<9.39$ & \nodata & None \\
J1939+2134 & 0.0 & $<4.40$ & $<5.68$ & \nodata & None \\
J1952+3252 & 0.4 & $<7.78$ & $<10.05$ & \nodata & None \\
J1954+2836 & 6.1 & $<18.52$ & $<23.91$ & \nodata & None \\
J1957+5033 & 0.0 & $<2.52$ & $<3.26$ & \nodata & None \\
J1958+2846 & 0.0 & $<7.72$ & $<9.97$ & \nodata & None \\
J1959+2048 & 0.0 & $<4.89$ & $<6.32$ & \nodata & None \\
J2017+0603 & 0.0 & $<2.97$ & $<3.83$ & \nodata & None \\
J2021+3651 & 0.1 & $<7.88$ & $<10.18$ & \nodata & None \\
J2021+4026 & 936.6 & $1196.46 \pm 26.76$ & $824.96 \pm 13.64$ & $2.25 \pm 0.01$ & None \\
J2028+3332 & 0.0 & $<4.57$ & $<5.90$ & \nodata & None \\
J2030+3641 & 0.0 & $<2.89$ & $<3.73$ & \nodata & None \\
J2030+4415 & 3.5 & $<28.40$ & $<36.66$ & \nodata & None \\
J2032+4127 & 91.3 & $192.51 \pm 51.56$ & $425.89 \pm 53.73$ & $1.84 \pm 0.08$ & None \\
J2043+2740 & 0.0 & $<2.58$ & $<3.33$ & \nodata & None \\
J2051$-$0827 & 0.0 & $<1.89$ & $<2.44$ & \nodata & None \\
J2055+2539 & 109.0 & $46.79 \pm 6.38$ & $21.45 \pm 2.28$ & $2.52 \pm 0.08$ & None \\
J2124$-$3358 & 106.5 & $20.21 \pm 3.88$ & $18.48 \pm 3.09$ & $2.13 \pm 0.10$ & None \\
J2139+4716 & 16.8 & $<9.29$ & $<11.99$ & \nodata & None \\
J2214+3000 & 0.0 & $<5.02$ & $<6.48$ & \nodata & None \\
J2238+5903 & 0.0 & $<6.19$ & $<7.99$ & \nodata & None \\
J2240+5832 & 0.0 & $<6.37$ & $<8.22$ & \nodata & None \\
J2302+4442 & 115.0 & $33.65 \pm 5.34$ & $18.69 \pm 2.23$ & $2.38 \pm 0.10$ & None \\
\enddata
\tablecomments{\todo[inline]{Put table comments}}
\end{deluxetable}

\clearpage
\begin{deluxetable}{l*{9}c}
%\tabletypesize{\tiny}
\setlength{\tabcolsep}{0.04in}
\tabletypesize{\scriptsize}
\tablewidth{0pt}
\rotate
\tablecaption{Energy bin spectral fit for the \todo[inline]{How many pulsars?} \lat-detected Pulsars
\label{tab:each_energy}
}
\tablehead{
\colhead{TeV source} & \colhead{$\text{TS (10--31 GeV)}$} & \colhead{F(10--31 GeV)} & \colhead{$\text{TS (31--100 GeV)}$} & \colhead{F(31--100 GeV)} & \colhead{$\text{TS (100--316 GeV)}$} & \colhead{F(100--316 GeV)}\\
\colhead{} &  \colhead{} & \colhead{($10^{-10}$ ph cm$^{-2}$ s$^{-1}$)} & \colhead{} & \colhead{($10^{-11}$ ph cm$^{-2}$ s$^{-1}$)} & \colhead{} & \colhead{($10^{-11}$ ph cm$^{-2}$ s$^{-1}$)}}
\startdata
HESS J1018-589 & 24.7 & 1.3 $\pm$ 0.5 $\pm$ 0.5 & 0.5 & $<$ 5.9 & 5.7 & $<$ 6.2 \\
HESS J1023-575 & 42.8 & 3.7 $\pm$ 0.8 $\pm$ 1.4 & 1.6 & $<$ 9.0 & 18.3 & 1.8 $\pm$ 0.5 $\pm$ 0.9\\
HESS J1119-614 & 17.1 & 1.6 $\pm$ 0.5 $\pm$ 0.5 & 1.0 & $<$ 7.5 & 11.0  & 1.3 $\pm$ 0.6 $\pm$ 0.6\\
HESS J1303-631 & 20.9 & 3.6 $\pm$ 0.9 $\pm$ 1.4 & 26.6 & 17.1 $\pm$ 0.5 $\pm$ 0.6& 7.8 & $<$ 10.7 \\
HESS J1356-645 & 0.2 & $<$ 9.4 & 14.7 & 1.4 $\pm$ 0.7 $\pm$ 0.4 & 10.7 & 6.5 $\pm$ 1.7 $\pm$ 1.7 \\
HESS J1418-609 & 28.7 & 3.7 $\pm$ 0.9 $\pm$ 1.2 & 1.7 & $<$ 10.5 & 0.2 & $<$ 5.4 \\
HESS J1420-607 & 18.1 & 2.4 $\pm$ 0.7 $\pm$ 0.7 & 12.0 & 6.8 $\pm$ 2.6 $\pm$ 3.0 & 12.0 & 4.0 $\pm$ 1.7 $\pm$ 2.1 \\
HESS J1507-622 & 18.2 & 1.4 $\pm$ 0.4 $\pm$ 0.4 & 2.8 & $<$ 7.0 & 0.5 & $<$ 3.9 &\\
HESS J1514-591 & 69.6 & 3.9 $\pm$ 0.7 $\pm$ 1.0 & 65.7 & 15.8 $\pm$ 3.9 $\pm$ 4.8 & 36.8 & 6.7 $\pm$ 2.5 $\pm$ 2.7\\
HESS J1614-518 & 73.4 & 7.9 $\pm$ 1.2 $\pm$ 2.6 & 52.4 & 27.6 $\pm$ 5.9 $\pm$ 11.6 & 27.1 & 9.1 $\pm$ 3.2 $\pm$ 4.6\\
HESS J1616-508 & 46.8 & 6.5 $\pm$ 1.2 $\pm$ 2.0 & 28.4 & 19.9 $\pm$ 5.4 $\pm$ 9.5 & 4.0 & $<$ 10.1 \\
HESS J1632-478 & 71.2 & 10.3 $\pm$ 1.5 $\pm$ 4.2 & 38.1 & 28.2 $\pm$ 6.4 $\pm$ 11.4 & 37.6 & 15.4 $\pm$ 4.4 $\pm$ 6.3 \\
HESS J1634-472 & 15.5 & 3.6 $\pm$ 1.0 $\pm$ 1.9 & 11.7  & 10.8 $\pm$ 3.7 $\pm$ 5.7 & 2.1 & $<$ 8.3 \\
HESS J1640-465 & 20.1 & 3.4 $\pm$ 0.9 $\pm$ 1.5 & 32.2 & 15.1 $\pm$ 4.5 $\pm$ 6.9 & 0 & $<$ 4.5 \\
HESS J1708-443 & 1131.3 & 22.0 $\pm$ 1.5 $\pm$ 3.2 & 22.0 & 5.0 $\pm$ 1.3 $\pm$ 2.0 & 0.0 & $<$ 6.3 \\
HESS J1804-216 & 83.7 & 9.6 $\pm$ 1.4 $\pm$ 2.5 & 37.1 & 24.6 $\pm$ 6.1 $\pm$ 10.7 & 20.8 & 10.4 $\pm$ 3.9 $\pm$ 5.1\\
HESS J1825-137 & 18.2 & 6.4 $\pm$ 1.7 $\pm$ 3.4 & 46.7 & 50.5 $\pm$ 9.4 $\pm$ 20.4 & 19.4 & 11.9 $\pm$ 4.4 $\pm$ 5.5 &\\
HESS J1834-087 & 21.9 & 4.4 $\pm$ 1.6 $\pm$ 1.9 & 7.0 & $<$ 18.7 & 2.5 & $<$ 9.7\\
HESS J1837-069 & 30.6 & 6.9 $\pm$ 1.4 $\pm$ 3.1 & 24.3 & 25.6 $\pm$ 6.7 $\pm$ 10.5 & 28.3 & 15.2 $\pm$ 4.6 $\pm$ 5.9 \\
HESS J1841-055 & 22.8 & 6.4 $\pm$ 1.6 $\pm$ 3.0 & 11.0 & 19.6 $\pm$ 7.1 $\pm$ 5.7 & 22.1 &14.5 $\pm$ 4.5 $\pm$ 7.0 \\
HESS J1848-018 & 16.0 & 5.8 $\pm$ 1.6 $\pm$ 3.1 & 4.2 & $<$ 26.0 & 0.4 & $<$ 10.1\\
HESS J1857+026 & 1.9 & $<$ 2.5 & 12.9 & 13.8 $\pm$ 4.5 $\pm$ 6.8 & 39.3 & 13.0 $\pm$ 4.5 $\pm$ 4.6\\
MGRO J0632+17 & 2144.4 & 27.6 $\pm$ 1.6 $\pm$ 9.8 & 13.0 & 3.0 $\pm$ 1.0 $\pm$ 1.7 & 0 & $<$ 3.8 \\
MGRO J1908+06 & 32.1  & 2.6 $\pm$ 0.5 $\pm$ 1.2 & 0.0  & $<$ 5.0 & 2.8 & 8.6 \\
MGRO J1958+2848 & 18.9 & 1.3 $\pm$ 0.5 $\pm$ 0.4 & 0 & $<$ 3.3 & 0 & $<$3.0\\
MGRO J2019+37 & 100.1 & 3.7 $\pm$ 0.7 $\pm$ 1.5 & 0 & $<$ 3.2 & 2.4 & $<$ 4.9 \\
MGRO J2031+41 & 66.7 & 4.6 $\pm$ 0.8 $\pm$ 1.3 & 5.2 & $<$ 9.4 & 0 & $<$ 3.6\\
MGRO J2228+61 & 108.8 & 2.7 $\pm$ 0.5 $\pm$ 0.5 & 8.0 & $<$ 7.0 & 0 & $<$2.3 \\
VER J0006+727 & 1181.0 & 12.3 $\pm$ 0.9 $\pm$ 2.0 & 38.4 &  3.9 $\pm$ 1.6 $\pm$ 1.6 & 1.5 & $<$ 3.2\\
VER J2016+372 & 25.9 & 1.6 $\pm$ 0.4 $\pm$ 0.5 & 3.4 & $<$ 5.9  & 3.2& $<$ 5.0 \\
\enddata
\tablecomments{\todo[inline]{Put table comments}}
\end{deluxetable}

\begin{deluxetable}{l*{6}c}
\tabletypesize{\scriptsize}
\tablecaption{Spectral fitting of pulsar wind nebula candidates with low energy component
\label{tab:cutoff_test}
}
\tablehead{\colhead{PSR} & \colhead{$\ts_\text{point}$} & \colhead{$\ts_\text{cutoff}$} & \colhead{$F_{0.1-316}$} & \colhead{$G_{0.1-316}$} & \colhead{$\Gamma$} & \colhead{$E_\text{cutoff}$}\\ \colhead{ } & \colhead{ } & \colhead{ } & \colhead{($10^{-9}$\ erg\,cm$^{-2}$\,s$^{-1}$)} & \colhead{($10^{-12}$\ erg\,cm$^{-2}$\,s$^{-1}$)} & \colhead{ } & \colhead{(GeV)}}
\startdata
J0007+7303 & 84.0 & 0.0 & \nodata & \nodata & \nodata & \nodata \\
J0034$-$0534 & 42.4 & 5.5 & \nodata & \nodata & \nodata & \nodata \\
J0218+4232 & 34.7 & 2.8 & \nodata & \nodata & \nodata & \nodata \\
J0340+4130 & 25.1 & 17.2 & $2.38 \pm 1.52$ & $4.95 \pm 1.47$ & $-1.20 \pm 3.36$ & $0.58 \pm 0.66$ \\
J0534+2200 & 4959.1 & 0.0 & \nodata & \nodata & \nodata & \nodata \\
J0633+1746 & 2842.4 & 176.1 & $711.67 \pm 31.00$ & $415.72 \pm 12.92$ & $1.40 \pm 0.10$ & $1.00 \pm 0.12$ \\
J0835$-$4510 & 304.7 & 23.7 & $260.77 \pm 22.71$ & $115.15 \pm 7.65$ & $1.84 \pm 0.17$ & $1.00 \pm 0.30$ \\
J1023$-$5746 & 83.0 & 0.0 & \nodata & \nodata & \nodata & \nodata \\
J1119$-$6127 & 123.2 & 0.0 & \nodata & \nodata & \nodata & \nodata \\
J1513$-$5908 & 122.6 & 0.0 & \nodata & \nodata & \nodata & \nodata \\
J1620$-$4927 & 39.1 & 43.8 & $80.75 \pm 20.97$ & $70.24 \pm 10.35$ & $0.48 \pm 0.39$ & $0.65 \pm 0.16$ \\
J1709$-$4429 & 30.7 & 7.4 & \nodata & \nodata & \nodata & \nodata \\
J1744$-$1134 & 74.4 & 13.7 & \nodata & \nodata & \nodata & \nodata \\
J1746$-$3239 & 47.6 & 33.3 & $64.84 \pm 16.74$ & $39.00 \pm 6.10$ & $0.79 \pm 0.61$ & $0.50 \pm 0.24$ \\
J1747$-$2958 & 30.3 & 12.6 & \nodata & \nodata & \nodata & \nodata \\
J1809$-$2332 & 29.0 & 10.8 & \nodata & \nodata & \nodata & \nodata \\
J1813$-$1246 & 53.3 & 3.4 & \nodata & \nodata & \nodata & \nodata \\
J1836+5925 & 5019.4 & 203.4 & $449.37 \pm 14.27$ & $330.04 \pm 8.76$ & $1.40 \pm 0.03$ & $1.64 \pm 0.06$ \\
J2021+4026 & 920.6 & 138.0 & $949.97 \pm 56.79$ & $586.25 \pm 21.87$ & $1.64 \pm 0.08$ & $1.81 \pm 0.26$ \\
J2032+4127 & 28.5 & 0.0 & \nodata & \nodata & \nodata & \nodata \\
J2055+2539 & 109.0 & 26.3 & $32.23 \pm 2.43$ & $17.45 \pm 1.03$ & $1.51 \pm 0.04$ & $1.00 \pm 0.04$ \\
J2124$-$3358 & 106.5 & 28.7 & $6.61 \pm 2.50$ & $9.86 \pm 1.60$ & $0.06 \pm 0.92$ & $0.87 \pm 0.43$ \\
J2302+4442 & 115.0 & 12.7 & \nodata & \nodata & \nodata & \nodata \\
\enddata
\tablecomments{\todo[inline]{Put table comments}}
\end{deluxetable}




\figref{cutoff_test} shows the cutoff test\ldots

\begin{figure}
  \ifdefined\bwfigures
  \plotone{cutoff_test_bw.eps}
  \else
  \plotone{cutoff_test_color.eps}
  \fi
  \caption{Cutoff test for some pulsars\dots}
  \label{fig:cutoff_test}
\end{figure}


\figref{variability} shows the variability test for each source
candidate. The distribution of \tsvar is plotted in 

\begin{deluxetable}{l*{2}c}
\tabletypesize{\scriptsize}
\tablecaption{Variability test
\label{tab:variability}
}
\tablehead{\colhead{PSR} & \colhead{$\ts_\text{point}$} & \colhead{$\ts_\text{var}$}}
\startdata
J0007+7303 & 84.0 & 187.5 \\
J0034$-$0534 & 42.4 & 25.6 \\
J0218+4232 & 34.7 & 35.8 \\
J0340+4130 & 25.1 & None \\
J0534+2200 & 4959.1 & 79.7 \\
J0633+1746 & 2842.4 & 8.7 \\
J0835$-$4510 & 304.7 & 39.4 \\
J1023$-$5746 & 83.0 & 209.3 \\
J1119$-$6127 & 123.2 & None \\
J1513$-$5908 & 122.6 & None \\
J1620$-$4927 & 39.1 & None \\
J1709$-$4429 & 30.7 & 324.1 \\
J1744$-$1134 & 74.4 & None \\
J1746$-$3239 & 47.6 & None \\
J1747$-$2958 & 30.3 & 333.5 \\
J1809$-$2332 & 29.0 & None \\
J1813$-$1246 & 53.3 & None \\
J1836+5925 & 5019.4 & 12.5 \\
J2021+4026 & 920.6 & 48.2 \\
J2032+4127 & 28.5 & None \\
J2055+2539 & 109.0 & 66.6 \\
J2124$-$3358 & 106.5 & 47.4 \\
J2302+4442 & 115.0 & 116.5 \\
\enddata
\tablecomments{\todo[inline]{Put table comments}}
\end{deluxetable}

\begin{figure}
  \ifdefined\bwfigures
  \plotone{variability_bw.eps}
  \else
  \plotone{variability_color.eps}
  \fi
  \caption{Distribution of \tsvar for each source candidate. \todo[inline]{Disclaimer about crab not being included}.}
  \label{fig:variability}
\end{figure}
