\subsection{Analysis of the \fermi-LAT data}
\label{subsec:analysis_off_peak}

We develope a procedure for characterizing any emission found
in off-peak phase internals for the pulsars in this catalog.  
This proceure used both the spectral and spatial characteristics
of any observed emission to determine the physical origin of the emission.

Pulsar wind nebula are expected in many cases to be spatially
extended. For example, \velax and HESS\;J1825-137 are PWN
that have been observed by the \lat to be spatially extended
\citep{LAT_collaboration_Vela_X_2010,LAT_collaboration_HESS_J1825_2011}.
On the other hand, not all pulsar wind nebula are expected to be
significantly spatially resolved at \gev energies due to the finite
instrument resolution of the LAT. For example, the Crab nebula appears
as a point-like source in the LAT but is distinuished in the off-peak
from the Crab pulsar by its hard spectrum for E$\gtrsim$1 \gev.
\citep{LAT_collaboration_crab_2010}.

On the other hand, a previous analysis by the Fermi LAT collaboration
found seven off-peak regions to have significant emission which is
point-like in nature that is characterized by a pulsar-like cutoff
spectrum \citep{LAT_collaboration_PWNCAT_2011}.  We can thefore use
either spatial extension or significant emission at high energy to
distinguish the PWN scenario and point-like emission with a cutoff
spectrum to distingiugh the pulsar scenario.

To perform this test, we used the likelihood fitting pacakge \pointlike to study
the spatial character of emission in the off-peak regions and \gtlike to
study the spectral character of the emission. These tools provide complementary
features and this method is very similar to the approach used in the
second \lat catalog \citep{LAT_Collaboration_2FGL_2012}
and a followup search for spatially extended sources 
\citep{LAT_collaboration_extended_search_2012}.

First, we assumed the potential emission to have a point-like spatial model
and we used \pointlike to fit the position of 
the off-peak region. The localization procedure is described in \cite{LAT_Collaboration_2FGL_2012}. 
Following the position fit, we used the best fit positions
obtained through \pointlike and performed a spectral analysis using \gtlike.

\todo[inline]{Describe assumed spectral model}

After fitting the position of a source, we use \gtlike to perform a likelihood ratio test for
the detection of the source. Here, \ts is defiend as
\begin{equation}
  \ts = 2 \log(\likelihood_\text{pt}/\likelihood_\text{bg})
\end{equation}
where $\likelihood_\text{pt}$ is the poission likelihood for a model including the source and $\likelihood_\text{bg}$
the likliehood for a model not including the source.
We set the threshold for detection of signfiicant emission at $\ts>25$, 
corresponding to a significance just over $4\sigma$ \citep{LAT_Collaboration_1FGL_2010}.

We then used \pointlike to test wheter the observed emisssion was spatially extended, assuming
a radially-symmetric Gaussian spatial model. 
\pointlike canb e used simulatenously fits the position and the extension of the assumed Gaussian source, following
the description in \citep{LAT_collaboration_extended_search_2012}. 
After the extension fit, we refit the spectrum of the spatially extended source using \gtlike
and performed a likelihood ratio for the significance of the extenstion of a source.
\begin{equation}
  \tsext = 2 \log(\likelihood_\text{ext}/\likelihood_\text{pt})
\end{equation}
We set the thresholt for detecting the significance of a spatially extended
soruce at $\tsext>16$, corresponding to a $4\sigma$ detection \citep{LAT_collaboration_extended_search_2012}. 


\citep{LAT_collaboration_extended_search_2012}. 


From \gtlike, we obtained  
\begin{equation}
  \tscut= 2 \log(\likelihood_\text{expcut}/\likelihood_\text{pt})
\end{equation}

%for this analysis, 
%we then attempted to fit the position of the region using \pointlike.
%for the regions that contained significant emission, we then
%
%extended source search will be referneced: \cite{extended_source_search}.
%
%methods for data analysis
%\begin{itemize}
%\item cut on pulsar phase
%\item perform localization or extension fitting using \gtlike using energies from 1 \gev to 316 \gev.
%\item if it is point-like, perform an extension upper limit analysis 
%\item perform spectral analysis using \gtlike for energies above 100 \mev to 316 \gev.
%\item there is a detection if  $\ts>25$ in the point-like source hypothesis after fitting the position of the point-like source. 
%\item consider the source to be extended if $\tsext>16$. similar to extended source search paper \todo{cite extended source search paper}.
%\item calculate \tscut for all energies.
%\item compute \tsvar
%\end{itemize}
%
%
%when to consider the source a pulsar or pwn:
%\begin{itemize}
%  \item if extended, then it is a pwn (cannot be a pulsar)
%  \item if it is significiant for $e>10$ \gev, it is a pwn (too hard to be a pulsar)
%  \item otherwise, if it has a cutoff, it is a pulsar candidate
%  \item for point-like emission that is not significantly cutoff, the emission  mechanism is uncertain.
%\end{itemize}
