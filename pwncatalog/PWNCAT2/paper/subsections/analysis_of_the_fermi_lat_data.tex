\subsection{Analysis of the \fermi-LAT data}

Extended Source Search will be referneced: \cite{extended_source_search}.

Methods for data analysis
\begin{itemize}
\item Cut on pulsar phase
\item Perform localization or extension fitting using \gtlike using energies from 1 \gev to 316 \gev.
\item If it is point-like, perform an extension upper limit analysis 
\item Perform spectral analysis using \gtlike for energies above 100 \mev to 316 \gev.
\item There is a detection if  $\ts>25$ in the point-like source hypothesis after fitting the position of the point-like source. 
\item Consider the source to be extended if $\tsext>16$. Similar to extended source search paper \todo{cite extended source search paper}.
\item Calculate \tscut for all energies.
\end{itemize}

\subsection{Variability}

\subsection{When to consider the source a pulsar or PWN.}
\begin{itemize}
  \item If extended, then it is a PWN (cannot be a pulsar)
  \item If it is significiant for $E>10$ \gev, it is a PWN (too hard to be a pulsar)
  \item Otherwise, if it has a cutoff, it is a Pulsar candidate
  \item For point-like emission that is not significantly cutoff, the emission  mechanism is uncertain.
\end{itemize}
