\subsection{Off-peak Analysis Method}
\label{subsec:off_peak_analysis}

We develope a procedure for characterizing any emission found
in off-peak phase internals for the pulsars in this catalog.  
This proceure used both the spectral and spatial characteristics
of any observed emission to determine the physical origin of the emission.

Pulsar wind nebula are expected in many cases to be spatially
extended. For example, \velax and HESS\;J1825-137 are PWN
that have been observed by the \lat to be spatially extended
\citep{LAT_collaboration_Vela_X_2010,LAT_collaboration_HESS_J1825_2011}.
On the other hand, not all pulsar wind nebula are expected to be
significantly spatially resolved at \gev energies due to the finite
instrument resolution of the LAT. For example, the Crab nebula appears
as a point-like source in the LAT but is distinuished in the off-peak
from the Crab pulsar by its hard spectrum for E$\gtrsim$1 \gev.
\citep{LAT_collaboration_crab_2010}.

On the other hand, a previous analysis by the Fermi LAT collaboration
found seven off-peak regions to have significant emission which is
point-like in nature that is characterized by a pulsar-like cutoff
spectrum \citep{LAT_collaboration_PWNCAT_2011}.  We can thefore use
either spatial extension or significant emission at high energy to
distinguish the PWN scenario and point-like emission with a cutoff
spectrum to distingiugh the pulsar scenario.

To perform this test, we used the likelihood fitting pacakge \pointlike
to study the spatial character of emission in the off-peak regions
and \gtlike in binned model to study the spectral character of
the emission. These tools provide complementary features and this
method is very similar to the approach used in the second \lat catalog
\citep{LAT_Collaboration_2FGL_2012} and a followup search for spatially
extended sources \citep{LAT_collaboration_extended_search_2012}.

For this analysis, we build a model of the sky consistent with the
second \lat catalog. We included as background sources all
nearby sources from the second 
and used the same background model as the catalog \citep{LAT_Collaboration_2FGL_2012}.
Our analysis energy range varied from 100 \mev to 316 \gev.
For this analysis, we removed all of the on-peak photons before binning the data
and scaled the exposure to fit the all-phase flux assuming 
constant emission with phase.

With this model of the sky, we used fit an assumed source in the off-peak
emission.  First, we assumed the potential emission to have a point-like
spatial model and (unless otherwise noted) a power-law spectral model.
We used \pointlike to fit the position of the off-peak region following
the procedure described in \cite{LAT_Collaboration_2FGL_2012}.  We used
the best fit positions obtained through \pointlike and performed a
spectral analysis using \gtlike.

After fitting the position of a source, we use \gtlike to perform a likelihood-ratio test for
the detection of the source. Here, \ts is defined as
\begin{equation}
  \ts = 2 \log(\likelihood_\text{pt}/\likelihood_\text{bg})
\end{equation}
where $\likelihood_\text{pt}$ is the Poission likelihood for a model including the source and $\likelihood_\text{bg}$
the likelihood for a model not including the source.
We set the threshold for detection of signfiicant emission at $\ts>25$, 
corresponding to a significance just over $4\sigma$ \citep{LAT_Collaboration_1FGL_2010}.

For the significantly-detected source, after fitting the position of the source
we test to see if the spectrum of the source is significantly 
cutoff following the description in 
\cite{LAT_collaboration_PWNCAT_2011}. After fitting the source using \gtlike
with both a power-law and exponentially-cutoff spectral model, we
define the likelihood ratio test for the cutoff test as
From \gtlike, we obtained  
\begin{equation}
  \tscut= 2 \log(\likelihood_\text{cutoff}/\likelihood_\text{pt})
\end{equation}
where $\likelihood_\text{cutoff}$ is the poisson likelihood for a model including the source
with a cutoff spectral model.
We set the threshold for detecting a significant cutoff at 
$\tscut>16$, corresponding to a $4\sigma$ detection \citep{LAT_collaboration_PWNCAT_2011} .

We used \pointlike to test whether the observed emisssion was
spatially extended, assuming a radially-symmetric Gaussian spatial
model.  \pointlike can simulatenously fit the position and the
extension of the assumed Gaussian source, following the description in
\citep{LAT_collaboration_extended_search_2012}.  After the extension fit,
we refit the spectrum of the spatially extended source using \gtlike
and performed a likelihood ratio for the significance of the extenstion
of a source.  Following \cite{LAT_collaboration_extended_search_2012},
we define
\begin{equation}
  \tsext = 2 \log(\likelihood_\text{ext}/\likelihood_\text{pt})
\end{equation}
where $\likelihood_\text{ext}$ is the Poisson likelihood assuming the
source is spatially extended. We set the threshold for detecting the
significance of a spatially extended source at $\tsext>16$, corresponding
to a $4\sigma$ detection \citep{LAT_collaboration_extended_search_2012}.

For sources that are not significantly deteced over the energy range,
we compute flux upper limits for the region assuming a fixed spectral
index of 2.0.  We also computed a pulsed upper limit assuming a canonical
pulsar spectrum with an index of -1.7 and a cutoff energy of 3 \gev.

To better assess the spectral charter of any emission in the off-peak
region, we performed a spectral analsyis in three smaller energy bins
(100 \mev to 1 \gev, 1 \gev to 10 \gev, and 10 \gev to 316 \gev). In each
energy bin, we fit the flux and the spectral index of the source. For
sources not singificantly detected, we compute a flux upper limit
(assuming a fixed spectral index of 2) in the energy bin.


Following the discovery of variable emission from the Crab nebula
by the LAT, it is interesting to search for other variable PWN
\citep{LAT_Collaboration_Crab_Flare_2011}.  Therefore, we testing all the
off-peak regions for variable emission.  We divided the 3 year time range
into 36 month-long intervals and fit the flux of the source indepnedently
in each time range. We compute this significance of the variability
using \tsvar following the same procedure as the second LAT catalog
\citep{LAT_Collaboration_2FGL_2012}.  Since we have 36 months of data,
the null distribution (assuming no variability) should follow $\chi^2$
distribution with 35 degrees of freedom, and we set the detection critera
for significant variability at $\tsvar>91.7$, corresponding to a $4\sigma$
significance detection threshold.

In many situations, this algorithm would fail due to not modeling the
position of nearby sources. We expect to be more sensitive to nearby
sources than the second \lat catalog both because of our expanded data
set (three years of observations instead of two) and also because, for
very bright pulsars, we are more sensitivite to other sources once the
pulsar has been removed.

Unfortuantly, in many situations, this algorithm failed due to
systematics associated with the modeling nearby sources. The large and
energy depdended point-spread function of the \lat causes the analysis
of any one source to be sensitively affected by the modeling of nearby
sources. Therefore, we had to, in many situations, iteratively improve
the model of a region by including new sources. This procedure 
involved generating maps of residual test statistic assuming the
presence of a new source (of a fixed spectral index of 2) and looking
for regions with $\ts\ge25$.  For these positions, we would include a
new source into our model, fit the position and spectrum of this source,
and iterate until there was no remaining $\ts\ge25$ emission.

Even so, there are still some linering regions which have significant
emission but which remain particular difficult to model and understand
the origion of the emission for.  These issues are most likely due to
systematics associated with the model of the galactic diffuse emission
in the region and issues associated with modeling nearby sources. We
will flag these problematic regions in our analysis.

\todo[inline]{Describe special case of Crab spectrum}
