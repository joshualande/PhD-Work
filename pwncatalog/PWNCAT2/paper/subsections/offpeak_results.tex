\subsection{Results}

Following the procedure described in \subsecref{off_peak},
we performed an analysis of the off-peak emission for all 107
pulsars in this catalog. In this paper, we will present
a condensed summary of the results. We will also
describe a supplemental \fits-format table which will contain
the full spectral and spatial results for each region.

Table 
First, we tested the sources to see if they
were spatially extended. The localization results are in \tabref{offpeak}.

\begin{deluxetable}{l*{7}c}
\tabletypesize{\scriptsize}
\tablecaption{Off-Peak Spatial and Spectral Results
\label{tab:offpeak}
}
\tablehead{\colhead{PSR} & \colhead{Phase} & \colhead{$\ts_\text{point}$} & \colhead{\tsext} & \colhead{$\ts_\text{cutoff}$} & \colhead{$F_{0.1-316}$} & \colhead{$\Gamma$} & \colhead{$E_\text{cutoff}$}\\ \colhead{ } & \colhead{ } & \colhead{ } & \colhead{ } & \colhead{ } & \colhead{($10^{-9}$\ erg\,cm$^{-2}$\,s$^{-1}$)} & \colhead{ } & \colhead{(MeV)}}
\startdata
J0007+7303 & 0.53 - 0.89 & 71.2 & 10.8 & 0.0 & $47.22 \pm 8.60$ & $2.61 \pm 0.14$ & \nodata \\
J0034$-$0534 & 0.21 - 0.68 & 42.8 & 0.0 & 4.9 & $16.82 \pm 4.58$ & $2.44 \pm 0.16$ & \nodata \\
J0101$-$6422 & 0.22 - 0.61 & 25.2 & 0.0 & 18.2 & $45.76 \pm 6.41$ & $-5.00 \pm 0.00$ & $100.02 \pm 1.46$ \\
J0102+4839 & 0.81 - 0.57 & 69.6 & 0.0 & 5.3 & $26.34 \pm 4.89$ & $2.42 \pm 0.10$ & \nodata \\
J0106+4855 & 0.67 - 0.03, 0.18 - 0.54 & 25.5 & 0.0 & 0.2 & $29.14 \pm 7.04$ & $2.80 \pm 0.17$ & \nodata \\
J0218+4232 & 0.82 - 0.21 & 50.1 & 0.0 & 6.6 & $55.94 \pm 11.20$ & $2.72 \pm 0.13$ & \nodata \\
J0340+4130 & 0.13 - 0.64 & 26.8 & 0.1 & 16.5 & $2.45 \pm 1.48$ & $0.93 \pm 2.51$ & $645.30 \pm 580.54$ \\
J0534+2200 & 0.59 - 0.87 & 5253.1 & 0.0 & 0.0 & $764.73 \pm 18.42$ & \nodata & \nodata \\
J0633+1746 & 0.83 - 0.93 & 3649.0 & 2.3 & 237.3 & $719.12 \pm 27.80$ & $-1.42 \pm 0.09$ & $998.24 \pm 116.74$ \\
J0734$-$1559 & 0.28 - 0.84 & 28.4 & 10.7 & 33.4 & $31.63 \pm 6.36$ & $1.77 \pm 0.40$ & $100.10 \pm 3.01$ \\
J0835$-$4510 & 0.81 - 0.03 & 506.0 & 241.9 & 0.0 & $431.14 \pm 22.25$ & $2.11 \pm 0.03$ & \nodata \\
J0908$-$4913 & 0.66 - 0.04, 0.17 - 0.54 & 35.5 & 9.3 & 74.1 & $40.13 \pm 37.59$ & $-1.20 \pm 0.71$ & $999.01 \pm 0.71$ \\
J1023$-$5746 & 0.67 - 0.03 & 84.8 & 57.7 & 14.1 & $230.75 \pm 6722.99$ & $2.04 \pm 0.72$ & \nodata \\
J1044$-$5737 & 0.55 - 0.97 & 27.9 & 187.0 & 0.0 & $243.05 \pm 2.30$ & $1.95 \pm 0.00$ & \nodata \\
J1105$-$6107 & 0.73 - 0.46 & 28.9 & 36.6 & 78.6 & $161.16 \pm 6644.93$ & $2.14 \pm 0.72$ & \nodata \\
J1112$-$6103 & 0.31 - 0.04 & 122.2 & 93.4 & 12.4 & $232.99 \pm 26.43$ & $2.12 \pm 0.04$ & \nodata \\
J1119$-$6127 & 0.59 - 0.18 & 40.7 & 18.3 & 0.0 & $54.59 \pm 3399.22$ & $2.16 \pm 0.70$ & \nodata \\
J1124$-$5916 & 0.69 - 0.05 & 86.2 & 0.0 & 24.2 & $26.39 \pm 19.44$ & $-0.79 \pm 0.72$ & $1000.00 \pm 0.71$ \\
J1410$-$6132 & 0.55 - 0.24 & 42.4 & 91.6 & 12.5 & $81.41 \pm 2783.70$ & $1.79 \pm 0.72$ & \nodata \\
J1513$-$5908 & 0.53 - 0.15 & 100.6 & 2.1 & 0.0 & $15.83 \pm 984.68$ & $1.74 \pm 0.72$ & \nodata \\
J1620$-$4927 & 0.54 - 0.98 & 27.9 & 0.5 & 39.9 & $72.80 \pm 22.94$ & $-0.86 \pm 0.25$ & $1000.00 \pm 173.04$ \\
J1744$-$1134 & 0.14 - 0.74 & 61.4 & 0.0 & 15.0 & $33.38 \pm 15.81$ & $2.25 \pm 0.09$ & \nodata \\
J1746$-$3239 & 0.41 - 0.99 & 53.8 & 7.2 & 42.0 & $61.29 \pm 54.13$ & $-1.18 \pm 0.71$ & $999.95 \pm 0.71$ \\
J1747$-$2958 & 0.66 - 0.1 & 53.6 & 0.0 & 102.6 & $146.08 \pm 125.15$ & $-1.17 \pm 0.71$ & $991.64 \pm 0.70$ \\
J1809$-$2332 & 0.53 - 0.91 & 31.5 & 13.0 & 15.2 & $85.80 \pm 49.10$ & $2.45 \pm 0.11$ & \nodata \\
J1813$-$1246 & 0.77 - 0.01 & 57.8 & 0.0 & 12.0 & $147.87 \pm 33.22$ & $2.46 \pm 0.05$ & \nodata \\
J1836+5925 & 0.76 - 0.92 & 10450.2 & 0.0 & 364.6 & $497.25 \pm 10.72$ & $-1.49 \pm 0.02$ & $2024.58 \pm 59.89$ \\
J2021+4026 & 0.25 - 0.41 & 1712.8 & 37.4 & 228.3 & $1248.80 \pm 20643.72$ & $2.23 \pm 0.72$ & \nodata \\
J2043+1711 & 0.79 - 0.06, 0.18 - 0.55 & 151.6 & 0.0 & 11.9 & $23.36 \pm 9.33$ & $2.18 \pm 0.08$ & \nodata \\
J2055+2539 & 0.37 - 0.87 & 117.0 & 0.0 & 30.4 & $30.05 \pm 31.95$ & $-1.45 \pm 0.70$ & $999.99 \pm 0.71$ \\
J2124$-$3358 & 0.09 - 0.69 & 177.7 & 0.0 & 27.0 & $10.88 \pm 3.78$ & $-0.61 \pm 0.64$ & $1000.01 \pm 437.18$ \\
J2302+4442 & 0.75 - 0.23 & 113.7 & 0.0 & 8.4 & $34.35 \pm 5.34$ & $2.36 \pm 0.09$ & \nodata \\
\enddata
\tablecomments{\todo[inline]{Put table comments}}
\end{deluxetable}



Next, we performed a spectral analysis over all energy using the best
fit morphology. XXX shows the results of the all energy
analysis of the off-peak emission for each pulsar.

Next, we fit a powerlaw independently in each energy
bin. XXX shows the results of the analysis in separate
energy bins of each pulsar.

Finally, we tested sources to see which were
variable. XXX shows the results of the cutoff test for
pulsars with significant low-energy emission.


\todo[inline]{Finally, a description of the acompanying fits table}



\figref{cutoff_test} shows the cutoff test\ldots

\begin{figure}
  \ifdefined\bwfigures
  \plotone{cutoff_test_bw.eps}
  \else
  \plotone{cutoff_test_color.eps}
  \fi
  \caption{Cutoff test for some pulsars\dots}
  \label{fig:cutoff_test}
\end{figure}


\todo[inline]{We tested all sources for variability.}

\todo[inline]{Describe upper limits}

Special cases:
\begin{itemize}
  \item Crab has a funny spectrum, we have (so far) fixed it to
  \item \velax Spatialmodel
\end{itemize}


Results:
\begin{itemize}
  \item List of all XXX detections detected
  \item For each pulsar, is it a 
    \begin{itemize}
      \item Magnetospheric emission
      \item PWN emission
      \item Region where the fit shows trouble for some reason.
    \end{itemize}
\end{itemize}

