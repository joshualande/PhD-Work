\subsection{Results}

After analyzing the off-peak emission using the pipeline described
in \subsecref{off_peak_analysis}, we consider any emission in the
off-peak region to be magnetospheric in nature if the emission is
not significantly-extended and has a significantly-cutoff spectrum.
We consider the emission to originate in the pulsar wind if it is
spatially extended or had a hard spectral index.  If the source is
point-like and had a soft spectrum, or if it is formally spatially
extended but if the best-fit extension is biased by nearby point-like
sources, we flag the source as confused and do not speculate on the
origin of the emission.  For these confused regions, we include in our
results the best fit spectrum at the position of the pulsar assuming a
point-like spectral model.

A condensed summary of the results of the pipeline can be found in
\tabref{off_peak_table}. This table includes the significance of the source,
extension, and cutoff as well as the spectrum of the source.
In addition, \figref{off_peak_seds} shows the best fit spectral
models for a selection of the significantly-detected sources.


\begin{deluxetable}{l*{7}c}
\tabletypesize{\scriptsize}
\tablecaption{Off-Peak Spatial and Spectral Results
\label{tab:off_peak_table}
}
\tablehead{\colhead{PSR} & \colhead{Phase} & \colhead{$\ts_\text{point}$} & \colhead{\tsext} & \colhead{$\ts_\text{cutoff}$} & \colhead{$F_{0.1-316}$} & \colhead{$\Gamma$} & \colhead{$E_\text{cutoff}$}\\ \colhead{ } & \colhead{ } & \colhead{ } & \colhead{ } & \colhead{ } & \colhead{($10^{-9}$\ erg\,cm$^{-2}$\,s$^{-1}$)} & \colhead{ } & \colhead{(MeV)}}
\startdata
J0007+7303 & 0.53 - 0.89 & 71.2 & 10.8 & 0.0 & $47.22 \pm 8.60$ & $2.61 \pm 0.14$ & \nodata \\
J0034$-$0534 & 0.21 - 0.68 & 42.8 & 0.0 & 4.9 & $16.82 \pm 4.58$ & $2.44 \pm 0.16$ & \nodata \\
J0101$-$6422 & 0.22 - 0.61 & 25.2 & 0.0 & 18.2 & $45.76 \pm 6.41$ & $-5.00 \pm 0.00$ & $100.02 \pm 1.46$ \\
J0102+4839 & 0.81 - 0.57 & 69.6 & 0.0 & 5.3 & $26.34 \pm 4.89$ & $2.42 \pm 0.10$ & \nodata \\
J0106+4855 & 0.67 - 0.03, 0.18 - 0.54 & 25.5 & 0.0 & 0.2 & $29.14 \pm 7.04$ & $2.80 \pm 0.17$ & \nodata \\
J0218+4232 & 0.82 - 0.21 & 50.1 & 0.0 & 6.6 & $55.94 \pm 11.20$ & $2.72 \pm 0.13$ & \nodata \\
J0340+4130 & 0.13 - 0.64 & 26.8 & 0.1 & 16.5 & $2.45 \pm 1.48$ & $0.93 \pm 2.51$ & $645.30 \pm 580.54$ \\
J0534+2200 & 0.59 - 0.87 & 5253.1 & 0.0 & 0.0 & $764.73 \pm 18.42$ & \nodata & \nodata \\
J0633+1746 & 0.83 - 0.93 & 3649.0 & 2.3 & 237.3 & $719.12 \pm 27.80$ & $-1.42 \pm 0.09$ & $998.24 \pm 116.74$ \\
J0734$-$1559 & 0.28 - 0.84 & 28.4 & 10.7 & 33.4 & $31.63 \pm 6.36$ & $1.77 \pm 0.40$ & $100.10 \pm 3.01$ \\
J0835$-$4510 & 0.81 - 0.03 & 506.0 & 241.9 & 0.0 & $431.14 \pm 22.25$ & $2.11 \pm 0.03$ & \nodata \\
J0908$-$4913 & 0.66 - 0.04, 0.17 - 0.54 & 35.5 & 9.3 & 74.1 & $40.13 \pm 37.59$ & $-1.20 \pm 0.71$ & $999.01 \pm 0.71$ \\
J1023$-$5746 & 0.67 - 0.03 & 84.8 & 57.7 & 14.1 & $230.75 \pm 6722.99$ & $2.04 \pm 0.72$ & \nodata \\
J1044$-$5737 & 0.55 - 0.97 & 27.9 & 187.0 & 0.0 & $243.05 \pm 2.30$ & $1.95 \pm 0.00$ & \nodata \\
J1105$-$6107 & 0.73 - 0.46 & 28.9 & 36.6 & 78.6 & $161.16 \pm 6644.93$ & $2.14 \pm 0.72$ & \nodata \\
J1112$-$6103 & 0.31 - 0.04 & 122.2 & 93.4 & 12.4 & $232.99 \pm 26.43$ & $2.12 \pm 0.04$ & \nodata \\
J1119$-$6127 & 0.59 - 0.18 & 40.7 & 18.3 & 0.0 & $54.59 \pm 3399.22$ & $2.16 \pm 0.70$ & \nodata \\
J1124$-$5916 & 0.69 - 0.05 & 86.2 & 0.0 & 24.2 & $26.39 \pm 19.44$ & $-0.79 \pm 0.72$ & $1000.00 \pm 0.71$ \\
J1410$-$6132 & 0.55 - 0.24 & 42.4 & 91.6 & 12.5 & $81.41 \pm 2783.70$ & $1.79 \pm 0.72$ & \nodata \\
J1513$-$5908 & 0.53 - 0.15 & 100.6 & 2.1 & 0.0 & $15.83 \pm 984.68$ & $1.74 \pm 0.72$ & \nodata \\
J1620$-$4927 & 0.54 - 0.98 & 27.9 & 0.5 & 39.9 & $72.80 \pm 22.94$ & $-0.86 \pm 0.25$ & $1000.00 \pm 173.04$ \\
J1744$-$1134 & 0.14 - 0.74 & 61.4 & 0.0 & 15.0 & $33.38 \pm 15.81$ & $2.25 \pm 0.09$ & \nodata \\
J1746$-$3239 & 0.41 - 0.99 & 53.8 & 7.2 & 42.0 & $61.29 \pm 54.13$ & $-1.18 \pm 0.71$ & $999.95 \pm 0.71$ \\
J1747$-$2958 & 0.66 - 0.1 & 53.6 & 0.0 & 102.6 & $146.08 \pm 125.15$ & $-1.17 \pm 0.71$ & $991.64 \pm 0.70$ \\
J1809$-$2332 & 0.53 - 0.91 & 31.5 & 13.0 & 15.2 & $85.80 \pm 49.10$ & $2.45 \pm 0.11$ & \nodata \\
J1813$-$1246 & 0.77 - 0.01 & 57.8 & 0.0 & 12.0 & $147.87 \pm 33.22$ & $2.46 \pm 0.05$ & \nodata \\
J1836+5925 & 0.76 - 0.92 & 10450.2 & 0.0 & 364.6 & $497.25 \pm 10.72$ & $-1.49 \pm 0.02$ & $2024.58 \pm 59.89$ \\
J2021+4026 & 0.25 - 0.41 & 1712.8 & 37.4 & 228.3 & $1248.80 \pm 20643.72$ & $2.23 \pm 0.72$ & \nodata \\
J2043+1711 & 0.79 - 0.06, 0.18 - 0.55 & 151.6 & 0.0 & 11.9 & $23.36 \pm 9.33$ & $2.18 \pm 0.08$ & \nodata \\
J2055+2539 & 0.37 - 0.87 & 117.0 & 0.0 & 30.4 & $30.05 \pm 31.95$ & $-1.45 \pm 0.70$ & $999.99 \pm 0.71$ \\
J2124$-$3358 & 0.09 - 0.69 & 177.7 & 0.0 & 27.0 & $10.88 \pm 3.78$ & $-0.61 \pm 0.64$ & $1000.01 \pm 437.18$ \\
J2302+4442 & 0.75 - 0.23 & 113.7 & 0.0 & 8.4 & $34.35 \pm 5.34$ & $2.36 \pm 0.09$ & \nodata \\
\enddata
\tablecomments{A condensed version of the spectral analysis of
sources detected significantly in the off-peak regions.
This table includes the significance of the source, extension, and 
spectral cutoff (\ts, \tsext, \tscut).  The table also includes
the best fit spectrum for these sources. For sources that are
significantly cutoff, the spectral index and cutoff energy are from the
fit of a cutoff spectrum.
}
\end{deluxetable}


\begin{figure}
  \ifdefined\bwfigures
  \plotone{off_peak_seds_bw.eps}
  \else
  \plotone{off_peak_seds_color.eps}
  \fi
  \caption{SEDs for several sources significantly-detected in the off-peak region.}
  \label{fig:seds_test}
\end{figure}


Consistent with \citep{LAT_Collaboration_Crab_Flare_2011}, we found
the Crab nebula to be highly variable with \tsvar=XXXXXX.  Besides that,
we found no significantly variable off-peak emission.  The results of
the variability test are contained in the auxiliary information.

The results of the spectral analysis in smaller energy bands described
in section \subsecref{peak_analysis} are included in the auxiliary
information. In addition, upper limits computed assuming a power law
spectral model and a canonical pulsar spectrum are included in the
auxiliary information.

In our pipeline, we found XXXXX off-peak regions to show significant PWNe.
We detected the Vela X PWN (associated with PSRJ0835$-$4510), the Crab Nebula (associated
with PSRJ0534+2200), and MSH 15$-$52 (associated with PSRJ1513-5908).
\begin{itemize}
  \item What to say about PSRJ1023-5746, which was a PWN in PWNCAT1, but here shows a complicated spectrum???
\end{itemize}

We detected XXXXX of the regions to show clear magnetopsheric pulsar emission.
These pulsars are ...

And Finally, we found XXXXX regions which where the analysis could not decisively determine the nature of the emission.
\begin{itemize}
  \item PSRJ0101-6422 was fit to have an unphysically soft spectrum. We suspect that source is an artifact of
  problems modeling the nearby diffuse emission.
  \item PSRJ1112-6103: Very close to MSH 11~62 (arXiv:1202.3371v1). Nearby source is 2FGL J1112.1-6040. Significant extension.
  \item When fitting PSRJ1119-6127, description of adding nearby point
  source to represent emission residual from o represent residual from
  PSRJ1112-6103 looking spatially
    extended.
  \item PSRJ1418-6058 and PSRJ1420-6048, very difficult to analyze because both very near each other.
  \item Something about the complicated Gamma Cygni PSR: PSRJ2021+4026
  \item PSRJ1105-6107: something about adding source to background model that is very nearby.
  \item PSRJ1648-4611 (what to do about nearby 2FGL source and other residual emission. Is this paper relevant: http://arxiv.org/pdf/1111.2043.pdf???)?
  \item What to do about emission from PSR J1023 at low energy???
  \item PSRJ0908-4913: what to say about far localization?
\end{itemize}

