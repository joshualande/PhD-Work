\subsection{Off-peak Phase Selection}


To study the off-peak emission of \lat-detected pulsars, we first
developed a new method for defining the off-peak emission.
The primary constraint for this method was that it was systematic,
computationally efficient and model independent, and that it correctly
removed the pulsed emission for already studied pulsars.

The method we developed is
\begin{itemize}
  \item First, deconstruct the pulsar phaseogram 
    using a Bayseian blocks representation of the data.
    \begin{itemize}
      \item \figref{off_peak_select} shows the off peak selection for some pulsars\dots
      \item Set the ncpPrior parameter to 5
    \end{itemize}
  \item Before beinning the data, first rotate the maximum phase range to 0
    so that the off-peak region will not overlap the phase edge.
  \item 
\end{itemize}
  required first representing the 


\begin{figure}
  \ifdefined\bwfigures
  \plotone{off_peak_select_bw.eps}
  \else
  \plotone{off_peak_select_color.eps}
  \fi
  \caption{Off peak selection for some pulsars\dots}
  \label{fig:off_peak_select}
\end{figure}


THe off peak phase range is defined in \tabref{phase_range}.


\begin{deluxetable}{l*{4}c}
  \tabletypesize{\scriptsize}
  \tablecaption{Timing Observatories, definition of the off-peak region, and pulsar distances.
  \label{tab:phase_range}
  }
  \input{tables/phase_range}
  \tablecomments{\todo[inline]{Put table comments}}
\end{deluxetable}

