\documentclass[12pt,preprint]{aastex}

\bibliographystyle{apj}

\newif\ifcolorfigure
\colorfiguretrue
%\colorfigurefalse



%#For adding line numbers:
\usepackage{lineno}
\linenumbers


%\usepackage{rotating}
\usepackage{amsmath}
\usepackage{graphicx}
\usepackage{xspace}
\usepackage{url}

% Note, hyperref has to come after other packages!
\usepackage{hyperref}

\usepackage{color}

\begin{document}

\title{PWNCAT2}
\shorttitle{PWNCAT2}

\keywords{
Catalogs;
Fermi Gamma-ray Space Telescope; 
Gamma rays: observations; 
pulsar wind nebula
}

\author{
The authors go here
}



\begin{abstract}
  Abstract goes here

\end{abstract}

\section{Introduction}


Introduction\dots

\section{Analysis Methods}


\section{Discussion}

Discussion



The \textit{Fermi} LAT Collaboration acknowledges generous ongoing support
from a number of agencies and institutes that have supported both the
development and the operation of the LAT as well as scientific data analysis.
These include the National Aeronautics and Space Administration and the
Department of Energy in the United States, the Commissariat \`a l'Energie Atomique
and the Centre National de la Recherche Scientifique / Institut National de Physique
Nucl\'eaire et de Physique des Particules in France, the Agenzia Spaziale Italiana
and the Istituto Nazionale di Fisica Nucleare in Italy, the Ministry of Education,
Culture, Sports, Science and Technology (MEXT), High Energy Accelerator Research
Organization (KEK) and Japan Aerospace Exploration Agency (JAXA) in Japan, and
the K.~A.~Wallenberg Foundation, the Swedish Research Council and the
Swedish National Space Board in Sweden.

Additional support for science analysis during the operations phase is
gratefully acknowledged from the Istituto Nazionale di Astrofisica in
Italy and the Centre National d'\'Etudes Spatiales in France.

The authors acknowledge the use of
HEALPix\footnote{\url{http://healpix.jpl.nasa.gov/}} \citep{healpix}.


\bibliography{pwncat2}


\end{document}
