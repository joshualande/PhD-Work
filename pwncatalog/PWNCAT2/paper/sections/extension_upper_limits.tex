\section{Validation of Extension Upper Limits}

For sources that are significantly detected but that do not have a
significantly-detectable extension, it can often interesting to derive an
upper limit on any possible extension for the source.  Here, we present
an algorithm for obtaining an upper limit on the extension
of a currently unresolved source using a a Bayesian method proposed by Helene \citep{Helene1983}.
The same algorithm has been used by the \fermi-LAT collaboration to
obtain upper limits on the flux of point-like sources not significantly
detected at \gev energies. See for example \cite{first_cat}.

In this method, we assume a particular spatial shape for
the currently-unresolved source (for example a uniform disk
spatial distribution) and we profile over the extension of the
currently-unresolved source. For simplicty, we assume this spatial shape
is defined by a single extension parameter $\sigma$.

For a given extension, the spectral parameters
of the source as well as other nearby sources and diffuse backgrounds
are maximized to obtain the likelihood as a function of extension $\likelihood(\sigma)$.
The 95\% confidence upper limits $\sigma_{ul}$ on the extension of a source
is obtained by finding the extension $\sigma$ such that
\begin{equation}
  \int_0^{\sigma_{ul}} \likelihood(\sigma) = 0.95
\end{equation}
assuming that the likelihood is normalized such that
\begin{equation}
  \int_0^{\infty} \likelihood(\sigma) = 1
\end{equation}

For compulational efficienciy, this upper limit algorithm was implemented in \pointlike 
which introduces negligible approximations what make this alogirhtm compultationally 
tractable \citep{extended_source_search}.
Something about how this is biased by not marginalizing over the position of the extended source,
but that this is not a huge effect

To validate this method, we performed a monte carlo study of the coverage
of the extension upper limits.  Because the method is Bayesian, it has
to overcover for arbitrarily small extensions.  On the other hand, it
is reasonable to expect the method to correctly cover for larger sources.

For simplicity, In our simulation we used the same time range and background model described in section 
4 of \cite{extended_source_serach}. In particular, our simulations were performed assuming the 2FGL
pointing history against an isotropic background consistent with the 
isotropic background measured by EGRET.

We were interseted in verying the extension upper limits in two distincy energy ranges, first
for sources just on the edge of detectability (as pointlike) which could potentially be extended.
We picked for spectral indices varying from 1.5 to 4 and extensions varying from 0\fdg0 to 2\fdg0.
We selected fluxes so that $\ts_\text{point} \sim 35$ for a given flux and extension.
\figref{extul_dim} shows this monte carlo study.

The second simulation concerns sources that are very significant, comparable in brightness to, for example,
the Crab Nebula but that cannot be spatially resolved.
\figref{extul_bright} shows this monte carlo study.

\begin{figure}
  \ifdefined\bwfigures
  \plotone{extul_dim_bw.eps}
  \else
  \plotone{extul_dim_color.eps}
  \fi
  \caption{Monte carlo study.}
\label{fig:extul_dim}
\end{figure}


\begin{figure}
  \ifdefined\bwfigures
  \plotone{extul_bright_bw.eps}
  \else
  \plotone{extul_bright_color.eps}
  \fi
  \caption{Extension upper limit stuff.}
\label{fig:extul_bright}
\end{figure}
