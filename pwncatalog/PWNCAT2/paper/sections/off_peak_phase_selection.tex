\section{Off-peak Phase Selection}


\figref{off_peak_select} shows the off peak selection for some pulsars\dots

\begin{figure}
  \ifdefined\bwfigures
  \plotone{off_peak_select_bw.eps}
  \else
  \plotone{off_peak_select_color.eps}
  \fi
  \caption{Off peak selection for some pulsars\dots}
  \label{fig:off_peak_select}
\end{figure}


To study the off-peak emission of \lat-detected pulsars, we first
developed a new method for defining the off-peak emission.
The primary constraint for this method was that it was systematic,
computationally efficient and model independent, and that it correctly
removed the pulsed emission for already studied pulsars.

The method we developed required first representing the 

THe off peak phase range is defined in \tabref{off_peak_phase_range}.


\begin{deluxetable}{lcccc}
  \tabletypesize{\scriptsize}
  \tablecaption{Timing Observatories, definition of the off-peak region, and pulsar distances.
  \label{tab:off_peak_phase_range}
  }
  \tablehead{\colhead{PSR} & \colhead{ObsID} & \colhead{PHASE} & \colhead{Distance} & \colhead{Observation period rejected (MJD)}\\ \colhead{ } & \colhead{ } & \colhead{ } & \colhead{ } & \colhead{ }}
\startdata
J0007+7303 & \nodata & 0.53-0.91 & \nodata & \nodata \\
J0030+0451 & \nodata & 0.71-0.05 & \nodata & \nodata \\
J0034$-$0534 & \nodata & 0.21-0.68 & \nodata & \nodata \\
J0106+4855 & \nodata & 0.24-0.54 & \nodata & \nodata \\
J0218+4232 & \nodata & 0.83-0.17 & \nodata & \nodata \\
J0248+6021 & \nodata & 0.56-0.12 & \nodata & \nodata \\
J0340+4130 & \nodata & 0.17-0.64 & \nodata & \nodata \\
J0357+3205 & \nodata & 0.37-0.85 & \nodata & \nodata \\
J0437$-$4715 & \nodata & 0.60-0.16 & \nodata & \nodata \\
J0534+2200 & \nodata & 0.60-0.84 & \nodata & \nodata \\
J0610$-$2100 & \nodata & 0.29-0.51 & \nodata & \nodata \\
J0613$-$0200 & \nodata & 0.57-0.05 & \nodata & \nodata \\
J0614$-$3329 & \nodata & 0.36-0.50 & \nodata & \nodata \\
J0622+3749 & \nodata & 0.31-0.87 & \nodata & \nodata \\
J0631+1036 & \nodata & 0.64-0.19 & \nodata & \nodata \\
J0633+0632 & \nodata & 0.65-0.96 & \nodata & \nodata \\
J0633+1746 & \nodata & 0.84-0.92 & \nodata & \nodata \\
J0659+1414 & \nodata & 0.41-0.04 & \nodata & \nodata \\
J0729$-$1448 & \nodata & 0.70-0.42 & \nodata & \nodata \\
J0734$-$1559 & \nodata & 0.33-0.83 & \nodata & \nodata \\
J0742$-$2822 & \nodata & 0.73-0.37 & \nodata & \nodata \\
J0751+1807 & \nodata & 0.75-0.29 & \nodata & \nodata \\
J0835$-$4510 & \nodata & 0.85-0.03 & \nodata & \nodata \\
J0908$-$4913 & \nodata & 0.17-0.53 & \nodata & \nodata \\
J0940$-$5428 & \nodata & 0.56-0.14 & \nodata & \nodata \\
J1016$-$5857 & \nodata & 0.62-0.01 & \nodata & \nodata \\
J1019$-$5749 & \nodata & 0.66-0.37 & \nodata & \nodata \\
J1023$-$5746 & \nodata & 0.67-0.01 & \nodata & \nodata \\
J1024$-$0719 & \nodata & 0.88-0.34 & \nodata & \nodata \\
J1028$-$5819 & \nodata & 0.77-0.08 & \nodata & \nodata \\
J1044$-$5737 & \nodata & 0.56-0.96 & \nodata & \nodata \\
J1048$-$5832 & \nodata & 0.67-0.03 & \nodata & \nodata \\
J1057$-$5226 & \nodata & 0.16-0.56 & \nodata & \nodata \\
J1105$-$6107 & \nodata & 0.69-0.03 & \nodata & \nodata \\
J1119$-$6127 & \nodata & 0.60-0.18 & \nodata & \nodata \\
J1135$-$6055 & \nodata & 0.44-0.86 & \nodata & \nodata \\
J1231$-$1411 & \nodata & 0.86-0.10 & \nodata & \nodata \\
J1357$-$6429 & \nodata & 0.79-0.01 & \nodata & \nodata \\
J1410$-$6132 & \nodata & 0.51-0.89 & \nodata & \nodata \\
J1413$-$6205 & \nodata & 0.58-0.02 & \nodata & \nodata \\
J1418$-$6058 & \nodata & 0.66-0.92 & \nodata & \nodata \\
J1420$-$6048 & \nodata & 0.57-0.05 & \nodata & \nodata \\
J1429$-$5911 & \nodata & 0.32-0.42 & \nodata & \nodata \\
J1459$-$6053 & \nodata & 0.33-0.67 & \nodata & \nodata \\
J1509$-$5850 & \nodata & 0.65-0.13 & \nodata & \nodata \\
J1513$-$5908 & \nodata & 0.52-0.12 & \nodata & \nodata \\
J1531$-$5610 & \nodata & 0.55-0.19 & \nodata & \nodata \\
J1600$-$3053 & \nodata & 0.53-0.09 & \nodata & \nodata \\
J1614$-$2230 & \nodata & 0.83-0.17 & \nodata & \nodata \\
J1620$-$4927 & \nodata & 0.54-0.98 & \nodata & \nodata \\
J1702$-$4128 & \nodata & 0.58-0.16 & \nodata & \nodata \\
J1709$-$4429 & \nodata & 0.75-0.07 & \nodata & \nodata \\
J1713+0747 & \nodata & 0.67-0.19 & \nodata & \nodata \\
J1718$-$3825 & \nodata & 0.01-0.19 & \nodata & \nodata \\
J1732$-$3131 & \nodata & 0.79-0.95 & \nodata & \nodata \\
J1741$-$2054 & \nodata & 0.47-0.97 & \nodata & \nodata \\
J1744$-$1134 & \nodata & 0.16-0.72 & \nodata & \nodata \\
J1746$-$3239 & \nodata & 0.42-0.98 & \nodata & \nodata \\
J1747$-$2958 & \nodata & 0.66-0.10 & \nodata & \nodata \\
J1803$-$2149 & \nodata & 0.58-0.02 & \nodata & \nodata \\
J1809$-$2332 & \nodata & 0.53-0.91 & \nodata & \nodata \\
J1813$-$1246 & \nodata & 0.78-0.01 & \nodata & \nodata \\
J1823$-$3021A & \nodata & 0.09-0.56 & \nodata & \nodata \\
J1826$-$1256 & \nodata & 0.26-0.52 & \nodata & \nodata \\
J1836+5925 & \nodata & 0.82-0.90 & \nodata & \nodata \\
J1846+0919 & \nodata & 0.42-0.88 & \nodata & \nodata \\
J1907+0602 & \nodata & 0.69-0.05 & \nodata & \nodata \\
J1939+2134 & \nodata & 0.09-0.47 & \nodata & \nodata \\
J1952+3252 & \nodata & 0.73-0.05 & \nodata & \nodata \\
J1954+2836 & \nodata & 0.67-0.98 & \nodata & \nodata \\
J1957+5033 & \nodata & 0.44-0.90 & \nodata & \nodata \\
J1958+2846 & \nodata & 0.64-0.92 & \nodata & \nodata \\
J1959+2048 & \nodata & 0.79-0.97 & \nodata & \nodata \\
J2017+0603 & \nodata & 0.76-0.20 & \nodata & \nodata \\
J2021+3651 & \nodata & 0.74-0.98 & \nodata & \nodata \\
J2021+4026 & \nodata & 0.26-0.36 & \nodata & \nodata \\
J2028+3332 & \nodata & 0.58-0.97 & \nodata & \nodata \\
J2030+3641 & \nodata & 0.71-0.21 & \nodata & \nodata \\
J2030+4415 & \nodata & 0.94-0.02 & \nodata & \nodata \\
J2032+4127 & \nodata & 0.68-0.92 & \nodata & \nodata \\
J2043+2740 & \nodata & 0.64-0.04 & \nodata & \nodata \\
J2051$-$0827 & \nodata & 0.77-0.24 & \nodata & \nodata \\
J2055+2539 & \nodata & 0.39-0.86 & \nodata & \nodata \\
J2124$-$3358 & \nodata & 0.14-0.58 & \nodata & \nodata \\
J2139+4716 & \nodata & 0.27-0.90 & \nodata & \nodata \\
J2214+3000 & \nodata & 0.64-0.74 & \nodata & \nodata \\
J2238+5903 & \nodata & 0.65-0.99 & \nodata & \nodata \\
J2240+5832 & \nodata & 0.70-0.46 & \nodata & \nodata \\
J2302+4442 & \nodata & 0.75-0.23 & \nodata & \nodata \\
\enddata
  \tablecomments{\todo[inline]{Put table comments}}
\end{deluxetable}

