\section{Results}

Results goes here\ldots

\tabref{off_peak_all_energy} shows the results of the all energy analysis of the off-peak emission for each pulsar.

\tabref{off_peak_each_energy} shows the results of the analysis in separate energy bins of each pulsar.

\tabref{off_peak_cutoff_test} shows the results of the cutoff test for pulsars with significant low-energy emission.


\begin{deluxetable}{lccc}
\tabletypesize{\scriptsize}
\tablecaption{All Energy spectral fit for the \todo[inline]{How many pulsars?} \lat-detected Pulsars
\label{tab:off_peak_all_energy}
}
\tablehead{\colhead{PSR} & \colhead{TS} & \colhead{$F_{0.1-316}$} & \colhead{$\Gamma$}\\ \colhead{ } & \colhead{ } & \colhead{($10^{-9} \text{ph}\,\text{cm}^{-2}\,\text{s}^{-1}$)} & \colhead{ }}
\startdata
J0007+7303 & 1.6 & $<18.76$ & \nodata \\
J0030+0451 & 14.0 & $<8.19$ & \nodata \\
J0034$-$0534 & 41.9 & $15.88 \pm 4.74$ & $2.41 \pm 0.19$ \\
J0106+4855 & 0.0 & $<6.80$ & \nodata \\
J0218+4232 & 33.9 & $49.22 \pm 20.43$ & $2.79 \pm 0.49$ \\
J0248+6021 & 18.5 & $<13.59$ & \nodata \\
J0340+4130 & 25.6 & $10.50 \pm 3.68$ & $2.13 \pm 0.15$ \\
J0357+3205 & 0.0 & $<2.97$ & \nodata \\
J0437$-$4715 & 0.0 & None & \nodata \\
J0534+2200 & 4957.5 & $559.70 \pm 19.47$ & $2.24 \pm 0.02$ \\
J0610$-$2100 & 0.0 & $<3.23$ & \nodata \\
J0613$-$0200 & 0.0 & $<3.37$ & \nodata \\
J0614$-$3329 & 15.6 & $<15.81$ & \nodata \\
J0622+3749 & 1.0 & $<7.81$ & \nodata \\
J0631+1036 & 15.1 & $<14.03$ & \nodata \\
J0633+0632 & 4.1 & $<10.20$ & \nodata \\
J0633+1746 & 2851.0 & $882.36 \pm 30.61$ & $2.28 \pm 0.03$ \\
J0659+1414 & 0.0 & $<1.77$ & \nodata \\
J0729$-$1448 & 0.0 & $<4.85$ & \nodata \\
J0734$-$1559 & 24.5 & $<12.39$ & \nodata \\
J0742$-$2822 & 0.0 & $<6.76$ & \nodata \\
J0751+1807 & 8.1 & $<5.70$ & \nodata \\
J0835$-$4510 & 286.0 & $288.52 \pm 22.98$ & $2.54 \pm 0.06$ \\
J0908$-$4913 & 21.7 & $<29.39$ & \nodata \\
J0940$-$5428 & 0.0 & $<1.73$ & \nodata \\
J1016$-$5857 & 0.0 & $<12.99$ & \nodata \\
J1019$-$5749 & 3.7 & $<14.67$ & \nodata \\
J1023$-$5746 & 68.5 & $96.55 \pm 27.38$ & $2.25 \pm 0.11$ \\
J1024$-$0719 & 0.0 & $<2.30$ & \nodata \\
J1028$-$5819 & 9.7 & $<29.20$ & \nodata \\
J1044$-$5737 & 0.0 & $<17.85$ & \nodata \\
J1048$-$5832 & 7.8 & $<19.16$ & \nodata \\
J1057$-$5226 & 0.8 & $<5.03$ & \nodata \\
J1105$-$6107 & 10.9 & $<32.30$ & \nodata \\
J1119$-$6127 & 110.5 & $77.87 \pm 15.89$ & $2.28 \pm 0.09$ \\
J1135$-$6055 & 4.2 & $<6.81$ & \nodata \\
J1231$-$1411 & 0.0 & $<3.21$ & \nodata \\
J1357$-$6429 & 2.9 & $<5.94$ & \nodata \\
J1410$-$6132 & 22.4 & $<50.77$ & \nodata \\
J1413$-$6205 & 2.8 & $<12.42$ & \nodata \\
J1418$-$6058 & 11.8 & $<49.69$ & \nodata \\
J1420$-$6048 & 0.0 & $<26.02$ & \nodata \\
J1429$-$5911 & 0.0 & $<12.98$ & \nodata \\
J1459$-$6053 & 0.3 & $<9.28$ & \nodata \\
J1509$-$5850 & 0.7 & $<10.88$ & \nodata \\
J1513$-$5908 & 117.9 & $16.05 \pm 6.39$ & $1.74 \pm 0.11$ \\
J1531$-$5610 & 0.0 & $<3.49$ & \nodata \\
J1600$-$3053 & 0.0 & $<1.85$ & \nodata \\
J1614$-$2230 & 0.0 & $<5.77$ & \nodata \\
J1620$-$4927 & 33.5 & $70.36 \pm 19.70$ & $2.20 \pm 0.11$ \\
J1702$-$4128 & 0.0 & $<6.30$ & \nodata \\
J1709$-$4429 & 14.8 & $<15.42$ & \nodata \\
J1713+0747 & 2.1 & $<4.93$ & \nodata \\
J1718$-$3825 & 0.0 & $<14.61$ & \nodata \\
J1732$-$3131 & 0.0 & $<8.69$ & \nodata \\
J1741$-$2054 & 12.0 & $<14.27$ & \nodata \\
J1744$-$1134 & 74.3 & $47.28 \pm 8.68$ & $2.35 \pm 0.08$ \\
J1746$-$3239 & 54.7 & $109.56 \pm 22.82$ & $2.54 \pm 0.15$ \\
J1747$-$2958 & 32.6 & $124.33 \pm 30.73$ & $2.36 \pm 0.11$ \\
J1803$-$2149 & 6.7 & $<29.50$ & \nodata \\
J1809$-$2332 & 24.8 & $<38.61$ & \nodata \\
J1813$-$1246 & 57.3 & $200.61 \pm 41.33$ & $2.59 \pm 0.14$ \\
J1823$-$3021A & 2.7 & $<5.11$ & \nodata \\
J1826$-$1256 & 20.0 & $<68.93$ & \nodata \\
J1836+5925 & 5020.0 & $561.42 \pm 17.71$ & $2.11 \pm 0.02$ \\
J1846+0919 & 0.0 & $<3.35$ & \nodata \\
J1907+0602 & 0.0 & $<7.55$ & \nodata \\
J1939+2134 & 0.0 & $<4.40$ & \nodata \\
J1952+3252 & 0.2 & $<7.71$ & \nodata \\
J1954+2836 & 6.1 & $<18.52$ & \nodata \\
J1957+5033 & 0.0 & $<2.52$ & \nodata \\
J1958+2846 & 0.0 & $<7.72$ & \nodata \\
J1959+2048 & 0.0 & $<4.89$ & \nodata \\
J2017+0603 & 0.0 & $<2.96$ & \nodata \\
J2021+3651 & 0.0 & $<7.59$ & \nodata \\
J2021+4026 & 909.4 & $1206.23 \pm 57.85$ & $2.29 \pm 0.02$ \\
J2028+3332 & 0.0 & $<4.58$ & \nodata \\
J2030+3641 & 0.0 & $<2.96$ & \nodata \\
J2030+4415 & 3.1 & $<26.78$ & \nodata \\
J2032+4127 & 11.7 & $<31.08$ & \nodata \\
J2043+2740 & 0.0 & $<2.58$ & \nodata \\
J2051$-$0827 & 0.0 & $<1.89$ & \nodata \\
J2055+2539 & 108.4 & $47.02 \pm 6.41$ & $2.53 \pm 0.08$ \\
J2124$-$3358 & 103.4 & $19.86 \pm 3.88$ & $2.13 \pm 0.10$ \\
J2139+4716 & 16.8 & $<9.29$ & \nodata \\
J2214+3000 & 0.0 & $<5.02$ & \nodata \\
J2238+5903 & 0.3 & $<6.62$ & \nodata \\
J2240+5832 & 0.0 & $<6.21$ & \nodata \\
J2302+4442 & 114.7 & $33.50 \pm 5.34$ & $2.38 \pm 0.10$ \\
\enddata
\tablecomments{\todo[inline]{Put table comments}}
\end{deluxetable}

\begin{deluxetable}{lcccccc}
\tabletypesize{\scriptsize}
\tablecaption{Energy bin spectral fit for the \todo[inline]{How many pulsars?} \lat-detected Pulsars
\label{tab:off_peak_each_energy}
}
\tablehead{\colhead{PSR} & \colhead{$\ts_{0.1-1}$} & \colhead{$F_{0.1-1}$} & \colhead{$\Gamma_{0.1-1}$} & \colhead{$\ts_{1-10}$} & \colhead{$F_{1-10}$} & \colhead{$\Gamma_{1-10}$} & \colhead{$\ts_{10-316}$} & \colhead{$F_{10-316}$} & \colhead{$\Gamma_{10-316}$}\\ \colhead{ } & \colhead{ } & \colhead{($10^{-9}\ \text{ph}\,\text{cm}^{-2}\,\text{s}^{-1}$)} & \colhead{ } & \colhead{ } & \colhead{($10^{-9}\ \text{ph}\,\text{cm}^{-2}\,\text{s}^{-1}$)} & \colhead{ } & \colhead{ } & \colhead{($10^{-9}\ \text{ph}\,\text{cm}^{-2}\,\text{s}^{-1}$)} & \colhead{ }}
\startdata
J0007+7303 & 41.5 & $24.69 \pm 4.11$ & None & 23.4 & $<1.41$ & None & 1.2 & $<0.15$ & None \\
J0030+0451 & 14.5 & $<16.93$ & None & 4.2 & $<0.60$ & None & 0.0 & $<0.12$ & None \\
J0034$-$0534 & 20.1 & $<15.78$ & None & 25.5 & $0.64 \pm 0.19$ & None & 0.0 & $<0.07$ & None \\
J0106+4855 & 1.5 & $<11.34$ & None & 2.8 & $<0.80$ & None & 0.0 & $<0.11$ & None \\
J0218+4232 & 54.3 & $30.70 \pm 4.56$ & None & 7.6 & $<1.13$ & None & 0.0 & $<0.10$ & None \\
J0248+6021 & 27.1 & $32.35 \pm 6.38$ & None & 4.3 & $<0.96$ & None & 2.5 & $<0.13$ & None \\
J0340+4130 & 0.6 & $<9.05$ & None & 36.3 & $1.16 \pm 0.26$ & None & 0.0 & $<0.07$ & None \\
J0357+3205 & 0.0 & $<6.04$ & None & 0.0 & $<0.40$ & None & 0.0 & $<0.09$ & None \\
J0437$-$4715 & 0.7 & $<5.11$ & None & 0.0 & $<0.21$ & None & 0.0 & $<0.05$ & None \\
J0534+2200 & 2065.0 & $432.14 \pm 12.30$ & None & 2101.7 & $28.41 \pm 1.28$ & None & 1227.0 & $5.35 \pm 0.47$ & None \\
J0610$-$2100 & 0.0 & $<6.16$ & None & 0.0 & $<0.52$ & None & 0.0 & $<0.15$ & None \\
J0613$-$0200 & 0.4 & $<10.87$ & None & 0.0 & $<0.38$ & None & 0.0 & $<0.10$ & None \\
J0614$-$3329 & 16.8 & $<28.41$ & None & 1.8 & $<1.09$ & None & 0.0 & $<0.22$ & None \\
J0622+3749 & 3.2 & $<10.50$ & None & 10.1 & $<0.92$ & None & 0.0 & $<0.07$ & None \\
J0631+1036 & 13.5 & $<30.33$ & None & 3.9 & $<1.11$ & None & 2.3 & $<0.15$ & None \\
J0633+0632 & 0.0 & $<16.00$ & None & 2.2 & $<1.44$ & None & 0.0 & $<0.17$ & None \\
J0633+1746 & 2432.9 & $770.12 \pm 23.09$ & None & 865.5 & $40.45 \pm 2.66$ & None & 0.0 & $<0.53$ & None \\
J0659+1414 & 0.0 & $<4.00$ & None & 0.0 & $<0.27$ & None & 0.0 & $<0.06$ & None \\
J0729$-$1448 & 5.7 & $<18.70$ & None & 0.1 & $<0.48$ & None & 0.0 & $<0.06$ & None \\
J0734$-$1559 & 36.7 & $33.00 \pm 5.75$ & None & 0.0 & $<0.65$ & None & 0.0 & $<0.08$ & None \\
J0742$-$2822 & 7.3 & $<20.56$ & None & 0.0 & $<0.49$ & None & 2.4 & $<0.12$ & None \\
J0751+1807 & 1.8 & $<7.59$ & None & 9.6 & $<0.71$ & None & 0.0 & $<0.06$ & None \\
J0835$-$4510 & 341.2 & $233.53 \pm 14.30$ & None & 42.5 & $5.63 \pm 1.10$ & None & 0.0 & $<0.39$ & None \\
J0908$-$4913 & 17.9 & $<55.19$ & None & 2.0 & $<1.79$ & None & 3.4 & $<0.26$ & None \\
J0940$-$5428 & 0.0 & $<2.40$ & None & 0.0 & $<0.32$ & None & 0.8 & $<0.12$ & None \\
J1016$-$5857 & 0.0 & $<18.70$ & None & 0.6 & $<1.53$ & None & 1.2 & $<0.23$ & None \\
J1019$-$5749 & 6.9 & $<38.33$ & None & 3.0 & $<1.44$ & None & 0.0 & $<0.10$ & None \\
J1023$-$5746 & 33.6 & $69.82 \pm 12.30$ & None & 31.0 & $3.91 \pm 0.82$ & None & 23.1 & $<0.73$ & None \\
J1024$-$0719 & 0.0 & $<5.49$ & None & 0.0 & $<0.30$ & None & 0.0 & $<0.08$ & None \\
J1028$-$5819 & 4.5 & $<46.61$ & None & 8.1 & $<2.84$ & None & 0.0 & $<0.26$ & None \\
J1044$-$5737 & 13.3 & $<45.47$ & None & 3.6 & $<1.65$ & None & 0.0 & $<0.13$ & None \\
J1048$-$5832 & 1.6 & $<28.62$ & None & 8.5 & $<2.11$ & None & 0.0 & $<0.14$ & None \\
J1057$-$5226 & 0.1 & $<10.22$ & None & 0.9 & $<0.58$ & None & 0.0 & $<0.12$ & None \\
J1105$-$6107 & 15.5 & $<60.17$ & None & 10.6 & $<3.20$ & None & 0.1 & $<0.19$ & None \\
J1119$-$6127 & 78.5 & $67.54 \pm 7.91$ & None & 63.2 & $3.39 \pm 0.52$ & None & 15.2 & $<0.29$ & None \\
J1135$-$6055 & 2.4 & $<24.56$ & None & 0.0 & $<0.67$ & None & 1.5 & $<0.12$ & None \\
J1231$-$1411 & 0.3 & $<10.39$ & None & 0.0 & $<0.36$ & None & 0.0 & $<0.14$ & None \\
J1357$-$6429 & 0.0 & $<12.05$ & None & 0.0 & $<0.64$ & None & 1.2 & $<0.27$ & None \\
J1410$-$6132 & 0.2 & $<33.67$ & None & 16.9 & $<5.27$ & None & 5.5 & $<0.54$ & None \\
J1413$-$6205 & 0.0 & $<12.32$ & None & 1.6 & $<1.83$ & None & 1.4 & $<0.22$ & None \\
J1418$-$6058 & 0.0 & $<30.95$ & None & 4.4 & $<4.23$ & None & 4.5 & $<0.60$ & None \\
J1420$-$6048 & 2.2 & $<41.83$ & None & 2.0 & $<2.88$ & None & 11.1 & $<0.47$ & None \\
J1429$-$5911 & 0.0 & $<21.81$ & None & 0.0 & $<1.93$ & None & 0.4 & $<0.42$ & None \\
J1459$-$6053 & 1.9 & $<30.07$ & None & 0.0 & $<0.82$ & None & 0.0 & $<0.22$ & None \\
J1509$-$5850 & 0.2 & $<24.81$ & None & 0.0 & $<1.13$ & None & 0.0 & $<0.18$ & None \\
J1513$-$5908 & 7.4 & $<39.24$ & None & 34.0 & $2.48 \pm 0.51$ & None & 83.8 & $0.54 \pm 0.11$ & None \\
J1531$-$5610 & 0.0 & $<5.32$ & None & 0.0 & $<0.64$ & None & 0.3 & $<0.15$ & None \\
J1600$-$3053 & 0.0 & $<3.47$ & None & 0.0 & $<0.35$ & None & 0.0 & $<0.06$ & None \\
J1614$-$2230 & 0.0 & $<10.16$ & None & 2.0 & $<0.84$ & None & 0.0 & $<0.11$ & None \\
J1620$-$4927 & 8.8 & $<67.59$ & None & 19.1 & $<5.47$ & None & 6.1 & $<0.41$ & None \\
J1702$-$4128 & 0.0 & $<10.01$ & None & 0.0 & $<0.93$ & None & 0.1 & $<0.21$ & None \\
J1709$-$4429 & 30.7 & $63.29 \pm 11.82$ & None & 0.1 & $<1.15$ & None & 0.0 & $<0.16$ & None \\
J1713+0747 & 3.9 & $<12.83$ & None & 0.6 & $<0.50$ & None & 0.0 & $<0.07$ & None \\
J1718$-$3825 & 0.3 & $<51.37$ & None & 0.0 & $<1.61$ & None & 0.1 & $<0.33$ & None \\
J1732$-$3131 & 0.0 & $<23.46$ & None & 0.0 & $<1.20$ & None & 0.0 & $<0.39$ & None \\
J1741$-$2054 & 5.0 & $<28.78$ & None & 2.3 & $<1.15$ & None & 3.3 & $<0.20$ & None \\
J1744$-$1134 & 32.2 & $33.65 \pm 6.21$ & None & 48.9 & $2.25 \pm 0.41$ & None & 0.4 & $<0.10$ & None \\
J1746$-$3239 & 60.7 & $76.79 \pm 10.21$ & None & 22.1 & $<3.44$ & None & 0.0 & $<0.08$ & None \\
J1747$-$2958 & 45.5 & $112.35 \pm 16.93$ & None & 28.0 & $4.80 \pm 1.03$ & None & 0.0 & $<0.15$ & None \\
J1803$-$2149 & 1.0 & $<44.97$ & None & 2.2 & $<2.71$ & None & 2.6 & $<0.41$ & None \\
J1809$-$2332 & 42.3 & $82.32 \pm 13.11$ & None & 7.0 & $<2.79$ & None & 0.0 & $<0.16$ & None \\
J1813$-$1246 & 61.2 & $129.65 \pm 17.44$ & None & 20.7 & $<5.41$ & None & 0.8 & $<0.33$ & None \\
J1823$-$3021A & 0.4 & $<12.31$ & None & 0.0 & $<0.40$ & None & 3.3 & $<0.18$ & None \\
J1826$-$1256 & 29.3 & $118.08 \pm 22.24$ & None & 4.5 & $<4.93$ & None & 0.3 & $<0.47$ & None \\
J1836+5925 & 3491.7 & $497.90 \pm 14.02$ & None & 2421.1 & $44.07 \pm 2.19$ & None & 0.0 & $<0.28$ & None \\
J1846+0919 & 0.0 & $<8.71$ & None & 0.0 & $<0.46$ & None & 0.0 & $<0.11$ & None \\
J1907+0602 & 0.3 & $<36.68$ & None & 0.0 & $<1.05$ & None & 0.0 & $<0.14$ & None \\
J1939+2134 & 0.0 & $<11.60$ & None & 0.0 & $<0.85$ & None & 0.0 & $<0.12$ & None \\
J1952+3252 & 2.3 & $<24.92$ & None & 0.0 & $<0.88$ & None & 0.0 & $<0.13$ & None \\
J1954+2836 & 2.9 & $<36.32$ & None & 5.5 & $<2.12$ & None & 0.1 & $<0.18$ & None \\
J1957+5033 & 0.0 & $<5.54$ & None & 0.0 & $<0.29$ & None & 0.1 & $<0.09$ & None \\
J1958+2846 & 0.0 & $<13.30$ & None & 0.4 & $<1.30$ & None & 0.0 & $<0.16$ & None \\
J1959+2048 & 0.6 & $<21.38$ & None & 0.0 & $<0.53$ & None & 0.0 & $<0.18$ & None \\
J2017+0603 & 0.8 & $<10.00$ & None & 0.0 & $<0.31$ & None & 0.0 & $<0.10$ & None \\
J2021+3651 & 1.5 & $<41.22$ & None & 0.0 & $<1.07$ & None & 0.0 & $<0.13$ & None \\
J2021+4026 & 1673.3 & $982.41 \pm 29.24$ & None & 914.2 & $59.76 \pm 3.15$ & None & 7.4 & $<1.24$ & None \\
J2028+3332 & 0.0 & $<13.05$ & None & 0.0 & $<0.75$ & None & 0.0 & $<0.09$ & None \\
J2030+3641 & 0.0 & $<11.83$ & None & 0.0 & $<0.42$ & None & 0.0 & $<0.09$ & None \\
J2030+4415 & 0.5 & $<52.40$ & None & 1.1 & $<2.94$ & None & 2.0 & $<0.62$ & None \\
J2032+4127 & 0.0 & $<33.94$ & None & 3.2 & $<2.81$ & None & 8.7 & $<0.71$ & None \\
J2043+2740 & 0.0 & $<5.44$ & None & 0.3 & $<0.34$ & None & 0.0 & $<0.10$ & None \\
J2051$-$0827 & 0.0 & $<3.77$ & None & 0.0 & $<0.29$ & None & 0.0 & $<0.09$ & None \\
J2055+2539 & 106.3 & $36.24 \pm 4.00$ & None & 23.0 & $<1.50$ & None & 0.0 & $<0.07$ & None \\
J2124$-$3358 & 16.6 & $<16.97$ & None & 107.2 & $2.01 \pm 0.30$ & None & 0.0 & $<0.07$ & None \\
J2139+4716 & 10.0 & $<18.78$ & None & 8.0 & $<1.02$ & None & 0.0 & $<0.05$ & None \\
J2214+3000 & 1.1 & $<18.13$ & None & 0.0 & $<0.48$ & None & 0.0 & $<0.28$ & None \\
J2238+5903 & 0.2 & $<16.55$ & None & 0.2 & $<0.87$ & None & 0.0 & $<0.10$ & None \\
J2240+5832 & 0.0 & $<8.74$ & None & 5.9 & $<1.07$ & None & 0.0 & $<0.06$ & None \\
J2302+4442 & 61.9 & $24.28 \pm 3.41$ & None & 61.1 & $1.47 \pm 0.26$ & None & 0.0 & $<0.08$ & None \\
\enddata
\tablecomments{\todo[inline]{Put table comments}}
\end{deluxetable}

\begin{deluxetable}{lcccc}
\tabletypesize{\scriptsize}
\tablecaption{Spectral fitting of pulsar wind nebula candidates with low energy component
\label{tab:off_peak_cutoff_test}
}
\tablehead{\colhead{PSR} & \colhead{$G_{0.1-316}$} & \colhead{$\Gamma$} & \colhead{$E_\text{cutoff}$} & \colhead{$\text{TS}_\text{cutoff}$}\\ \colhead{ } & \colhead{($10^{-12}$\,erg\,cm$^{-2}$\,s$^{-1}$)} & \colhead{ } & \colhead{(GeV)} & \colhead{ }}
\startdata
J0034$-$0534 & $6.06 \pm 1.59$ & $1.49 \pm 0.67$ & $1.52 \pm 1.17$ & 5.3 \\
J0633+1746 & $415.30 \pm 12.92$ & $1.41 \pm 0.10$ & $1.00 \pm 0.13$ & 177.0 \\
J1813$-$1246 & $65.41 \pm 3.93$ & $1.68 \pm 0.03$ & $1.00 \pm 0.05$ & 2.5 \\
J1836+5925 & $330.12 \pm 8.76$ & $1.40 \pm 0.03$ & $1.64 \pm 0.07$ & 203.4 \\
J2021+4026 & $585.23 \pm 16.60$ & $1.64 \pm 0.03$ & $1.83 \pm 0.07$ & 124.2 \\
J2055+2539 & $15.57 \pm 2.25$ & $0.67 \pm 0.71$ & $0.47 \pm 0.20$ & 29.1 \\
J2124$-$3358 & $9.80 \pm 1.57$ & $0.15 \pm 0.84$ & $0.91 \pm 0.42$ & 27.6 \\
\enddata
\tablecomments{\todo[inline]{Put table comments}}
\end{deluxetable}

The localization results are in \tabref{off_peak_localization}

\begin{deluxetable}{lccccc}
\tabletypesize{\scriptsize}
\tablecaption{Localization results
\label{tab:off_peak_localization}
}
\tablehead{\colhead{PSR} & \colhead{$\ts_\text{point}$} & \colhead{GLON} & \colhead{GLAT} & \colhead{Pos Err} & \colhead{Offset} & \colhead{\tsext} & \colhead{Extension}\\ \colhead{ } & \colhead{ } & \colhead{(deg)} & \colhead{(deg)} & \colhead{ } & \colhead{(deg)} & \colhead{ } & \colhead{(deg)}}
\startdata
J0007+7303 & 82.8 & 119.66 & 10.46 & None & 0.00 & 10.6 & $<2.62$ \\
J0034$-$0534 & 42.4 & 111.53 & -68.03 & 0.06 & 0.04 & 0.0 & $<0.12$ \\
J0218+4232 & 34.7 & 139.56 & -17.53 & 0.08 & 0.05 & 0.0 & $<0.15$ \\
J0340+4130 & 25.1 & 153.81 & -11.00 & 0.06 & 0.04 & 0.0 & $<0.13$ \\
J0534+2200 & 4959.1 & 184.55 & -5.79 & 0.01 & 0.01 & 0.0 & $<0.02$ \\
J0633+1746 & 2842.4 & 195.12 & 4.22 & 0.02 & 0.05 & 3.3 & $<0.09$ \\
J0835$-$4510 & 304.7 & 263.46 & -3.15 & 0.08 & 0.37 & 295.3 & $0.73 \pm 0.06$ \\
\enddata
\tablecomments{\todo[inline]{Put table comments}}
\end{deluxetable}

\todo[inline]{Add table on PWN Variability}


\figref{cutoff_test} shows the cutoff test\ldots

\begin{figure}
  \ifdefined\bwfigures
  \plotone{cutoff_test_bw.eps}
  \else
  \plotone{cutoff_test_color.eps}
  \fi
  \caption{Cutoff test for some pulsars\dots}
  \label{fig:cutoff_test}
\end{figure}
