\documentclass[12pt]{beamer}
\usetheme[navbar=false, bkgimage=false, shadow=true]{Fermi}

\usepackage{graphicx}

\usepackage{amsmath}
\usepackage{xspace}
\usepackage{listings}
\lstset{basicstyle=\scriptsize,frame=single,showstringspaces=false}


\newcommand{\python}{\ensuremath{\mathtt{python}}\xspace}
\newcommand{\gtlike}{\ensuremath{\mathtt{gtlike}}\xspace}
\newcommand{\pointlike}{\ensuremath{\mathtt{pointlike}}\xspace}
\newcommand{\gtobssim}{\ensuremath{\mathtt{gtobssim}}\xspace}
\newcommand{\gtmktime}{\ensuremath{\mathtt{gtmktime}}\xspace}
\newcommand{\fermi}{\textit{Fermi}\xspace}
\newcommand{\roimc}{\texttt{roi\_monte\_carlo}\xspace}
\newcommand{\FileSpectrumMap}{\texttt{FileSpectrumMap}\xspace}
\newcommand{\mapcube}{\texttt{MapCube}\xspace}
\newcommand{\ftone}{\texttt{ft1}\xspace}


\title{\pointlike's MC Simulation Package}

\author{Joshua Lande}
\date{June 27, 2012}

\begin{document}

\fermititle


\begin{frame}{Motivation (Ext. Srs. Search paper)}
\begin{columns}
  \column{.4\textwidth} 
  \includegraphics[scale=0.4]{plots/ext_src_table.pdf}
  \column{.6\textwidth} 
  \includegraphics[scale=0.45]{plots/ext_src_plot.pdf}

  $\sim90,000$ simulations/model!
\end{columns}
\end{frame}

\begin{frame}{\gtobssim Overview}
  \begin{itemize}
    \item Input to \gtobssim:
      \begin{itemize}
        \item XML File
        \item Ft2 file/source list
        \item templates for certain spectral and spatial models (more soon\dots)
      \end{itemize}
    \item After running \gtobssim
      \begin{itemize}
        \item Remove bad time intervals from simulated data
        \item Apply zenith angle cut to simulated data
      \end{itemize}
    \item \em{Building the \gtobssim XML file can be error prone}
    \item \em{Cutting simulated data can be error prone}
  \end{itemize}
\end{frame}

\begin{frame}{\pointlike's MC Simulation package}
  \begin{itemize}
    \item I developed a wrapper around \gtobssim to automate
      otherwise time consuming, tedious, or error-prone tasks
    \item Built around \pointlike, an alternate maximum
      likelihood package written in \python
      \begin{itemize}
      \item Uses as input a list of \pointlike objects
        \item Builds the XML file for \gtobssim
        \item Converts unsupported models into required templates.
        \item Automatically removes bad time intervals + \texttt{zmax} cut
      \end{itemize}
    \item Code is in \pointlike package: \texttt{uw.like.roi\_monte\_carlo.py}.
  \end{itemize}
\end{frame}


\begin{frame}[fragile]
  \frametitle{Point Sources}

  \begin{lstlisting}[language=XML]
<source name="source" flux="0.03">
  <spectrum escale="MeV">
    <particle name="gamma">
      <power_law emin="20" emax="1000000" gamma="1.9"/>
    </particle>
    <celestial_dir ra="193.98" dec="-5.82"/>
  </spectrum>
</source>
  \end{lstlisting}

  \begin{itemize}
    \item Supported Spectral Models
  \begin{itemize}
    \item power law, (dark matter) line, broken powerlaw, and file function
    \item \url{http://fermi.gsfc.nasa.gov/ssc/data/analysis/scitools/other_sources.html}
  \end{itemize}
\item Problematic for all other spectral models!
  \end{itemize}
\end{frame}

\begin{frame}[fragile]
  \frametitle{Point Sources (cont)}
  \begin{itemize}
    \item Any spectrum can be simulated using \texttt{FileSpectrum}:
  \end{itemize}

  \begin{lstlisting}[language=XML]
<source name="FileSpectrum">
   <spectrum escale="MeV" >
      <SpectrumClass name="FileSpectrum" params="flux=0.,
      specFile=$(FERMI_DIR)/spectrum.dat"/>
      <celestial_dir ra="194.04" dec="-5.789"/>
   </spectrum>
</source>
  \end{lstlisting}

  \begin{itemize}
    \item \roimc will automatically build a \texttt{FileSpectrum} for any otherwise-unsupported
      model
    \item \gtobssim Requires integral of spectral model (done automatically by \roimc)
    \item WARNING! \texttt{FileSpectrum} objects cannot contain 0 pixels (stripped out by \roimc)
  \end{itemize}
\end{frame}


\begin{frame}[fragile]
  \frametitle{Diffuse Sources Sources}
  \begin{itemize}
  \item \mapcube model to simulate diffuse background:
  \end{itemize}
\begin{lstlisting}[language=XML]
<source name="map_cube_source">
   <spectrum escale="MeV">
      <SpectrumClass name="MapCube" params="1., 
        $(FERMI_DIR)/mapcube.fits "/>
      <use_spectrum frame="galaxy"/>
   </spectrum>
</source>
\end{lstlisting}
  \begin{itemize}
    \item Requires 3D integral of fits file
    \item Integration automatic by \roimc
  \end{itemize}
\end{frame}

\begin{frame}[fragile]
  \frametitle{Building the XML file (Isotropic DIffuse Sources)}
\begin{itemize}
\item \FileSpectrumMap for simulation the isotropic diffuse:
\end{itemize}

\begin{lstlisting}[language=XML]
<source name="isotropic">
   <spectrum escale="GeV" flux="1.">
      <SpectrumClass name="FileSpectrumMap" 
      params="flux=17,fitsFile=$(FERMI_DIR)/iso_spatial.fits,
specFile=$(FERMI_DIR)/iso_spectral.dat,emin=100.,emax=1100"/>
      <use_spectrum frame="galaxy"/>
   </spectrum>
</source>
\end{lstlisting}

\begin{itemize}
  \item Must integrate isotropic spectrum
  \item Must generate allsky spatial \texttt{fits} file predicting 1 
  \item Must add energy range from isotropic file
  \item All done automatically by \roimc
\end{itemize}
\end{frame}

\begin{frame}[fragile]
\frametitle{Extended Sources}
\begin{lstlisting}[language=XML]
<source name="gaussian_source">
   <spectrum escale="MeV">
      <SpectrumClass name="GaussianSource" 
        params="0.1,2.1,45,30,3,0.5,45,30,2e5"/>
      <use_spectrum frame="galaxy"/>
   </spectrum>
</source>
\end{lstlisting}
  \begin{itemize}
    \item \gtobssim only natively supports an Elliptical Gaussian spatial model with a power law spectral model.
    \item WARNING, the ellipse angle is defined west of celestial north)!
  \end{itemize}
\end{frame}

\begin{frame}[fragile]
\frametitle{Extended Sources (cont)}

  \begin{itemize}
    \item Any extended source can be represented by a \FileSpectrumMap
  \end{itemize}
\begin{lstlisting}[language=XML]
<source name="filespectrummap_test">
   <spectrum escale="GeV" flux="1.">
      <SpectrumClass name="FileSpectrumMap" params="
        flux=17,
        fitsFile=$(FERMI_DIR)/spatial.fits,
        specFile=$(FERMI_DIR)/spectral.dat,
        emin=100, emax=1100"/>
      <use_spectrum frame="galaxy"/>
   </spectrum>
</source>
\end{lstlisting}
  \begin{itemize}
    \item Have to:
    \begin{itemize}
    \item Build fits template for spatial model
    \item Build text file for spectral model
    \item Integrate spectral model
    \end{itemize}
  \item Process automatic by \roimc for any of \pointlike's extended
    sources (disk, Gauss, NFW, SpatialMap, \dots).
  \end{itemize}
\end{frame}


\begin{frame}{Common Gotcha's (Energy Dispersion)}
  \begin{itemize}
        \item Energy dispersion means photons with energies outside simulation range can
          disperse into energy range
        \item All spectral models must be simulated for energies well outside of simulation range
        \item Handled automatically by \roimc. 
        \item Parameter \texttt{roi\_pad} (default=2) will pad a given
          amount to energy of all spectral models.
        \item \texttt{acutal\_emin} = \texttt{simulation\_emin}/\texttt{roi\_pad}
        \item \texttt{acutal\_emax} = \texttt{simulation\_emax}*\texttt{roi\_pad}
      \end{itemize}
\end{frame}

\begin{frame}{Common Gotcha's (\ftone cuts)}
  \begin{itemize}
    \item Must remove bad time intervals
      \begin{itemize}
        \item \gtobssim takes as input an FT2 file (does not store GTIs)
        \item But \gtlike analysis uses \texttt{ltcube} only over good time intervals
        \item No ScienceTool for applying \texttt{gtmktime} unless you know exact filter applied to \ftone file
          which generated ltcube
        \item \roimc uses a compbiation of \texttt{pyfits} and \texttt{gtmktime} to apply exact GTIs
          from an \ftone or ltcube file.
      \end{itemize}
    \item zenith angle cut must be consistent with zmax flag in \texttt{gtltcube}
      \begin{itemize}
        \item Flag in \roimc to automatically apply \texttt{zmax} after simulation.
      \end{itemize}
  \end{itemize}
\end{frame}

\begin{frame}{All sky vs region simulations}
  \begin{itemize}
    \item \gtobssim will simulate over all sky for allsky \mapcube files.
    \item \texttt{use\_ac} parameter is applied AFTER the simulation!
    \item This is very inefficient when simulating only a particular region in the sky
    \item As far as I can tell, non-allsky mapcubes will cause strange projection effects
    \item lonMin and lonMax parameters for spatial models does not work correct.
  \end{itemize}
\end{frame}

\begin{frame}{\mapcube Cutting}
  \includegraphics[scale=0.4]{plots/allsky_cubes.png}
  \begin{itemize}
    \item My solution for simulation small regions of the sky is to set to 0 pixels
      far away from ROI
    \item Done automaticlly by \roimc
    \item Dramatic speedup for simulations of small regions
    \item Also, cut out energy bins in \mapcube far away from simulation energy range.
    \end{itemize}
\end{frame}

\begin{frame}[fragile]
  \frametitle{\roimc Usage}
  \begin{itemize}
    \item First, build a list of pointlike soruces
    \item Most easily, you can use \pointlike's XML parser:
\begin{lstlisting}[language=Python]
from uw.utilities.xml_parsers import parse_sources
ps,ds=parse_sources(xmlfile)
sources=ps+ds
\end{lstlisting}
\item You can also build source programatically with \pointlike:
\begin{lstlisting}[language=Python]
  from uw.like.pointspec_helpers import PointSource
  from uw.like.Models import PowerLaw
  skydir = SkyDir(34,-100, SkyDir.EQUATORIAL)
  model = PowerLaw(norm=1e-10, index=2)
  ps=PointSource(name='ps', model=model, skydir=skydir)
\end{lstlisting}
  \end{itemize}
\end{frame}

\begin{frame}[fragile]
\frametitle{Run the Simulation}

\begin{lstlisting}[language=Python]
from skymaps import SkyDir
roi_dir = SkyDir(30, 0.5, SkyDir.GALACTIC)

from uw.like.roi_monte_carlo import MonteCarlo
mc = MonteCarlo(
    sources=sources,
    seed=0,
    emin=1e3,
    emax=1e4,
    roi_dir=roi_dir,
    maxROI=10,
    irf='P7SOURCE_V6',
    ft1='ft1.fits',
    ft2='ft2.fits',
    gtifile='ltcube.fits',
    zmax=100,
)
mc.simulate()
\end{lstlisting}


\end{frame}

\begin{frame}{Conclusion}
  \begin{itemize}
    \item \roimc part of \pointlike
    \item Automatically distributed with the ScienceTools
    \item \roimc documented on Confluence:
      \url{https://confluence.slac.stanford.edu/x/MIAcBw}
    \item The code has been successfully used by several people (see work by Stephan Zimmer)
  \end{itemize}
\end{frame}

\end{document}
