\documentclass[12pt,preprint]{aastex}

\bibliographystyle{apj}

\newif\ifcolorfigure
\colorfiguretrue
%\colorfigurefalse



%#For adding line numbers:
\usepackage{lineno}
\linenumbers


%\usepackage{rotating}
\usepackage{amsmath}
\usepackage{graphicx}
\usepackage{xspace}
\usepackage{url}

% Note, hyperref has to come after other packages!
\usepackage{hyperref}

% some units
\newcommand{\kev}{\text{Kev}\xspace}
\newcommand{\mev}{\text{MeV}\xspace}
\newcommand{\gev}{\text{GeV}\xspace}
\newcommand{\tev}{\text{TeV}\xspace}
\newcommand{\sr}{\text{sr}\xspace}
\newcommand{\s}{\text{s}\xspace}
\newcommand{\ph}{\text{ph}\xspace}
\newcommand{\cm}{\text{cm}\xspace}
\newcommand{\km}{\text{km}\xspace}
\newcommand{\phflux}{\ensuremath{\ph\ \cm^{-2}\s^{-1}}\xspace}
\newcommand{\tsext}{{\ensuremath{\text{TS}_{\text{ext}}}}\xspace}
\newcommand{\tsinc}{\ensuremath{\text{TS}_{\text{inc}}}\xspace}

\newcommand{\likelihood}{\ensuremath{\mathcal{L}}\xspace}

\newcommand{\chandra}{\text{{\em Chandra}}\xspace}
\newcommand{\swiftxrt}{\text{{\em Swift}/XRT}\xspace}
\newcommand{\rosat}{\text{{\em ROSAT}}\xspace}
\newcommand{\suzaku}{\text{{\em Suzaku}}\xspace}
\newcommand{\asca}{\text{{\em ASCA}}\xspace}
\newcommand{\einstein}{\text{{\em Einstein}}\xspace}
\newcommand{\xmmnewton}{\text{{\em XMM-Newton}}\xspace}
\newcommand{\fermi}{\textit{Fermi}\xspace}


\newcommand{\rsixeight}{{\ensuremath{\text{r}_{68}}}\xspace}

\newcommand{\tsextpointlike}{\ensuremath{\tsext_{,\pointlike}}\xspace}
\newcommand{\tsextgtlike}{\ensuremath{\tsext_{,\gtlike}}\xspace}

\newcommand{\ts}{\text{TS}\xspace}
\newcommand{\glon}{\text{GLON}\xspace}
\newcommand{\glat}{\text{GLAT}\xspace}
\newcommand{\altdiff}{\text{alt,diff}\xspace}
\newcommand{\altpsf}{\text{alt,psf}\xspace}
\newcommand{\sys}{\text{sys}\xspace}
\newcommand{\stat}{\text{stat}\xspace}

% the program names
\newcommand{\gtlike}{\ensuremath{\mathtt{gtlike}}\xspace}
\newcommand{\pointlike}{\ensuremath{\mathtt{pointlike}}\xspace}
\newcommand{\gtobssim}{\ensuremath{\mathtt{gtobssim}}\xspace}
\newcommand{\minuit}{\ensuremath{\mathtt{minuit}}\xspace}

\usepackage{color}

%\newcommand{\hl}[1]{{\color{red}#1}}
\newcommand{\hl}[1]{#1}


\begin{document}

\title{Search for Spatially Extended \fermi-LAT Sources Using Two Years of Flight
Data}
\shorttitle{Search for Extended LAT Sources}

\keywords{
Catalogs;
Fermi Gamma-ray Space Telescope; 
Gamma rays: observations; 
ISM: supernova remnants;
Methods: statistical;
pulsar wind nebula
}

\author{
J.~Lande\altaffilmark{2},
S.~Funk\altaffilmark{2},
M.~Ackermann (???),
A.~Allafort\altaffilmark{2},
M.-H.~Grondin\altaffilmark{53},
% Keith Bechtol
% Marshall Roth?
% Marianne Lemoine-Goumard
% Francesco Giordano
% Toby Burnett
% Jean Ballet
% Matthew Kerr
\ldots
\altaffiltext{2}{W. W. Hansen Experimental Physics Laboratory, Kavli Institute for Particle Astrophysics and Cosmology, Department of Physics and SLAC National Accelerator Laboratory, Stanford University, Stanford, CA 94305, USA}
\altaffiltext{53}{Institut f\"ur Astronomie und Astrophysik, Universit\"at T\"ubingen, D 72076 T\"ubingen, Germany}
}


\begin{abstract}
We present a new method for quantifying the spatial extension of sources
with the Large Area Telescope (LAT), the primary science instrument on
the {\em \fermi Gamma-ray Space Telescope} (\fermi).  Spatial extension
is an important characteristic for correctly associating LAT sources with
their counterparts at other wavelengths. We perform a series of Monte
Carlo simulations to validate this tool and calculate the LAT's threshold
for detecting the spatial extension of sources.  We then 
test all sources in the two-year source catalog for extension. We
report the detection of nine spatially extended sources in addition to
the twelve spatially extended sources reported in the second \fermi-LAT
Catalog (2FGL).
\end{abstract}

\section{Introduction}

\hl{

A number of astrophysical sources classes including supernova remnant
(SNRs), pulsar wind nebulae (PWNe), molecular clouds, normal galaxies,
galaxy clusters could be expected to spatially resolvable at \gev
energies.  Dark matter satellites are also hypothesized to be seen
as spatially extended at \gev energies.  The LAT Collaboration has
previously reported on five SNRs which are spatially extended beyond
the LAT PSF at \gev energies (IC443, W28, W44, W51C, and RX\,J1713.7$-$3946
\citep{ic443,w28,w44,w51c,rx_j1713_lat}). In addition, three extended PWNe
were detected as being extended: MSH\,15$-$52, Vela~X, and HESS\,J1825$-$137
\citep{msh1552,velax,fermi_hess_j1825}. Two close-by galaxies the Large
and Small Magalenic Clouds and one radio galaxy Centarus A were spatially
resolved at \gev energies \citep{lmc,smc,cen_a_lat}.  Furthermore,
a number of other sources previously detected at \gev energies are
positionally coincident with sources that exhibit extension at other
wavelengths and could be expected to be spatially resolvable at \gev
energies.

At \tev energies many sources have been recently detected as
begin spatially extended using air Cherenkov detectors.  Most
prominent was a survey of the Galactic plane using the High Energy
Stereoscopic System (H.E.S.S) which reported 14 spatially extended
sources with extensions varying from $\sim0\fdg1$ to $\sim0\fdg25$
\citep{hess_plane_survey}.  In fact, within our Galaxy only very few
sources detected at \tev energies (most notably the $\gamma$-ray binaries
LS\,5039 \citep{HESSLS5039}, LS I+61$-$303 \citep{MAGICLSI, VERITASLSI}
HESS\,J0632+057 \citep{HESS0632}, and the Crab nebula \citep{crab_weekes})
have no significant extension detected.  High energy $\gamma$-rays from
these sources are produced by the decay of $\pi^0$s produced by hadronic
interactions with interstellar matter and by relativistic electrons
due to Inverse Compton (IC) scattering and Bremsstrahlung radiation
\citep{blandford_and_eichler_1987}.  It is likely that the \gev and
\tev emission from these sources originates from the same population of
high-energy particles and so at least some of these sources should be
detected as extended sources at \gev energies.  Studying 
these \tev sources at \gev energies would help to
determine the emission mechanisms producing these high energy photons.

The Large Area Telescope (LAT) is a pair conversion telescope on the
{\em \fermi Gamma-ray Space Telescope} (\fermi) that been surveying the
$\gamma$-ray sky since 2008 June.  The LAT has broad energy coverage
(20 \mev to $>300$ \gev), wide field of view ($\sim 2.4 \sr$), and large
effective area ($\sim 8000\ \cm^2$ at $>1 \gev$).  
Using one year of all-sky surveying data, the LAT Collaboration published
a catalog of 1451 sources significantly detected at \gev energies called 1FGL \citep{first_cat}.
Using two years of data, a second catalog called 2FGL reported on 1873
sources \citep{second_cat}.
The counterparts of many of these sources can be spatially resolved
when observed at other frequencies but detecting the spatial extension
of these sources at \gev energies is difficult because the size of the
point-spread function (PSF) of the LAT is comparable to the typical size
of many of these sources.

The capability to spatially resolve \gev $\gamma$-ray
sources is important for several reasons.  First, multiple potential
counterparts detected at other frequencies exist within the localization
confidence region of a typical LAT source along the Galactic plane.
Finding a coherent source extension across different energy bands can
help to associate a LAT source to an otherwise confused counterpart.
Furthermore, dark matter satellite in our galaxy have been predicted to be
detected and spatially extended at \gev energies and could be discovered
outside the galactic plane \citep{pre_luanch_dark_matter_fermi}.  Also,
due to the strong energy dependence of the LAT PSF, the spatial and
spectral information about a source do not simply decouple. An inaccurate
spatial model will bias the spectral model of the source. Specifically,
modeling a spatially extended source as point-like will systematically
shift a spectral analysis to softer indices. Furthermore, correctly
modeling a source's extension is important for improving the model of
the sky and removing excess residuals, for example in the region around
surrounding the Large Magellanic Clouds \citep{first_cat}.  Such excess
residuals potentially bias the significance and measured spectrum of
neighboring sources in the densely populated Galactic plane.


Each of the previously detected extended LAT sources was spatially
resolved by a dedicated analysis of the particular source so
we expect that a systematic scan of all LAT detected sources
could uncover additional spatially extended sources.  For these
reasons, in Section~\ref{analysis_methods_section} we present
a new method for analyzing spatially extended LAT sources. In
Section~\ref{monte_carlo_validation} we demonstrate that this tool
can be used for an unbiased statistical test of the extension of
LAT sources and in Section~\ref{extension_sensitivity} we calculate
the LAT's detection threshold to resolve the extension of a source.
In Section~\ref{extended_source_search_method} we describe a search for
new spatially extended LAT sources and in Section~\ref{validate_known}
we reanalyzing the twelve previously detected extended sources
included in 2FGL.  In Section~\ref{test_2lac_sources} we further
demonstrate that our detection method does not misidentify
point sources as being extended 
by testing the extension of Active
galactic nucleus (AGN) believed to be unresolvable. Finally, in
Section~\ref{new_ext_srcs_section} we present the detection of the
extension of nine spatially extended sources that were reported in 2FGL
but not previously spatially resolved.

}

\section{Analysis Methods}
\label{analysis_methods_section}

Morphological studies of sources using the LAT are challenging
because of the strongly energy-dependent PSF that is comparable in
size to the extension of many sources expected to be detected at
\gev energies.  Additional complications arise for sources along
the Galactic plane due to systematic uncertainties in the Galactic
diffuse emission.  The LAT's PSF is limited at its lowest detectable
energies by multiple scattering in the silicon strip tracking section
of the detector and is several degrees at 100 \mev.  The PSF improves
with energy approaching a 68\% containment radius of $\sim0\fdg2$ at
the highest energies (when averaged over the acceptance of the LAT)
and is limited by the granularity of the silicon strips in the tracker
\citep{atwood_LAT_mission,on_orbit_calibration,lat_on_orbit_psf}.\footnote{More
information about the performance of the LAT can be found at the \fermi
Science Support Center (FSSC,\url{http://fermi.gsfc.nasa.gov}).} However,
since most high energy astrophysical sources have spectra that decrease
rapidly with increasing energy, the improved resolution of the higher
energy photons is offset by their low statistics. Therefore sophisticated
analysis techniques are required to maximize our sensitivity
to these extended sources.

\subsection{The \pointlike Package}

A new analysis tool has been developed to address the unique requirements
for studying spatially extended sources with the LAT. The tool performs
a maximum likelihood analysis in which the Poisson likelihood to find
the observing counts is maximized given a parametrized spatial and
spectral model of the source and its surrounding region.  The sky is
divided into cubes in space and energy using the healpix representation
of the sky \citep{healpix} and the likelihood is maximized over all bins
in a region.  The extension of a source can be modeled by a geometric
shape (e.g. a disk or a two-dimensional Gaussian) and the source's position, extension,
and spectrum can be simultaneously fit.

This type of analysis is not feasible using the standard LAT likelihood
analysis tool \gtlike\footnote{\gtlike is distributed publicly by the
FSSC.} because it can only fit the spectral parameters of the model
unless a more sophisticated iterative procedure is used to also test
various source morphologies.  We note that \gtlike has been used in the
past in several studies of source extension in the LAT Collaboration
\citep{lmc,smc,w28,w51c}.  In these studies, a profile based upon a
set of \gtlike maximum likelihood fits at fixed extensions was used
to build a profile of the likelihood as a function of extension.
This approach is not optimal because the position, extension, and
spectrum of the source must be simultaneously fit to find the best fit
parameters and to correctly compute the statistical significance of
a detection.  Furthermore since the \gtlike likelihood profile approach
is computationally intensive, no large-scale Monte Carlo simulations
have been run to validate it.

The approach presented here is based on a second maximum likelihood
fitting package developed in the LAT Collaboration called \pointlike
\citep{first_cat,matthew_kerr_thesis}.  The choice to base the
spatial extension fitting on \pointlike rather than \gtlike was made
on considerations of computing time.  The \pointlike algorithm was
optimized for speed to handle larger numbers of sources efficiently
which is important for our catalog scan and for being able
to perform large-scale Monte Carlo simulations to validate the tool.
Details on the \pointlike package can be
found in \cite{matthew_kerr_thesis}.  We extended the code to allow a
simultaneous fit of the source extension together with the position and
the spectral parameters.

\subsection{Extension Fitting}
\label{extension_fitting}

In \pointlike, it is assumed that the spatial and spectral
model of an extended source are separable, i.e. that the source
model $M(l,b,E)=S(l,b)\times X(E)$ where $S(l,b)$ is the spatial
distribution and $X(E)$ is the spectral distribution.  To fit an extended source,
\pointlike convolves the extended source shape with the PSF (as a
function of energy) and uses the \minuit library 
\citep{minuit_documentation}
to maximize the
likelihood by simultaneously varying the position and extension of
the source.  As will be described in
Section~\ref{monte_carlo_validation}, simultaneously fitting the position
and extension is important to correctly calculate the statistical
significance of the detection of extension.  For each position and
extension, the spectral parameters of the sky model are refit.  To avoid
projection effects, the fitted source position parameters are not the
source's longitude and latitude but instead the source's displacement
in a rotated reference frame.

The significance of the extension of a source can be calculated from the
likelihood ratio test of a model with a spatially extended source and
a model with a point-like source. The test statistic for this procedure
is defined as
\begin{equation}
  \tsext=2\log(\likelihood_\text{ext}/\likelihood_\text{ps}) 
\end{equation}
where \likelihood is the Poisson likelihood.
\pointlike calculates \tsext by fitting a source first with a spatially
extended model and then as a point source.  The interpretation
of \tsext in terms of a statistical significance is discussed in
Section~\ref{monte_carlo_validation}.

For extended sources with an assumed radially symmetric shape,
we can optimize the calculation by performing one
of the integrals analytically.
The expected photon 
distribution can be written as
\begin{equation}
  \text{PDF}(\vec r) = \int  \text{PSF}(|\vec r - \vec r'|)I_\text{src}(\vec r') r' dr' d\phi'
\end{equation}
For the LAT, the PSF can be parameterized by a King function \citep{king_function}:
\begin{equation}
  \text{PSF}(r) = 
  \frac{1}{2\pi\sigma^2}
  \left(1-\frac{1}{\gamma}\right)
  \left(1+\frac{u}{\gamma}\right)^{-\gamma},
\end{equation}
where $u=(r/\sigma)^2/2$ and $\sigma$ and $\gamma$ are free parameters
\citep{matthew_kerr_thesis}.  For radially symmetric extended sources,
the angular part of the integral can be evaluated analytically
\begin{align}
  \text{PDF}(u) & = \int_0^\infty r' dr'
  I_\text{src}(v) 
  \int_0^{2\pi} d\phi' 
  \text{PSF}(\sqrt{2\sigma^2(u+v-2\sqrt{uv}\cos(\phi-\phi'))})
  \\
  & = \int_0^\infty dv
  I_\text{src}(v) 
  \left(\frac{\gamma-1}{\gamma}\right)
  \left( \frac{\gamma}{\gamma + u + v}\right)^\gamma 
  \times {}_2F_1 \left(\gamma/2,\frac{1+\gamma}{2},1,\frac{4uv}{(\gamma+u+v)^2}\right).
\end{align}
where $v=(r'/\sigma)^2/2$ and ${}_2F_1$ is the Gaussian hypergeometric
function.  This convolution formula reduces the expected photon
distribution to a single numerical integral.

There will always be a small numerical discrepancy between the expected
photon distribution derived from a true point source and a very small
extended source due to numerical error in the convolution.  In most
situations, this error is insignificant.  But in particular for
very bright sources, this numerical error has the potential to bias the
test statistic for the extension test. Therefore, when calculating
\tsext, we compare the likelihood fitting the source with an extended
spatial model to the likelihood when fixing its extension to $10^{-10\,\circ}$.

We estimate the error on the extension of a source by fixing
the position of the source and varying the extension until the
likelihood has fallen by \onehalf, corresponding to a $1\sigma$ error
\citep{Statistical_methods_book}.  This method is shown schematically in
Figure~\ref{four_plots_ic443} which shows the change in the log of the
likelihood when varying the extension of the SNR IC443.  The localization
error is calculated by fixing the extension and spectrum of the source
and fitting to the likelihood function a 2D Gaussian as a function
of position.

\subsection{\gtlike Analysis Validation}
\label{gtlike_crosscheck}

\pointlike is important for LAT analyses that require many iterations
such as source localization and extension fitting.  On the other hand,
because \gtlike makes fewer approximations in calculating the likelihood
we expect the spectral parameters found with \gtlike to be slightly more
precise.  Furthermore, because \gtlike is the standard likelihood analysis
package, it has been more extensively validated for spectral analysis.
For those reasons, in the following analysis we used \pointlike to
determine the position and extension of a source and subsequently derived
the spectrum using \gtlike. Both \gtlike and \pointlike can be used to
estimate the statistical significance of the extension of a source and we
required that both methods agree for a source to be considered extended.
We found good agreement between the two methods.  Unless explicitly
mentioned, all \ts, \tsext, and spectral parameters were calculated using
\gtlike with the best-fit positions and extension found by \pointlike.

\subsection{Dual Localization}
\label{dual_localization_method}

There is a degeneracy between a spatially extended source and multiple
point sources separated by angular distances comparable to or smaller than
the size of the LAT PSF.  To assess the possibility of source confusion,
we use \pointlike to simultaneously fit the position of two point sources
in the region of extended source candidates.

We define \tsinc as twice the increase in the log of the likelihood fitting the
region as two point sources compared to fitting the region as one point
source: 
\begin{equation}
  \tsinc=2\log(\likelihood_\text{2pts}/\likelihood_\text{ps}).
\end{equation} 
\tsinc can not be directly compared to \tsext to see which
model is more significant because the models are not nested
\citep{statistics_with_care}. Even though the comparison of \tsext with
\tsinc is not a calibrated test, we find the cases $\tsinc \ll \tsext$
or $\tsinc\gg\tsext$ suggestive and we only consider a source to be
extended if $\tsext>\tsinc$.
Similar to the case of extended sources described in Section~\ref{gtlike_crosscheck},
the spectra of the two point sources can be refit using \gtlike.
We quote the spectral values obtained from \gtlike using the best fit
positions found using \pointlike.  

\subsection{Comparing Source Sizes}

\label{compare_source_size}

We tested two different models for the
surface brightness profile, either a 2D Gaussian
\begin{equation}\label{gauss_pdf}
  I(x,y)=\tfrac{1}{2\pi\sigma^2}\exp\left(-(x^2+y^2)/2\sigma^2\right)
\end{equation}
or a uniform disk
\begin{equation}\label{disk_pdf}
  I(x,y)=
  \begin{cases}
    \frac{1}{\pi\sigma^2} & x^2+y^2\le\sigma^2 \\
    0                      & x^2+y^2>\sigma^2
  \end{cases}
\end{equation}
Although these shapes are significantly different, when convolved
with the PSF, their profiles are very similar.  
This is demonstrated in 
Figure~\ref{compare_disk_gauss}
by showing $I_\text{src}$, the PSF, and the PDF for
a radially symmetric uniform brightness and a Gaussian surface brightness of size 0\fdg5.\footnote{To
allow a valid comparison between Gaussian and disk shaped morphologies,
we define the source size for this test as the radius containing
68\% of the intensity ($\rsixeight$).  $\rsixeight_\text{,Gaussian}=1.51\sigma$
and $\rsixeight_\text{,disk}=0.82\sigma$ where $\sigma$
is defined in Equation~\ref{gauss_pdf} and
Equation~\ref{disk_pdf} respectively.} This plot shows that the LAT 
has little sensitivity to
the exact structure of an extended source.
Therefore, in our
search for extended sources, we use only a uniform
disk as our spatial extension shape. We then quote the radius of the
disk edge as the size of the source.

\section{Validation of Analysis Method}

\label{monte_carlo_validation}

We test the false detection probability of \pointlike by fitting the extension of
point-like sources.
\cite{mattox_egret} discuss that the test statistic distribution
for a likelihood ratio test on the existence of a source at
a given position is 
\begin{equation}\label{ts_ext_distribution}
  P(\ts)=\onehalf(\chi^2_1(\ts)+\delta(\ts)).
\end{equation}
The particular form of Equation \ref{ts_ext_distribution} is
due to the null hypothesis (source flux $\Phi=0$) residing
on the edge of parameter space and the model hypothesis
adding a single degree of freedom. It is plausible to
expect a similar distribution of the test statistic
in the test for source extension since the same conditions
apply (with the source flux $\Phi$ replaced by the source radius $r$ and
$r<0$ being unphysical).
To validate this claim, we performed Monte Carlo
simulations
to calculate empirical distributions for $\tsext$ and
compared them to Equation~\ref{ts_ext_distribution}.

We simulated point sources with various spectral forms using
the LAT on-orbit simulation tool, 
\gtobssim\footnote{\gtobssim is distributed publicly by the FSSC.} and fit them with \pointlike using both point
and extended source hypotheses.  These point sources were simulated with a power-law
spectral model with integrated fluxes above 100 \mev ranging from $3\times10^{-9}$ 
to $1\times10^{-6}$ \phflux in six discrete steps and spectral
indices ranging from 1.5 to 3 in four discrete steps.  These values
were picked to to represent typical source parameters of LAT-detected
sources. The point sources were simulated on top of an isotropic
background with an integrated flux above 100 \mev of $1.5\times10^{-5}$ \phflux
taken to be the same as the isotropic spectrum measured by EGRET
\citep{sreekumar_isotropic}.  The Monte Carlo simulation was performed
over a one-year observation period using a representative rocking profile and a
representative livetime fraction of 0.8.  The reconstruction was performed
using 1 \gev to 100 \gev photons and the Pass~7\_V6 (P7\_V6) Source Instrument
Response Function (IRFs, \cite{lat_on_orbit_psf}).  For each 
significantly detected point source ($\ts\ge25$), we used \pointlike
to fit it as an extended source and calculate \tsext.


For each set of spectral parameters, $\sim30,000$ statistically independent
simulations were performed. For the dimmer spectral models, many of the
simulations left the source undetected ($\ts<25$)
and were discarded.  Table~\ref{ts_ext_num_sims}
shows the different spectral models used in our study as well as the
number of simulations.  The cumulative density of \tsext is plotted in
Figure~\ref{ts_ext_mc}. The $\chi^2_1/2$ distribution of
Equation~\ref{ts_ext_distribution} is overlaid for comparison.

Our study shows broad agreement between simulations and
Equation~\ref{ts_ext_distribution}. Nevertheless, the agreement is not
perfect.  It should be noted that the discrepancy seems to be worst for
bright sources where numerical errors in the convolution
are most apparent.  Another possible
reason for the departure from Equation~\ref{ts_ext_distribution} 
is that \pointlike ignores energy dispersion which will change the
PSF shape as a function of energy. We emphasize that many of
the empirical distributions lie to the left of the theoretical curve so
using the theoretical distribution will lead to an underestimate of the
statistical significance of a detection. Therefore, we are confident that
$\sqrt{\tsext}$ can be used as a conservative measure of the statistical
significance of a source's extension and use it in the following analysis.

\section{Extended Source Detection Threshold}
\label{extension_sensitivity}

We calculated the LAT's detection threshold to detect that a spatially extended
sources is extended. We define the detection threshold as the flux at which the value
of $\tsext$ averaged over many statistical realizations of a source
is $\langle\tsext\rangle=16$, corresponding to a $4\sigma$ detection
(see section~\ref{monte_carlo_validation}).

We used a simulation setup similar to that described in
Section~\ref{monte_carlo_validation}, but instead of point sources
we simulated extended
sources with a radially symmetric uniform surface
brightness. Additionally, we
simulated our sources over the two-year time range included
in 2FGL.
Each extension and
spectral index, we selected a flux range which bracketed $\tsext=16$
and performed an extension test for $>100$ independent realizations of
ten flux values in this range.
We calculated $\langle\tsext\rangle=16$ by fitting a line to the flux
and $\tsext$ values in this narrow range.

Figure~\ref{index_sensitivity} shows the threshold for sources of four
spectral indices from 1.5 to 3 and extension varying from $\sigma=0\fdg1$
to $2\fdg0$.  The LAT's flux threshold for a significant detection
of source extension drops quickly with
increasing source size and reaches a minimum around 0\fdg5. 
Figure~\ref{index_sensitivity} shows
the threshold using photons with energies between 100 \mev and 100 \gev
and also using only 1 \gev to 100 \gev photons.
Except for very large ($>1\deg$) sources, our detection threshold is
not substantially improved by including photons with energies between 100 \mev and
1 \gev.  This is also demonstrated in Figure~\ref{four_plots_ic443}
which shows \tsext for the SNR IC443 computed independently in twelve
energy bins between 100 \mev and 100 \gev. For IC443, which has a
spectral index $\sim2.4$, almost the entire increase in likelihood
modeling the source as being extended comes
from energies above 1 \gev.  On the other hand, other systematic errors
become increasingly important at low energy. For our extension search,
we therefore use only photons with energies above 1 \gev.

\hl{
Figure~\ref{all_sensitivity} shows the flux threshold
as a function of source extension for different background levels (1x,
10x, and 100x the nominal background), different spectral indices, and
two different energy bands (1 \gev to 100 \gev and 10 \gev to 100 \gev).  The detection
threshold is higher for sources in regions of higher background.  For a
fixed 1 \gev to 100 \gev or 10 \gev to 100 \gev flux, the LAT's detection threshold
only weakly depends upon the spectral index.  This effect is most
pronounced when using only photons with energies above 10 \gev.  Overlaid
on Figure~\ref{all_sensitivity} are the LAT detected extended sources
that will be discussed in Sections~\ref{validate_known}
and \ref{new_ext_srcs_section}.  The extension thresholds are tabulated
in Table~\ref{all_sensitivity_table}.

Finally, Figure~\ref{time_sensitivity} shows the LAT's projected
detection threshold to extension after 10 years against 10 times
the isotropic background. This background is representative of the
background near the Galactic plane.  For small extended sources, our
detection threshold improves by a factor larger than the square root
of the exposure because at high energies, where we are most sensitive
to extension, the background levels are in the Poisson instead of the
Gaussian regime.  For large extended sources, the relevant background
is over a larger spatial range and so the improvement is closer to the
expected factor corresponding to the square root of the exposure.
}


\section{Extended Source Search Method}
\label{extended_source_search_method}

We test all sources in 2FGL for spatial extension.
2FGL included
twelve previously published spatially extended sources but not
attempt was made to fit the extension of these sources. Other than these sources,
all 2FGL sources were modeled as point sources and 2FGL did not attempt
to resolve the extension of new sources.

Our analysis technique is closely related to that 
of the 2FGL catalog. We used the same two-year dataset from 2008 August 4
to 2010 August 1 and we used the same P7\_V6 Source class
event selection and IRFs \citep{lat_on_orbit_psf}.  
The same 
models were used to describe the background from Galactic diffuse, isotropic, and
Earth limb emission.  To account for possible residuals in the diffuse
emission model, the galactic emission was scaled by a power-law
and the isotropic component's normalization was left free.

As was shown in Section~\ref{extension_sensitivity}, we gain little in sensitivity using photons with energies
below 1 \gev. On the other hand, the large PSF at low energy makes us
more susceptible to systematic errors arising from source confusion due
to multiple point sources and modeling of the
Galactic diffuse emission. In addition, the Galactic diffuse emission
is more pronounced at lower energies due to its steep energy spectrum
\citep{intermediate_diffuse_lat}.
For that reason, we performed our search using only photons with
energies between 1 \gev and 100 \gev. 

We also performed a search for extended sources using only 10 \gev to 100 \gev photons. 
Even this approach tests the same
sources, it is complimentary because the Galactic diffuse
emission is even less dominant above 10 \gev. Furthermore, source
confusion becomes less of a problem since most LAT sources are not
significantly detected at energies exclusively above 10 \gev. So we
expect to be able to resolve harder sources in more complicated regions. The $>10$ \gev
analysis is
especially
beneficial for regions near pulsars which are not significantly detected 
above 10 \gev. A similar procedure was
used to detect HESS\,J1825$-$137 and MSH\,15$-$52 with the LAT
\citep{msh1552,fermi_hess_j1825}.

We tested each source for extension using
\pointlike
assuming the source had a uniform radially symmetric surface brightness
and a power-law spectral model.
We used a circular $10\deg$ region of interest (ROI) centered on our source and
included all catalog sources within $15\deg$ of the source of interest
in our background model.
We refit the spectral parameters of sources within $2\deg$ of the source
to avoid potential biases of the extension parameters of the source of
interest due to close-by background sources.

Finally, when analyzing each region, we automatically removed from
our background model other 2FGL sources which were within 0\fdg5 of
the source of interest from the background model. This was done due
to a concern that extended sources are included in 2FGL as multiple
point sources. These spurious sources could distort the extension
fit.  Instead, when a source is found to be significantly extended
(i.e. $\tsext>16$), we perform the dual localization procedure (described
in Section~\ref{dual_localization_method} to compare the extended source
hypothesis to the hypothesis of two independent point sources. Only
sources with $\tsext>\tsinc$ are considered as extended.

\subsection{Additional Analysis}

We expect most spatially resolvable extended sources to be located
in our Galaxy and thus to be concentrated along the Galactic plane.
Unfortunately, the \gev emission in the Galactic plane is dominated by
extremely structured diffuse emission from the interactions of cosmic rays
with the interstellar medium.  Finding sources on top of this emission
is difficult \citep{first_diffuse_paper} as has been discussed in 1FGL
and 2FGL \citep{first_cat,second_cat}.  Furthermore, the Galactic plane
is crowded and it is often difficult to correctly model nearby sources.
Because of this, finding a source with $\tsext>16$ is not a sufficient
criteria alone for claiming a detection of source extension.  For each
extended source, we perform several analysis crosschecks.

For each candidate, we generated a map of residual \ts by adding a new
source of spectral index 2 into the region at each pixel and finding the
increase in likelihood when fitting its flux. Residual \ts maps are useful
to look for residual structure in the sky model.  Figure~\ref{res_tsmaps}
shows a residual \ts map for the extended source IC443.  The residual
\ts map indicates that the spatially extended model for IC443 is a
significantly better description of the observed photons.  We also
generated plots of the sum of all counts within a given distance of
the source and compared it to the model predictions of a point source.
An example radial integral plot is shown for the extended source
IC443 in Figure~\ref{four_plots_ic443}.  For each source, we also made
diffuse-emission-subtracted smoothed counts maps (show for IC443 in
Figure~\ref{four_plots_ic443}).

% 2FGL J1856.2+0450c = P72Y3047 for
We then inspected each extended source candidate to identify cases in
which the extension fit is clearly influenced by large-scale residuals in
the diffuse emission and hence the extension measurement is unreliable.
An example of such a case is shown in Figure~\ref{example_bad_fit}. It
shows a Galactic and isotropic diffuse-emission-subtracted smoothed
counts map of 2FGL\,J1856.2+0450c using 1 \gev to 100 \gev photons.
In this region along the Galactic plane, there appears to be large-scale
residual in the diffuse emission. As a result, 2FGL\,J1856.2+0450c is fit
to an extension of 1\fdg83 and the result is statistically significant
with \tsext=45.4. However, by looking at the residuals it is clear
that this complicated region is not fit well even though the source's
extension is statistically significant. We do not report sources that
fail this inspection as extended.
% information came from
% /nfs/slac/g/ki/ki03/lande/extended_catalog/2FGL/v15/standard_analysis/spectral_emin_1000_v1/P72Y3047/v1/results_followup_P72Y3047.yaml

For the remaining candidates, we took the spatial and spectral model found
using \pointlike and re-determined the spectral parameters using \gtlike.
We used the `binned likelihood' mode of \gtlike on a $14\deg\times14\deg$
ROI with a pixel size of 0\fdg03.  We obtained a second measure of \tsext
from \gtlike for the extension and position of the source as determined
by \pointlike.  We only considered a source to be significantly extended
if $\tsext>16$ with both \pointlike and \gtlike.

Because of the high source density in the Galactic plane, we often had
to iteratively improve the model of our background sources to obtain a
better fit of the candidate source.  Several catalog sources would often
fill in the emission of the extended source and had to be removed from
the background model of the region.  Similarly, background sources were
often insignificant in the fit energy range and had to be removed from
the sky model.  The position of background sources would often have
to be refit once the extended source was fit.  When the model of the
extended source coupled strongly with nearby sources, we iteratively
fit the extended source and all nearby sources until the fit converged.
For each extended source, we describe the modifications of the background
model compared to the model used in the creation of the 2FGL catalog
that were required.

\subsection{Systematic Errors on Extension}
\label{systematic_errors_on_extension}

% Alternative PSF

We estimate a systematic error on the extension of a source due to
uncertainty in our knowledge of the LAT PSF.  Before launch, the LAT
PSF was determined by detector simulations which were verified in
accelerator tests \citep{atwood_LAT_mission}. However, in-flight data
revealed a discrepancy above a few \gev in the PSF compared to the 
angular distribution of photons from bright AGN.  A publication on this issue is in
preparation \citep{lat_on_orbit_psf}.  Subsequently, the PSF was fit
empirically to bright AGN and this empirical parameterization is the
default PSF used in the P7\_V6 IRFs.  To account for this uncertainty in
our knowledge of the PSF, we refit our extended source candidates using
the pre-flight Monte Carlo representation of the PSF and consider the difference
in extension found using the two PSFs as a systematic error on the
extension of a source.  The same approach was used in \citep{ic443}.
At high energies, we believe that our parameterization of the PSF from
bright AGN is substantially better than the Monte Carlo representation
of the PSF so this estimate of systematic errors is conservative.

% Alternate Diffuse

We estimate a second systematic error on the extension of a source
due to uncertainty in the model of the Galactic diffuse emission by
using an alternative diffuse model. An alternative model was based
upon GALPROP\footnote{GALPROP is a software package for calculating the
Galactic $\gamma$-ray emission based on a model of cosmic-ray propagation
in the Galaxy. See \url{http://galprop.stanford.edu/} for details
and references} and used in the LAT analysis of the isotropic diffuse
emission \citep{isotropic_lat}.  The intensities of various components
to the galactic diffuse emission were then fitted individually using a
spatial distribution of the intensities within the ROI as predicted by
the model.  We distinguish contributions from CR interactions with the
molecular hydrogen, the atomic+ionized hydrogen, residual gas traced
by dust \citep{isabelle_dark_gass}, and the interstellar radiation
field. We further split the contributions from interactions with molecular
and atomic hydrogen to the Galactic diffuse emission according to the
distance from the Galactic center in which they are produced. Hence, we
replace the standard diffuse emission model by 18 individually fitted
templates to describe individual components of the diffuse emission.
A similar crosscheck was used in the LAT Collaboration's analysis of 
RX\,J1713.7$-$3946 \citep{rx_j1713_lat}.

It is not expected that this diffuse model is superior to the standard
LAT model obtained through an all-sky fit.  However, adding degrees of freedom to the background model can
remove likely spurious sources that correlate with features in the
Galactic diffuse emission.  Therefore, this tests systematics that may
be due to incorrect modeling of the diffuse emission in the region.

We do not except the systematic error due to uncertainties in the PSF to
be correlated to the systematic due to uncertainty in the Galactic diffuse
emission so the total systematic error on the extension of a source so
we obtain the total systematic error by adding them in quadrature.

\section{Analysis of Extended Sources Identified in 2FGL}
\label{validate_known}

% 6 SNRs 

We first present on our analysis of the twelve extended sources
included in 2FGL.
\citep{second_cat}.
Six extended SNRs were included in 2FGL: IC443, W28, W30, W44, W51C, and the
Cygnus Loop \citep{ic443,w28,w44,w51c,cygnus_loop_lat}.\footnote{A
detailed publication by the LAT
Collaboration about W30 is still in preparation.}
\gev emission from W30 was also studied in \cite{castro_and_slane_2010}.
Using 1 \gev to 100 \gev photons, our analysis significantly detected the
extension of all six SNRs.
% 2 galaxies

Two nearby satellite galaxies of the Milky Way the Large Magalenic Cloud (LMC)
and the Small Magalenic
Cloud (SMC) were included in 2FGL as spatially extended sources \citep{lmc,smc}.  They were significantly
detected using 1 \gev to 100 \gev photons. Our
fit extension is comparable to the published result, but we note that
previous LAT Collaboration publication on LMC used a more complicated two 2D Gaussian surface
brightness spatial model when fitting it (\citep{lmc}).

% 3 PWN
Three PWNe, MSH\,15$-$52, Vela X, and HESS\,J1825$-$137 were
identified in 2FGL \citep{msh1552,velax,fermi_hess_j1825}.  
HESS\,J1825$-$137 was significantly detected using 10 \gev to 100 \gev photons.
To improve the model of this source, we removed from our background model the
nearby catalog source 2FGL\,J1823.1$-$1338c which is part of the extended
source.  To avoid confusion with the nearby bright pulsar PSR\,J1509$-$5850, MSH\,15$-$52 must be
analyzed at high energies.  Using photons with energies above 10 \gev,
we fit the extension of MSH\,15$-$52 to be consistent with the published
size with an extension significance of \tsext=6.5.  

% failed to detect vela x + cen a
Our analysis was unable to resolve Vela X which would have required first
removing the pulsed photons from the Vela pulsar which is beyond the
scope of this paper.  Our analysis also failed to detect a significant
extension for the Centaurus A Lobes \citep{cen_a_lat}. This is because
the source's emission is significantly different from a uniform
radially symmetric surface brightness.

Our analysis of these sources is summarized in
Table~\ref{known_extended_sources}.  This table includes the best fit
position and extension of these sources when fitting them as disk-shaped
sources with a radially symmetric uniform surface brightness. It also
includes the best fit spectral parameters for each source.  The position
and extension of Vela X and the Centaurus A Lobes are taken from
\cite{velax,cen_a_lat} and are included in this list for completeness.

\section{Test of 2LAC Sources}
\label{test_2lac_sources}

To validate our method, we test LAT sources associated with AGN for
extension.  \gev emission from AGN is believed to originate from the
cores of kiloparsec-scale jets of distant galaxies.  Therefore AGN are
not expected to be spatially resolvable by the LAT and provide a good
calibration source to demonstrate that our extension detection method
does not misidentify point sources as being extended.  We note that
megaparsec-scale $\gamma$-ray halos around AGNs have been hypothesized
\citep{pair_halo_paper} to be resolvable by the LAT. However, no such
halo has been discovered in the LAT data so far.

Following 1FGL, the LAT Collaboration published the First LAT AGN Catalog
(1LAC), a list of sources that had a high probability association with
AGN \citep{first_agn_cat}.  1FGL had 709 1FGL sources associated with 671
distinct AGN at high latitude ($|b|>10\deg$).  Using two years of data
and 2FGL, the Second LAT AGN Catalog (2LAC) associated 1016 2FGL sources
with AGN \citep{second_agn_cat}.  To avoid systematic problems with AGN
classification, we selected 885 out of the 1016 AGN which made it into
the clean AGN sub-sample.  An AGN association is considered clean only
if it has a high probability of association $P\ge 80\%$, if it is the
only AGN associated with the catalog source, and if there are no flags
on the source in 2FGL. These last two conditions are important for our
analysis. Source confusion may look like a spatially extended source
and flagged catalog sources may correlate with unmodeled structure in
the diffuse emission.

Of the 885 clean AGN, we select 783 of these 2FGL sources which
are significantly-detected above 1 \gev and fit each of them for extension.
A histogram of the \tsext values computed for these AGN is
shown in Figure~\ref{agn_ts_ext}. Overlaid on the plot is the
$\chi^2/2$ distribution of Equation~\ref{ts_ext_distribution}.
The \tsext distribution for AGN shows good agreement with the
theoretical distribution.  Two sources had $\tsext>10$.  One was due
to the incorrect removal of a nearby catalog sources from
the sky model (see
Section~\ref{extended_source_search_method}) and the other was due to a
failure of convergence of the point hypothesis.  This result demonstrates
that we can use \tsext as a measure of the statistical significance of
the detection of the extension of a source.

We should clarify that the LAT PSF used in this study was determined
empirically by fitting the observed shape of bright AGN (see
Section~\ref{systematic_errors_on_extension}). Finding that the AGN we
test are not extended is not surprising.  This validation analysis is
not suitable to reject any hypotheses about the existence of megaparsec-scale
halos around AGN.

\section{New Extended Sources}
\label{new_ext_srcs_section}

% Notes about extended Catalog sources
% Catalog used for analysis is P72Y_uw23.fits
% Compare to final catalog gll_psc_v04.fit
%          1FGL Name - Preliminary 2FGL - 2FGL Name
% 1FGL J1628.6-2419c - P72Y2516         - 2FGL J1627.0-2425c
% 1FGL J0823.3-4248  - P72Y1212         - 2FGL J0823.0-4246
% 1FGL J1711.7-3944c - P72Y2674         - 2FGL J1712.4-3941

% 1FGL J1613.6-5100c - P72Y2472         - 2FGL J1615.0-5051 
% 1FGL J1614.7-5138c - P72Y2473         - 2FGL J1615.2-5138
% 1FGL J1632.9-4802c - P72Y2540         - 2FGL J1632.4-4753c
% 1FGL J2020.0+4049  - P72Y3281         - 2FGL J2021.5+4026
% 1FGL J1837.5-0659c - P72Y2974         - 2FGL J1837.3-0700c
% N/A                - P72Y1287         - 2FGL J0851.7-4635

Nine extended sources not included in 2FGL were found using this search.
Three were found in our search using 1 \gev 100 \gev photons and six were
found in our search using 10 \gev to 100 \gev photons.  The fit properties of
these nine sources is summarized in Table~\ref{new_ext_srcs_table}.
This table includes for each source the best fit position, extension,
spectrum, source significance, and significance of extension.

The results of our investigation of systematic uncertainties of this
measurement are presented in 
Table~\ref{alt_diff_model_results}.  It
shows a comparison between the fit with a single extended source
hypothesis and the fit assuming the emission originates from two
independent sources and  the
results of the extension fit using variations of the PSF and the galactic
diffuse model described in Section~\ref{systematic_errors_on_extension}.
There is good agreement between \tsext and the fit size using
the standard analysis, the alternative diffuse models, and the alternative PSF.
This suggests that the sources are robust against features in the diffuse
model and uncertainties in the angular resolution.

\subsection{2FGL\,J0823.0$-$4246}
\label{section_2FGL_J0823.0-4246}

% 1FGL J0823.3-4248  - P72Y1212         - 2FGL J0823.0-4246
% Deleted 
%  * P72Y1214 - 2FGL J0823.4-4305
%  * P72Y1210 - 2FGL J0821.0-4254

The source 2FGL\,J0823.0$-$4246 was found 
to have an 
extension of $0\fdg37\pm0\fdg03_\stat\pm0\fdg02_\sys$ 
with an extension
significance of $\tsext=46.3$
using 1 \gev to 100 \gev photons.  This source was fit to a position of
$(l,b)=(260\fdg32,-3\fdg28)$.  
This source is coincident with the one-year
catalog source 1FGL\,J0823.3$-$4248.
\hl{
Figure~\ref{1FGL_J0823.3-4248} shows a
counts map of this source
and Figure~\ref{snr_seds} shows the LAT spectrum of it.
}
To get a good fit of this source, we removed from our background
model the nearby catalog
sources 2FGL\,J0823.4$-$4305 and 2FGL\,J0821.0$-$4254 which are part of the
extended source.  These modifications are
shown in Figure~\ref{1FGL_J0823.3-4248}.

\hl{
This extended source is spatially coincident with the middle-aged
SNR Puppis A.  Puppis A has been studied in detail in radio
(\cite{puppis_a_vla}, and references therein) and  X-ray 
(\cite{rosat_puppis_a,suzaku_puppis_a}, and references therein)
Mosaic \rosat observations of Puppis A produced the highest
resolution X-ray image of this source
and contours corresponding to this
image are overlaid on Figure~\ref{1FGL_J0823.3-4248} and match the
inferred \gev size \citep{rosat_puppis_a}.
Puppis A's distance was estimated at 2.2 kpc \citep{reynoso_1995,reynoso_2003}
which leads to a 1 \gev to 100 \gev luminosity of $\sim 3\times 10^{34}$ ergs/\s.

No molecular clouds have been observed directly adjacent to Puppis A 
\citep{co_eastern_puppis_a}.
This is similar to the LAT detected SNR the Cyngus Loop \citep{cygnus_loop_lat}.
The luminosity of Puppis A is also smaller than that of 
other SNRs believed to interact with molecular clouds
\citep{w51c,ic443,w44,w28,w49b_lat}.
The \gev emission could therefore be explained by the interaction
of accelerated particles by interstellar matter or fields, similar
to the emission mechanism suggested for the Cygnus Loop SNR
\citep{cygnus_loop_lat}.
}


% Puppis A:
%   Flux = PowerLaw(norm=6.21e-14,index=2.22,e0=10000).i_flux(cgs=True,emin=1e3,emax=1e5,e_weight=1)
%        = 4.7804242225163421e-11 ergs cm^-2 s-1
%   Luminosity =  Flux * 4*pi*r^2 = 4.78e-11 ergs cm^-2 s^-1 * 4*pi * (2.2 kpc )**2 in ergs/s
%              = 2.8e34 ergs/s from 1.0-100 gev
%              = 5.6e34 ergs/s from 0.1-100 gev
%   Luminosities
%   Cygnus Loop:
%     L = 10^33 ergs/s between 1-100 \gev
%   W51C : 
%     http://arxiv.org/pdf/0910.0908
%     L > 1 x 10^36 ergs/s (0.2-50GeV) - 
%   IC443:
%     http://arxiv.org/pdf/1002.2198v1
%     L = 1.2e35 ergs/s (from 0.2-50 gev)
%     d = 1.5 kpc
%     From my (possibly incorrect) powerlaw analysis 
%       In [30]: PowerLaw(norm=4.69e-13,index=2.21,e0=1e4).i_flux(cgs=True,emin=1e3,emax=1e5,e_weight=1)
%       Out[30]: 3.5968626651391638e-10 ergs/cm^2/s
%       L = 3.5968626651391638e-10/cm^2 * 4*pi*(1.5kpc)**2 = 1e35 (1-100gev)
%   W44:
%     http://www.sciencemag.org/content/327/5969/1103.full.pdf
%     From my (possibly incorrect) powerlaw analysis 
%       PowerLaw(norm=2.7e-13,index=2.63,e0=1e4).i_flux(cgs=True,emin=1e3,emax=1e5,e_weight=1)
%       Out[27]: 2.7681360351439016e-10 ergs/cm^2/s
%     Distance = 3kpc
%     L = 3e35 ergs/s (1-100 gev)
%   W28
%     http://iopscience.iop.org/0004-637X/718/1/348/
%     From my (possibly incorrect) powerlaw analysis 
%       In [28]: PowerLaw(norm=2.23e-13,index=2.6,e0=1e4).i_flux(cgs=True,emin=1e3,emax=1e5,e_weight=1)
%       Out[28]: 2.221059203350944e-10 ergs/cm^2/s
%     Distance = 2kpc
%     L = 1e35 ergs/s (1-100 gev)
%   W49B
%     http://iopscience.iop.org/0004-637X/722/2/1303
%     L = 10^36 erg/s (0.2 - 200 gev)
%
    
\subsection{2FGL\,J1627.0$-$2425c}
\label{section_2FGL_J1627.0-2425c}

% 1FGL J1628.6-2419c - P72Y2516         - 2FGL J1627.0-2425c
% No modifications to background were needed

The source 2FGL\,J1627.0$-$2425c was found  to
have an extension of $0\fdg41\pm0\fdg05_\stat\pm0\fdg02_\sys$ with
an extension significance of $\tsext=31.1$
using 1 \gev to 100 \gev photons.  The best fit position is
$(l,b)=(353\fdg08, 16\fdg78)$.  This source is coincident with the one
year catalog source 1FGL\,J1628.6$-$2419c.  A counts map showing this source
is seen in Figure \ref{1FGL_J1628.6-2419c}.  

\hl{
This source is in a region of remarkably complicated diffuse emission.
Even though it is $16\deg$ from the Galactic plane, this source is on
top of the core of the Ophiuchus molecular cloud which contains massive
star-forming regions that are bright in the infrared.  The region also has
abundant molecular and atomic gas traced by CO and HI but also plenty of
dark gas found only by its association with dust emission
\citep{isabelle_dark_gass}. Embedded star-forming regions make it even
more challenging to measure the column density of dust.  Infared and 
${}^{12}\text{CO}$ ($J=1\rightarrow 0$)
contours are overlaid on Figure~\ref{1FGL_J1628.6-2419c}. There is good
spatial correlation with the \gev emission \citep{iras_rho_ophiuci,co_rho_ophiuci}
so this source might represent an inadequacy in the diffuse
$\gamma$-ray model.
}

\subsection{2FGL\,J1712.4$-$3941}
\label{section_2FGL_J1712.4-3941}

% 1FGL J1711.7-3944c - P72Y2674         - 2FGL J1712.4-3941
% New Source:
%  * SkyDir(346.85102135118683,0.24359779858357899,SkyDir.GALACTIC),
% Delete:
%  * P72Y2685 - Not in 2FGL
%  * P72Y2680 - Not in 2FGL
%  * P72Y2689 - Not in 2FGL

The source 2FGL\,J1712.4$-$3941 was found 
photons to have an extension $0\fdg56\pm0\fdg04_\stat\pm0\fdg02_\sys$
with an extension significance of $\tsext=39.6$
using 1 \gev to 100 \gev.  This source was
fit to a position of $(l,b)=(347\fdg25,-0\fdg54)$.  this source
is coincident with the one-year catalog source 1FGL\,J1711.7$-$3944c.
Figure \ref{2FGL_J1712.4-3941} shows a smoothed counts map of this source.

This source is spatially coincident with the SNR RX\,J1713.7$-$3946
and was recently reported by \citep{rx_j1713_lat}.  
Figure~\ref{2FGL_J1712.4-3941} overlays H.E.S.S. \tev contours of SNR 
RX\,J1713.7$-$3946 from \citep{rx_j1713_hess}.  To analyze this source,
we used the same background model as the recent LAT publication.
2FGL\,J1715.4$-$4024c is spatially coincident with Source A and was
moved to $(\text{RA},\text{Dec})=(258.84,-40.46)$. Source B was added
at $(\text{RA},\text{Dec})=(258.71,-38.70)$ and Source C was added at
$(\text{RA},\text{Dec})=(257.47,-39.75)$.

\subsection{2FGL\,J0851.7$-$4635}
\label{section_2FGL_J0851.7-4635}

% (no 1FGL) - P72Y1287         - 2FGL J0851.7-4635
% Delete:
% * P72Y1291 - 2FGL J0853.5-4711
% * P72Y1296 - 2FGL J0855.4-4625
% * P72Y1274 - 2FGL J0848.5-4535
% * P72Y1259 - 2FGL J0842.9-4721
% * P72Y1300 - 2FGL J0858.0-4815
% * P72Y1310 - 2FGL J0901.7-4655 
% Modify:
% * P72Y1293 - 2FGL J0854.7-4501 - SkyDir(266.235,0.493,SkyDir.GALACTIC))
%              (initial Position - SkyDir(265.5705,-0.0099)

The source 2FGL\,J0851.7$-$4635 was found 
to have an
extension of $1\fdg13\pm0\fdg08_\stat\pm0\fdg05_\sys$ 
with an extension
significance of $\tsext=87.2$
using 10 \gev to 100 \gev photons.  This source was fit to a position of
$(l,b)=(266\fdg29,-1\fdg43)$.  Figure~\ref{Vela_Jr} shows a counts
map of this source.

2FGL\,J0851.7$-$4635 is spatially coincident with the SNR Vela Jr.
Overlaid on Figure~\ref{Vela_Jr} are contours of Vela Jr. as seen in
\tev by H.E.S.S \citep{vela_jr_hess}.  The \gev and \tev morphology
match well.  A detailed papers by the LAT Collaboration on Vela Jr. is
in preparation.

To get a good fit of this source, removed from
our background model the nearby catalog
sources 2FGL\,J0853.5$-$4711, 2FGL\,J0848.5$-$4535, and 2FGL\,J0855.4$-$4625
which are part of the extended source.  In addition, we relocalized
the position of the nearby catalog source 2FGL\,J0854.7$-$4501 to
$(l,b)=(266\fdg24,0\fdg49)$ to better fit its position at high energies
in the presence of of the extended source.  In addition, we removed from
our model the
further away catalog sources 2FGL\,J0858.0$-$4815 and 2FGL\,J0901.7$-$4655
because they were not significant above 10 \gev.  

\subsection{2FGL\,J1615.0$-$5051}
\label{section_2FGL_J1615.0-5051}

% For both 2FGL J1615.0-5051 and 2FGL J1615.2-5138 
% Extended Sources:
%   * 1FGL J1613.6-5100c - P72Y2472         - 2FGL J1615.0-5051 % extended source
%   * 1FGL J1614.7-5138c - P72Y2473         - 2FGL J1615.2-5138 % extended source
% Deleted
%  *P72Y2490 - 2FGL J1620.6-5111c
%  *P72Y2487 - 2FGL J1619.7-5040c
%  *P72Y2496 - 2FGL J1622.8-5006

The source 2FGL\,J1615.0$-$5051 was found 
photons to have an extension of $0\fdg33\pm0\fdg04_\stat\pm0\fdg01_\sys$
with an extension significance of \tsext=16.3
using 10 \gev to 100 \gev.  This source
was fit to a position of $(l,b)=(332\fdg38,-0\fdg14)$.
This source is coincident with the one-year catalog source
1FGL\,J1613.6$-$5100c. Figure~\ref{1FGL_J1613.6-5100c} shows a counts map
of this source.

This source is less than $1\deg$ away from 2FGL\,J1615.2$-$5138 which is
also spatially extended (see Section~\ref{section_2FGL_J1615.2-5138}).
To get a good fit of both sources, we modeled both sources as
being spatially extended and iteratively fit the position and extension
of each source until obtaining a global best fit.  Before doing this,
We removed from our background model the source 2FGL\,J1614.9$-$5212 because it is
part of 2FGL\,J1615.2$-$5138's. Furthermore, we removed from our model the nearby catalog
sources 2FGL\,J1619.7$-$5040c and 2FGL\,J1620.6$-$5111c because they were not
significant above 10 \gev.  These modifications are further described
in the caption to Figure~\ref{1FGL_J1613.6-5100c}.  

2FGL\,J1615.2$-$5138 is spatially coincident with the extended
\tev source HESS\,J1616$-$508 \citep{hess_plane_survey}.  In
Figure~\ref{1FGL_J1613.6-5100c}, contours of HESS\,J1616$-$508 are overlaid
on 2FGL\,J1615.0$-$5051.  The H.E.S.S. experiment measured an
extension $0.136\pm 0.008$ when fitting this source with an elliptical
2D Gaussian surface brightness.  This size corresponds to a 68\% containment
radius of $\rsixeight=0\fdg21\pm0\fdg01$. This size is comparable to the LAT
size $\rsixeight=0\fdg27\pm0\fdg03$ (see Section~\ref{compare_source_size}).
Figure~\ref{hess_seds} shows that the LAT spectrum of 2FGL\,J1615.0$-$5051
connects smoothly to the H.E.S.S spectrum of HESS\,J1616$-$508.

\hl{
HESS\,J1616$-$508 is located in the region of two SNRs, RCW103
(G332.4-04) and Kes~32 (G332.4+0.1) but is not spatially coincident
with either of them \citep{hess_plane_survey}.  HESS\,J1616$-$508 is near
three pulsars PSR\,J1614$-$5048, PSR\,J1616$-$5109, and PSR\,J1617$-$5055
but only PSR\,J1617$-$5055 is energetically favored as being the \tev
emitter \citep{discovery_of_PSR_J1617-5055,integral_HESS_J1616-508}.
\cite{hess_plane_survey} speculated that a PWN powered by this young
pulsar could be responsible for the emission of HESS\,J1616$-$508.
Because HESS\,J1616$-$508 is $9\arcmin$ away from PSR\,J1617$-$5055, this would
require an assymetric X-ray PWNe to power the \tev emission. However,
\chandra ACIS observations revealed only an underluminous PWN of
about $\sim1\arcmin$ size around the pulsar which was not oriented
towards the \tev emission, rendering this association as uncertain
\citep{discovery_of_pwn_for_PSR_J1617-5055}.  No other promising
counterparts were observed in observations of X-ray and soft gamma-ray
emission by \suzaku \citep{suzakzu_HESS_J1616-508}, \swiftxrt,
IBIS/ISGRBI, BeppoSAX and \xmmnewton \citep{integral_HESS_J1616-508}.
Finally, \cite{discovery_of_pwn_for_PSR_J1617-5055} found diffuse
emission towards the center of HESS\,J1616$-$508 using archival radio and
infared observations.  Deeper observations will likely be necessary to
understand this region.
}

\subsection{2FGL\,J1615.2$-$5138}
\label{section_2FGL_J1615.2-5138}

The source 2FGL\,J1615.2$-$5138 was found 
photons to have an extension of $0\fdg42\pm0\fdg03_\stat\pm0.01_\sys$
with an extension significance of \tsext=48.0
using 10 \gev to 100 \gev.  This source was fit to a
position of $(l,b)=(331\fdg66,-0\fdg66)$.  This source is coincident
with the one-year catalog source 1FGL\,J1614.7$-$5138c.  Because 2FGL
J1615.2$-$5138 is close to 2FGL\,J1615.0$-$5051, the same model described
in Section~\ref{section_2FGL_J1615.0-5051} was used to analyze both
sources. Both sources can be seen in Figure~\ref{1FGL_J1613.6-5100c}.

This source is spatially coincident with the extended
\tev source HESS\,J1614$-$518 \citep{hess_plane_survey}. In
Figure~\ref{1FGL_J1613.6-5100c}, contours of HESS\,J1614$-$518 are overlaid
on 2FGL\,J1615.2$-$5138.  The H.E.S.S. experiment measured a 2D Gaussian
extension of $\sigma=0\fdg23\pm0\fdg02$ and $\sigma=0.15\pm0.02$
in the semi-major and semi-minor axis. This size corresponds
to a 68\% containment size of $\rsixeight=0\fdg35\pm0\fdg03$
and $0.23\pm0.03$.  This elliptical size matches the LAT size
$\rsixeight=0\fdg35\pm0\fdg03$.  Figure~\ref{hess_seds} shows
that the LAT spectrum of 2FGL\,J1615.2$-$5138 connects smoothly to
the H.E.S.S spectrum of HESS\,J1614$-$518.  Further data collected by
H.E.S.S. in 2007 helped to resolve a double peaked structure in the
H.E.S.S. data but no spectral variation across this source, suggesting
that the emission is not the confusion of physically separate sources
\citep{closer_look_hess_j1614-518}.  The source was also detected by
CANGAROO-III \citep{cangaroo_j1614-518}.

\hl{
There are five nearby pulsars, but none are luminous enough to
provide the energy output required to power the $\tev$ 
emission \citep{closer_look_hess_j1614-518}.  HESS\,J1614$-$518
is spatially coincident with a young open cluster Pismis 22
\citep{hess_1614_landi_atel,closer_look_hess_j1614-518}.
\suzaku detected two
promising X-ray candidates. Source A is an extended source consistent
with the peak of HESS\,J1614$-$518 and source B coincident with Pismis 22
and towards the center but in a relatively dim region of HESS\,J1614$-$518
\citep{suazku_hess_j1614_518}.  Three hypothesis have been presented to
explain this emission; either source A is an SNR powering the $\gamma$-ray
emission, source A is a PWN powered by an undiscovered pulsar in either
source A or B, and finally that the emission may arise from hadronic 
acceleration in the stellar winds of Pisim 22 \citep{cangaroo_j1614-518}.
}

\subsection{2FGL\,J1632.4$-$4753c}
\label{section_2FGL_J1632.4-4753c}


% 1FGL J1632.9-4802c - P72Y2540         - 2FGL J1632.4-4753c
% Kept:
% * P72Y2539 - 2FGL J1632.4-4820c
% Deleted
% * P72Y2535 - 2FGL J1631.7-4720c
% * P72Y2556 - 2FGL J1638.0-4703c
% * P72Y2528 - 2FGL J1630.2-4752
% * P72Y2543 - 2FGL J1634.4-4743c
% * P72Y2521 - 2FGL J1628.1-4857c
% * P72Y2527 - 2FGL J1630.1-4615
% * P72Y2562 - 2FGL J1639.8-4921c
% Modify
% * P72Y2547 - 2FGL J1635.4-4717c - SkyDir(337.230,0.346,SkyDir.GALACTIC)
%              (initial Position  - SkyDir(337.1396,0.1433)
% * P72Y2550 - 2FGL J1636.3-4740c - SkyDir(336.968,-0.066,SkyDir.GALACTIC))
%              (initial Position  - SkyDir(336.9639,-0.236)

The source 2FGL\,J1632.4$-$4753c was found 
to
have an extension of $0\fdg44\pm0\fdg04_\stat\pm0\fdg03_\sys$ 
with an extension
significance of \tsext=64.5
using 10 \gev to 100 \gev photons.  This source was fit to a position of
$(l,b)=(336\fdg41,0\fdg22)$.  This source is coincident with the one
year catalog source 1FGL\,J1632.9$-$4802c.  Figure~\ref{1FGL_J1632.9-4802c}
shows a counts map of this source.

To get a good fit of this source, we removed from our background model
three catalog sources 2FGL\,J1631.7$-$4720c, 2FGL\,J1630.2$-$4752,
2FGL J1634.4$-$4743c.4-4820c that were part of the extended source..
We then iteratively relocalized the source 2FGL\,J1635.4$-$4717c
to $(l,b)=(337\fdg23,0\fdg35)$ and 2FGL\,J1636.3$-$4740c to
$(l,b)=(336\fdg97,-0\fdg07)$ while fitting the extension of
2FGL\,J1632.4$-$4753c.  In addition we removed from our model four
farther away two-year catalog sources 2FGL\,J1638.0$-$4703c, 
2FGL\,J1628.1$-$4857c, 2FGL\,J1630.1$-$4615, 2FGL\,J1639.8$-$4921c because they
were not significant above 10 \gev.  These modifications are shown
in Figure~\ref{1FGL_J1632.9-4802c}.  

This extended source is spatially coincident with the extended
\tev source HESS\,J1632$-$478 \citep{hess_plane_survey}.
In Figure~\ref{1FGL_J1632.9-4802c}, contours of  HESS\,J1632$-$478
are overlaid on 2FGL\,J1635.4$-$4717c.  H.E.S.S measured a 
extension of $\sigma=0.21\pm0.05$ and $0.06\pm0.04$ along the
semi-major and semi-minor axes when fitting this source with an
elliptical 2D Gaussian surface brightness.  This corresponds to a 68\%
containment size $\rsixeight=0\fdg31\pm0\fdg08$ and $0\fdg09\pm0\fdg06$
along the semi-major and semi-minor axis. This size is consistent with
the LAT size $\rsixeight=0\fdg36\pm0\fdg04$.  Figure~\ref{hess_seds}
shows that the LAT spectrum of 2FGL\,J1635.4$-$4717c connects smoothly to
the H.E.S.S spectrum of HESS\,J1632$-$478.

\hl{
\cite{hess_plane_survey} argued that HESS\,J1632$-$478
is positionally coincident with the hard X-ray source
IGR\,J1632-4751 observed by INTEGRAL, \xmmnewton, and \asca
\citep{Igr_J16320-4751_circ,xmm_newton_IGR_J16320-4751,asca_plane_survey},
but this source is suspected to be an galactic X-Ray Binary so the $\gamma$-ray
extension disfavors the association.  Further observations by
\xmmnewton reveal extended emission of size $\sim32\arcsec\times15\arcsec$
and inside point-like emission coincident with the peak
H.E.S.S. emission \citep{hess_j1632_478_xmm_newton}.  They found
in archival MGPS-2 data a spatially coincident extended radio source
\citep{most_survey_galactic_plane}.  The positional match argues for a
single synchrotron processes producing X-ray and \gev and \tev radiation,
likely due to a PWNe.  The increased \gev and \tev size compared
to the X-ray size has previously been observed in several aging PWNe
including HESS\,J1825$-$137 \citep{hess_j1825_xmm_newton,hess_j1825_hess},
HESS\,J1640$-$465 \citep{hess_plane_survey,xmm_newton_hess_j_1640-466},
and Vela X \citep{vela_x_rosat,vela_x_hess}.  This can be explained by
a different synchrotron cooling time for the electrons producing X-rays
and \tev $\gamma$-rays.
}

\subsection{2FGL\,J1837.3$-$0700c}
\label{section_2FGL_J1837.3-0700c}

% 1FGL J1837.5-0659c - P72Y2986 - 2FGL J1837.3-0700c

% Delete:
% * P72Y2982 - Not in 2FGL
% * P72Y2979 - 2FGL J1835.5-0649 - SkyDir(25.0503,0.3894)
% * P72Y2993 - 2FGL J1839.0-0539 - SkyDir(26.4904,0.1632)

% Modify:
% * P72Y2974 - 2FGL J1834.7-0705c - SkyDir(24.716,0.500,SkyDir.GALACTIC)
%              (initial Position  - SkyDir(25.0953,-0.0887)
% * P72Y2985 - 2FGL J1836.8-0623c - SkyDir(25.574,0.320,SkyDir.GALACTIC)
%              (initial Position  - SkyDir(25.5925,0.3103)
% * P72Y2994 - 2FGL J1839.3-0558c - SkyDir(26.080,0.229,SkyDir.GALACTIC)
%              (initial Position  - SkyDir(26.2301,-0.0394)

The source 2FGL\,J1837.3$-$0700c was found 
to
have an extension of $0\fdg35\pm0\fdg08_\stat\pm0\fdg03_\sys$ with an
extension significance of $\tsext=18.8$
using 10 \gev to 100 \gev photons.  This source was fit to a position
of $(l,b)=(25\fdg08,0\fdg13)$.  This source is coincident with the one
year catalog source 1FGL\,J1837.5$-$0659c.  Figure~\ref{1FGL_J1837.5-0659c}
shows a counts map of this source.

This source is in a complicated region. There are three nearby
catalog sources 2FGL\,J1834.7$-$0705c, 2FGL\,J1836.8$-$0623c, and
2FGL\,J1839.3$-$0558c.  To get a good fit of 2FGL\,J1837.3$-$0700c, we
relocalized 2FGL\,J1834.7$-$0705c to $(l,b)=(24\fdg77,0\fdg50)$,
2FGL\,J1836.8$-$0623c to $(l,b)=(25\fdg57,0\fdg32)$, and
2FGL\,J1839.3$-$0558c to $(l,b)=(26\fdg08,0\fdg23)$.  We removed from
our background model the
nearby catalog source 2FGL\,J1835.5$-$0649 which is part of the extended
source and also the farther away catalog source 2FGL\,J1839.0$-$0539
because it was not significant above 10 \gev. These modifications are
shown in Figure~\ref{1FGL_J1837.5-0659c}.  

This source is spatially coincident with the \tev source HESS\,J1837$-$069
\citep{hess_plane_survey}.  In Figure~\ref{1FGL_J1837.5-0659c}, contours
of HESS\,J1837$-$069 are overlaid on 2FGL\,J1837.3$-$0700c. H.E.S.S. measured
an extension of $\sigma=0.12\pm0.02$ and $0.05\pm0.02$
along the semi-major and semi-minor axis when fitting this source
with an elliptical 2D Gaussian surface brightness.  This corresponds
to a 68\% containment radius of $\rsixeight=0\fdg18\pm0\fdg03$ and
$0\fdg08\pm0\fdg03$ along the semi-major and semi-minor axis. The
size is comparable to LAT which fit a 68\% containment radius of
$\rsixeight=0\fdg29\pm0\fdg07$.  the size difference is not significant
(less than 2 sigma).  Figure~\ref{hess_seds} shows that the LAT spectrum
connects smoothly to the H.E.S.S spectrum of HESS\,J1837$-$069.

\hl{
HESS\,J1837$-$069 is coincident
with the hard and steady X-ray source AX\,J1838.0$-$0655
\citep{einstein_galactic_plane_survey,hard_x-ray_asca,integral_AX_J1838.0-0655,swift_follow_up,pulsations_HESS_J1837-069,suzaku_HESS_J1837-069}.
This source was discovery by RXTE to be a pulsar sufficiently
luminous to power the \tev emission.  AX\,J1838.0$-$0655 was spatially
resolved by \chandra to be a bright point source surrounded by
a $\sim2\arcmin$ nebula \citep{pulsations_HESS_J1837-069} and so the
$\gamma$-ray emission may be powered by this pulsar.  A second
X-ray point source AX\,J1837.3$-$0652 is in the region of HESS\,J1837$-$069
\citep{hard_x-ray_asca,swift_follow_up,pulsations_HESS_J1837-069,suzaku_HESS_J1837-069}.
It was also resolved into point-like and diffuse component
although no pulsations have yet been detected from it
\citep{pulsations_HESS_J1837-069}.  If AX\,J1838.0$-$0655 is a pulsar/PWN
powering HESS\,J1837$-$069, some of the \tev emission may also come from
AX\,J1837.3$-$0652.
}


\subsection{2FGL\,J2021.5+4026}
\label{section_2FGL J2021.5+4026}


% 1FGL J2020.0+4049  - P72Y3281         - 2FGL J2021.5+4026
% Delete:
% * P72Y3282 - 2FGL J2019.1+4040
% * P72Y3292 - 2FGL J2022.8+3843c
% * P72Y3287 - 2FGL J2020.0+4159
% * P72Y3262 - 2FGL J2013.8+4115c
% * P72Y3260 - 2FGL J2012.4+3955c
% New source
%  * SkyDir(78.854,2.670,SkyDir.GALACTIC)

The source 2FGL\,J2021.5+4026 was found 
to have an extension of $0\fdg59\pm0\fdg03_\stat\pm0\fdg02_\sys$
with an extension significance of $\tsext=116.4$
using 10 \gev to 100 \gev photons.  This source was
fit to a position of $(l,b)=(78\fdg18,2\fdg19)$.  This source
is coincident with the one-year catalog source 1FGL\,J2020.0+4049.
\hl{
Figure~\ref{1FGL_J2020.0+4049} shows a counts map of this source.
and Figure~\ref{snr_seds} shows that the LAT spectrum of it.
}

To get a good fit of this source, we removed from our background
model the nearby catalog
source 2FGL\,J2019.1+4040 which were part of the extended source.
Further, we found it necessary to add an additional point source not in
the two-year catalog into our background model.  The new source is was
localized to a position of $(l,b)=(78\fdg85,2\fdg67)$ and had $\ts=13.5$ .
Although this source is not very significant, not including it into the
background model significantly biased the fit extension.
In addition, we removed from our model the four further away catalog
sources 2FGL\,J2022.8+3843c, 2FGL\,J2020.0+4159, 2FGL\,J2013.8+4115c,
and 2FGL\,J2012.4+3955c from the region because they were not significant
above 10 \gev.  These modifications are further described in the caption to
Figure~\ref{1FGL_J2020.0+4049}.


\hl{
This extended sources is coincident with 2CG\,078+01 and
3EG\,J2020+4017 detected by COS B and EGRET respectively
\citep{second_cos_b_catalog,third_egret_catalog}.  This source is
spatially coincident with the $\gamma$-Cygni SNR and has been speculated
to be be the interaction of accelerated particles in the SNR interacting
with dense molecular clouds \citep{pollock_1985,gaisser_1998}.  But this
association was disfavored when the LAT \gev emission from this source
was detected to be pulsed 
(PSR\,J2021+4026,\cite{first_lat_pulsar_cat}).

Milagro detected a $4.2\sigma$
excess at energies $\sim 30 \tev$ from the location of
this source in the previous LAT bright $\gamma$-ray source
list. \cite{lat_bsl,milagro_bright_source_list}.  Veritas also detected
an extended source VER\,J2019+407 coincident with the SNR above 200
\gev and suggested the \tev emission could be a shock-cloud interaction
in $\gamma$-Cygni \citep{veritas_gamma_cygni}.  Using just 10 \gev to 100
\gev photons to avoid PSR\,J2021+4026, we resolve an
extended source whose inferred \gev size well matches the radio size of
$\gamma$-Cygni, as can be seen in Figure~\ref{1FGL_J2020.0+4049}
which overlays contours of $\gamma$-Cygni at 408MHz from the Canadian
Galactic Plane Survey \citep{canadian_galactic_plane_survey}.
}

% Radio observations \dots
% X-ray observations \dots
% Age \ldots
% Distance: 1.5 kpc Landecker et al 1980 (from integral paper)
% Optical: (Mavromatakis 2003 (from integral paper)
% EGRET observations: 3EG J2020+4027
% LAT Pulsar detected
%   PSR J2021+4026
%   Prestended at scineghe2010.
%     http://scineghe2010.ts.infn.it/allegati/talks/ThursdaySept9/13_Razzano.pdf
%   Presented at Fermi Symposium 2011:
%     https://confluence.slac.stanford.edu/download/attachments/98831433/2011_3FermiSymp_CygPsr_ver2.pdf
%     "PSR J2021+4026 is a 265 ms pulsar found by Fermi LAT using"
%     "This discovery help to better clarify the nature of the
%      unidentified source 3EG J2020+4027, thought to be associated
%      with SNR 78.2+2.1, a.k.a. the $\gamm$-Cygni SNR.
%      Unfortunately, so far this pulsar lacks a radio counterpart, and
%      observations with Chandra and XMM show a potential X-ray
%      counterpart [8,9,15], but no pulsations have been detected so
%      far."

% Milagro analysis
% http://arxiv.org/abs/astro-ph/0611691
% Milagro $4.2\sigma$ excess.

% Veritas observations of $\gamma$-Cygni: 
%   http://arxiv.org/pdf/0912.4492v1
%   extended: VER J019+407
%     Veritas observations of $\gamm$-Cygni veritas_gamma_cygni,


\section{Discussion}

2FGL reported the detection of twelve sources
previously resolved by the LAT at \gev energies. Using
two years of LAT data and a new analysis method, we present on 
the detection of nine additional extended sources. We also reanalyze
the spatial shape of the extended sources in 2FGL.  Finding a coherent
source extension with different frequencies, especially at \tev energies,
allows us to firmly associate the LAT source in many cases.

The 21 extended LAT sources are located close to the Galactic
plane and plotted in Figure~\ref{allsky_extended_sources}.  Most of the
extended sources are indeed expected to be of Galactic origin as the
distance of extragalactic sources (with exception of the local group
Galaxies) is typically too large to resolve them in $\gamma$-rays.


Figure~\ref{compare_index_2FGL} shows a comparison of the spectral index
of LAT detected extended sources and of all sources in 2FGL. This, and
Table~\ref{known_extended_sources} and~\ref{new_ext_srcs_table},
show that the LAT observes a population of hard (index $<2$)
extended sources at energies above 10 \gev which connects to \tev sources.
Figure~\ref{hess_seds} shows the LAT energy spectrum of four of the
extended sources which were first discovered in the H.E.S.S. Galactic
plane survey \citep{hess_plane_survey} and have a matching size to LAT
extended sources.  The LAT energy spectrum connects smoothly to \tev.
This is also true of HESS\,J1825$-$137 \citep{fermi_hess_j1825} and 
RX\,J1713.7$-$3946 \citep{rx_j1713_lat}. 

For the LAT extended sources also seen in the \tev energy range,
Figure~\ref{gev_vs_tev_plot} shows a comparison of their sizes.  There is
a good correlation between the size of the sources in the \gev and \tev
energy ranges. Even
so, the size of PWNe are expected to vary across the \gev and \tev energy
range and the size of HESS\,J1825$-$137 was seen by H.E.S.S. to vary
between 200 \gev and 20 \tev \citep{energy_dependent_hess_j1825}.
HESS\,J1825$-$137 is significantly larger in \gev than \tev
\citep{fermi_hess_j1825}.  It is interesting to compare the \gev and
\tev size of other PWN candidates, but definitively measuring this 
would require a more in-depth analysis of the LAT data using the same 
elliptical surface brightness model.

Figure~\ref{gev_vs_tev_histogram} compares the size of the 21 extended
LAT sources to the 42 extended H.E.S.S. sources.  Because of the large
field of view and all-sky coverage, the LAT can more easily measure
larger sources including the LMC and SMC.  On the other hand, the smaller
angular resolution of Air Cherenkov detectors allows them to measure a
population of extended sources below our sensitivity limit (currently at
about $\sim0\fdg2$).  \fermi has a five year nominal mission lifetime with a
goal of ten years of operation.  As Figure~\ref{time_sensitivity} shows,
our sensitivity to these smaller sources will improve by a factor greater
than the square root of the exposure.  With increased exposure, the LAT will likely begin to
resolve these smaller \tev sources.


\bibliography{extended_catalog}

\clearpage
\begin{figure}
    \ifcolorfigure
    \plotone{ic443_plots/four_plots_ic443_color.eps}
    \else
    \plotone{ic443_plots/four_plots_ic443_bw.eps}
    \fi
    % taken from /u/gl/lande/work/fermi/extended_catalog/2FGL/plots_for_paper/four_plots_ic443/v1/run.py
    \caption{
    Counts and test statistic profiles for the SNR IC443. (a) \ts
    vs. extension of the source. (b) \tsext for individual energy
    bands. (c) observed radial profile of counts in comparison to the
    expected profiles for a spatially extended source (solid and colored
    red in the online version) and for a point source (dashed and colored
    blue in the online version).  (d) smoothed count map after subtraction
    of the diffuse emission compared to the LAT PSF (inset).  Plots (a),
    (c), and (d) use only 1 \gev to 100 \gev photons.  Plots (c) and (d) use
    only photons which converted in the front part of the tracker and
    have an improved angular resolution.
    }
    \label{four_plots_ic443}
\end{figure}

\clearpage
\begin{figure}
    \ifcolorfigure
      \plotone{mc_plots/compare_disk_gauss_color.eps}
    \else
      \plotone{mc_plots/compare_disk_gauss_bw.eps}
    \fi
    % this plot came from /u/gl/lande/work/fermi/extended_catalog/2FGL/plots_for_paper/compare_disk_gauss/v3/plot.py
    \caption{
    A comparison of a 2D Gaussian and disk-like surface brightness profiles
    of extended sources before and after convolving with the PSF for two
    energy ranges.  The solid black line is the PSF that would be observed
    for a power-law source of spectral index 2. The dashed line
    and the dash-dot line are 
    the brightness profile of a Gaussian with $\rsixeight=0\fdg5$
    and the convolution of this profile with the LAT PSF respectively
    (colored red in the online version)
    The dash-dot-dot and the dot-dot line are the brightness profile of a uniform disk
    with $\rsixeight=0\fdg5$ and the convolution 
    of this profile with the LAT PSF respectively (colored blue in the online version).
    }\label{compare_disk_gauss}
  \end{figure}


\clearpage
\begin{deluxetable}{rrrr}
\tabletypesize{\normalsize}
\tablecaption{Monte Carlo Validation Parameters
\label{ts_ext_num_sims}
}
\tablewidth{0pt}
\tablehead{

\colhead{Spectral Index}&
\colhead{Flux\tablenotemark{(a)}}&
\colhead{$N_\text{sims}$}&
\colhead{$\langle\ts\rangle$}\\
\colhead{}&
\colhead{(\phflux)}&
\colhead{}&
\colhead{}\\
}

\startdata
      1.5 &          $10^{-6}$ &           31952 &  92862 \\
          &  $3\times 10^{-7}$ &           31962 &  22169 \\
          &          $10^{-7}$ &           31977 &   5806 \\
          &  $3\times 10^{-8}$ &           31991 &   1270 \\
          &          $10^{-8}$ &           31940 &    301 \\
          &  $3\times 10^{-9}$ &           30324 &     62 \\
      \hline
        2 &          $10^{-6}$ &           31872 &  22067 \\
          &  $3\times 10^{-7}$ &           31890 &   4898 \\
          &          $10^{-7}$ &           31858 &   1097 \\
          &  $3\times 10^{-8}$ &           31632 &    236 \\
          &          $10^{-8}$ &           27491 &    103 \\
      \hline
      2.5 &          $10^{-6}$ &           31822 &   4706 \\
          &  $3\times 10^{-7}$ &           31822 &    889 \\
          &          $10^{-7}$ &           31169 &    176 \\
          &  $3\times 10^{-8}$ &           21591 &     41 \\
      \hline                                                
      3   &          $10^{-6}$ &           31763 &    929 \\
          &  $3\times 10^{-7}$ &           31665 &    161 \\
          &          $10^{-7}$ &           19271 &     40 \\
\enddata

\tablenotetext{(a)}{Integral 1 \gev to 100 \gev flux.}

\tablecomments{
    A list of th spectral models of simulated point sources which were
    tested for extension.  For each model, the number of statistically
    independent simulations and the average value of \ts is also
    tabulated.  The models spans representative spectral parameters.
}
\end{deluxetable}

\clearpage
\begin{figure}
    \ifcolorfigure
    \plotone{mc_plots/ts_ext_emin_1000_color.eps}
    \else
    \plotone{mc_plots/ts_ext_emin_1000_bw.eps}
    \fi
    % this plot came from /u/gl/lande/work/fermi/extended_catalog/monte_carlo/ts_ext/v4/plot.py
    \caption{
    Distribution of the test statistic \tsext of a likelihood ratio
    test when fitting the extension of a point source.  The four plots
    represent simulated point sources of different spectral indices and
    the different lines (colored in the online version) 
    represent point sources with different 100 \mev
    to 100 \gev integral fluxes.  The dashed line (colored red in online version)
    is the cumulative
    density function of Equation~\ref{ts_ext_distribution}.
    }\label{ts_ext_mc}
  \end{figure}

\clearpage

\begin{figure}
    \ifcolorfigure
    \plotone{mc_plots/index_sensitivity_color.eps}
    \else
    \plotone{mc_plots/index_sensitivity_bw.eps}
    \fi
    % this plot came from /u/gl/lande/work/fermi/extended_catalog/monte_carlo/sensitivity/v12/plot/plot_vs_index.py
    \caption{
    The detection threshold ($\langle\tsext\rangle=16$) to resolve a
    uniform disk extended source for a two-year exposure.  All sources
    have an assumed power-law spectrum and the different colors correspond
    to different simulated spectral indices.  The solid line is the
    detection threshold using photons with energy between 100 \mev and 100
    \gev while the dashed line is the threshold using 1 \gev to 100 \gev photons.
    }\label{index_sensitivity}
  \end{figure}

\clearpage
\begin{figure}
    \ifcolorfigure
    \plotone{mc_plots/all_sensitivity_color.eps}
    \else
    \plotone{mc_plots/all_sensitivity_bw.eps}
    \fi
    % this plot came from /u/gl/lande/work/fermi/extended_catalog/monte_carlo/sensitivity/v12/plot/plot_all.py
    \caption{the LAT detection threshold for four spectral indices and
    three background (1x, 10x, and 100x the Sreekumar-like isotropic
    spectrum) for a two-year exposure. The left plots are the detection
    threshold when using 1 \gev to 100 \gev photons and the right plots are
    the detection threshold when using 10 \gev 100 \gev photons.  The flux
    is integrated only in the selected energy range.  Overlaid on this
    plot are the LAT detected extended sources placed by the magnitude of
    the nearby Galactic diffuse emission and the energy range they were
    analyzed with.  The stars (colored red in the electronic version) are
    sources with a spectral index closer to 1.5, the triangles (colored
    blue) an index closer to 2, and the circles (colored green) an index
    closer to 2.5 The circle in plot (d) below the sensitivity line is
    MSH\,15$-$52 which was found by our analysis to have $\tsext=6.5$.
    }\label{all_sensitivity} 
  \end{figure}

  \clearpage
  \thispagestyle{empty}
\begin{deluxetable}{rrrrrrrrrrrrrrrrrrrrrrr}
\tablecolumns{22}
\tabletypesize{\scriptsize}
\rotate
\tablewidth{0pt}
\tablecaption{Extension Detection Threshold
\label{all_sensitivity_table}
}
\tablehead{
\colhead{$\gamma$}&       
\colhead{BG}&
\colhead{$0.1$}&
\colhead{$0.2$}&
\colhead{$0.3$}&
\colhead{$0.4$}&
\colhead{$0.5$}&
\colhead{$0.6$}&
\colhead{$0.7$}&
\colhead{$0.8$}&
\colhead{$0.9$}&
\colhead{$1.0$}&
\colhead{$1.1$}&
\colhead{$1.2$}&
\colhead{$1.3$}&
\colhead{$1.4$}&
\colhead{$1.5$}&
\colhead{$1.6$}&
\colhead{$1.7$}&
\colhead{$1.8$}&
\colhead{$1.9$}&
\colhead{$2.0$}
}
\startdata
\multicolumn{22}{c}{E$>$1 \gev} \\
\hline
     1.5 &       x1 &      148.1 &       23.3 &       11.3 &        8.0 &        7.2 &        6.9 &        6.7 &        6.8 &        7.1 &        7.4 &        7.6 &        7.9 &        8.1 &        8.5 &        9.2 &        9.9 &        9.1 &        9.2 &        9.0 &       10.3 \\
         &      x10 &      148.4 &       29.0 &       18.7 &       15.2 &       15.4 &       15.0 &       16.1 &       16.0 &       16.8 &       17.7 &       18.2 &       19.3 &       20.9 &       22.5 &       23.8 &       24.8 &       21.3 &       22.8 &       23.4 &       23.7 \\
         &     x100 &      186.8 &       55.0 &       43.4 &       40.7 &       41.0 &       41.8 &       40.9 &       40.9 &       42.7 &       43.6 &       38.4 &       39.9 &       40.6 &       38.4 &       36.9 &       36.3 &       37.1 &       38.8 &       37.2 &       37.6 \\
       2 &       x1 &      328.4 &       43.4 &       18.9 &       13.4 &       11.2 &       10.4 &       10.2 &       10.2 &       10.2 &       10.4 &       10.7 &       10.9 &       11.2 &       11.5 &       12.4 &       12.6 &       13.0 &       13.4 &       14.0 &       14.4 \\
         &      x10 &      341.0 &       55.9 &       32.3 &       27.6 &       26.5 &       25.4 &       25.6 &       25.9 &       27.4 &       26.8 &       27.8 &       28.7 &       29.8 &       30.1 &       31.0 &       31.5 &       31.7 &       34.0 &       34.3 &       35.9 \\
         &     x100 &      420.5 &      128.3 &       90.2 &       77.3 &       73.3 &       70.8 &       67.5 &       64.3 &       64.2 &       64.1 &       62.8 &       63.6 &       61.7 &       61.9 &       58.4 &       59.0 &       61.4 &       63.3 &       60.1 &       58.1 \\
     2.5 &       x1 &      627.1 &       75.6 &       29.8 &       19.3 &       15.5 &       13.5 &       12.8 &       12.6 &       12.5 &       12.5 &       12.6 &       12.9 &       12.9 &       13.1 &       13.5 &       13.7 &       14.3 &       14.8 &       15.2 &       15.8 \\
         &      x10 &      638.9 &       99.1 &       52.1 &       39.1 &       34.6 &       33.0 &       32.5 &       32.5 &       32.8 &       33.2 &       34.1 &       34.3 &       34.5 &       35.1 &       36.6 &       36.9 &       35.5 &       36.0 &       36.5 &       37.3 \\
         &     x100 &      795.0 &      262.1 &      140.9 &      104.3 &       90.4 &       81.2 &       77.2 &       75.1 &       69.7 &       70.9 &       66.5 &       65.6 &       64.9 &       64.0 &       58.9 &       58.1 &       60.2 &       58.4 &       57.5 &       55.8 \\
       3 &       x1 &      841.5 &      110.6 &       43.2 &       25.5 &       18.7 &       16.1 &       14.4 &       13.6 &       13.3 &       13.2 &       13.1 &       13.1 &       13.4 &       13.6 &       13.5 &       13.8 &       14.2 &       14.4 &       14.8 &       15.4 \\
         &      x10 &      921.6 &      151.3 &       69.1 &       47.8 &       40.7 &       37.1 &       35.5 &       34.5 &       35.1 &       35.5 &       35.3 &       35.3 &       35.4 &       35.5 &       36.8 &       37.6 &       35.3 &       35.4 &       36.3 &       36.6 \\
         &     x100 &     1124.1 &      282.9 &      181.1 &      119.8 &      100.7 &       91.1 &       84.3 &       77.9 &       73.3 &       71.8 &       67.6 &       66.4 &       65.5 &       63.9 &       59.0 &       58.6 &       58.8 &       57.5 &       55.4 &       54.4 \\
\cutinhead{E$>$10 \gev}
     1.5 &       x1 &       44.6 &        8.0 &        4.3 &        3.2 &        2.7 &        2.6 &        2.5 &        2.5 &        2.4 &        2.5 &        2.5 &        2.6 &        2.7 &        2.8 &        2.9 &        2.9 &        3.1 &        3.2 &        3.3 &        3.4 \\
         &      x10 &       45.2 &        9.2 &        5.8 &        5.0 &        4.9 &        4.9 &        5.0 &        5.2 &        5.3 &        5.7 &        5.9 &        6.3 &        6.6 &        6.5 &        6.8 &        7.6 &        7.8 &        8.2 &        8.5 &        8.7 \\
         &     x100 &       47.3 &       13.4 &       11.6 &       10.6 &       10.8 &       10.8 &       12.0 &       12.7 &       13.2 &       13.7 &       15.3 &       16.1 &       17.2 &       18.2 &       18.9 &       19.5 &       20.4 &       21.0 &       21.7 &       22.9 \\
       2 &       x1 &       49.7 &        8.4 &        4.4 &        3.3 &        2.8 &        2.6 &        2.6 &        2.6 &        2.6 &        2.6 &        2.7 &        2.7 &        2.8 &        2.9 &        3.0 &        3.2 &        3.2 &        3.4 &        3.5 &        3.5 \\
         &      x10 &       48.6 &        9.5 &        6.0 &        5.2 &        5.0 &        5.2 &        5.2 &        5.3 &        5.4 &        5.8 &        6.4 &        6.6 &        7.0 &        7.1 &        7.5 &        8.0 &        8.3 &        8.6 &        9.0 &        9.2 \\
         &     x100 &       51.8 &       14.7 &       11.8 &       11.5 &       11.5 &       11.9 &       13.2 &       14.0 &       14.3 &       15.3 &       16.2 &       16.9 &       18.4 &       19.2 &       19.8 &       21.0 &       22.0 &       22.8 &       23.2 &       24.3 \\
     2.5 &       x1 &       53.1 &        9.1 &        4.5 &        3.3 &        2.8 &        2.7 &        2.6 &        2.5 &        2.5 &        2.6 &        2.7 &        2.7 &        2.8 &        2.8 &        2.9 &        3.1 &        3.2 &        3.3 &        3.5 &        3.6 \\
         &      x10 &       53.7 &       10.5 &        6.3 &        5.4 &        5.1 &        5.1 &        5.3 &        5.4 &        5.7 &        6.0 &        6.3 &        6.6 &        6.8 &        6.9 &        7.5 &        8.1 &        8.3 &        8.6 &        8.9 &        9.2 \\
         &     x100 &       57.0 &       15.6 &       12.7 &       11.9 &       11.8 &       12.2 &       13.1 &       14.3 &       14.6 &       15.2 &       16.3 &       17.0 &       18.8 &       19.2 &       19.9 &       21.0 &       21.9 &       22.3 &       23.3 &       23.7 \\
       3 &       x1 &       55.5 &        9.4 &        4.8 &        3.4 &        2.9 &        2.7 &        2.6 &        2.5 &        2.5 &        2.5 &        2.6 &        2.7 &        2.7 &        2.8 &        2.9 &        3.0 &        3.1 &        3.2 &        3.4 &        3.4 \\
         &      x10 &       56.0 &       10.5 &        6.2 &        5.3 &        5.1 &        5.1 &        5.1 &        5.3 &        5.5 &        5.7 &        5.9 &        6.4 &        6.4 &        6.6 &        7.0 &        7.8 &        8.0 &        8.3 &        8.6 &        8.9 \\
         &     x100 &       60.3 &       16.2 &       12.7 &       11.7 &       11.8 &       12.2 &       12.6 &       13.8 &       14.2 &       14.6 &       15.8 &       16.5 &       17.6 &       18.5 &       19.4 &       19.8 &       20.7 &       21.0 &       21.8 &       22.5 \\
\enddata
\tablecomments{
      The detection threshold to resolve a uniform disk spatially extended
      sources using two years of data for sources of varying energy
      ranges, spectral indices, and background levels.  The extended
      sources were simulated against a Sreekumar-like isotropic spectrum
      and the second column is the factor that the simulated background
      was scaled by. The remaining columns are varying sizes of the source
      in degrees assuming a uniform surface brightness. The table quotes
      integral fluxes in the analyzed energy range (1 \gev to 100 \gev or 10 \gev 100
      \gev) in units of $10^{-10}\phflux$.
      }
\end{deluxetable}




\begin{figure}
    \ifcolorfigure
      \plotone{mc_plots/time_sensitivity_color.eps}
    \else
      \plotone{mc_plots/time_sensitivity_bw.eps}
    \fi
    % this plot came from /u/gl/lande/work/fermi/extended_catalog/monte_carlo/sensitivity/v12/plot/plot_vs_time.py
    \caption{
    \hl{
    The LAT's projected detection threshold to extension after 10 years for a power-law
    source of spectral index 2 against 10 times the isotropic background
    in the energy range from 1 \gev to 100 \gev (solid line colored red in the electronic
    version) and 10 \gev to 100 \gev
    (dashed line colored blue in the electronic version). The solid gray region is the 
    detection threshold assuming the sensitivity improves from 2 to 10
    years by the square root of the exposure (top edge) and linearly with
    exposure (bottom edge).  The lower plot shows the factor increase in
    sensitivity.  For small extended sources, our detection threshold
    to the extension of a source will improve by a factor better than the square root of 
    the exposure.
    }
    }\label{time_sensitivity}
  \end{figure}


\begin{figure}
  \ifcolorfigure
    \plotone{ic443_plots/res_tsmap_ic443_color.eps}
    \else
    \plotone{ic443_plots/res_tsmap_ic443_bw.eps}
    \fi
  % plot from /u/gl/lande/work/fermi/extended_catalog/2FGL/plots_for_paper/res_tsmap_ic443/v1

  \caption{
  A test statistic map generated for the region around the SNR 
  IC443 using 1 \gev to 100 \gev photons.  (a) test statistic map after
  subtracting IC443 modeled as a point source. (b) same as (a), but
  IC443 modeled as extended source. Crosses represent sources included
  in the model of the region.  Such maps were generated for all extended
  source candidates.}
  \label{res_tsmaps}
\end{figure}

\clearpage
\begin{figure}
    \ifcolorfigure
    \plotone{source_plots/example_bad_fit_color.eps}
    \else
    \plotone{source_plots/example_bad_fit_bw.eps}
    \fi
    % taken from /u/gl/lande/work/fermi/extended_catalog/2FGL/plots_for_paper/example_bad_fit/v3/run.py
    \caption{
    A diffuse-emission-subtracted 1 \gev to 100 \gev counts map of the region
    around 2FGL\,J1856.2+0450c smoothed by a 0\fdg1 2D Gaussian kernel. The
    plus and circle (colored red in the online version) represent the
    center and size of the best fit radially symmetric source with
    a uniform intensity profile.  The black crosses represent the
    position of background 2FGL sources.  The result is statistically
    significant, but the extension encompasses many catalog sources and
    the emission does not look to be uniform. Instead, this source is
    probably fitting large-scale diffuse residual features. Although
    the fit is statistically significant, it likely corresponds to
    residual features of inaccurately modeled diffuse emission picked
    up by the fit.  We manually discard candidates that appear like this.
    }
    \label{example_bad_fit}
\end{figure}


\begin{deluxetable}{lrrrrrrrr}
\tablecolumns{9}
\rotate
\tabletypesize{\footnotesize}
\tablewidth{0pt}
\tablecaption{Extension fit for the twelve extended sources included in 2FGL
\label{known_extended_sources}
}
\tablehead{
\colhead{Name}&
\colhead{\glon}&
\colhead{\glat}&
\colhead{$\sigma$}&
\colhead{\ts}&
\colhead{\tsext}&
\colhead{Pos Err}&
\colhead{Flux\tablenotemark{(a)}}&
\colhead{Index}\\
\colhead{}&
\colhead{(deg.)}&
\colhead{(deg.)}&
\colhead{(deg.)}&
\colhead{}&
\colhead{}&
\colhead{(deg.)}&
\colhead{(\phflux)}&
\colhead{}
}

\startdata
\multicolumn{9}{c}{E$>$1 \gev} \\
\hline
SMC                  &     302.68 &     -44.81 & $  1.75 \pm   0.07 \pm   0.02 $ &       94.8 &       67.4 &   0.12 & $    3.3 \pm     0.4$ & $   2.41 \pm    0.17$  \\
LMC                  &     279.10 &     -32.61 & $  1.74 \pm   0.05 \pm   0.13 $ &     1101.3 &      860.5 &   0.05 & $   15.5 \pm     0.6$ & $   2.48 \pm    0.06$  \\
IC443                &     189.05 &       3.04 & $  0.36 \pm   0.01 \pm   0.04 $ &    10719.8 &      510.4 &   0.01 & $   64.8 \pm     1.2$ & $   2.23 \pm    0.02$  \\
Vela X               &     263.34 &      -3.11 & $                         0.88$ &            &            &        &                       &                        \\
Centarus A           &     309.52 &      19.42 &                        $\sim10$ &            &            &        &                       &                        \\
W28                  &       6.50 &      -0.27 & $  0.43 \pm   0.02 \pm   0.03 $ &     1324.8 &      177.4 &   0.01 & $   58.0 \pm     1.8$ & $   2.63 \pm    0.03$  \\
W30                  &       8.61 &      -0.20 & $  0.36 \pm   0.02 \pm   0.02 $ &      465.4 &       73.3 &   0.02 & $   30.7 \pm     1.6$ & $   2.59 \pm    0.04$  \\
W44                  &      34.69 &      -0.38 & $  0.36 \pm   0.01 \pm   0.02 $ &     1903.3 &      217.7 &   0.01 & $   73.6 \pm     1.8$ & $   2.68 \pm    0.02$  \\
W51C                 &      49.13 &      -0.45 & $  0.28 \pm   0.02 \pm   0.05 $ &     1819.5 &      115.7 &   0.01 & $   39.3 \pm     1.3$ & $   2.35 \pm    0.03$  \\
Cygnus Loop          &      74.22 &      -8.46 & $  1.72 \pm   0.05 \pm   0.07 $ &      356.5 &      356.5 &   0.06 & $   11.1 \pm     0.7$ & $   2.53 \pm    0.11$  \\
\cutinhead{E$>$10 \gev}
MSH\,15$-$52         &     320.38 &      -1.22 & $  0.20 \pm   0.04 \pm   0.03 $ &       76.2 &        6.5 &   0.03 & $    0.6 \pm     0.7$ & $   2.27 \pm    0.73$  \\
HESS\,J1825$-$137    &      17.56 &      -0.46 & $  0.65 \pm   0.03 \pm   0.01 $ &       83.6 &       55.9 &   0.05 & $    1.8 \pm     0.2$ & $   1.74 \pm    0.19$  \\
\enddata

\tablenotetext{(a)}{
Integrated in the fit energy range (either 1 \gev 100 \gev or 10 \gev 100 \gev).
}

\tablecomments{
All sources were fit using a spatial model assuming a uniform radially
symmetric intensity distribution. \glon and \glat are Galactic longitude
and latitude of the best fit extended source respectively.  The first
error on $\sigma$ is statistical and the second is systematic (see
Section~\ref{systematic_errors_on_extension}).  Pos Err is the error on
the position of the source.  Vela X and the Centarus A Lobes were
not fit by our analysis but are include for completeness.
}
\end{deluxetable}


\clearpage
\begin{figure}
    \ifcolorfigure
      \plotone{source_plots/agn_color.eps}
    \else
      \plotone{source_plots/agn_bw.eps}
    \fi
    % this plot came from /u/gl/lande/work/fermi/extended_catalog/2FGL/agn/v6/agn.py
    \caption{The cumulative density of \tsext for 783 of the clean
    AGN in 2LAC which are significant above 1 \gev calculated with \pointlike (the dashed line
    colored blue in the electronic version)
    and with \gtlike (the solid line colored red in the electronic version).  AGNs are too far
    and too small to be resolved by the LAT. Therefore, the cumulative
    density of $\tsext$ is expected to follow a $\chi^2/2$ distribution
    (Equation~\ref{ts_ext_distribution}) and is the dash-dotted line (colored
    red in the electronic version).
    }\label{agn_ts_ext}
  \end{figure}


  % 1FGL J1628.6-2419c - P72Y2516         - 2FGL J1627.0-2425c
  % 1FGL J0823.3-4248  - P72Y1212         - 2FGL J0823.0-4246
  % 1FGL J1614.7-5138c - P72Y2473         - 2FGL J1615.2-5138
  % 1FGL J1632.9-4802c - P72Y2540         - 2FGL J1632.4-4753c
  % 1FGL J2020.0+4049  - P72Y3281         - 2FGL J2021.5+4026
  % 1FGL J1837.5-0659c - P72Y2974         - 2FGL J1837.3-0700c
  % N/A                - P72Y1287         - 2FGL J0851.7-4635


\clearpage
\thispagestyle{empty}
\begin{deluxetable}{lrrrrrrrrr}
\tablecolumns{10}
\tabletypesize{\small}
\rotate
\tablewidth{0pt}
\tablecaption{Extension fit for the nine new extended sources
\label{new_ext_srcs_table}
}
\tablehead{
\colhead{Name}&
\colhead{\glon}&
\colhead{\glat}&
\colhead{$\sigma$}&
\colhead{\ts}&
\colhead{\tsext}&
\colhead{Pos Err}&
\colhead{Flux\tablenotemark{(a)}}&
\colhead{Index}&
\colhead{Counterpart}\\
\colhead{}&
\colhead{(deg.)}&
\colhead{(deg.)}&
\colhead{(deg.)}&
\colhead{}&
\colhead{}&
\colhead{(deg.)}&
\colhead{(\phflux)}&
\colhead{}&
\colhead{}
}

\startdata
\multicolumn{10}{c}{E$>$1 \gev} \\
\hline
2FGL\,J0823.0$-$4246   &     260.32 &      -3.28 & $  0.37 \pm   0.03 \pm   0.02 $ &      320.9 &       46.3 &   0.02 & $    8.5 \pm     0.7$ & $   2.20 \pm    0.09$ &             Puppis A \\
2FGL\,J1627.0$-$2425c  &     353.08 &      16.78 & $  0.41 \pm   0.05 \pm   0.02 $ &      144.5 &       31.1 &   0.04 & $    6.5 \pm     0.6$ & $   2.49 \pm    0.14$ &            Ophiuchus \\
2FGL\,J1712.4$-$3941   &     347.25 &      -0.54 & $  0.56 \pm   0.04 \pm   0.01 $ &       75.0 &       39.6 &   0.05 & $    4.2 \pm     0.9$ & $   1.47 \pm    0.12$ &     RX\,J1713.7$-$3946 \\
\cutinhead{E$>$10 \gev}
2FGL\,J0851.7$-$4635   &     266.29 &      -1.43 & $  1.13 \pm   0.08 \pm   0.05 $ &      116.1 &       87.2 &   0.07 & $    1.3 \pm     0.2$ & $   1.76 \pm    0.21$ &             Vela Jr. \\
2FGL\,J1615.0$-$5051   &     332.38 &      -0.14 & $  0.33 \pm   0.04 \pm   0.01 $ &       53.4 &       16.3 &   0.04 & $    1.1 \pm     0.2$ & $   2.24 \pm    0.28$ &      HESS\,J1616$-$508 \\
2FGL\,J1615.2$-$5138   &     331.66 &      -0.66 & $  0.42 \pm   0.03 \pm   0.01 $ &       76.6 &       48.0 &   0.05 & $    1.2 \pm     0.2$ & $   1.77 \pm    0.24$ &      HESS\,J1614$-$518 \\
2FGL\,J1632.4$-$4753c  &     336.41 &       0.22 & $  0.44 \pm   0.04 \pm   0.03 $ &      127.8 &       64.5 &   0.04 & $    1.9 \pm     0.2$ & $   2.29 \pm    0.21$ &      HESS\,J1632$-$478 \\
2FGL\,J1837.3$-$0700c  &      25.08 &       0.13 & $  0.35 \pm   0.08 \pm   0.03 $ &       46.2 &       18.8 &   0.07 & $    1.0 \pm     0.2$ & $   1.63 \pm    0.29$ &      HESS\,J1837$-$069 \\
2FGL\,J2021.5+4026     &      78.18 &       2.19 & $  0.59 \pm   0.03 \pm   0.02 $ &      222.2 &      116.4 &   0.04 & $    1.8 \pm     0.2$ & $   2.31 \pm    0.19$ &       $\gamma$-Cygni \\
\enddata

\tablenotetext{(a)}{
Integrated in the fit energy range (either 1 \gev to 100 \gev or 10 \gev 100 \gev).
}

\tablecomments{
    The columns in this table
    are equivalent to the columns in Table~\ref{known_extended_sources}.
}
\end{deluxetable}



\clearpage
\thispagestyle{empty}
\begin{deluxetable}{lrrrrrrrrrr}
  \tablecolumns{11}
  \rotate
  \tablewidth{0pt}
  \tablecaption{Dual localization, alternative PSF, and alternative diffuse results
  \label{alt_diff_model_results}
  }
  \tablehead{
  \colhead{Name}&
  \colhead{$\ts_\pointlike$}&
  \colhead{$\ts_\gtlike$}&
  \colhead{$\ts_\altdiff$}&
  \colhead{$\tsext_\pointlike$}&
  \colhead{$\tsext_\gtlike$}&
  \colhead{$\tsext_\altdiff$}&
  \colhead{$\sigma$}&
  \colhead{$\sigma_\altdiff$}&
  \colhead{$\sigma_\altpsf$}&
  \colhead{$\tsinc$}\\
  \colhead{}&
  \colhead{}&
  \colhead{}&
  \colhead{}&
  \colhead{}&
  \colhead{}&
  \colhead{}&
  \colhead{(deg.)}&
  \colhead{(deg.)}&
  \colhead{(deg.)}&
  \colhead{}
  }
  \startdata
  \multicolumn{11}{c}{E$>$1 \gev} \\
  \hline
  2FGL\,J0823.0$-$4246   &                350.9 &                320.9 &                352.5 &                 66.0 &                 46.3 &                 53.6 &                 0.37 &                 0.39 &                 0.38 &                 22.1 \\
  2FGL\,J1627.0$-$2425c  &                170.2 &                144.5 &                112.6 &                 43.9 &                 31.1 &                 23.9 &                 0.41 &                 0.40 &                 0.39 &                 20.0 \\
  2FGL\,J1712.4$-$3941   &                 80.9 &                 75.0 &                 43.4 &                 47.4 &                 39.6 &                 22.2 &                 0.56 &                 0.56 &                 0.54 &                  6.4 \\
  \cutinhead{E$>$10 \gev}
  2FGL\,J0851.7$-$4635   &                116.7 &                116.1 &                122.3 &                 87.1 &                 87.2 &                 90.4 &                 1.13 &                 1.16 &                 1.17 &                 16.1 \\
  2FGL\,J1615.0$-$5051   &                 52.4 &                 53.4 &                 55.6 &                 17.5 &                 16.3 &                 17.4 &                 0.33 &                 0.32 &                 0.32 &                 11.9 \\
  2FGL\,J1615.2$-$5138   &                 76.3 &                 76.6 &                 86.3 &                 44.0 &                 48.0 &                 52.6 &                 0.42 &                 0.43 &                 0.43 &                 37.0 \\
  2FGL\,J1632.4$-$4753c  &                126.6 &                127.8 &                120.7 &                 63.9 &                 64.5 &                 64.1 &                 0.44 &                 0.44 &                 0.47 &                 40.6 \\
  2FGL\,J1837.3$-$0700c  &                 45.4 &                 46.2 &                 39.0 &                 18.5 &                 18.8 &                 16.6 &                 0.35 &                 0.34 &                 0.38 &                 12.6 \\
  2FGL\,J2021.5+4026     &                234.3 &                222.2 &                235.6 &                135.9 &                116.4 &                121.4 &                 0.59 &                 0.60 &                 0.60 &                 24.3 \\
  \enddata


\tablecomments{
$\ts_\pointlike$, $\ts_\gtlike$, and $\ts_\altdiff$ are the test
statistic values from \pointlike, \gtlike, and with the alternate
diffuse model respectively.  $\tsext_\pointlike$, $\tsext_\gtlike$,
and $\tsext_\altdiff$ are the test statistic values of the extension
test from \pointlike, \gtlike, and with the alternate diffuse
model respectively.  $\sigma$, $\sigma_\altdiff$, and $\sigma_\altpsf$
are the fit extension with the standard analysis,
alternate diffuse model, and alternate PSF respectively.  \tsinc is
the test statistic for the two point source test.
}
\end{deluxetable}


\begin{figure}
    \ifcolorfigure
      \plotone{source_plots/source_Puppis_A_color.eps}
    \else
      \plotone{source_plots/source_Puppis_A_bw.eps}
    \fi
  % this plot came from 
  % /u/gl/lande/work/fermi/extended_catalog/2FGL/plots_for_paper/source_plots/1FGL_J0823.3-4248/v7/run.py
  \caption{A diffuse-emission-subtracted 1 \gev to 100
  \gev counts map of the region around 2FGL\,J0823.0$-$4246 smoothed
  by a 0\fdg1 2D Gaussian kernel.  
  The triangle (colored red in the online version)
  is the 2FGL position
  of this source.  The cross and circle (colored red) are the best fit position
  and extension of this source assuming a radially symmetric uniform
  surface brightness.  The two stars (colored green) are the positions of the
  sources 2FGL\,J0823.4$-$4305 and 2FGL\,J0821.0$-$4254 which were removed
  from out background model
  because they are part of the extended source.  The lower right inset
  is the model predicted emission from a point source with the same
  spectrum as 2FGL\,J0823.4$-$4305 smoothed by the same kernel.
  This source is spatially coincident with the Puppis A SNR. The
  light blue contours correspond to the X-ray image of Puppis A observed by 
  \rosat
  \citep{rosat_puppis_a}. 
  }\label{1FGL_J0823.3-4248}
\end{figure}

\begin{figure}
    \ifcolorfigure
      \plotone{source_plots/source_Ophiuchus_color.eps}
    \else
      \plotone{source_plots/source_Ophiuchus_bw.eps}
    \fi
  % this plot came from 
  % /u/gl/lande/work/fermi/extended_catalog/2FGL/plots_for_paper/source_plots/1FGL_J1628.6-2419c/v7/run.py
  \caption{
    \hl{
  A diffuse-emission-subtracted 1 \gev to 100 \gev counts map of (a) the region
  around 2FGL\,J1627.0$-$2425 smoothed by a 0\fdg1 2D Gaussian kernel (b)
  also the emission from the background source 2FGL\,J1625.7$-$2526
  subtracted.  The red star is the 2FGL position of this source.
  The red cross and circle are the best position and extension of this
  source assuming a radially symmetric uniform surface brightness. The
  contours in (a) correspond to the 100 micrometer image observed by
  IRAS \citep{iras_rho_ophiuci}.  The contours in (b) correspond to
  ${}^{12}\text{CO}$ ($J=1\rightarrow 0$) emission integrated from -8 \km/\s
  to 20 \km/\s.  They are from \cite{co_rho_ophiuci}, were cleaned using
  the moment-masking technique \citep{masking_moment_2011}, and have
  been smoothed by a 0\fdg25 2D Gaussian kernel.
    }
  }\label{1FGL_J1628.6-2419c}
\end{figure}

\begin{figure}
    \ifcolorfigure
      \plotone{source_plots/source_RX_J1713.7-3946_color.eps}
    \else
      \plotone{source_plots/source_RX_J1713.7-3946_bw.eps}
    \fi
    % this plot came from
  % /u/gl/lande/work/fermi/extended_catalog/2FGL/plots_for_paper/source_plots/RX_J1713.7-3946/v6/run.py
  \caption{A diffuse-emission-subtracted 1 \gev to 100 \gev
  counts map of the region around 2FGL\,J1712.4$-$3941 (a) smoothed by
  a 0\fdg25 2D Gaussian kernel and (b) with the emission 
  from the background sources subtracted.
  This source is spatially
  coincident with RX\,J1713.7$-$3946 was was recently reported by
  \citep{rx_j1713_lat}.  The light blue contours correspond to the \tev image
  observed by H.E.S.S. \citep{rx_j1713_hess}.  The region was analyzed
  with the same background model as \citep{rx_j1713_lat}.  Source A (blue
  cross) is spatially coincident with 2FGL\,J1715.4$-$4024c (blue star).
  The gray crosses represent from left to right the position of source B
  and C which were added to the background model. 
  }\label{2FGL_J1712.4-3941}
\end{figure}



\begin{figure}
    \ifcolorfigure
      \plotone{source_plots/source_Vela_Jr_color.eps}
    \else
      \plotone{source_plots/source_Vela_Jr_bw.eps}
    \fi
  % this plot came from 
  % /u/gl/lande/work/fermi/extended_catalog/2FGL/plots_for_paper/source_plots/Vela_Jr/v7/run.py
  \caption{A diffuse-emission-subtracted 10 \gev 100
  \gev counts map of the region around 2FGL\,J0851.7$-$4635 smoothed
  by a 0\fdg25 2D Gaussian kernel. The red star is the 2FGL position of this source.  The red cross
  and circle are the best fit position and extension of this source
  assuming a radially
  symmetric uniform surface brightness.  The three green stars
  inside the extension circle 
  are (from left to right) 2FGL\,J0853.5$-$4711, 2FGL\,J0855.4$-$4625,
  and 2FGL\,J0848.5$-$4535 and were removed from our background model.
  The farther away green stars are (from left to
  right) 2FGL\,J0901.7$-$4655 and 2FGL\,J0858.0$-$4815 which were removed
  from our model
  because they were not significant above 10 \gev.  The blue cross
  is the relocalized position of 2FGL\,J0854.7$-$4501.  This extended
  source is spatially coincident with the Vela Jr SNR.  The light
  blue contours correspond to the \tev image of Vela Jr. observed by H.E.S.S
  \citep{vela_jr_hess}.
  }\label{Vela_Jr}
\end{figure}

\begin{figure}
    \ifcolorfigure
      \plotone{source_plots/source_HESS_J1614-518_color.eps}
    \else
      \plotone{source_plots/source_HESS_J1614-518_bw.eps}
    \fi
  % this plot came from 
  % /u/gl/lande/work/fermi/extended_catalog/2FGL/plots_for_paper/source_plots/1FGL_J1613.6-5100c/v7/run.py
  \caption{
    A diffuse-emission-subtracted 10 \gev to 100
    \gev counts map of the region around 2FGL\,J1615.0$-$5051 (upper
    left) and 2FGL\,J1615.2$-$5138 (lower right) smoothed by a 0\fdg1
    2D Gaussian kernel.  The red stars are the catalog positions of these
    sources.  The red stars and circles are the best fit positions and
    extensions of these sources 
    assuming a radially
    symmetric uniform surface brightness.
    The green star inside the extension of 2FGL\,J1615.2$-$5138  is
    2FGL\,J1614.9$-$5212 which was removed from our background model
    The green stars to the left are 2FGL\,J1619.7$-$5040c
    and 2FGL\,J1620.6$-$5111c which were removed from our model because they were
    not significant above 10\gev. These are shown as green stars.
    The light blue
    contours correspond to the \tev image 
    of the extended sources
    HESS\,J1616$-$508 (left) and the extended source HESS\,J1614$-$518
    (right)
    observed by H.E.S.S
    \citep{hess_plane_survey}
    2FGL\,J1615.0$-$5051 is spatially consistent with HESS\,J1616$-$508 and
    2FGL\,J1615.2$-$5138 is spatially consistent with HESS\,J1614$-$518.
  }\label{1FGL_J1613.6-5100c}
\end{figure}

\begin{figure}
    \ifcolorfigure
      \plotone{source_plots/source_HESS_J1632-478_color.eps}
    \else
      \plotone{source_plots/source_HESS_J1632-478_bw.eps}
    \fi
  % this plot came from 
  % /u/gl/lande/work/fermi/extended_catalog/2FGL/plots_for_paper/source_plots/1FGL_J1632.9-4802c/v7/run.py
  \caption{A diffuse-emission-subtracted 10 \gev to 100
  \gev counts map of the region around 2FGL\,J1632.4$-$4753c smoothed by
  a 0\fdg1 2D Gaussian kernel.  This source is in a crowded region.
  The red star is the 2FGL position of this source.  The red
  cross and circle are the best fit position and extension 2FGL\,J1632.4$-$4753c 
  assuming a radially
  symmetric uniform surface brightness.
  The three
  green crosses inside the extension are 2FGL\,J1631.7$-$4720c, 
  2FGL\,J1630.2$-$4752, and 2FGL\,J1634.4$-$4743c.4-4820c
  which were removed from our background model.
  The blue stars and crosses are the catalog
  positions and the relocalized positions of (from left to right)
  2FGL\,J1635.4$-$4717c and 2FGL\,J1636.3$-$4740c.  The farther away green
  stars are other catalog sources which were removed from our model because they are
  not significant above 10 \gev.  This extended source is spatially
  coincident with the extended H.E.S.S source HESS\,J1632-478.
  The light blue contours correspond to the \tev image observed by H.E.S.S.
  \citep{hess_plane_survey}.
  }\label{1FGL_J1632.9-4802c}
\end{figure}


\begin{figure}
    \ifcolorfigure
      \plotone{source_plots/source_HESS_J1837-069_color.eps}
    \else
      \plotone{source_plots/source_HESS_J1837-069_bw.eps}
    \fi
  % this plot came from 
  % /u/gl/lande/work/fermi/extended_catalog/2FGL/plots_for_paper/source_plots/1FGL_J1837.5-0659c/v7/run.py
  \caption{
  A diffuse-emission-subtracted 10 \gev to 100 \gev counts map of
  the region around 2FGL\,J1837.3$-$0700c (a) smoothed by a 0\fdg1 2D Gaussian
  kernel and (b) with the emission from the background 
  sources subtracted.  The red
  star is the 2FGL position of this source. The red cross and circle
  are the best fit position and extension  of 2FGL\,J1837.3$-$0700c 
  assuming a radially
  symmetric uniform surface brightness. The blue stars and crosses are the
  catalog position and the relocalized position of (from left to right)
  2FGL\,J1839.3$-$0558c, 2FGL\,J1836.8$-$0623c, and 2FGL\,J1834.7$-$0705c.
  The green cross inside the extension is 2FGL\,J1835.5$-$0649 which
  was removed from our background model. The farther
  away green star is 2FGL J1839.0$-$0539 which was removed from
  our model because
  it is not significant above 10 \gev.  This source is spatially
  coincident with the \tev source HESS\,J1837$-$069.  The light blue
  contours correspond to the \tev image observed by H.E.S.S.
  \citep{hess_plane_survey}.}\label{1FGL_J1837.5-0659c}
\end{figure}


\begin{figure}
    \ifcolorfigure
      \plotone{source_plots/source_Gamma_Cygni_color.eps}
    \else
      \plotone{source_plots/source_Gamma_Cygni_bw.eps}
    \fi
  % this plot came from 
  % /u/gl/lande/work/fermi/extended_catalog/2FGL/plots_for_paper/source_plots/Gamma_Cygni/v7/run.py
  \caption{A Diffuse subtracted 10 \gev to 100 \gev counts map of the region
  around 2FGL\,J2021.5+4026 smoothed by a 0\fdg1 2D Gaussian kernel. The
  triangle (colored red in the online version)
  represents the 2FGL position of 2FGL J2021.5+4026.  The plus
  and circle (colored red) 
  represent the best fit position and extension 2FGL\,J2021.5+4026
  assuming a radially symmetric uniform surface brightness.  The 
  star (colored green) 
  inside the extension represents the position of 
  2FGL\,J2019.1+4040 which was removed
from our background model. The diamond
  (colored purple) is the
  position a source not in the two-year catalog which was added to the
  region. 2FGL\,J2021.5+4026 is spatially coincident with the $\gamma$
  Cygni SNR.  The contours (colored light blue)
  correspond to the 408MHz image of
  $\gamma$ Cygni observed by the Canadian Galactic Plane Survey.
  }\label{1FGL_J2020.0+4049}
\end{figure}

\clearpage
\begin{figure}
    \ifcolorfigure
      \plotone{summary_plots/snr_seds_color.eps}
    \else
      \plotone{summary_plots/snr_seds_bw.eps}
    \fi
    % this plot came from /u/gl/lande/work/fermi/extended_catalog/2FGL/seds/v3/plot_seds/plot_snr_seds.py
    \caption{
    \hl{
    The spectral energy distribution of the extended sources 
    Puppis A (2FGL\,J0823.0$-$4246) and $\gamma$-Cygni 
    (2FGL\,J2021.5+4026).
    The lines (colored red in the online version)
    are the best fit power-law spectral model of
    these sources.
    The LAT spectral errors are statistical only.}
    }
    \label{snr_seds}
  \end{figure}

\clearpage
\begin{figure}
    \ifcolorfigure
      \plotone{summary_plots/hess_seds_color.eps}
    \else
      \plotone{summary_plots/hess_seds_bw.eps}
    \fi
    % this plot came from /u/gl/lande/work/fermi/extended_catalog/2FGL/seds/v3/plot_seds/plot_hess_seds.py
    \caption{
    The spectral energy distribution of the four extended sources
    associated with extended \tev sources.  The black points with
    circular markers are the LAT data points. The points with dashed
    markers (colored red in the electronic version) are the H.E.S.S
    points of the associated sources.  Plot (a) shows the LAT SED of
    2FGL\,J1615.0$-$5051 together with the H.E.S.S. SED of HESS\,J1616$-$508.
    Plot (b) shows 2FGL\,J1615.2$-$5138 and HESS\,J1614$-$518. Plot (c)
    shows 2FGL\,J1632.4$-$4753c and HESS\,J1632$-$478. Plot (d) shows
    2FGL\,J1837.3$-$0700c and HESS\,J1837$-$069. The H.E.S.S. data points
    are from \citep{hess_plane_survey} Both LAT and H.E.S.S. spectral
    errors are statistical only.}
    \label{hess_seds}
  \end{figure}

\clearpage
  \begin{figure}
      \ifcolorfigure
      \plotone{summary_plots/allsky_extended_sources_color.eps}
      \else
      \plotone{summary_plots/allsky_extended_sources_bw.eps}
      \fi
      % this plot came from /u/gl/lande/work/fermi/extended_catalog/2FGL/plots_for_paper/allsky/v2/run.py
      \caption{A plot of all spatially extended sources detected by the LAT
      in the \gev energy range
      with two years of data.  The twelve extended sources included in
      2FGL are indicated by the circular markers (colored red in the online
      version).  The nine new extended sources analyzed in this paper are
      the triangular markers (colored orange in the online version).}
\label{allsky_extended_sources}
  \end{figure}


\clearpage
\begin{figure}
    \ifcolorfigure
      \plotone{summary_plots/gev_vs_tev_plot_color.eps}
    \else
      \plotone{summary_plots/gev_vs_tev_plot_bw.eps}
      \fi
    % this plot came from /u/gl/lande/work/fermi/extended_catalog/2FGL/plots_for_paper/compare_tev_gev_sizes/v3/run.py
    \caption{
    A comparison of the \gev and \tev sizes of LAT extended
    sources detected by \tev instruments.  The \tev extensions
    of W30, 2FGL\,J1837.3$-$0700c, 2FGL\,J1632.4$-$4753c,
    2FGL\,J1615.0$-$5051, and 2FGL\,J1615.2$-$5138 \citep{hess_plane_survey}.
    The \tev extensions of MSH\,15$-$52, HESS\,J1825$-$137, Vela X,
    Vela Jr., RX\,J1713.7$-$3946 and W28 are from 
    \citep{msh_15_52_hess,hess_j1825_hess,vela_x_hess,vela_jr_hess,rx_j1713_hess,w28_hess}.
    The \tev extension of IC443 is from \citep{ic443_veritas}
    and W51c is from \citep{w51c_with_magic_at_fermi_symposium}.
    The LAT extension of Vela X is from \citep{velax}.  Except for 
RX\,J1713.7$-$3946 Vela Jr., the \tev sources were fit assuming a 2D Gaussian
    surface brightness so the plotted \gev and \tev extensions were first
    converted to \rsixeight (see Section~\ref{compare_source_size}).
    Because of their large size, the shape of RX\,J1713.7$-$3946 and Vela Jr.
    were not directly fit in \tev. Here, we take their
    sizes to be the same as the LAT size and 
    do not converted it to \rsixeight. The LAT extension errors are
    the statistical and systematic errors added in quadrature. The \tev size of MSH\,15$-$52, HESS\,J1614$-$518,
    HESS\,J1632$-$478, and HESS\,J1837$-$069 have only been reported with
    an elliptical 2D Gaussian fit and so the plotted size is the average
    of semi-major and semi-minor axis.
    }\label{gev_vs_tev_plot}
  \end{figure}

\clearpage
\begin{figure}
    \ifcolorfigure
      \plotone{summary_plots/gev_vs_tev_histogram_color.eps}
    \else
      \plotone{summary_plots/gev_vs_tev_histogram_bw.eps}
    \fi
    % this plot came from /u/gl/lande/work/fermi/extended_catalog/2FGL/plots_for_paper/extension_histogram/v4/extension_histogram.py
    \caption{
    The distribution of the \gev size of the 22 extended LAT 
    sources (colored blue in the electronic version) and
    the \tev size of the 42 extended H.E.S.S. sources
    (colored red in the electronic version).
    The extension of the LAT sources comes from
    Table~\ref{known_extended_sources}, Table~\ref{new_ext_srcs_table},
    and from \cite{velax}. 
    The \tev extension of
    the extended H.E.S.S. sources comes from the H.E.S.S. Source
    Catalog \citep{hesscat}.
%   The \tev extension of
%   the 42 extended H.E.S.S. sources comes from the H.E.S.S. Source
%   Catalog\footnote{
%   The H.E.S.S. source catalog can be found at \url{http://www.mpi-hd.mpg.de/hfm/HESS/pages/home/sources/}.}
    Except for RX\,J1713.7$-$3946 and Vela Jr.,
    the H.E.S.S. sources were fit with a 2D Gaussian surface
    brightness model and
    so the LAT and H.E.S.S. sizes are first converted to \rsixeight.
    (see Section~\ref{compare_source_size}). Because the spatial
    morphology of RX\,J1713.7$-$3946 and Vela Jr. is poorly matched by a
    2D Gaussian surface brightness model, the \gev and
    \tev extensions are included assuming a uniform surface brightness.
    }\label{gev_vs_tev_histogram}
  \end{figure}

\clearpage
\begin{figure}
    \ifcolorfigure
      \plotone{summary_plots/compare_index_2FGL_color.eps}
    \else
      \plotone{summary_plots/compare_index_2FGL_bw.eps}
    \fi
    % plot taken from /u/gl/lande/work/fermi/extended_catalog/2FGL/plots_for_paper/compare_index_2fgl/v3/run.py
    \caption{
    The distribution of spectral indices of all 1873 2FGL sources 
    (colored red in the electronic version)
    and the spatially extended sources (colored blue in the electronic
    version). The
    spectral indices of LAT extended sources are taken from
    Table~\ref{known_extended_sources} and \ref{new_ext_srcs_table}.
    The index of Centarus A is taken to be 2.58 from 
    \cite{second_cat} and the index of Vela X is taken to be 2.41
    from \cite{velax}. }\label{compare_index_2FGL}
  \end{figure}



\end{document}
