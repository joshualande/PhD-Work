\documentclass[12pt,preprint]{aastex}

% New text (from v3.0 to v4.0):
%  * Improve MC null distribution tests by using 8 bins per decade.
%  * New study of null hypothesis distribution fitting only above E>10GeV.
%  * Regenerate source confusion plot to show TS_ext-TS2pts
%  * Remove figure 8 from draft v3.0  (the one describing the spectral index difference of the two point 
%    source fit of extended sources) since everybody asked me to.
%  * Rerun of agn analysis to fix sources that failed to converge + using 8bins per decade + using 2FGL skymodel.
%  * For new source anlaysis:
%  ** use 8 insetad of 4 bins per decade
%  ** Freeze instead of delete insignificant 2FGL sources. This should make the text clearer because sources are only
%     deleted if they are part of the extended emission.
%  ** Add two nearby point-like sources to analysis of HESS J1632-478.
%  ** Redo analysis of RX J1713 to be E>10GeV to avoid issues of diffuse uncertainty at low energy.
%  * Better clarification about previous gtlike validation of extended source analysis + Citation of Francesco's work.
%  * Removed from the GeV/TeV extension size plots RX J1713+Vela Jr becuase they were not directly fit at TeV energies.

\bibliographystyle{apj}

\newif\ifcolorfigure
\colorfiguretrue
%\colorfigurefalse



%#For adding line numbers:
\usepackage{lineno}
\linenumbers


%\usepackage{rotating}
\usepackage{amsmath}
\usepackage{graphicx}
\usepackage{xspace}
\usepackage{url}

% Note, hyperref has to come after other packages!
\usepackage{hyperref}

% some units
\newcommand{\kev}{\text{Kev}\xspace}
\newcommand{\mev}{\text{MeV}\xspace}
\newcommand{\gev}{\text{GeV}\xspace}
\newcommand{\tev}{\text{TeV}\xspace}
\newcommand{\s}{\text{s}\xspace}
\newcommand{\ph}{\text{ph}\xspace}
\newcommand{\cm}{\text{cm}\xspace}
\newcommand{\km}{\text{km}\xspace}
\newcommand{\phflux}{\ensuremath{\text{ph}\,\text{cm}^{-2}\,\text{s}^{-1}}\xspace}
\newcommand{\tsext}{{\ensuremath{\text{TS}_{\text{ext}}}}\xspace}
\newcommand{\tsinc}{\ensuremath{\text{TS}_{\text{2pts}}}\xspace}

\newcommand{\likelihood}{\ensuremath{\mathcal{L}}\xspace}

\newcommand{\chandra}{\text{{\em Chandra}}\xspace}
\newcommand{\swiftxrt}{\text{{\em Swift}/XRT}\xspace}
\newcommand{\rosat}{\text{{\em ROSAT}}\xspace}
\newcommand{\suzaku}{\text{{\em Suzaku}}\xspace}
\newcommand{\asca}{\text{{\em ASCA}}\xspace}
\newcommand{\einstein}{\text{{\em Einstein}}\xspace}
\newcommand{\xmmnewton}{\text{{\em XMM-Newton}}\xspace}
\newcommand{\fermi}{\textit{Fermi}\xspace}


\newcommand{\rsixeight}{{\ensuremath{\text{r}_{68}}}\xspace}

\newcommand{\tsextpointlike}{\ensuremath{\tsext_{,\pointlike}}\xspace}
\newcommand{\tsextgtlike}{\ensuremath{\tsext_{,\gtlike}}\xspace}

\newcommand{\ts}{\text{TS}\xspace}
\newcommand{\glon}{\text{GLON}\xspace}
\newcommand{\glat}{\text{GLAT}\xspace}
\newcommand{\altdiff}{\text{alt,diff}\xspace}
\newcommand{\altpsf}{\text{alt,psf}\xspace}
\newcommand{\sys}{\text{sys}\xspace}
\newcommand{\stat}{\text{stat}\xspace}

% the program names
\newcommand{\gtlike}{\ensuremath{\mathtt{gtlike}}\xspace}
\newcommand{\pointlike}{\ensuremath{\mathtt{pointlike}}\xspace}
\newcommand{\gtobssim}{\ensuremath{\mathtt{gtobssim}}\xspace}
\newcommand{\minuit}{\ensuremath{\mathtt{minuit}}\xspace}

\newcommand{\degree}{^\circ\xspace}

%\newcommand{\newtext}[1]{{\textbf #1}}
\newcommand{\newtext}[1]{{\bfseries \color{red}#1}}

\usepackage{color}

\begin{document}

\title{Search for Spatially Extended \fermi-LAT Sources Using Two Years of Data}
\shorttitle{Search for Extended LAT Sources}

\keywords{
Catalogs;
Fermi Gamma-ray Space Telescope; 
Gamma rays: observations; 
ISM: supernova remnants;
Methods: statistical;
pulsar wind nebula
}

\author{
J.~Lande\altaffilmark{1,2}, 
M.~Ackermann\altaffilmark{3,4}, 
A.~Allafort\altaffilmark{1}, 
K.~Bechtol\altaffilmark{1}, 
T.~H.~Burnett\altaffilmark{5}, 
A.~Drlica-Wagner\altaffilmark{1}, 
S.~Funk\altaffilmark{1,6}, 
F.~Giordano\altaffilmark{7,8}, 
M.-H.~Grondin\altaffilmark{9,10}, 
M.~Kerr\altaffilmark{1}, 
M.~Lemoine-Goumard\altaffilmark{11,12}
}
\altaffiltext{1}{W. W. Hansen Experimental Physics Laboratory, Kavli Institute for Particle Astrophysics and Cosmology, Department of Physics and SLAC National Accelerator Laboratory, Stanford University, Stanford, CA 94305, USA}
\altaffiltext{2}{email: joshualande@gmail.com}
\altaffiltext{3}{Deutsches Elektronen Synchrotron DESY, D-15738 Zeuthen, Germany}
\altaffiltext{4}{email: markus.ackermann@desy.de}
\altaffiltext{5}{Department of Physics, University of Washington, Seattle, WA 98195-1560, USA}
\altaffiltext{6}{email: funk@slac.stanford.edu}
\altaffiltext{7}{Dipartimento di Fisica ``M. Merlin" dell'Universit\`a e del Politecnico di Bari, I-70126 Bari, Italy}
\altaffiltext{8}{Istituto Nazionale di Fisica Nucleare, Sezione di Bari, 70126 Bari, Italy}
\altaffiltext{9}{Max-Planck-Institut f\"ur Kernphysik, D-69029 Heidelberg, Germany}
\altaffiltext{10}{Landessternwarte, Universit\"at Heidelberg, K\"onigstuhl, D 69117 Heidelberg, Germany}
\altaffiltext{11}{Universit\'e Bordeaux 1, CNRS/IN2p3, Centre d'\'Etudes Nucl\'eaires de Bordeaux Gradignan, 33175 Gradignan, France}
\altaffiltext{12}{Funded by contract ERC-StG-259391 from the European Community}

\begin{abstract}
Spatial extension is an important characteristic for correctly
associating LAT sources with their counterparts at other wavelengths
and for obtaining an unbiased model of their spectra.  We present a new
method for quantifying the spatial extension of sources detected by the Large
Area Telescope (LAT), the primary science instrument on the {\em \fermi
Gamma-ray Space Telescope} (\fermi).  We perform a series of Monte Carlo
simulations to validate this tool and calculate the LAT threshold for
detecting the spatial extension of sources.  We then test all sources
in the second \fermi-LAT catalog (2FGL) for extension. 
We report the
detection of seven spatially extended sources in addition to the twelve
spatially extended sources reported in 2FGL and two others studied
in dedicated publications.
\end{abstract}

\section{Introduction}

A number of astrophysical source classes including supernova remnants
(SNRs), pulsar wind nebulae (PWNe), molecular clouds, normal galaxies,
and galaxy clusters are expected to be spatially resolvable at
\gev energies.  Dark matter satellites are also hypothesized to
be spatially extended.  The Large Area Telescope
(LAT) on the {\em \fermi Gamma-ray Space Telescope} (\fermi) has
detected seven SNRs which are significantly extended
at \gev energies: 
W51C, W30, 
IC443, W28, W44, RX\,J1713.7$-$3946,
and the Cygnus Loop
\citep{w51c,w30_lat,ic443,w28,w44,rx_j1713_lat,cygnus_loop_lat}. In addition, 
three extended PWN have been detected: MSH\,15$-$52, Vela~X, and
HESS\,J1825$-$137 \citep{msh1552,velax,fermi_hess_j1825}. Two
nearby galaxies, the Large and Small Magellanic Clouds, and one
radio galaxy, Centaurus A, were spatially resolved at \gev energies
\citep{lmc,smc,cen_a_lat}.  A number of additional sources detected
at \gev energies are positionally coincident with sources that exhibit
extension at other wavelengths, large enough to be spatially
resolvable by the LAT at \gev energies.

The current generation of air Cherenkov detectors have
made it apparent that many sources can be spatially resolved
at even higher energies.
Most
prominent was a survey of the Galactic plane using the High Energy
Stereoscopic System (H.E.S.S) which reported 14 spatially extended
sources with extensions varying from $\sim0\fdg1$ to $\sim0\fdg25$
\citep{hess_plane_survey}.  In fact, within our Galaxy only very few
sources detected at \tev energies, most notably the $\gamma$-ray binaries
LS\,5039 \citep{HESSLS5039}, LS I+61$-$303 \citep{MAGICLSI, VERITASLSI},
HESS\,J0632+057 \citep{HESS0632}, and the Crab nebula \citep{crab_weekes},
have no detectable extension.  High-energy $\gamma$-rays from
these sources are produced by the decay of $\pi^0$s produced by hadronic
interactions with interstellar matter and by relativistic electrons
due to Inverse Compton (IC) scattering and Bremsstrahlung radiation.  
It is plausible that the \gev and
\tev emission from these sources originates from the same population of
high-energy particles and so at least some of these sources should be
detected at \gev energies.  Studying 
these \tev sources at \gev energies would help to
determine the emission mechanisms producing these high energy photons.

The LAT is a pair conversion telescope that has been surveying the
$\gamma$-ray sky since 2008 August.  The LAT has broad energy coverage
(20 \mev to $>300$ \gev), wide field of view ($\sim 2.4$ sr), and large
effective area ($\sim 8000\ \cm^2$ at $>1$ \gev) 
Additional information about the performance of the LAT can be found in
\cite{atwood_LAT_mission}.
Using 2 years of all-sky surveying data, the LAT Collaboration published
a catalog of 1873
sources significantly detected at \gev energies called 2FGL \citep{second_cat}.
The possible counterparts of many of these sources can be spatially resolved
when observed at other frequencies but detecting the spatial extension
of these sources at \gev energies is difficult because the size of the
point-spread function (PSF) of the LAT is comparable to the typical size
of many of these sources.

The capability to spatially resolve \gev $\gamma$-ray
sources is important for several reasons.  
Finding a coherent source extension across different energy bands can
help to associate a LAT source to an otherwise confused counterpart.
Furthermore, some of the dark matter substructure in our Galaxy 
is predicted to be spatially extended at \gev energies \citep{pre_luanch_dark_matter_fermi}.  
Characterization of spatial extension could help to identify this substructure.
Also,
due to the strong energy dependence of the LAT PSF, the spatial and
spectral characterization of a source cannot be
decoupled. An inaccurate
spatial model will bias the spectral model of the source and vice versa. Specifically,
modeling a spatially extended source as point-like will systematically
shift a spectral analysis to softer indices. Furthermore, correctly
modeling source extension is important for 
understanding an entire region of the sky. For example,
an imperfect model of the spatially extended LMC introduced
significant residuals in the surrounding region \citep{first_cat,second_cat}.
Such residuals can bias the significance and measured spectrum of
neighboring sources in the densely populated Galactic plane.

Previous analysis of extended LAT sources was performed as dedicated
studies of individual sources so we expect that a systematic scan of
all LAT detected sources could uncover additional spatially extended
sources.  For these reasons, in Section~\ref{analysis_methods_section}
we present a new systematic method for analyzing spatially extended
LAT sources.  In Section~\ref{monte_carlo_validation}, we demonstrate
that this method can be used to test the statistical significance of the
extension of a LAT source and in Section~\ref{extension_sensitivity}
we calculate the LAT detection threshold to resolve the extension
of a source.  In Section~\ref{dual_localization_method}, we
study the ability of the LAT to distinguish between a spatially
extended source and the confusion of nearby point-like sources.
In Section~\ref{test_2lac_sources}, we further demonstrate that our
detection method does not misidentify point-like sources as being
extended by testing the extension of active galactic nuclei (AGN)
believed to be unresolvable.  In Section~\ref{validate_known},
we systematically reanalyze the twelve extended sources included
in 2FGL and in Section~\ref{systematic_errors_on_extension}
we describe a way to estimate systematic errors on the extension of a source.
In Section~\ref{extended_source_search_method}, we
describe a search for new spatially extended LAT sources. Finally,
in Section~\ref{new_ext_srcs_section} we present the detection of the
extension of nine spatially extended sources that were reported in 2FGL
but as being point-like.  Two of these sources have been previously analyzed in
dedicated publications.


\section{Analysis Methods}
\label{analysis_methods_section}

Morphological studies of sources using the LAT are challenging
because of the strongly energy-dependent PSF that is comparable in
size to the extension of many sources expected to be detected at
\gev energies.  Additional complications arise for sources along
the Galactic plane due to systematic uncertainties in the Galactic
diffuse emission.  The LAT PSF is limited at its lowest detectable
energies by multiple scattering in the silicon strip tracking section
of the detector and is several degrees at 100 \mev.  The PSF improves
with energy approaching a 68\% containment radius of $\sim0\fdg2$ at
the highest energies (when averaged over the acceptance of the LAT)
and is limited by the granularity of the silicon strips in the tracker
\citep{atwood_LAT_mission,on_orbit_calibration,lat_on_orbit_psf}.\footnote{More
information about the performance of the LAT can be found at the \fermi
Science Support Center (FSSC, \url{http://fermi.gsfc.nasa.gov}).} However,
since most high energy astrophysical sources have spectra that decrease
rapidly with increasing energy, there are typically fewer higher
energy photons with improved angular resolution. Therefore sophisticated
analysis techniques are required to maximize the LATs sensitivity
to these extended sources.

\subsection{Modeling Extended Sources in the \pointlike Package}

A new analysis tool has been developed to address the unique requirements
for studying spatially extended sources with the LAT.
The Poisson likelihood to find
the observing counts is maximized given a parametrized spatial and
spectral model of the source and its surrounding region.  The sky is
divided into cubes of space and energy using the healpix representation
of the sky \citep{healpix} and the likelihood is maximized over all bins
in a region.  The extension of a source can be modeled by a geometric
shape (e.g. a disk or a two-dimensional Gaussian) and the position, extension,
and spectrum of the source
can be simultaneously fit.

This type of analysis is not feasible using the standard LAT likelihood
analysis tool \gtlike\footnote{\gtlike is distributed publicly by the
FSSC.} because it can only fit the spectral parameters of the model
unless a more sophisticated iterative procedure is used to also test
various source morphologies.  We note that \gtlike has been used in the
past in several studies of source extension in the LAT Collaboration
\citep{lmc,smc,w28,w51c}.  In these studies, a profile based upon a
set of \gtlike maximum likelihood fits at fixed extensions was used
to build a profile of the likelihood as a function of extension.
\newtext{
The \gtlike likelihood profile approach has been show to correctly
reproduce the extension of simulated extended sources assuming that the
true position is known \citep{francesco_2011}.  But it is not optimal
because the position, extension, and spectrum of the source must be
simultaneously fit to find the best fit parameters and to maximize the
statistical significance of the detection.  Furthermore, because the \gtlike
approach is computationally intensive, no large-scale Monte Carlo
simulations have been run to calculate its false detection rate.
}

The approach presented here is based on a second maximum likelihood
fitting package developed in the LAT Collaboration called \pointlike
\citep{first_cat,matthew_kerr_thesis}.  The choice to base the
spatial extension fitting on \pointlike rather than \gtlike was made
on considerations of computing time.  The \pointlike algorithm was
optimized for speed to handle larger numbers of sources efficiently,
which is important for our catalog scan and for being able
to perform large-scale Monte Carlo simulations to validate the analysis.
Details on the \pointlike package can be
found in \cite{matthew_kerr_thesis}.  We extended the code to allow a
simultaneous fit of the source extension together with the position and
the spectral parameters.

\subsection{Extension Fitting}
\label{extension_fitting}

In \pointlike, it is assumed that the spatial and spectral model
of an extended source are separable, i.e. that the source model
$M(l,b,E)=S(l,b)\times X(E)$ where $S(l,b)$ is the spatial distribution
and $X(E)$ is the spectral distribution.  To fit an extended source,
\pointlike convolves the extended source shape with the PSF (as a function
of energy) and uses the \minuit library \citep{minuit_documentation}
to maximize the likelihood by simultaneously varying the position,
extension, and spectrum of the source.  As will be described in
Section~\ref{monte_carlo_validation}, simultaneously fitting the
position, extension, and spectrum is important to maximize
the statistical significance of the detection of extension of a source.
To avoid projection effects, the longitude and latitude 
of the source are
not directly fit but instead the displacement of the source in a rotated
reference frame.

The significance of the extension of a source can be calculated from the
likelihood ratio test of a hypothesis with a spatially extended source and
a hypothesis with a point-like source. The test statistic for this procedure
is defined as
\begin{equation}
  \tsext=2\log(\likelihood_\text{ext}/\likelihood_\text{ps}) 
\end{equation}
where \likelihood is the Poisson likelihood.
\pointlike calculates \tsext by fitting a source first with a spatially
extended model and then as a point-like source.  The interpretation
of \tsext in terms of a statistical significance is discussed in
Section~\ref{monte_carlo_validation}.

For extended sources with an assumed radially-symmetric shape,
we optimized the calculation by performing one
of the integrals analytically.
The expected photon 
distribution can be written as
\begin{equation}
  \text{PDF}(\vec r) = \int  \text{PSF}(|\vec r - \vec r'|)I_\text{src}(\vec r') r' dr' d\phi'
\end{equation}
where $I_\text{src}(\vec r')$ is the spatial distribution of the
source.
\newtext{
The LAT's PSF is currently parameterized 
in the Pass~7\_V6 (P7\_V6) Source Instrument
Response Function \citep[IRFs,][]{lat_on_orbit_psf} by a King function \citep{king_function}:
}
\begin{equation}
  \text{PSF}(r) = 
  \frac{1}{2\pi\sigma^2}
  \left(1-\frac{1}{\gamma}\right)
  \left(1+\frac{u}{\gamma}\right)^{-\gamma},
\end{equation}
where $u=(r/\sigma)^2/2$ and $\sigma$ and $\gamma$ are free parameters
\citep{matthew_kerr_thesis}.  For radially-symmetric extended sources,
the angular part of the integral can be evaluated analytically
\begin{align}
  \text{PDF}(u) & = \int_0^\infty r' dr'
  I_\text{src}(v) 
  \int_0^{2\pi} d\phi' 
  \text{PSF}(\sqrt{2\sigma^2(u+v-2\sqrt{uv}\cos(\phi-\phi'))})
  \\
  & = \int_0^\infty dv
  I_\text{src}(v) 
  \left(\frac{\gamma-1}{\gamma}\right)
  \left( \frac{\gamma}{\gamma + u + v}\right)^\gamma 
  \times {}_2F_1 \left(\gamma/2,\frac{1+\gamma}{2},1,\frac{4uv}{(\gamma+u+v)^2}\right).
\end{align}
where $v=(r'/\sigma)^2/2$ and ${}_2F_1$ is the Gaussian hypergeometric
function.  This convolution formula reduces the expected photon
distribution to a single numerical integral.

There will always be a small numerical discrepancy between the expected
photon distribution derived from a true point-like source and a very small
extended source due to numerical error in the convolution.  In most
situations, this error is insignificant.  But in particular for
very bright sources, this numerical error has the potential to bias the
test statistic for the extension test. Therefore, when calculating
\tsext, we compare the likelihood fitting the source with an extended
spatial model to the likelihood when the extension is
fixed to a very small value ($10^{-10}$ degrees in radius 
for a uniform surface brightness profile).

We estimate the error on the extension of a source by fixing
the position of the source and varying the extension until the
likelihood has fallen by \onehalf, corresponding to a $1\sigma$ error
\citep{Statistical_methods_book}.  Figure~\ref{four_plots_ic443}
demonstrates this method by showing the change in the log of the
likelihood when varying the extension of the SNR IC443.  The localization
error is calculated by fixing the extension and spectrum of the source
to their best fit value and then
fitting to the likelihood function a 2D Gaussian as a function
of position. This localization error algorithm is further described in
\cite{second_cat}.

\subsection{\gtlike Analysis Validation}
\label{gtlike_crosscheck}

\pointlike is important for LAT analyses that require many iterations
such as source localization and extension fitting.  On the other hand,
because \gtlike makes fewer approximations in calculating the likelihood
we expect the spectral parameters found with \gtlike to be slightly more
precise.  Furthermore, because \gtlike is the standard likelihood analysis
package, it has been more extensively validated for spectral analysis.
For those reasons, in the following analysis we used \pointlike to
determine the position and extension of a source and subsequently derived
the spectrum using \gtlike. Both \gtlike and \pointlike can be used to
estimate the statistical significance of the extension of a source and we
required that both methods agree for a source to be considered extended.
There was good agreement between the two methods.  Unless explicitly
mentioned, all \ts, \tsext, and spectral parameters were calculated using
\gtlike with the best-fit positions and extension found by \pointlike.


\subsection{Comparing Source Sizes}

\label{compare_source_size}

We considered two different models that can be used to model the
surface brightness profile for extended sources, either a 2D Gaussian
profile
\begin{equation}\label{gauss_pdf}
  I_\text{Gaussian}(x,y)=\tfrac{1}{2\pi\sigma^2}\exp\left(-(x^2+y^2)/2\sigma^2\right)
\end{equation}
or a uniform surface brightness profile
\begin{equation}\label{disk_pdf}
  I_\text{disk}(x,y)=
  \begin{cases}
    \frac{1}{\pi\sigma^2} & x^2+y^2\le\sigma^2 \\
    0                      & x^2+y^2>\sigma^2.
  \end{cases}
\end{equation}
Although these shapes are significantly different, 
Figure~\ref{compare_disk_gauss} shows that their PDF 
are similar for a source that has a 
0\fdg5
radius
typical of LAT detected extended sources.
To
allow a valid comparison between the Gaussian and the uniform surface
brightness morphologies,
we define the source size 
as the radius containing
68\% of the intensity ($\rsixeight$).  $\rsixeight_\text{,Gaussian}=1.51\sigma$
and $\rsixeight_\text{,disk}=0.82\sigma$ where $\sigma$
is defined in Equation~\ref{gauss_pdf} and
Equation~\ref{disk_pdf} respectively.
For this example, $\rsixeight=0\fdg5$ so
$\sigma_\text{disk}=0.61\degree$ and 
$\sigma_\text{Gaussian}=0.33\degree$.

For smaller and fainter extended sources, the differences in the PDF for
different spatial models are lost in the noise and the LAT is not sensitive
to the detailed spatial structure of these sources.  Therefore, in our
search for extended sources we use only a radially-symmetric uniform
surface brightness profile as our spatial extension shape. Unless otherwise noted,
we quote the radius of the edge ($\sigma$) as the size of the source.

\section{Validation of Analysis Method}

\label{monte_carlo_validation}

We tested the false detection probability for the extended source test
by fitting the extension of point-like sources.  \cite{mattox_egret}
discuss that the test statistic distribution for a likelihood ratio test
on the existence of a source at a given position is
\begin{equation}\label{ts_ext_distribution}
  P(\ts)=\onehalf(\chi^2_1(\ts)+\delta(\ts)).
\end{equation}
The particular form of Equation \ref{ts_ext_distribution} is due to the
null hypothesis (source flux $\Phi=0$) residing on the edge of parameter
space and the model hypothesis adding a single degree of freedom and
leads to the often quoted result that the square root of the test
statistic is the number of $\sigma$ of the detection. It is plausible
to expect a similar distribution of the test statistic in the test for
source extension since the same conditions apply (with the source flux
$\Phi$ replaced by the source radius $r$ and $r<0$ being unphysical).
We performed Monte Carlo simulations to calculate empirical distributions
for $\tsext$ and we compared them to Equation~\ref{ts_ext_distribution}.

We simulated point-like sources with various spectral forms using
the LAT on-orbit simulation tool
\gtobssim\footnote{\gtobssim is distributed publicly by the FSSC.} and fit the sources
with \pointlike using both point-like
and extended source hypotheses.  These point-like sources were simulated with a power-law
spectral model with integrated fluxes above 100 \mev ranging from $3\times10^{-9}$ 
to $1\times10^{-6}$ \phflux in six discrete steps and spectral
indices ranging from 1.5 to 3 in four discrete steps.  These values
were picked to represent typical source parameters of LAT-detected
sources. The point-like sources were simulated on top of an isotropic
background with a power-law spectral model with
integrated flux above 100 \mev of $1.5\times10^{-5}$ \phflux
and spectral index 2.1.
This was
taken to be the same as the isotropic spectrum measured by EGRET
\citep{sreekumar_isotropic}.  The Monte Carlo simulation was performed
over a one-year observation period using a representative rocking profile and a
representative livetime fraction of 0.8.  The reconstruction was performed
using the P7\_V6 Source class event selection and IRFs \citep{lat_on_orbit_psf}. For each 
significantly detected point-like source ($\ts\ge25$), we used \pointlike
to fit the source as an extended source and calculate \tsext.
\newtext{
This entire procedure was performed twice, once fitting in the 1 \gev
to 100 \gev energy range and once fitting in the 10 \gev to 100 \gev
energy range.

For each set of spectral parameters, $\sim20,000$ statistically independent
simulations were performed. For dimmer spectral models, many of the
simulations left the source insignificant ($\ts<25$)
and were discarded.  Table~\ref{ts_ext_num_sims}
shows the different spectral models used in our study as well as the
number of simulations and the average point-like source
significance.  The cumulative density of \tsext is plotted in
Figure~\ref{ts_ext_mc_1000} and Figure~\ref{ts_ext_mc_10000} 
and compared to the $\chi^2_1/2$ distribution of
Equation~\ref{ts_ext_distribution}.

Our study shows broad agreement between simulations and
Equation~\ref{ts_ext_distribution}. To the extent that there is
disagreement, the simulations tended to produce smaller than expected
values of \tsext which would make the formal significance conservative.
Considering the distribution in Figure~\ref{ts_ext_mc_1000} and
Figure~\ref{ts_ext_mc_10000}, the choice of a threshold \tsext set to 16
(correspond to a formal $4\sigma$ significance) is reasonable.
}


\section{Extended Source Detection Threshold}
\label{extension_sensitivity}

We calculated the LAT flux threshold to detect that a spatially extended
source is extended. We define the detection threshold as the flux at
which the value of $\tsext$ averaged over many statistical realizations is
$\langle\tsext\rangle=16$ for a source of a given flux and true extension
\newtext{
(corresponding to a formal $4\sigma$ significance).
}

We used a simulation setup similar to that described in
Section~\ref{monte_carlo_validation}, but instead of point-like sources
we simulated extended sources with a radially-symmetric uniform surface
brightness profile. Additionally, we simulated our sources over the two-year
time range included in 2FGL.  For each extension and spectral index,
we selected a flux range which bracketed $\tsext=16$ and performed an
extension test for $>100$ independent realizations of ten fluxes in
the range.  We calculated $\langle\tsext\rangle=16$ by fitting a line
to the flux and $\tsext$ values in the narrow range.

Figure~\ref{index_sensitivity} shows the threshold for sources of four
spectral indices from 1.5 to 3 and extensions varying from $\sigma=0\fdg1$
to $2\fdg0$.  
The threshold is high for small extensions when the
source is 
small compared to the size of the PSF. 
It drops quickly with increasing source size and reaches
a minimum around 0\fdg5. 
The threshold increases
for large extended sources because the source becomes
increasingly diluted by the background.
Figure~\ref{index_sensitivity} shows
the threshold using photons with energies between 100 \mev and 100 \gev
and also using only photons with energies between 1 \gev and 100 \gev.
Except for very large or very soft
sources, our detection threshold is
not substantially improved by including photons with energies between 100 \mev and
1 \gev.  This is also demonstrated in Figure~\ref{four_plots_ic443}
which shows \tsext for the SNR IC443 computed independently in twelve
energy bins between 100 \mev and 100 \gev. For IC443, which has a
spectral index $\sim2.4$ and an extension $\sim0\fdg35$, 
almost the entire increase in likelihood
modeling the source as being extended comes
from energies above 1 \gev.  Furthermore, other systematic errors
become increasingly important at low energy. For our extension search
(Section~\ref{extended_source_search_method}),
we therefore used only photons with energies above 1 \gev.

Figure~\ref{all_sensitivity} shows the flux threshold as a function of
source extension for different background levels ($1\times$, $10\times$,
and $100\times$ the nominal background), different spectral indices,
and two different energy ranges (1 \gev to 100 \gev and 10 \gev to
100 \gev).  The detection threshold is higher for sources in regions of
higher background.  When studying sources only at energies above 1 \gev,
the LAT detection threshold (defined as the 1 \gev to 100 \gev flux at
which $\langle\tsext\rangle=16$) depends less strongly on the 
spectral index of the source. 
The index dependence of the detection threshold is
even weaker when considering only photons with energies above 10 \gev
because the PSF changes little from 10 \gev to 100 \gev.
Overlaid on Figure~\ref{all_sensitivity} are the LAT detected extended
sources that will be discussed in Sections~\ref{validate_known} and
\ref{new_ext_srcs_section}.  The extension thresholds are tabulated in
Table~\ref{all_sensitivity_table}.

Finally, Figure~\ref{time_sensitivity} shows the projected
detection threshold of the LAT to extension with a 10 year
exposure against 10 times
the isotropic background measured by EGRET. This background is representative of the
background near the Galactic plane.  For small extended sources, our
detection threshold improves by a factor larger than the square root
of the exposure because at high energies, where we are most sensitive
to extension, the background levels are in the Poisson instead of the
Gaussian regime.  For large extended sources, the relevant background
is over a larger spatial range and so the improvement is closer to the
expected factor corresponding to the square root of the exposure.


\section{Testing Against Source Confusion}
\label{dual_localization_method}

It is impossible to discriminate with LAT data exclusively between a
spatially extended source and multiple point-like sources separated by
angular distances comparable to or smaller than the size of the LAT PSF.
To assess the plausibility of source confusion for situations
in which $\tsext>16$, we developed an algorithm in \pointlike to
simultaneously fit the position and spectrum of two point-like sources.

After simultaneously fitting the two positions and two spectra,
we define \tsinc as twice the increase in the log of the likelihood
fitting the region as two point-like sources compared to fitting the
region as one point-like source:
\begin{equation}
  \tsinc=2\log(\likelihood_\text{2pts}/\likelihood_\text{ps}).
\end{equation} 
\tsinc cannot be quantitatively compared to \tsext using a simple
likelihood ratio test to evaluate which model is significantly better
because the models are not nested \citep{statistics_with_care}.
Even though the comparison of \tsext with \tsinc is not a calibrated
test, $\tsext>\tsinc$ indicates that the likelihood for the extended
source hypothesis is higher than for two point-like sources and we only
consider a source to be extended if $\tsext>\tsinc$.

We assessed the power of the $\tsext>\tsinc$ test with a Monte Carlo
study.  We simulated one spatially extended source and fit it as both
an extended source and as two point-like sources.  We then simulated two
point-like sources and fit them with the same two hypotheses. By comparing
the distribution of \tsinc and \tsext for the two cases, we evaluated
how good the $\tsext>\tsinc$ test is at rejecting the possibility of
source confusion as well as how likely it is to incorrectly reject 
with this test that an extended source is 
spatially extended.  All sources were simulated using
the same two-year time range included in 2FGL and all fit values were
calculated using \pointlike. The simulations were performed against
an background 10 times the isotropic background measured by EGRET,
representative of the background near the Galactic plane.

We did this study first in the energy range from 1 \gev to 100 \gev by
simulating extended sources of flux $4\times10^{-9} \phflux$ integrated
from 1 \gev to 100 \gev with a power-law spectral model with
spectral index 2.  This flux was picked to be representative of the
new extended sourced that were discovered in the following analysis
when looking in the 1 \gev to 100 \gev energy range
(see Section~\ref{new_ext_srcs_section}).
We simulated these sources using a uniform surface brightness profile
with extensions varying
up to $1\degree$.  
\newtext{
Figure~\ref{confusion_plot}a shows 
the distribution of $\tsext-\tsinc$ as a
function of the simulated extension of the source
for 200 statistically independent simulations.
}


Figure~\ref{confusion_plot}c shows the same plot but when fitting
simulations of two point-like sources each with half of the flux of
the spatially extended source and with the same spectral index as the
extended source.  Finally, figure~\ref{confusion_plot}e shows the same
plot with each point-like source having the same flux but different
spectral indices.  One point-like source had a spectral index of 1.5
and the other an index of 2.5.

The same three plots are shown in figure~\ref{confusion_plot}b,
figure~\ref{confusion_plot}d, and figure~\ref{confusion_plot}f but this
time when analyzing a source of flux $10^{-9} \phflux$ (integrated from
10 \gev to 100 \gev) only in the 10 \gev to 100 \gev energy range.
This spectrum is typical of the new extended sources discovered
using only photons with energies between 10 \gev and 100 \gev (see
Section~\ref{new_ext_srcs_section}).

Several interesting conclusions can be made from this study. The overall
behavior of these plots is what one would expect. \tsext is on average
greater than \tsinc when fitting simulated extended sources whereas
\tsinc is greater than \tsext when fitting two simulated point-like sources.
Furthermore, the distributions of \tsext and \tsinc separate better
when the extended source is bigger or when the two point-like sources are
farther apart.

The distributions also show that for simulated
extended sources with typical
sizes ($\sigma\sim0\fdg5$), one can often obtain almost as good 
an increase in likelihood
with two sources as with one extended source ($\tsinc\sim\tsext$).
This is because although
two point-like sources represent an incorrect spatial model, 
the two point-like sources have four additional degrees
of freedom (two spatial and two spectral parameters)
than one extended source (one extension parameter)
and can therefore easy model
the extended source and statistical fluctuations in the data.
This effect is most pronounced when using photons with energies between 1
\gev and 100 \gev where the PSF is even larger.  
On the other hand,
when testing the two simulated point-like sources for extension our test
did a better job at separating the distribution of \tsinc and \tsext
and found that \tsinc was larger than \tsext.
This is because the one additional extension parameter is less able to
model the second source or statistical fluctuations.

From this, we conclude that the $\tsext>\tsinc$ test is 
powerful at discarding cases in which the true emission is derived
from two point-like sources instead of from a spatially extended
source. On the other hand, because of the large values of \tsinc
in figure~\ref{confusion_plot}a and figure~\ref{confusion_plot}b when
fitting spatially extended sources, we caution that even in the case of
their being spatially extended emission the fit of two point-like sources
can often seem (by comparing the likelihoods) to be almost as good.

From this Monte Carlo study, we can see the limits of a LAT analysis
of spatially extended sources.  Section~\ref{monte_carlo_validation}
showed that we have a statistical test that finds when a LAT source is
not well described by the PSF.  But this test does not uniquely prove
that the emission originates from spatially extended emission instead
of from multiple unresolved sources.  Demanding that $\tsext>\tsinc$
is a powerful second test to avoid cases of simple confusion of two
point-like sources. But it could always be the case that an extended
source candidate is actually the superposition of multiple point-like or
extended sources that could be resolved with deeper observations of the
region.  There is nothing about this conclusion unique to LAT analysis,
but the broad PSF of the LAT and the density of sources expected to be
\gev emitters in the galactic plane makes this issue more significant
for LAT analysis.  When possible, multiwavelength information should be
used to help select the best model of the sky.

Even though this dedicated monte carlo study was performed using
\pointlike, for the analysis of LAT data we took the best fit positions
of the two point-like sources and refit them using \gtlike.  We quote
\tsinc calculated with \gtlike using the best fit positions of the two
point-like sources found using \pointlike.

\section{Test of 2LAC Sources}
\label{test_2lac_sources}


For all following LAT analysis, we used the same two-year dataset
that was used in 2FGL spanning from 2008 August 4 to 2010 August 1. We
applied the same acceptance cuts and we used the same P7\_V6 Source class
event selection and IRFs \citep{lat_on_orbit_psf}.  
When analyzing sources in \pointlike, we used a circular $10\degree$ region of
interest (ROI) centered on our source.  
When refitting the region in \gtlike using the best fit spatial and
spectral models from \pointlike, we used the `binned likelihood' mode of
\gtlike on a $14\degree\times14\degree$ ROI with a pixel size of 0\fdg03.

Unless explicitly
mentioned, we used the same background model as 2FGL to represent the
Galactic diffuse, isotropic, and Earth limb emission.  To account for
possible residuals in the diffuse emission model, the galactic emission
was scaled by a power-law and the normalization
of the isotropic component
was left free.  
Unless explicitly mentioned,
we used all 2FGL sources within $15\degree$ of our source as our list
of background sources and we refit the spectral parameters of all sources
within $2\degree$ of the source.



To validate our method, we tested LAT sources associated with AGN for
extension.  \gev emission from AGN is believed to originate from the
cores of kiloparsec-scale jets of distant galaxies.  Therefore AGN are
not expected to be spatially resolvable by the LAT and provide a good
calibration source to demonstrate that our extension detection method
does not misidentify point-like sources as being extended.  We note that
megaparsec-scale $\gamma$-ray halos around AGNs have been hypothesized
to be resolvable by the LAT \citep{pair_halo_paper}. However, no such
halo has been discovered in the LAT data so far.

Following 2FGL, the LAT Collaboration published the Second LAT AGN
Catalog (2LAC), a list a list of high latitude ($|b|>10\degree$) sources
that had a high probability association with AGN \citep{second_agn_cat}.
2LAC associated 1016 2FGL sources with AGN.  To avoid systematic problems
with AGN classification, we selected only the 885 AGN which made it into
the clean AGN sub-sample.  An AGN association is considered clean only
if it has a high probability of association $P\ge 80\%$, if it is the
only AGN associated with the catalog source, and if there are no flags
on the source in 2FGL. These last two conditions are important for our
analysis. Source confusion may look like a spatially extended source
and flagged catalog sources may correlate with unmodeled structure in
the diffuse emission.

\newtext{
Of the 885 clean AGN, we selected the 733 of these 2FGL sources which
were significantly detected above 1 \gev and fit each of them for
extension.  The cumulative density of \tsext for these AGN is compared
to the $\chi^2_1/2$ distribution of Equation~\ref{ts_ext_distribution}
and is shown in Figure~\ref{agn_ts_ext}.  The \tsext distribution
for the AGN shows reasonable agreement with the theoretical
distribution and no AGN was found to be significantly extended
($\tsext>16$).  The observed discrepancy from the theoretical
distribution is likely due to small systematics
in our model of the LAT PSF and the Galactic diffuse emission (see
Section~\ref{systematic_errors_on_extension}).  The discrepancy could
also in a few cases be due to confusion with a nearby undetected source.
The overall agreement with the expected distribution demonstrates that
we can use \tsext as a measure of the statistical significance of the
detection of the extension of a source.
}

We should clarify that the LAT PSF used in this study was determined
empirically by fitting the observed shape of bright AGN (see
Section~\ref{systematic_errors_on_extension}). Finding that the AGN we
test are not extended is not surprising.  This validation analysis is
not suitable to reject any hypotheses about the existence of megaparsec-scale
halos around AGN.

\section{Analysis of Extended Sources Identified in 2FGL}
\label{validate_known}

As further validation of our extended source method for studying
spatially extended sources, we reanalyzed the twelve spatially extended
sources which were included in 2FGL \citep{second_cat}.  Even though
each of these sources required a dedicated analysis and they
were fit with a variety of spatial models, it is valuable to show that
these sources are significantly extended using our systematic 
method that assumed a radially-symmetric uniform surface
brightness profile.  On the other hand, for some of
these sources a uniform disk spatial model does not well describe the
observed extended emission and so the dedicated LAT publications provides
a qualitatively better model of these sources.


% 6 SNRs 

Six extended SNRs were
included in 2FGL: W51C, IC443, W28, W30, W44, the Cygnus Loop,
and W30
\citep{w51c,ic443,w28,w44,cygnus_loop_lat,w30_lat}.
Using photons
with energies between
1 \gev and 100 \gev, our analysis significantly detected
the extension of all six of these SNRs.

% 2 galaxies

Two nearby satellite galaxies of the Milky Way the Large Magellanic Cloud (LMC)
and the Small Magellanic
Cloud (SMC) were included in 2FGL as spatially extended sources \citep{lmc,smc}.  They were significantly
detected using photons with energies between
1 \gev and 100 \gev. Our
fit extension is comparable to the published result, but we note that
the previous LAT Collaboration publication on LMC used a more complicated two 2D Gaussian surface
brightness profile when fitting it \citep{lmc}.

% 3 PWN
Three PWNe, MSH\,15$-$52, Vela X, and HESS\,J1825$-$137 were
identified in 2FGL \citep{msh1552,velax,fermi_hess_j1825}.  
HESS\,J1825$-$137 was significantly detected using photons
with energies between 10 \gev and 100 \gev.
To improve our analysis of this source, we had to remove
from our background model 
2FGL\,J1823.1$-$1338c which was included in 2FGL to
represent part of the extended emission.
To avoid confusion with the nearby bright pulsar PSR\,J1509$-$5850, MSH\,15$-$52 had
to be
analyzed at high energies.  Using photons with energies above 10 \gev,
we fit the extension of MSH\,15$-$52 to be consistent with the published
size but with an extension significance of \tsext=6.6.

% failed to detect vela x + cen a
Our analysis was unable to resolve Vela X which would have required first
removing the pulsed photons from the Vela pulsar which was beyond the
scope of this paper.  Our analysis also failed to detect a significant
extension for the Centaurus A Lobes because
the shape of the source is significantly different from a uniform
radially-symmetric surface brightness profile \citep{cen_a_lat}.

Our analysis of these sources is summarized in
Table~\ref{known_extended_sources}.  This table includes the best fit
position and extension of these sources when fitting them 
with a radially-symmetric uniform surface brightness profile. It also
includes the best fit spectral parameters for each source.  The position
and extension of Vela X and the Centaurus A Lobes were taken from
\cite{velax,cen_a_lat} and are included in this list for completeness.

\section{Systematic Errors on Extension}
\label{systematic_errors_on_extension}

% Alternative PSF

For the extended sources presented in this paper,
we developed two criteria for estimating systematic errors
on the extension of the sources.
First, we estimated a systematic error due to
uncertainty in our knowledge of the LAT PSF.  
Before launch, the LAT PSF
was determined by detector simulations which were verified in accelerator
tests \citep{atwood_LAT_mission}. However, in-flight data revealed
a discrepancy above a few \gev in the PSF compared to the angular
distribution of photons from bright AGN \citep{lat_on_orbit_psf}.
Subsequently, the PSF was fit empirically to bright AGN and this
empirical parameterization is used in the P7\_V6 IRFs.  To account for
our uncertainty in our knowledge of the PSF, we refit our extended source
candidates using the pre-flight Monte Carlo representation of the PSF
and consider the difference in extension found using the two PSFs as a
systematic error on the extension of a source.  The same approach was used
in \cite{ic443}.  At high energies, we believe that our parameterization
of the PSF from bright AGN is substantially better than the Monte Carlo
representation of the PSF so this systematic error is conservative.

% Alternate Diffuse

We estimated a second systematic error on the extension of a source
due to uncertainty in our model of the Galactic diffuse emission by
using an alternative model which takes as input templates
calculated by
GALPROP\footnote{GALPROP is a software package for calculating the
Galactic $\gamma$-ray emission based on a model of cosmic-ray propagation
in the Galaxy \citep{galprop1998,galprop2011}. 
See also \url{http://galprop.stanford.edu/} for details.} 
but then fits each template locally in
the surrounding region.
The particular GALPROP model that was used as input is described in
the LAT analysis of the isotropic diffuse
emission \citep{isotropic_lat}.  
The intensities of various components
of the Galactic diffuse emission were fitted individually using a
spatial distribution of the intensities within the ROI as predicted by
the model.  We distinguished contributions from CR interactions with the
molecular hydrogen, the atomic+ionized hydrogen, residual gas traced
by dust \citep{isabelle_dark_gass}, and the interstellar radiation
field. We further split the contributions from interactions with molecular
and atomic hydrogen to the Galactic diffuse emission according to the
distance from the Galactic center in which they are produced. Hence, we
replaced the standard diffuse emission model by 18 individually fitted
templates to describe individual components of the diffuse emission.
A similar crosscheck was used in an analysis of RX\,J1713.7$-$3946 
by the LAT Collaboration \citep{rx_j1713_lat}.

It is not expected that this diffuse model is superior to the standard
LAT model obtained through an all-sky fit.  However, adding degrees of
freedom to the background model can remove likely spurious sources that
correlate with features in the Galactic diffuse emission.  Therefore,
this tests systematics that may be due to imperfect modeling of the
diffuse emission in the region. Nevertheless, this alternative diffuse
model does not test all systematics related to the diffuse emission model.
In particular, because the alternate model uses the same underlying gas
maps it is unable to asses systematics due to mischaracterization of local
diffuse structure due to lack lack of resolution of the underlying maps.

We do not expect the systematic error
due to uncertainties in the PSF to be correlated with the systematic
error due to uncertainty in the Galactic diffuse emission. Therefore,
the total systematic error on the extension of a source was obtained by
adding the two errors in quadrature.

\section{Extended Source Search Method}
\label{extended_source_search_method}


Now that we have demonstrated that we understand the statistical
issues associated with analyzing spatially extended sources
(Section~\ref{monte_carlo_validation} and~\ref{test_2lac_sources}) and
that our method can correctly analyze the extended sources included in
2FGL (Section~\ref{validate_known}), we apply this method to search for
new spatially extended \gev sources.

Ideally, we would apply a completely blind and uniform search that
tests the extension of each catalog source in the presence of all other
2FGL sources to find a complete list of all spatially extended sources.
As our test of AGN in Section~\ref{test_2lac_sources} showed, at high
galactic latitude where the source density is not as great and the
diffuse emission is less structured, this method works well.

But this is infeasible in the galactic plane where we are most likely expect
to discover new spatially extended sources.  In the galactic plane,
this analysis is challenged by our imperfect model of the diffuse
emission and by issues related to source confusion.  The Monte Carlo
study in Section~\ref{dual_localization_method}
showed that the overall likelihood would greatly increase when fitting
a spatially extended source as two point-like sources so we expect
that spatially extended LAT sources are modeled in 2FGL as
multiple point-like sources. Furthermore, the position of other nearby sources
in the region near an extended source could be biased by not correctly
modeling the
sources' extension.  Furthermore, 2FGL produced a
list of sources significant at energies above 100 \mev whereas we are
most sensitive to spatial extension at higher energies.  Therefore,
we expect that at higher energies our analysis would be complicated
by 2FGL sources no longer significant and by 2FGL
sources whose position was biased by diffuse emission at lower energies.

To account for these issues, we first produced a large list of
possibly extended sources employing a very liberal search criteria
and then refined the analysis of the promising candidates on a case
by case basis.  Our strategy tested all point-like 2FGL sources for
extension assuming they had a uniform radially-symmetric uniform surface
brightness profile and a power-law spectral model.  Although not all
extended sources are expected to have a shape very similar to a uniform
surface brightness profile, Section~\ref{compare_source_size} showed that
for many spatially extended sources the wide PSF of the LAT and limited
statistics makes this a reasonable approximation.  Additionally, our Monte
Carlo study and AGN validation (Sections~\ref{monte_carlo_validation}
and~\ref{test_2lac_sources}) demonstrate that we can use this shape
for a statistically rigorous test. Because this spatial model only adds
one additional degree of freedom, this test should be very sensitive.
On the other hand, choosing this spatial model biases us against
finding extended sources that are not well described by uniform emission
such as shell-type supernova remnants.

Before testing for extension,
we automatically removed from the background model all other 2FGL
sources within 0\fdg5 of the source.  This distance is somewhat arbitrary,
but was picked in hopes of finding extended sources with sizes on the
order of $\sim1\degree$ or smaller. On the other hand, by removing
these nearby background sources we expect to also incorrectly flag 
in our preliminary list point-like sources that are 
confused with
nearby sources.  To screen out obvious cases of source
confusion, we performed the dual localization procedure described in
Section~\ref{dual_localization_method} to compare the extended source
hypothesis to the hypothesis of two independent point-like sources.

As was shown in Section~\ref{extension_sensitivity}, we gain little in
sensitivity using photons with energies below 1 \gev. On the other hand,
the broad PSF at low energy makes us more susceptible to systematic
errors arising from source confusion due to nearby soft point-like sources
and by uncertainties in our modeling of the Galactic diffuse emission. 
In addition, the Galactic
diffuse emission is more pronounced at lower energies due to its steep
energy spectrum compared to the typical spectrum of extended sources \citep{intermediate_diffuse_lat}.
For that reason,
we performed our search using only photons with energies between 1 \gev
and 100 \gev.

We also performed a second search for extended sources using only photons
with energies between 10 \gev and 100 \gev.  Although this approach
tests the same sources, it is complementary because the Galactic
diffuse emission is even less dominant above 10 \gev. Furthermore,
source confusion is less of a problem since many LAT sources are not
significantly detected at energies exclusively above 10 \gev. Therefore,
we expected to be able to resolve harder sources in more complicated
regions by restricting our analysis to higher energies.  The $>10$ \gev
analysis is especially beneficial for regions near pulsars which are
not significantly detected above 10 \gev. A similar procedure was used
to detect the spatial extension of HESS\,J1825$-$137 and MSH\,15$-$52
with the LAT \citep{msh1552,fermi_hess_j1825}.

When we applied this test to the 1861 point-like sources in 2FGL, our
search found 117 extended source candidates in the 1 \gev to 100 \gev
energy range and 11 extended source candidates in the 10 \gev to 100
\gev energy range. Most of the extended sources flagged above 10 \gev were
also flagged above 1 \gev and many of these situations represent multiple
point-like 2FGL sources all fitting the same emission region. For example,
the sources 2FGL\,J1630.2-4752, 2FGL\,J1632.4-4753c 2FGL\,J1634.4-4743c,
and 2FGL\,J1636.3-4740c were all found to be spatially extended in the
10 \gev to 100 \gev energy range even though they all fit to similar
positions and sizes.  For these situations, we manually discarded all
but one of the 2FGL sources.

Similarly, many of these sources were confused with nearby
point-like sources or influence by large-scale residuals in the
diffuse emission.  To help determine which of these fits found
truly extended sourced and when the extension was
influenced by source confusion and diffuse emission, we generated a
series of diagnostic plots.  For each candidate, we generated a map
of the residual test statistic by adding a new point-like source of spectral index 2 into the
region at each position and finding the increase in likelihood when fitting
its flux. Figure~\ref{res_tsmaps} shows this map around the
most significantly extended source IC443 when it is modeled both as a
point-like sources and an extended source.  The residual test statistic
map indicates
that the spatially extended model for IC443 is a significantly better
description of the observed photons and that there is not significant
residual in the region after modeling the source as being spatially extended.
We also generated maps of the sum of all counts within a given distance of
the source and compared it to the model predictions assuming the emission
originated from a point-like source.  An example radial integral plot
is shown for the extended source IC443 in Figure~\ref{four_plots_ic443}.
For each source, we also made diffuse-emission-subtracted smoothed counts
maps (shown for IC443 in Figure~\ref{four_plots_ic443}).

It was very clear by visual inspection that in many
cases our analysis was strongly influence by large-scale residual in the
diffuse emission and hence the extension measure was unreliable.  This was
especially true in our analysis of sources in the 1 \gev to 100 \gev
energy range.  An example of such a case is 2FGL\,J1856.2+0450c analyzed
in the 1 \gev to 100 \gev energy range. Figure~\ref{example_bad_fit}
shows a diffuse-emission-subtracted smoothed counts map for
this source with the best
fit extension of the source overlaid. There appears to be large-scale
residual in the diffuse emission in this region along the Galactic plane.
As a result, 2FGL\,J1856.2+0450c is fit to an extension of $\sim2\degree$
and the result is statistically significant with \tsext=45.4. However,
by looking at the residuals it is clear that this complicated region is
not well modeled. We manually discard sources like this.
% information came from
% /nfs/slac/g/ki/ki03/lande/extended_catalog/2FGL/v15/standard_analysis/spectral_emin_1000_v1/P72Y3047/v1/results_followup_P72Y3047.yaml
% (2FGL J1856.2+0450c = P72Y3047 for)

Because of these systematic issues, we only selected extended source
candidates in regions that did not appear dominated by these issues and
where there was a multiwavelength
counterpart. Because of these systematic issues, this search can not be
expected to be complete and it is likely that there are other spatially
extended sources that this method missed.

For each candidate that was not biased by neighboring point-like
sources or by large-scale residuals in the diffuse emission model, we
improved the model of the region by decided on a case by case basis which
background point-like sources should be kept.  We kept in our model of the sky
sources that we
believed represented physically distinct sources and we removed 
sources that we believed were included in 2FGL to account
for residual induced by not modeling the extension of the source.
\newtext{
Soft nearby point-like 2FGL sources that were not significant at higher energies
were frozen to the spectrum predicted by 2FGL.
}
When deciding
which background sources to keep and which to remove, we used as much as possible
multiwavelength information about possibly extended source counterparts
to help guild our choice. For each extended source presented in 
Section~\ref{new_ext_srcs_section}, we will describe any modification from 2FGL
of the background model that were required.

The best fit position of nearby point-like sources can be influenced by
the spatial model of a nearby extended source and vice versa.  Similarly,
the best fit position of nearby point-like sources in 2FGL can be biased
by systematic errors at lower energies.
Therefore, after selecting the list of background sources, we iteratively
refit the position and spectrum of nearby background sources as well as
the position and extension of the analyzed spatially extended
sources until the overall fit globally converged.  For each extended
source, we will describe the position of any relocalized background
sources.

After obtaining the overall best fit positions and extensions of all
of the sources in the region using \pointlike, we refit the spectral
parameters of the region using \gtlike.  Using \gtlike, we obtained a
second measure of \tsext.  We only consider a source to be extended when
both \pointlike and \gtlike agree that $\tsext>16$. 
We further required that $\tsext>16$ using the alternate diffuse
model presented in section~\ref{systematic_errors_on_extension}.
We then replaced the spatially extended source with two point-like
sources and refit the position and spectrum of the
two point-like sources to calculate \tsinc.
We only consider a source to be spatially extended, instead of being
the result of confusion of two point-like sources, if $\tsext>\tsinc$.
As was show in Section~\ref{dual_localization_method}, this test is
fairly powerful at removing situations in which the emission actually
originates from two distinct point-like sources instead of one spatially
extended source.  On the other hand, it is still possible that improved
observations could resolve additional structure or new sources that we are
currently not sensitive to. Considering the very complicated morphologies
of extended sources observed at other wavelengths and the high density
of possible sources that are expected to emit at \gev energies, it is
likely that in some of these regions further observations will resolve
that the emission is significantly more complicated than the simple
radially-symmetric uniform surface brightness profile that we assume.


\section{New Extended Sources}
\label{new_ext_srcs_section}

% Notes about extended Catalog sources
% Catalog used for analysis is P72Y_uw23.fits
% Compare to final catalog gll_psc_v04.fit
%          1FGL Name - Preliminary 2FGL - 2FGL Name
% 1FGL J1628.6-2419c - P72Y2516         - 2FGL J1627.0-2425c
% 1FGL J0823.3-4248  - P72Y1212         - 2FGL J0823.0-4246

% 1FGL J1613.6-5100c - P72Y2472         - 2FGL J1615.0-5051 
% 1FGL J1614.7-5138c - P72Y2473         - 2FGL J1615.2-5138
% 1FGL J1632.9-4802c - P72Y2540         - 2FGL J1632.4-4753c
% 1FGL J1711.7-3944c - P72Y2674         - 2FGL J1712.4-3941
% 1FGL J2020.0+4049  - P72Y3281         - 2FGL J2021.5+4026
% 1FGL J1837.5-0659c - P72Y2974         - 2FGL J1837.3-0700c
% N/A                - P72Y1287         - 2FGL J0851.7-4635 (Vela Jr)

Nine extended sources not included in 2FGL were found by our
extended source search. Two of these have been previously
studied in dedicated publications: RX\,J1713.7$-$3946 and Vela
Jr. \citep{rx_j1713_lat,vela_jr_lat}
Two of these sources were
found when using photons with energies between 1 \gev and 100 \gev and seven
were found when using photons with energies between 10 \gev and 100 \gev.
For the sources found at energies above 10 \gev, we restrict our
analysis to higher energies because of the issues
of source confusion and diffuse emission modeling described in
section~\ref{extended_source_search_method}.
The spectral and spatial properties of these nine sources is summarized
in Table~\ref{new_ext_srcs_table} and the results of our investigation of
systematic errors are presented in Table~\ref{alt_diff_model_results}.
Table~\ref{alt_diff_model_results}
also
compares the likelihood assuming the source is spatially
extended to the likelihood assuming 
that the emission originates from two independent point-like
sources. For these new extended sources, $\tsext>\tsinc$ 
so we conclude that 
the \gev emission does not originate from two physically
distinct point-like sources (see Section~\ref{dual_localization_method}).  
Table~\ref{alt_diff_model_results} also includes the
results of the extension fit using variations of the PSF and the galactic
diffuse model described in Section~\ref{systematic_errors_on_extension}.
There is good agreement between \tsext and the fit size using the standard
analysis, the alternative diffuse models, and the alternative PSF.
This suggests that the sources are robust against features in the diffuse
emission model and uncertainties in the angular resolution of the LAT.

\subsection{2FGL\,J0823.0$-$4246}
\label{section_2FGL_J0823.0-4246}

% 1FGL J0823.3-4248  - P72Y1212         - 2FGL J0823.0-4246

2FGL\,J0823.0$-$4246 was found by our search to be an extended source
candidate in the 1 \gev to 100 \gev energy range and is spatially
coincident with the SNR Puppis A.  Figure~\ref{1FGL_J0823.3-4248} shows
a counts map of this source and overlays and \rosat X-ray contours of
Puppis A.  There are two nearby catalog sources 2FGL\,J0823.4$-$4305
and 2FGL\,J0821.0$-$4254 that are also coincident with the SNR but that
do not appear to represent physically distinct sources.
We concluded that
these nearby point-like sources
were included in 2FGL to account for residual induced
by not modeling the extension of this source and
removed them from our model of the sky.
After removing these sources, 2FGL\,J0823.0$-$4246 was found to have an extension of
$0\fdg37\pm0\fdg03_\stat\pm0\fdg02_\sys$ with an extension significance
of $\tsext=48.0$.  Figure~\ref{snr_seds} shows the spectrum of
this source.

Puppis A has been studied in detail in radio (\cite{puppis_a_vla}, and
references therein) and  X-ray (\cite{rosat_puppis_a,suzaku_puppis_a},
and references therein).  The fit extension of 2FGL\,J0823.0$-$4246
matches well its inferred size at \gev energies.  The distance of Puppis A was
estimated at 2.2 kpc \citep{reynoso_1995,reynoso_2003} and leads to a 1
\gev to 100 \gev luminosity of $\sim 3\times 10^{34}$ ergs$\,\s^{-1}$.
No molecular clouds have been observed directly adjacent to Puppis A
\citep{co_eastern_puppis_a}, similar to the LAT detected Cyngus Loop SNR
\citep{cygnus_loop_lat}.  The luminosity of Puppis A is also smaller
than that of other SNRs believed to interact with molecular clouds
\citep{w51c,ic443,w44,w28,w49b_lat}.


% Puppis A:
%   Flux = PowerLaw(norm=6.21e-14,index=2.22,e0=10000).i_flux(cgs=True,emin=1e3,emax=1e5,e_weight=1)
%        = 4.7804242225163421e-11 ergs cm^-2 s-1
%   Luminosity =  Flux * 4*pi*r^2 = 4.78e-11 ergs cm^-2 s^-1 * 4*pi * (2.2 kpc )**2 in ergs/s
%              = 2.8e34 ergs/s from 1.0-100 gev
%              = 5.6e34 ergs/s from 0.1-100 gev
%   Luminosities
%   Cygnus Loop:
%     L = 10^33 ergs/s between 1-100 \gev
%   W51C : 
%     http://arxiv.org/pdf/0910.0908
%     L > 1 x 10^36 ergs/s (0.2-50GeV) - 
%   IC443:
%     http://arxiv.org/pdf/1002.2198v1
%     L = 1.2e35 ergs/s (from 0.2-50 gev)
%     d = 1.5 kpc
%     From my (possibly incorrect) powerlaw analysis 
%       In [30]: PowerLaw(norm=4.69e-13,index=2.21,e0=1e4).i_flux(cgs=True,emin=1e3,emax=1e5,e_weight=1)
%       Out[30]: 3.5968626651391638e-10 ergs/cm^2/s
%       L = 3.5968626651391638e-10/cm^2 * 4*pi*(1.5kpc)**2 = 1e35 (1-100gev)
%   W44:
%     http://www.sciencemag.org/content/327/5969/1103.full.pdf
%     From my (possibly incorrect) powerlaw analysis 
%       PowerLaw(norm=2.7e-13,index=2.63,e0=1e4).i_flux(cgs=True,emin=1e3,emax=1e5,e_weight=1)
%       Out[27]: 2.7681360351439016e-10 ergs/cm^2/s
%     Distance = 3kpc
%     L = 3e35 ergs/s (1-100 gev)
%   W28
%     http://iopscience.iop.org/0004-637X/718/1/348/
%     From my (possibly incorrect) powerlaw analysis 
%       In [28]: PowerLaw(norm=2.23e-13,index=2.6,e0=1e4).i_flux(cgs=True,emin=1e3,emax=1e5,e_weight=1)
%       Out[28]: 2.221059203350944e-10 ergs/cm^2/s
%     Distance = 2kpc
%     L = 1e35 ergs/s (1-100 gev)
%   W49B
%     http://iopscience.iop.org/0004-637X/722/2/1303
%     L = 10^36 erg/s (0.2 - 200 gev)
%
    
\subsection{2FGL\,J1627.0$-$2425c}
\label{section_2FGL_J1627.0-2425c}

% 1FGL J1628.6-2419c - P72Y2516         - 2FGL J1627.0-2425c
% No modifications to background were needed

2FGL\,J1627.0$-$2425c was found by our search to
have an extension of $0\fdg42\pm0\fdg05_\stat\pm0\fdg16_\sys$ with
an extension significance of $\tsext=32.4$
using photons with energies between 1 \gev and 100 \gev.  
Figure \ref{1FGL_J1628.6-2419c} shows a counts map of this source.

This source is in a region of remarkably complicated diffuse emission.
Even though it is $16\degree$ from the Galactic plane, this source is on
top of the core of the Ophiuchus molecular cloud which contains massive
star-forming regions that are bright in the infrared.  The region also has
abundant molecular and atomic gas traced by CO and HI but also plenty of
dark gas found only by its association with dust emission
\citep{isabelle_dark_gass}. Embedded star-forming regions make it even
more challenging to measure the column density of dust.  Infared and 
${}^{12}\text{CO}$ ($J=1\rightarrow 0$)
contours are overlaid on Figure~\ref{1FGL_J1628.6-2419c} and show good
spatial correlation with the \gev emission \citep{iras_rho_ophiuci,co_rho_ophiuci}.
This source might 
represent $\gamma$-ray emission from the interactions of cosmic rays with
interstellar gas which has not been accounted for in the LAT diffuse
emission model.


\subsection{2FGL\,J0851.7$-$4635}
\label{section_2FGL_J0851.7-4635}

% (no 1FGL) - P72Y1287         - 2FGL J0851.7-4635

2FGL\,J0851.7$-$4635 was found by our search to be an extended source
candidate in the 10 \gev to 100 \gev energy range and is spatially
coincident with the SNR Vela Jr. This source was recently studied by the LAT
Collaboration in \cite{vela_jr_lat}.  Figure~\ref{Vela_Jr} shows a counts
map of the source.  Overlaid on Figure~\ref{Vela_Jr} are \tev
contours of Vela Jr. \citep{vela_jr_hess}.
There are three
point-like catalog sources 2FGL\,J0853.5$-$4711, 2FGL\,J0848.5$-$4535,
and 2FGL\,J0855.4$-$4625 which correlate with the 
multiwavelength
emission of
this SNR but do not appear to be physically distinct sources.
They were most likely included in 2FGL to account for residual
induced by not modeling the extension Vela Jr. and were removed
from our model of the sky.  

With this model of the background, 2FGL\,J0851.7$-$4635 was found to
have an extension of $1\fdg15\pm0\fdg08_\stat\pm0\fdg02_\sys$ with
an extension significance of $\tsext=86.8$.  The LAT size matches
well the \tev morphology of Vela Jr.  While fitting the extension
of 2FGL\,J0851.7$-$4635, we iteratively relocalized the position
of the nearby point-like catalog source 2FGL\,J0854.7$-$4501 to
$(l,b)=(266\fdg24,0\fdg49)$ to better fit its position at high energies.


\subsection{2FGL\,J1615.0$-$5051}
\label{section_2FGL_J1615.0-5051}

% For both 2FGL J1615.0-5051 and 2FGL J1615.2-5138 
% Extended Sources:
%   * 1FGL J1613.6-5100c - P72Y2472         - 2FGL J1615.0-5051 % extended source
%   * 1FGL J1614.7-5138c - P72Y2473         - 2FGL J1615.2-5138 % extended source

2FGL\,J1615.0$-$5051 and 2FGL\,J1615.2$-$5138 were both found to be
extended source candidates in the 10 \gev to 100 \gev energy range. Because
they are less than $1\degree$ away from each other, they needed to be analyzed
simultaneously.  2FGL\,J1615.0$-$5051 is spatially coincident with
the extended \tev source HESS\,J1616$-$508 and 2FGL\,J1615.2$-$5138
is coincident with the extended \tev source HESS\,J1614$-$518.
Figure~\ref{1FGL_J1613.6-5100c} shows a counts map of these sources and
overlays the \tev contours of HESS\,J1616$-$508 and HESS\,J1614$-$518
\citep{hess_plane_survey}.  The figure shows that the 2FGL source
2FGL\,J1614.9$-$5212 is very close to 2FGL\,J1615.2$-$5138 and correlates
with the same extended \tev source as 2FGL\,J1615.2$-$5138.  We concluded
that this source was included in 2FGL to account for residual induced
by not modeling the extension of 2FGL\,J1615.2$-$5138 and removed
it from our model of the sky.  

With this model of the sky, we iteratively fit the extensions of
2FGL\,J1615.0$-$5051 and 2FGL\,J1615.2$-$5138.
2FGL\,J1615.0$-$5051 was found to have an extension of
$0\fdg32\pm0\fdg04_\stat\pm0\fdg01_\sys$ with an extension significance
of \tsext=16.7.  
2FGL\,J1615.2$-$5138 was found
 to have an extension of $0\fdg42\pm0\fdg04_\stat\pm0.02_\sys$
with an extension significance of \tsext=46.5.
To test for the possibility that
2FGL\,J1615.2$-$5138 is not spatially extended but instead composed
of two point-like sources (one of them represented in 2FGL by
2FGL\,J1614.9$-$5212), we refit 2FGL\,J1615.2$-$5138 as two point-like
sources.  Because \tsinc is less than \tsext, we conclude that
this emission does not originate from two nearby point-like sources.

The \tev counterpart of 2FGL\,J1615.0$-$5051 was fit
with a radially-symmetric Gaussian surface brightness profile to have an
extension $0\fdg136\pm0\fdg008$.  This \tev size corresponds to a 68\%
containment radius of $\rsixeight=0\fdg21\pm0\fdg01$, comparable to the
LAT size $\rsixeight=0\fdg26\pm0\fdg03$.  Figure~\ref{hess_seds} shows
that the spectrum of 2FGL\,J1615.0$-$5051 at \gev energies connects to
the spectrum of HESS\,J1616$-$508 at \tev energies.

HESS\,J1616$-$508 is located in the region of two SNRs RCW103 (G332.4-04)
and Kes~32 (G332.4+0.1) but is not spatially coincident with either
of them \citep{hess_plane_survey}.  HESS\,J1616$-$508 is near three
pulsars PSR\,J1614$-$5048, PSR\,J1616$-$5109, and PSR\,J1617$-$5055.
\citep{discovery_of_PSR_J1617-5055,integral_HESS_J1616-508}.  Only
PSR\,J1617$-$5055 is energetically capable of powering the \tev emission
and \cite{hess_plane_survey} speculated that HESS\,J1616$-$508 could
be a PWN powered by this young pulsar.  Because HESS\,J1616$-$508 is
$9\arcmin$ away from PSR\,J1617$-$5055, this would require an asymmetric
X-ray PWNe to power the \tev emission.  \chandra ACIS observations
revealed an underluminous PWN of size $\sim1\arcmin$ around the
pulsar that was not oriented towards the \tev emission rendering this
association uncertain \citep{discovery_of_pwn_for_PSR_J1617-5055}.
No other promising counterparts were observed at X-ray and soft
gamma-ray energies by \suzaku \citep{suzakzu_HESS_J1616-508}, \swiftxrt,
IBIS/ISGRBI, BeppoSAX and \xmmnewton \citep{integral_HESS_J1616-508}.
\cite{discovery_of_pwn_for_PSR_J1617-5055} discovered some additional
diffuse emission towards the center of HESS\,J1616$-$508 using archival
radio and infared observations. Deeper observations will likely be
necessary to understand this $\gamma$-ray source.

\subsection{2FGL\,J1615.2$-$5138}
\label{section_2FGL_J1615.2-5138}

2FGL\,J1615.2$-$5138 is spatialy coincident with 
the extended \tev source HESS\,J1614$-$518.
H.E.S.S. measured a 2D Gaussian extension of $\sigma=0\fdg23\pm0\fdg02$
and $\sigma=0\fdg15\pm0\fdg02$ in the semi-major and
semi-minor axis. This corresponds to a 68\% containment size of
$\rsixeight=0\fdg35\pm0\fdg03$ and $0\fdg23\pm0\fdg03$,  consistent with
the LAT size $\rsixeight=0\fdg34\pm0\fdg03$.  Figure~\ref{hess_seds}
shows that the spectrum of 2FGL\,J1615.2$-$5138 at \gev energies connects
to the spectrum of HESS\,J1614$-$518 at \tev energies.  Further data
collected by H.E.S.S. in 2007 resolve a double peaked structure at
\tev energies but no spectral variation across this source, suggesting
that the emission is not the confusion of physically separate sources
\citep{closer_look_hess_j1614-518}.  This double peaked structure is
also hinted at in the LAT counts map in Figure~\ref{1FGL_J1613.6-5100c}
but is not very significant.  The \tev source was also detected by
CANGAROO-III \citep{cangaroo_j1614-518}.

There are five nearby pulsars, but none are luminous enough to
provide the energy output required to power the $\gamma$-ray
emission \citep{closer_look_hess_j1614-518}.  HESS\,J1614$-$518
is spatially coincident with a young open cluster Pismis 22
\citep{hess_1614_landi_atel,closer_look_hess_j1614-518}.  \suzaku detected
two promising X-ray candidates. Source A is an extended source consistent
with the peak of HESS\,J1614$-$518 and source B coincident with Pismis 22
and towards the center but in a relatively dim region of HESS\,J1614$-$518
\citep{suazku_hess_j1614_518}.  Three hypothesis have been presented to
explain this emission; either source A is an SNR powering the $\gamma$-ray
emission, source A is a PWN powered by an undiscovered pulsar in either
source A or B, and finally that the emission may arise from hadronic
acceleration in the stellar winds of Pisim 22 \citep{cangaroo_j1614-518}.

\subsection{2FGL\,J1632.4$-$4753c}
\label{section_2FGL_J1632.4-4753c}


% 1FGL J1632.9-4802c - P72Y2540         - 2FGL J1632.4-4753c


2FGL\,J1632.4$-$4753c was found by our search to be an extended source
candidate in the 10 \gev to 100 \gev energy range but is in a crowded
region of the sky.  It is spatially coincident with the \tev source
HESS\,J1632$-$478.  Figure~\ref{1FGL_J1632.9-4802c}a shows a counts
map of this source and overlays \tev contours of HESS\,J1632$-$478
\citep{hess_plane_survey}.  There are five nearby point-like 2FGL
sources that appear to  represent physically distinct sources: 2FGL
J1630.2-4752, 2FGL J1631.7-4720c, 2FGL J1632.4-4820c, 2FGL J1635.4-4717c,
2FGL J1636.3-4740c that were included in our background model.  On the
other hand, one point-like 2FGL source 2FGL\,J1634.4$-$4743c  correlates
with the extended \tev source and does not appear to be physically
separate.  We therefore removed this source from our model of the
background.  Figure~\ref{1FGL_J1632.9-4802c}b shows the same region with
the background sources subtracted.  With this model, 2FGL\,J1632.4$-$4753c
was found to have an extension of $0\fdg35\pm0\fdg04_\stat\pm0\fdg02_\sys$
with an extension significance of \tsext=26.9.  While fitting the
extension of 2FGL\,J1632.4$-$4753c, we iteratively relocalized
2FGL\,J1635.4$-$4717c to $(l,b)=(337\fdg23,0\fdg35)$ and
2FGL\,J1636.3$-$4740c to $(l,b)=(336\fdg97,-0\fdg07)$.

H.E.S.S measured a extension of $\sigma=0\fdg21\pm0\fdg05$ and
$0\fdg06\pm0\fdg04$ along the semi-major and semi-minor axes when
fitting HESS\,J1632$-$478 with an elliptical 2D Gaussian surface
brightness profile.  This corresponds to a 68\% containment size
$\rsixeight=0\fdg31\pm0\fdg08$ and $0\fdg09\pm0\fdg06$ along
the semi-major and semi-minor axis, consistent with the LAT size
$\rsixeight=0\fdg29\pm0\fdg04$.  Figure~\ref{hess_seds} shows that
the spectrum of 2FGL\,J1632.4$-$4753c at \gev energies connects to the
spectrum of HESS\,J1632$-$478 at \tev energies.

\cite{hess_plane_survey} argued that HESS\,J1632$-$478
is positionally coincident with the hard X-ray source
IGR\,J1632-4751 observed by \asca, INTEGRAL, and \xmmnewton
\citep{asca_plane_survey,Igr_J16320-4751_circ,xmm_newton_IGR_J16320-4751},
but this source is suspected to be a galactic X-Ray Binary so the
$\gamma$-ray extension disfavors the association.  Further observations
by \xmmnewton discovered point-like emission coincident with the peak
of the H.E.S.S. source surrounded by extended emission of size
$\sim32\arcsec\times15\arcsec$ \citep{hess_j1632_478_xmm_newton}.
They found in archival MGPS-2 data a spatially coincident extended
radio source \citep{most_survey_galactic_plane} and argued for a single
synchrotron process producing the radio, X-ray, and \tev emission,
likely due to a PWNe.  The increased size at \tev energies compared
to X-ray energies has previously been observed in several aging PWNe
including HESS\,J1825$-$137 \citep{hess_j1825_xmm_newton,hess_j1825_hess},
HESS\,J1640$-$465 \citep{hess_plane_survey,xmm_newton_hess_j_1640-466},
and Vela X \citep{vela_x_rosat,vela_x_hess} and can be explained by
different synchrotron cooling times for the electrons that produce X-rays
and $\gamma$-rays.


\subsection{2FGL\,J1712.4$-$3941}
\label{section_2FGL_J1712.4-3941}

% 1FGL J1711.7-3944c - P72Y2674         - 2FGL J1712.4-3941

2FGL\,J1712.4$-$3941 was found by our search to be spatially extended
using photons with energies between 1 \gev and 100 \gev.  This source
is spatially coincident with the SNR RX\,J1713.7$-$3946 and was
recently studied by the LAT Collaboration in \cite{rx_j1713_lat}.
To avoid issues related to uncertainties in the nearby Galactic
diffuse emission at lower energy, we restricted our analysis only
to energies above 10 \gev.  Figure~\ref{2FGL_J1712.4-3941} shows a
smoothed counts map of the source. Above 10 \gev, the \gev emission
nicely coorelates with the \tev contours of RX\,J1713.7$-$3946
\cite{rx_j1713_hess} and 2FGL\,J1712.4$-$3941 fit to an extension of
$0\fdg56\pm0\fdg04_\stat\pm0\fdg02_\sys$ with an extension significance
of $\tsext=38.5$.

\subsection{2FGL\,J1837.3$-$0700c}
\label{section_2FGL_J1837.3-0700c}

% 1FGL J1837.5-0659c - P72Y2986 - 2FGL J1837.3-0700c


2FGL\,J1837.3$-$0700c was found by our search to be an extended source
candidate in the 10 \gev to 100 \gev energy range and is spatially
coincident with the \tev source HESS\,J1837$-$069.  This source is
in a complicated region.  Figure~\ref{1FGL_J1837.5-0659c}a shows a
smoothed counts map of the region and overlays the \tev contours of
HESS\,J1837$-$069 \citep{hess_plane_survey}.  There are two very nearby
point-like 2FGL sources 2FGL\,J1836.8$-$0623c, and 2FGL\,J1839.3$-$0558c
that clearly represent distinct sources.  On the other hand, there is
another source 2FGL\,J1835.5$-$0649 located between the three sources that
appears to correlate with the \tev morphology of HESS\,J1837$-$069 but
does not appear to represent a physically distinct source.  We concluded
that this source was included in 2FGL to account for residual induced by
not modeling the extension of this source and removed it from our model
of the sky.  Figure~\ref{1FGL_J1837.5-0659c}b shows a
counts map of this region after subtracted these background sources.

With this model, 2FGL\,J1837.3$-$0700c was found to have an
extension of $0\fdg33\pm0\fdg07_\stat\pm0\fdg05_\sys$ with an extension
significance of $\tsext=18.5$. 
While fitting the extension of 2FGL\,J1837.3$-$0700c,
we iteratively relocalized the two closest
background sources along with the extension of 2FGL\,J1837.3$-$0700c but
their positions did not significantly change.  2FGL\,J1834.7$-$0705c
moved to $(l,b)=(24\fdg77,0\fdg50)$, 2FGL\,J1836.8$-$0623c moved
to $(l,b)=(25\fdg57,0\fdg32)$. With this best fit sky model, we tested for
source confusion by fitting the source instead as two point-like sources.
Because \tsinc is less than \tsext, we concluded that this emission
does not originate from two nearby point-like sources.

H.E.S.S. measured an extension of
$\sigma=0\fdg12\pm0\fdg02$ and $0\fdg05\pm0\fdg02$ 
of the coincident \tev source HESS\,J1837$-$069 
along the semi-major and semi-minor axis when fitting this source
with an elliptical 2D Gaussian surface brightness profile.  This corresponds
to a 68\% containment radius of $\rsixeight=0\fdg18\pm0\fdg03$ and
$0\fdg08\pm0\fdg03$ along the semi-major and semi-minor axis. The
size is not significantly different from the LAT 68\% containment
radius of $\rsixeight=0\fdg27\pm0\fdg07$ (less than $2\sigma$).
Figure~\ref{hess_seds} shows that the spectrum of 2FGL\,J1837.3$-$0700c
at \gev energies connects to the spectrum of HESS\,J1837$-$069 at \tev
energies.

HESS\,J1837$-$069 is coincident with the hard and steady X-ray source
AX\,J1838.0$-$0655 \citep{hard_x-ray_asca}.  This source was discovered
by RXTE to be a pulsar (PSR J1838-0655) 
sufficiently luminous to power the \tev emission
and was resolved by \chandra to be a bright point-like source surrounded
by a $\sim2\arcmin$ nebula \citep{pulsations_HESS_J1837-069}. The
$\gamma$-ray emission may be powered by this pulsar.  The hard spectral
index and spatial extension of 2FGL\,J1837.3$-$0700c disfavor a pulsar
origin of the LAT emission and suggest instead that the \gev and \tev
emission both originate from the pulsar's wind.  There is another
X-ray point-like source AX\,J1837.3$-$0652 near HESS\,J1837$-$069
\citep{hard_x-ray_asca} that was also resolved into a point-like
and diffuse component \citep{pulsations_HESS_J1837-069}.  Although no
pulsations have been detected from it, it could also be a pulsar powering
some of the $\gamma$-ray emission.


\subsection{2FGL\,J2021.5+4026}
\label{section_2FGL J2021.5+4026}


% 1FGL J2020.0+4049  - P72Y3281         - 2FGL J2021.5+4026



The source 2FGL\,J2021.5+4026 is coincident with 2CG\,078+01 and
3EG\,J2020+4017 \citep{second_cos_b_catalog,third_egret_catalog} and also
with the $\gamma$-Cygni SNR. This $\gamma$-ray source has been speculated
to originate from the interaction of accelerated particles in the SNR
with dense molecular clouds \citep{pollock_1985,gaisser_1998}, but this
association was disfavored when the LAT \gev emission from this source
was detected to be pulsed \citep[PSR\,J2021+4026,][]{first_lat_pulsar_cat}.
This pulsar was also been observed by AGILE \citep{gamma_cygni_agile}.

Looking at the same region at energies above 10 \gev, the pulsar is
no longer significant but we instead found in our search an extended
source candidate.  Figure~\ref{1FGL_J2020.0+4049} shows a counts map
of this source and overlays radio contours of $\gamma$-Cygni from the
Canadian Galactic Plane Survey \citep{canadian_galactic_plane_survey}.
There is good spatial overlap between the SNR and the \gev emission.

There is a nearby source 2FGL\,J2019.1+4040 that correlates with the radio
emission of $\gamma$-Cygni and does not appear to represent a physically
distinct source.  We concluded that it was included in 2FGL to account
for residual induced by not modeling the extension of $\gamma$-Cygni and
removed it from our model of the sky.  With this model, 2FGL\,J2021.5+4026
was found to have an extension of $0\fdg59\pm0\fdg03_\stat\pm0\fdg02_\sys$
with an extension significance of $\tsext=116.4$.  Figure~\ref{snr_seds}
shows its spectrum.  The inferred size of this source at \gev energies
well matches the radio size of $\gamma$-Cygni.  Milagro detected
a $4.2\sigma$ excess at energies $\sim 30$ \tev from this location
\citep{lat_bsl,milagro_bright_source_list}.  Veritas also detected an
extended source VER\,J2019+407 coincident with the SNR above 200 \gev
and suggested that the \tev emission could be a shock-cloud interaction
in $\gamma$-Cygni \citep{veritas_gamma_cygni}.


% Radio observations \dots
% X-ray observations \dots
% Age \ldots
% Distance: 1.5 kpc Landecker et al 1980 (from integral paper)
% Optical: (Mavromatakis 2003 (from integral paper)
% EGRET observations: 3EG J2020+4027
% LAT Pulsar detected
%   PSR J2021+4026
%   Prestended at scineghe2010.
%     http://scineghe2010.ts.infn.it/allegati/talks/ThursdaySept9/13_Razzano.pdf
%   Presented at Fermi Symposium 2011:
%     https://confluence.slac.stanford.edu/download/attachments/98831433/2011_3FermiSymp_CygPsr_ver2.pdf
%     "PSR J2021+4026 is a 265 ms pulsar found by Fermi LAT using"
%     "This discovery help to better clarify the nature of the
%      unidentified source 3EG J2020+4027, thought to be associated
%      with SNR 78.2+2.1, a.k.a. the $\gamma$-Cygni SNR.
%      Unfortunately, so far this pulsar lacks a radio counterpart, and
%      observations with Chandra and XMM show a potential X-ray
%      counterpart [8,9,15], but no pulsations have been detected so
%      far."

% Milagro analysis
% http://arxiv.org/abs/astro-ph/0611691
% Milagro $4.2\sigma$ excess.

% Veritas observations of $\gamma$-Cygni: 
%   http://arxiv.org/pdf/0912.4492v1
%   extended: VER J019+407
%     Veritas observations of $\gamma$-Cygni veritas_gamma_cygni,


\section{Discussion}

Twelve extended sources were included in 2FGL and two additional extended
sources were studied in dedicated publications.  Using two years of
LAT data and a new analysis method, we presented the detection of seven
additional extended sources.  We also reanalyze the spatial shape of the
twelve extended sources in 2FGL and the two additional sources.  The 21
extended LAT sources are located primarily along the Galactic plane
and their locations are shown in Figure~\ref{allsky_extended_sources}.
Most of the LAT detected extended sources are expected to be of Galactic
origin as the distance of extragalactic sources (with the exception of
the local group Galaxies) is typically too large to be able to resolve
them at $\gamma$-rays energies.

For the LAT extended sources also seen at \tev energies,
Figure~\ref{gev_vs_tev_plot} shows that there is a good correlation
between the size of the sources at \gev and \tev energies. Even so,
the size of PWNe are expected to vary across the \gev and \tev energy
range and the size of HESS\,J1825$-$137 is significantly larger at
\gev than \tev energies \citep{fermi_hess_j1825}.  It is interesting
to compare the size of other PWN candidates at \gev and \tev energies,
but definitively measuring this would require a more in-depth analysis
of the LAT data using the same elliptical surface brightness profile.

Figure~\ref{gev_vs_tev_histogram} compares the size of the 21 extended
LAT sources to the 42 extended H.E.S.S. sources.\footnote{The 
\tev extension of
the 42 extended H.E.S.S. sources comes from the H.E.S.S. Source
Catalog. The H.E.S.S. source catalog can be found at \url{http://www.mpi-hd.mpg.de/hfm/HESS/pages/home/sources/}.}
Because of the large
field of view and all-sky coverage, the LAT can more easily measure
larger sources.  On the other hand, the 
better
angular resolution of air Cherenkov detectors allows them to measure a
population of extended sources below our detection threshold limit (currently 
about $\sim0\fdg2$).  \fermi has a five year nominal mission lifetime with
a goal of ten years of operation.  As Figure~\ref{time_sensitivity} shows,
the low background of the LAT at high energies allows its sensitivity 
to
these smaller sources to improve by a factor greater than the square root
of the exposure.  With increasing exposure, the LAT will likely begin to
detect and resolve some of these smaller \tev sources.

Figure~\ref{compare_index_2FGL} compares the spectral index of LAT
detected extended sources and of all sources in 2FGL. This, and
Table~\ref{known_extended_sources} and~\ref{new_ext_srcs_table},
show that the LAT observes a population of hard extended sources
at energies above 10 \gev.  Figure~\ref{hess_seds} shows that
the spectra of four of these sources (2FGL\,J1615.0$-$5051,
2FGL\,J1615.2$-$5138, 2FGL\,J1632.4$-$4753c, and 2FGL\,J1837.3$-$0700c)
at \gev energies connects to the spectra of their H.E.S.S. counterparts
at \tev energies. This is also true of Vela Jr., HESS\,J1825$-$137
\citep{fermi_hess_j1825}, and RX\,J1713.7$-$3946 \citep{rx_j1713_lat}.
It is likely that the \gev and \tev emission from these sources originates
from the same population of high-energy particles.

Many of the \tev detected extended sources now seen at \gev energies are
currently unidentified and further multiwavelength follow-up observations
will be necessary to understand these particle accelerators.  But the
extension of the spectrum of these \tev sources towards lower energies
by LAT observations can help to determine the origin and nature of the
high-energy emission seen from these sources.



The \textit{Fermi} LAT Collaboration acknowledges generous ongoing support
from a number of agencies and institutes that have supported both the
development and the operation of the LAT as well as scientific data analysis.
These include the National Aeronautics and Space Administration and the
Department of Energy in the United States, the Commissariat \`a l'Energie Atomique
and the Centre National de la Recherche Scientifique / Institut National de Physique
Nucl\'eaire et de Physique des Particules in France, the Agenzia Spaziale Italiana
and the Istituto Nazionale di Fisica Nucleare in Italy, the Ministry of Education,
Culture, Sports, Science and Technology (MEXT), High Energy Accelerator Research
Organization (KEK) and Japan Aerospace Exploration Agency (JAXA) in Japan, and
the K.~A.~Wallenberg Foundation, the Swedish Research Council and the
Swedish National Space Board in Sweden.

Additional support for science analysis during the operations phase is gratefully
acknowledged from the Istituto Nazionale di Astrofisica in Italy and the Centre National d'\'Etudes Spatiales in France.

This research has made use of
pywcsgrid2, an open-source plotting package for
Python\footnote{\url{http://leejjoon.github.com/pywcsgrid2/}}. The authors acknowledge the
use of HEALPix\footnote{\url{http://healpix.jpl.nasa.gov/}} \citep{healpix}.



\bibliography{extended_catalog}

\clearpage
\begin{figure}
    \ifcolorfigure
    \plotone{ic443_plots/four_plots_ic443_color.eps}
    \else
    \plotone{ic443_plots/four_plots_ic443_bw.eps}
    \fi
    % taken from /u/gl/lande/work/fermi/extended_catalog/2FGL/plots_for_paper/four_plots_ic443/v2/run.py
    \caption{
    Counts maps and test statistic profiles for the SNR IC443. (a) \ts
    vs. extension of the source. (b) \tsext for individual energy
    bands. (c) observed radial profile of counts in comparison to the
    expected profiles for a spatially extended source (solid and colored
    red in the online version) and for a point-like source (dashed and colored
    blue in the online version).  (d) smoothed count map after subtraction
    of the diffuse emission compared to the smoothed
    LAT PSF (inset). Both were smoothed by a 0\fdg1 2D Gaussian kernel.
    Plots (a),
    (c), and (d) use only 
    photons with energies between
    1 \gev and 100 \gev.  Plots (c) and (d) use
    only photons which converted in the front part of the tracker and
    have an improved angular resolution.
    }
    \label{four_plots_ic443}
\end{figure}

\clearpage
\begin{figure}
    \ifcolorfigure
      \plotone{mc_plots/compare_disk_gauss_color.eps}
    \else
      \plotone{mc_plots/compare_disk_gauss_bw.eps}
    \fi
    % this plot came from /u/gl/lande/work/fermi/extended_catalog/2FGL/plots_for_paper/compare_disk_gauss/v3/plot.py
    \caption{
    A comparison of a 2D Gaussian and uniform surface brightness profiles
    of extended sources before and after convolving with the PSF for two
    energy ranges.  The solid black line is the PSF that would be observed
    for a power-law source of spectral index 2. The dashed line
    and the dash-dotted lines are 
    the brightness profile of a Gaussian with $\rsixeight=0\fdg5$
    and the convolution of this profile with the LAT PSF respectively
    (colored red in the online version).
    The dash-dot-doted and the dot-doted lines are the brightness profile
    of a uniform disk with $\rsixeight=0\fdg5$ and the convolution
    of this profile with the LAT PSF respectively (colored blue in the online version).
    }\label{compare_disk_gauss}
  \end{figure}


\clearpage
\begin{deluxetable}{rrrrrr}
\tabletypesize{\normalsize}
\tablecaption{Monte Carlo Spectral Parameters
\label{ts_ext_num_sims}
}
\tablewidth{0pt}
\tablehead{
\colhead{Spectral Index}&
\colhead{Flux\tablenotemark{(a)}}&
\colhead{$N_{1-100\gev}$}&
\colhead{$\langle\ts\rangle_{1-100\gev}$}&
\colhead{$N_{10-100\gev}$}&
\colhead{$\langle\ts\rangle_{10-100\gev}$}\\
\colhead{}&
\colhead{(\phflux)}&
\colhead{}&
\colhead{}&
\colhead{}&
\colhead{}
}

\startdata
      1.5 &  $3\times 10^{-7}$ &           18938 &  22233    &     18938   &    8084     \\
          &          $10^{-7}$ &           19079 &   5827    &     19079   &    2258     \\
          &  $3\times 10^{-8}$ &           19303 &   1276    &     19303   &     541     \\
          &          $10^{-8}$ &           19385 &    303    &     19381   &     142     \\
          &  $3\times 10^{-9}$ &           18694 &     62    &     12442   &      43     \\
      \hline
        2 &          $10^{-6}$ &           18760 &  22101    &     18760   &    3033     \\
          &  $3\times 10^{-7}$ &           18775 &   4913    &     18775   &     730     \\
          &          $10^{-7}$ &           18804 &   1170    &     18803   &     192     \\
          &  $3\times 10^{-8}$ &           18836 &    224    &     15256   &      50     \\
          &          $10^{-8}$ &           17060 &     50    &   \nodata   & \nodata     \\
      \hline
      2.5 &  $3\times 10^{-6}$ &           18597 &  19036    &     18597   &     786    \\
          &          $10^{-6}$ &           18609 &   4738    &     18608   &     208    \\
          &  $3\times 10^{-7}$ &           18613 &    954    &     15958   &      53    \\
          &          $10^{-7}$ &           18658 &    203    &   \nodata   & \nodata    \\
          &  $3\times 10^{-8}$ &           14072 &     41    &   \nodata   & \nodata    \\
      \hline                                                
        3 &          $10^{-5}$ &           18354 &  19466    &     18354   &     215    \\
          &  $3\times 10^{-6}$ &           18381 &   4205    &     15973   &      54    \\
          &          $10^{-6}$ &           18449 &    966    &   \nodata   & \nodata    \\
          &  $3\times 10^{-7}$ &           18517 &    174    &   \nodata   & \nodata    \\
          &          $10^{-7}$ &           13714 &     41    &   \nodata   & \nodata    \\
\enddata

\tablenotetext{(a)}{Integral 100 \mev to 100 \gev flux.}

\tablecomments{
\newtext{
    A list of the spectral models of the simulated point-like sources
    which were tested for extension.  For each model, the number of
    statistically independent simulations and the average value of \ts is
    also tabulated.  The models span representative spectral parameters.
    }
}
\end{deluxetable}

\clearpage
\begin{figure}
    \ifcolorfigure
    \plotone{mc_plots/ts_ext_emin_1000_color.eps}
    \else
    \plotone{mc_plots/ts_ext_emin_1000_bw.eps}
    \fi
    % this plot came from /u/gl/lande/work/fermi/extended_catalog/monte_carlo/ts_ext/v4/plot.py
    \caption{
    \newtext{
    Distribution of the test statistic for the extension test when fitting
    the sources in the 1 GeV to 100 GeV energy range.
    The four plots
    represent simulated point-like sources of different spectral indices and
    the different lines (colored in the online version) 
    represent point-like sources with different 100 \mev
    to 100 \gev integral fluxes.  The dashed line (colored red)
    is the cumulative
    density function of Equation~\ref{ts_ext_distribution}.
    }
    }\label{ts_ext_mc_1000}
  \end{figure}

\clearpage
\begin{figure}
    \ifcolorfigure
    \plotone{mc_plots/ts_ext_emin_10000_color.eps}
    \else
    \plotone{mc_plots/ts_ext_emin_10000_bw.eps}
    \fi
    % this plot came from /u/gl/lande/work/fermi/extended_catalog/monte_carlo/ts_ext/v4/plot.py
    \caption{
    \newtext{
    The same plot as Figure~\ref{ts_ext_mc_10000} but fitting in the 10 \gev to 100 \gev energy 
    range.
    }
    }\label{ts_ext_mc_10000}
  \end{figure}


\clearpage
\begin{figure}
    \ifcolorfigure
    \plotone{mc_plots/index_sensitivity_color.eps}
    \else
    \plotone{mc_plots/index_sensitivity_bw.eps}
    \fi
    % this plot came from /u/gl/lande/work/fermi/extended_catalog/monte_carlo/sensitivity/v12/plot/plot_vs_index.py
    \caption{
    The detection threshold to resolve a uniform surface brightness profile
    extended source for a two-year exposure.  All sources have an
    assumed power-law spectrum and the different line styles (colors in
    the electronic version) correspond to different simulated spectral
    indices.  The lines with no markers correspond to the detection
    threshold using photons with energies between 100 \mev and 100 \gev
    while the lines with star-shaped markers correspond to the threshold
    using photons with energies between 1 \gev and 100 \gev.
    }\label{index_sensitivity}
  \end{figure}

\clearpage
\begin{figure}
    \ifcolorfigure
    \plotone{mc_plots/all_sensitivity_color.eps}
    \else
    \plotone{mc_plots/all_sensitivity_bw.eps}
    \fi
    % this plot came from /u/gl/lande/work/fermi/extended_catalog/monte_carlo/sensitivity/v12/plot/plot_all.py
    \caption{the LAT detection threshold for four spectral indices
    and three background ($1\times$, $10\times$, and $100\times$ the
    Sreekumar-like isotropic background) for a two-year exposure. The
    left plots are the detection threshold when using 
    photons with energies between
    1 \gev and 100 \gev
    and the right plots are the detection threshold when using
    photons with energies between
    10 \gev and 100 \gev.  The flux is integrated only in the
    selected energy range.  Overlaid on this plot are the LAT detected
    extended sources placed by the magnitude of the nearby Galactic
    diffuse emission and the energy range they were analyzed with.
    The star-shaped markers (colored red in the electronic version)
    are sources with a spectral index closer to 1.5, the triangular
    markers (colored blue) an index closer to 2, and the circular markers
    (colored green) an index closer to 2.5.  The circular marker in plot
    (d) below the sensitivity line is MSH\,15$-$52.
    }\label{all_sensitivity} 
  \end{figure}

  \clearpage
  \thispagestyle{empty}
\begin{deluxetable}{rrrrrrrrrrrrrrrrrrrrrrr}
\tablecolumns{22}
\tabletypesize{\scriptsize}
\rotate
\tablewidth{0pt}
\tablecaption{Extension Detection Threshold
\label{all_sensitivity_table}
}
\tablehead{
\colhead{$\gamma$}&       
\colhead{BG}&
\colhead{$0.1$}&
\colhead{$0.2$}&
\colhead{$0.3$}&
\colhead{$0.4$}&
\colhead{$0.5$}&
\colhead{$0.6$}&
\colhead{$0.7$}&
\colhead{$0.8$}&
\colhead{$0.9$}&
\colhead{$1.0$}&
\colhead{$1.1$}&
\colhead{$1.2$}&
\colhead{$1.3$}&
\colhead{$1.4$}&
\colhead{$1.5$}&
\colhead{$1.6$}&
\colhead{$1.7$}&
\colhead{$1.8$}&
\colhead{$1.9$}&
\colhead{$2.0$}
}
\startdata
\multicolumn{22}{c}{E$>$1 \gev} \\
\hline
     1.5 &      $1\times$ &      148.1 &       23.3 &       11.3 &        8.0 &        7.2 &        6.9 &        6.7 &        6.8 &        7.1 &        7.4 &        7.6 &        7.9 &        8.1 &        8.5 &        9.2 &        9.9 &        9.1 &        9.2 &        9.0 &       10.3 \\
         &     $10\times$ &      148.4 &       29.0 &       18.7 &       15.2 &       15.4 &       15.0 &       16.1 &       16.0 &       16.8 &       17.7 &       18.2 &       19.3 &       20.9 &       22.5 &       23.8 &       24.8 &       21.3 &       22.8 &       23.4 &       23.7 \\
         &    $100\times$ &      186.8 &       55.0 &       43.4 &       40.7 &       41.0 &       41.8 &       40.9 &       40.9 &       42.7 &       43.6 &       38.4 &       39.9 &       40.6 &       38.4 &       36.9 &       36.3 &       37.1 &       38.8 &       37.2 &       37.6 \\
       2 &      $1\times$ &      328.4 &       43.4 &       18.9 &       13.4 &       11.2 &       10.4 &       10.2 &       10.2 &       10.2 &       10.4 &       10.7 &       10.9 &       11.2 &       11.5 &       12.4 &       12.6 &       13.0 &       13.4 &       14.0 &       14.4 \\
         &     $10\times$ &      341.0 &       55.9 &       32.3 &       27.6 &       26.5 &       25.4 &       25.6 &       25.9 &       27.4 &       26.8 &       27.8 &       28.7 &       29.8 &       30.1 &       31.0 &       31.5 &       31.7 &       34.0 &       34.3 &       35.9 \\
         &    $100\times$ &      420.5 &      128.3 &       90.2 &       77.3 &       73.3 &       70.8 &       67.5 &       64.3 &       64.2 &       64.1 &       62.8 &       63.6 &       61.7 &       61.9 &       58.4 &       59.0 &       61.4 &       63.3 &       60.1 &       58.1 \\
     2.5 &      $1\times$ &      627.1 &       75.6 &       29.8 &       19.3 &       15.5 &       13.5 &       12.8 &       12.6 &       12.5 &       12.5 &       12.6 &       12.9 &       12.9 &       13.1 &       13.5 &       13.7 &       14.3 &       14.8 &       15.2 &       15.8 \\
         &     $10\times$ &      638.9 &       99.1 &       52.1 &       39.1 &       34.6 &       33.0 &       32.5 &       32.5 &       32.8 &       33.2 &       34.1 &       34.3 &       34.5 &       35.1 &       36.6 &       36.9 &       35.5 &       36.0 &       36.5 &       37.3 \\
         &    $100\times$ &      795.0 &      262.1 &      140.9 &      104.3 &       90.4 &       81.2 &       77.2 &       75.1 &       69.7 &       70.9 &       66.5 &       65.6 &       64.9 &       64.0 &       58.9 &       58.1 &       60.2 &       58.4 &       57.5 &       55.8 \\
       3 &      $1\times$ &      841.5 &      110.6 &       43.2 &       25.5 &       18.7 &       16.1 &       14.4 &       13.6 &       13.3 &       13.2 &       13.1 &       13.1 &       13.4 &       13.6 &       13.5 &       13.8 &       14.2 &       14.4 &       14.8 &       15.4 \\
         &     $10\times$ &      921.6 &      151.3 &       69.1 &       47.8 &       40.7 &       37.1 &       35.5 &       34.5 &       35.1 &       35.5 &       35.3 &       35.3 &       35.4 &       35.5 &       36.8 &       37.6 &       35.3 &       35.4 &       36.3 &       36.6 \\
         &    $100\times$ &     1124.1 &      282.9 &      181.1 &      119.8 &      100.7 &       91.1 &       84.3 &       77.9 &       73.3 &       71.8 &       67.6 &       66.4 &       65.5 &       63.9 &       59.0 &       58.6 &       58.8 &       57.5 &       55.4 &       54.4 \\
\cutinhead{E$>$10 \gev}
     1.5 &      $1\times$ &       44.6 &        8.0 &        4.3 &        3.2 &        2.7 &        2.6 &        2.5 &        2.5 &        2.4 &        2.5 &        2.5 &        2.6 &        2.7 &        2.8 &        2.9 &        2.9 &        3.1 &        3.2 &        3.3 &        3.4 \\
         &     $10\times$ &       45.2 &        9.2 &        5.8 &        5.0 &        4.9 &        4.9 &        5.0 &        5.2 &        5.3 &        5.7 &        5.9 &        6.3 &        6.6 &        6.5 &        6.8 &        7.6 &        7.8 &        8.2 &        8.5 &        8.7 \\
         &    $100\times$ &       47.3 &       13.4 &       11.6 &       10.6 &       10.8 &       10.8 &       12.0 &       12.7 &       13.2 &       13.7 &       15.3 &       16.1 &       17.2 &       18.2 &       18.9 &       19.5 &       20.4 &       21.0 &       21.7 &       22.9 \\
       2 &      $1\times$ &       49.7 &        8.4 &        4.4 &        3.3 &        2.8 &        2.6 &        2.6 &        2.6 &        2.6 &        2.6 &        2.7 &        2.7 &        2.8 &        2.9 &        3.0 &        3.2 &        3.2 &        3.4 &        3.5 &        3.5 \\
         &     $10\times$ &       48.6 &        9.5 &        6.0 &        5.2 &        5.0 &        5.2 &        5.2 &        5.3 &        5.4 &        5.8 &        6.4 &        6.6 &        7.0 &        7.1 &        7.5 &        8.0 &        8.3 &        8.6 &        9.0 &        9.2 \\
         &    $100\times$ &       51.8 &       14.7 &       11.8 &       11.5 &       11.5 &       11.9 &       13.2 &       14.0 &       14.3 &       15.3 &       16.2 &       16.9 &       18.4 &       19.2 &       19.8 &       21.0 &       22.0 &       22.8 &       23.2 &       24.3 \\
     2.5 &      $1\times$ &       53.1 &        9.1 &        4.5 &        3.3 &        2.8 &        2.7 &        2.6 &        2.5 &        2.5 &        2.6 &        2.7 &        2.7 &        2.8 &        2.8 &        2.9 &        3.1 &        3.2 &        3.3 &        3.5 &        3.6 \\
         &     $10\times$ &       53.7 &       10.5 &        6.3 &        5.4 &        5.1 &        5.1 &        5.3 &        5.4 &        5.7 &        6.0 &        6.3 &        6.6 &        6.8 &        6.9 &        7.5 &        8.1 &        8.3 &        8.6 &        8.9 &        9.2 \\
         &    $100\times$ &       57.0 &       15.6 &       12.7 &       11.9 &       11.8 &       12.2 &       13.1 &       14.3 &       14.6 &       15.2 &       16.3 &       17.0 &       18.8 &       19.2 &       19.9 &       21.0 &       21.9 &       22.3 &       23.3 &       23.7 \\
       3 &      $1\times$ &       55.5 &        9.4 &        4.8 &        3.4 &        2.9 &        2.7 &        2.6 &        2.5 &        2.5 &        2.5 &        2.6 &        2.7 &        2.7 &        2.8 &        2.9 &        3.0 &        3.1 &        3.2 &        3.4 &        3.4 \\
         &     $10\times$ &       56.0 &       10.5 &        6.2 &        5.3 &        5.1 &        5.1 &        5.1 &        5.3 &        5.5 &        5.7 &        5.9 &        6.4 &        6.4 &        6.6 &        7.0 &        7.8 &        8.0 &        8.3 &        8.6 &        8.9 \\
         &    $100\times$ &       60.3 &       16.2 &       12.7 &       11.7 &       11.8 &       12.2 &       12.6 &       13.8 &       14.2 &       14.6 &       15.8 &       16.5 &       17.6 &       18.5 &       19.4 &       19.8 &       20.7 &       21.0 &       21.8 &       22.5 \\
\enddata
\tablecomments{
      The detection threshold to resolve spatially extended
      sources with a uniform surface brightness profile for a two-year exposure
      The threshold is calculated for
      sources of varying energy
      ranges, spectral indices, and background levels.  The sensitivity
      was calculated against
      a Sreekumar-like isotropic background
      and the second column is the factor that the simulated background
      was scaled by. The remaining columns are varying sizes of the source. 
      The table quotes
      integral fluxes in the analyzed energy range (1 \gev to 100 \gev or 10 \gev to 100
      \gev) in units of $10^{-10}$ \phflux.
      }
\end{deluxetable}




\begin{figure}
    \ifcolorfigure
      \plotone{mc_plots/time_sensitivity_color.eps}
    \else
      \plotone{mc_plots/time_sensitivity_bw.eps}
    \fi
    % this plot came from /u/gl/lande/work/fermi/extended_catalog/monte_carlo/sensitivity/v12/plot/plot_vs_time.py
    \caption{
    The projected detection threshold of the LAT to extension after a ten-year exposure
    for a power-law
    source of spectral index 2 against 10 times the isotropic background
    in the energy range from 1 \gev to 100 \gev (solid line colored red in the electronic
    version) and 10 \gev to 100 \gev
    (dashed line colored blue). The shaded gray regions represent the
    detection threshold assuming the sensitivity improves from 2 to 10
    years by the square root of the exposure (top edge) and linearly with
    exposure (bottom edge).  The lower plot shows the factor increase in
    sensitivity.  For small extended sources, our detection threshold
    to the extension of a source will improve by a factor larger than the square root of 
    the exposure.
    }\label{time_sensitivity}
  \end{figure}

\clearpage
\begin{figure}
    \ifcolorfigure
    \plotone{mc_plots/confusion_plot_color.eps}
    \else
    \plotone{mc_plots/confusion_plot_bw.eps}
    \fi
    % this plot came from 
    \caption{
    \newtext{
    The distribution of $\tsext-\tsinc$ 
    (a) and (b) when fitting simulated spatially extended sources with
    radially-symmetric uniform intensity surface profiles and varying
    the size of the source and (c), (d), (e), and (f) when fitting two
    simulated point-like sources and varying their separation distance.
    (c), and (d) represent simulations of two point-like sources with
    the same spectral index and (e), and (f) represent simulations of
    two point-like sources with different spectral indices.  (a), (c),
    and (e) fit the simulated sources in the 1 \gev to 100 \gev energy
    range and (b), (d), and (f) when fitting the simulated sources
    in the 10 \gev to 100 \gev energy range. The plus-shaped markers
    (colored red in the electronic version) are fits where $\tsext\ge16$
    that would be flagged as extended.
    }
    }\label{confusion_plot}
  \end{figure}


\clearpage
\begin{figure}
    \ifcolorfigure
      \plotone{source_plots/agn_color.eps}
    \else
      \plotone{source_plots/agn_bw.eps}
    \fi
    % this plot came from /u/gl/lande/work/fermi/extended_catalog/2FGL/agn/v6/agn.py
    \caption{The cumulative density of \tsext for the 733 clean
    AGN in 2LAC that were significant above 1 \gev calculated with \pointlike (dashed line
    colored blue in the electronic version)
    and with \gtlike (solid line colored red).  AGN are too far
    and too small to be resolved by the LAT. Therefore, the cumulative
    density of $\tsext$ is expected to follow a $\chi^2_1/2$ distribution
    (Equation~\ref{ts_ext_distribution}, the dash-dotted line colored
    red).
    }\label{agn_ts_ext}
  \end{figure}

\begin{figure}
  \ifcolorfigure
    \plotone{ic443_plots/res_tsmap_ic443_color.eps}
    \else
    \plotone{ic443_plots/res_tsmap_ic443_bw.eps}
    \fi
  % plot from /u/gl/lande/work/fermi/extended_catalog/2FGL/plots_for_paper/res_tsmap_ic443/v2

  \caption{
  A test statistic map generated for the region around the SNR 
  IC443 using 
  photons with energies between
  1 \gev and 100 \gev.  (a) test statistic map after
  subtracting IC443 modeled as a point-like source. (b) same as (a), but
  IC443 modeled as an extended source. The cross represents the best
  }
  \label{res_tsmaps}
\end{figure}

\clearpage
\begin{figure}
    \ifcolorfigure
    \plotone{source_plots/example_bad_fit_color.eps}
    \else
    \plotone{source_plots/example_bad_fit_bw.eps}
    \fi
    % taken from /u/gl/lande/work/fermi/extended_catalog/2FGL/plots_for_paper/example_bad_fit/v3/run.py
    \caption{
    A diffuse-emission-subtracted 1 \gev to 100 \gev counts map of the
    region around 2FGL\,J1856.2+0450c smoothed by a 0\fdg1 2D Gaussian
    kernel. The plus-shaped marker and circle (colored red in
    the online version) represent the center and size of the best fit
    radially-symmetric source with a uniform intensity profile.  The black
    crosses represent the position of other 2FGL sources.  The extension
    is statistically significant, but the extension encompasses many 2FGL
    sources and the emission does not look to be uniform. Although
    the fit is statistically significant, it likely corresponds to
    residual features of inaccurately modeled diffuse emission picked
    up by the fit. 
    }
    \label{example_bad_fit}
\end{figure}


\begin{deluxetable}{lrrrrrrrr}
\tablecolumns{9}
\rotate
\tabletypesize{\footnotesize}
\tablewidth{0pt}
\tablecaption{Analysis of the twelve extended sources included in 2FGL
\label{known_extended_sources}
}
\tablehead{
\colhead{Name}&
\colhead{\glon}&
\colhead{\glat}&
\colhead{$\sigma$}&
\colhead{\ts}&
\colhead{\tsext}&
\colhead{Pos Err}&
\colhead{Flux\tablenotemark{(a)}}&
\colhead{Index}\\
\colhead{}&
\colhead{(deg.)}&
\colhead{(deg.)}&
\colhead{(deg.)}&
\colhead{}&
\colhead{}&
\colhead{(deg.)}&
\colhead{}&
\colhead{}
}

\startdata
\multicolumn{9}{c}{E$>$1 \gev} \\
\hline
SMC                       &     302.59 &     -44.42 & $  1.32 \pm   0.15 \pm   0.31 $ &       95.0 &       52.9 &   0.14 & $    2.7 \pm     0.3$ & $   2.48 \pm    0.19$  \\
LMC                       &     279.26 &     -32.31 & $  1.37 \pm   0.04 \pm   0.11 $ &     1127.9 &      909.9 &   0.04 & $   13.6 \pm     0.6$ & $   2.43 \pm    0.06$  \\
IC443                     &     189.05 &       3.04 & $  0.35 \pm   0.01 \pm   0.04 $ &    10692.9 &      554.4 &   0.01 & $   62.4 \pm     1.1$ & $   2.22 \pm    0.02$  \\
Vela X                    &     263.34 &      -3.11 & $                         0.88$ &            &            &        &                       &                        \\
Centaurus A               &     309.52 &      19.42 &                        $\sim10$ &            &            &        &                       &                        \\
W28                       &       6.50 &      -0.27 & $  0.42 \pm   0.02 \pm   0.05 $ &     1330.8 &      163.8 &   0.01 & $   56.5 \pm     1.8$ & $   2.60 \pm    0.03$  \\
W30                       &       8.61 &      -0.20 & $  0.34 \pm   0.02 \pm   0.02 $ &      464.8 &       76.0 &   0.02 & $   29.1 \pm     1.5$ & $   2.56 \pm    0.05$  \\
W44                       &      34.69 &      -0.39 & $  0.35 \pm   0.02 \pm   0.02 $ &     1917.0 &      224.8 &   0.01 & $   71.2 \pm     0.5$ & $   2.66 \pm    0.00$  \\
W51C                      &      49.12 &      -0.45 & $  0.27 \pm   0.02 \pm   0.04 $ &     1823.4 &      118.9 &   0.01 & $   37.2 \pm     1.3$ & $   2.34 \pm    0.03$  \\
Cygnus Loop               &      74.21 &      -8.48 & $  1.71 \pm   0.05 \pm   0.06 $ &      357.9 &      246.0 &   0.06 & $   11.4 \pm     0.7$ & $   2.50 \pm    0.10$  \\
\cutinhead{E$>$10 \gev}
MSH\,15$-$52              &     320.39 &      -1.22 & $  0.21 \pm   0.04 \pm   0.04 $ &       76.3 &        6.6 &   0.03 & $    0.6 \pm     0.1$ & $   2.20 \pm    0.22$  \\
HESS\,J1825$-$137         &      17.57 &      -0.45 & $  0.65 \pm   0.04 \pm   0.02 $ &       82.9 &       44.9 &   0.05 & $    1.8 \pm     0.8$ & $   1.83 \pm    0.73$  \\
\enddata

\tablenotetext{(a)}{
Integral Flux in units of $10^{-9}$ \phflux and integrated in the fit
energy range (either 1 \gev to 100 \gev or 10 \gev to 100 \gev).
}

\tablecomments{
All sources were fit using a spatial model assuming a uniform radially
symmetric intensity distribution. \glon and \glat are Galactic longitude
and latitude of the best fit extended source respectively.  The first
error on $\sigma$ is statistical and the second is systematic (see
Section~\ref{systematic_errors_on_extension}).  Pos Err is the error on
the position of the source.  Vela X and the Centaurus A Lobes were
not fit in our analysis but are include for completeness.
}
\end{deluxetable}




  % 1FGL J1628.6-2419c - P72Y2516         - 2FGL J1627.0-2425c
  % 1FGL J0823.3-4248  - P72Y1212         - 2FGL J0823.0-4246
  % 1FGL J1614.7-5138c - P72Y2473         - 2FGL J1615.2-5138
  % 1FGL J1632.9-4802c - P72Y2540         - 2FGL J1632.4-4753c
  % 1FGL J2020.0+4049  - P72Y3281         - 2FGL J2021.5+4026
  % 1FGL J1837.5-0659c - P72Y2974         - 2FGL J1837.3-0700c
  % N/A                - P72Y1287         - 2FGL J0851.7-4635


\clearpage
\thispagestyle{empty}
\begin{deluxetable}{lrrrrrrrrr}
\tablecolumns{10}
\tabletypesize{\small}
\rotate
\tablewidth{0pt}
\tablecaption{Extension fit for the nine additional extended sources
\label{new_ext_srcs_table}
}
\tablehead{
\colhead{Name}&
\colhead{\glon}&
\colhead{\glat}&
\colhead{$\sigma$}&
\colhead{\ts}&
\colhead{\tsext}&
\colhead{Pos Err}&
\colhead{Flux\tablenotemark{(a)}}&
\colhead{Index}&
\colhead{Counterpart}\\
\colhead{}&
\colhead{(deg.)}&
\colhead{(deg.)}&
\colhead{(deg.)}&
\colhead{}&
\colhead{}&
\colhead{(deg.)}&
\colhead{}&
\colhead{}&
\colhead{}
}

\startdata
\multicolumn{10}{c}{E$>$1 \gev} \\
\hline
2FGL\,J0823.0$-$4246      &     260.32 &      -3.28 & $  0.37 \pm   0.03 \pm   0.02 $ &      322.2 &       48.0 &   0.02 & $    8.4 \pm     0.6$ & $   2.21 \pm    0.09$ &                  Puppis A \\
2FGL\,J1627.0$-$2425c     &     353.07 &      16.80 & $  0.42 \pm   0.05 \pm   0.16 $ &      139.9 &       32.4 &   0.04 & $    6.3 \pm     0.6$ & $   2.50 \pm    0.14$ &                 Ophiuchus \\
\cutinhead{E$>$10 \gev}
2FGL\,J0851.7$-$4635      &     266.31 &      -1.43 & $  1.15 \pm   0.08 \pm   0.02 $ &      116.6 &       86.8 &   0.07 & $    1.3 \pm     0.2$ & $   1.74 \pm    0.21$ &                  Vela Jr. \\
2FGL\,J1615.0$-$5051      &     332.37 &      -0.13 & $  0.32 \pm   0.04 \pm   0.01 $ &       50.4 &       16.7 &   0.04 & $    1.0 \pm     0.2$ & $   2.19 \pm    0.28$ &         HESS\,J1616$-$508 \\
2FGL\,J1615.2$-$5138      &     331.66 &      -0.66 & $  0.42 \pm   0.04 \pm   0.02 $ &       76.1 &       46.5 &   0.04 & $    1.1 \pm     0.2$ & $   1.79 \pm    0.26$ &         HESS\,J1614$-$518 \\
2FGL\,J1632.4$-$4753c     &     336.52 &       0.12 & $  0.35 \pm   0.04 \pm   0.02 $ &       64.4 &       26.9 &   0.04 & $    1.4 \pm     0.2$ & $   2.66 \pm    0.30$ &         HESS\,J1632$-$478 \\
2FGL\,J1712.4$-$3941      &     347.26 &      -0.53 & $  0.56 \pm   0.04 \pm   0.02 $ &       59.4 &       38.5 &   0.05 & $    1.2 \pm     0.2$ & $   1.87 \pm    0.22$ &        RX\,J1713.7$-$3946 \\
2FGL\,J1837.3$-$0700c     &      25.08 &       0.13 & $  0.33 \pm   0.07 \pm   0.05 $ &       47.0 &       18.5 &   0.07 & $    1.0 \pm     0.2$ & $   1.65 \pm    0.29$ &         HESS\,J1837$-$069 \\
2FGL\,J2021.5+4026        &      78.24 &       2.20 & $  0.63 \pm   0.05 \pm   0.04 $ &      237.2 &      128.9 &   0.05 & $    2.0 \pm     0.2$ & $   2.42 \pm    0.19$ &            $\gamma$-Cygni \\
\enddata

\tablenotetext{(a)}{
Integral Flux in units of $10^{-9}$ \phflux and integrated in the fit
energy range (either 1 \gev to 100 \gev or 10 \gev to 100 \gev).
}

\tablecomments{
    The columns in this table have the same meaning as those in
    Table~\ref{known_extended_sources}.
    RX\,J1713.7$-$3946 and Vela Jr. were previously studied in dedicated publications \citep{rx_j1713_lat,vela_jr_lat}.
}
\end{deluxetable}



\clearpage
\thispagestyle{empty}
\begin{deluxetable}{lrrrrrrrrrr}
  \tabletypesize{\small}
  \tablecolumns{11}
  \rotate
  \tablewidth{0pt}
  \tablecaption{Dual localization, alternative PSF, and alternative diffuse results
  \label{alt_diff_model_results}
  }
  \tablehead{
  \colhead{Name}&
  \colhead{$\ts_\pointlike$}&
  \colhead{$\ts_\gtlike$}&
  \colhead{$\ts_\altdiff$}&
  \colhead{$\tsext_\pointlike$}&
  \colhead{$\tsext_\gtlike$}&
  \colhead{$\tsext_\altdiff$}&
  \colhead{$\sigma$}&
  \colhead{$\sigma_\altdiff$}&
  \colhead{$\sigma_\altpsf$}&
  \colhead{$\tsinc$}\\
  \colhead{}&
  \colhead{}&
  \colhead{}&
  \colhead{}&
  \colhead{}&
  \colhead{}&
  \colhead{}&
  \colhead{(deg.)}&
  \colhead{(deg.)}&
  \colhead{(deg.)}&
  \colhead{}
  }
  \startdata
\multicolumn{11}{c}{E$>$1 \gev} \\
\hline
2FGL\,J0823.0$-$4246      &                331.9 &                322.2 &                356.0 &                 60.0 &                 48.0 &                 56.0 &                 0.37 &                 0.39 &                 0.39 &                 23.0 \\
2FGL\,J1627.0$-$2425c     &                154.8 &                139.9 &                105.7 &                 39.4 &                 32.4 &                 24.8 &                 0.42 &                
 0.40 &                 0.58 &                 24.5 \\
\cutinhead{E$>$10 \gev}
2FGL\,J0851.7$-$4635      &                115.2 &                116.6 &                123.1 &                 83.9 &                 86.8 &                 89.8 &                 1.15 &                 1.16 &                 1.17 &                 15.5 \\
2FGL\,J1615.0$-$5051      &                 48.2 &                 50.4 &                 56.6 &                 15.2 &                 16.7 &                 17.8 &                 0.32 &                 0.33 &                 0.32 &                 13.1 \\
2FGL\,J1615.2$-$5138      &                 75.0 &                 76.1 &                 83.8 &                 42.9 &                 46.5 &                 54.1 &                 0.42 &                 0.43 &                 0.43 &                 35.1 \\
2FGL\,J1632.4$-$4753c     &                 64.5 &                 64.4 &                 66.8 &                 23.0 &                 26.9 &                 25.5 &                 0.35 &                 0.36 &                 0.37 &                 10.9 \\
2FGL\,J1712.4$-$3941      &                 59.8 &                 59.4 &                 39.9 &                 38.4 &                 38.5 &                 30.7 &                 0.56 &                 0.55 &                 0.53 &                  2.7 \\
2FGL\,J1837.3$-$0700c     &                 44.5 &                 47.0 &                 39.2 &                 17.6 &                 18.5 &                 16.1 &                 0.33 &                 0.32 &                 0.38 &                 10.8 \\
2FGL\,J2021.5+4026        &                239.1 &                237.2 &                255.8 &                139.1 &                128.9 &                138.0 &                 0.63 &                 0.65 &                 0.59 &                 37.3 \\
  \enddata


\tablecomments{
$\ts_\pointlike$, $\ts_\gtlike$, and $\ts_\altdiff$ are the test
statistic values from \pointlike, \gtlike, and \gtlike with the alternate
diffuse model respectively.  $\tsext_\pointlike$, $\tsext_\gtlike$,
and $\tsext_\altdiff$ are the test statistic values of the extension
test from \pointlike, \gtlike, and \gtlike with the alternate diffuse
model respectively.  $\sigma$, $\sigma_\altdiff$, and $\sigma_\altpsf$
are the fit extension with the standard analysis, the alternate diffuse
model, and the alternate PSF respectively.  \tsinc is the test statistic
for the two point-like source hypothesis test 
(see Section~\ref{dual_localization_method}).
}
\end{deluxetable}


\begin{figure}
    \ifcolorfigure
      \plotone{source_plots/source_Puppis_A_color.eps}
    \else
      \plotone{source_plots/source_Puppis_A_bw.eps}
    \fi
  % this plot came from 
  % /u/gl/lande/work/fermi/extended_catalog/2FGL/plots_for_paper/source_plots/1FGL_J0823.3-4248/v8/run.py
  \caption{A diffuse-emission-subtracted 1 \gev to 100 \gev counts map
  of 2FGL\,J0823.0$-$4246 smoothed by a 0\fdg1 2D
  Gaussian kernel.  The triangular marker 
  (colored red in the online version)
  represents the 2FGL position of this source.  The plus-shaped
  marker and the circle (colored
  red) represent 
  the best fit position and extension of this source assuming a
  radially-symmetric uniform surface brightness profile.  The two 
  star-shaped markers (colored
  green) represent 2FGL sources that were
  removed from the background model.
  The lower right inset is the model predicted emission from a point-like
  source with the same spectrum as 2FGL\,J0823.4$-$4305 smoothed by the
  same kernel.  This source is spatially coincident with the Puppis A
  SNR. The light blue contours correspond to the X-ray image of Puppis
  A observed by \rosat \citep{rosat_puppis_a}.
  }\label{1FGL_J0823.3-4248}
\end{figure}

\clearpage
\begin{figure}
    \ifcolorfigure
      \plotone{summary_plots/snr_seds_color.eps}
    \else
      \plotone{summary_plots/snr_seds_bw.eps}
    \fi
    % this plot came from /u/gl/lande/work/fermi/extended_catalog/2FGL/seds/v3/plot_seds/plot_snr_seds.py
    \caption{
    The spectral energy distribution of the extended sources 
    Puppis A (2FGL\,J0823.0$-$4246) and $\gamma$-Cygni 
    (2FGL\,J2021.5+4026).
    The lines (colored red in the online version)
    are the best fit power-law spectral models of
    these sources.
    The spectral errors are statistical only.
    }
    \label{snr_seds}
  \end{figure}

\begin{figure}
    \ifcolorfigure
      \plotone{source_plots/source_Ophiuchus_color.eps}
    \else
      \plotone{source_plots/source_Ophiuchus_bw.eps}
    \fi
  % this plot came from 
  % /u/gl/lande/work/fermi/extended_catalog/2FGL/plots_for_paper/source_plots/1FGL_J1628.6-2419c/v8/run.py
  \caption{
  A diffuse-emission-subtracted 1 \gev to 100 \gev counts map of (a) the region
  around 2FGL\,J1627.0$-$2425 smoothed by a 0\fdg1 2D Gaussian kernel (b)
  also the emission from 2FGL\,J1625.7$-$2526
  subtracted.  The triangular marker 
  (colored red in the
  online version) represents the 2FGL position of this source.
  \newtext{
  The plus-shaped marker and the circle (colored red) 
  represent the best fit position and extension of this
  source assuming a radially-symmetric uniform surface brightness profile
  and the black cross represents a background 2FGL source. 
  }
  The
  contours in (a) correspond to the 100 micrometer image observed by
  IRAS \citep{iras_rho_ophiuci}.  The contours in (b) correspond to
  ${}^{12}\text{CO}$ ($J=1\rightarrow 0$) emission integrated from -8 \km/\s
  to 20 \km/\s.  They are from \cite{co_rho_ophiuci}, were cleaned using
  the moment-masking technique \citep{masking_moment_2011}, and have
  been smoothed by a 0\fdg25 2D Gaussian kernel.
  }\label{1FGL_J1628.6-2419c}
\end{figure}



\begin{figure}
    \ifcolorfigure
      \plotone{source_plots/source_Vela_Jr_color.eps}
    \else
      \plotone{source_plots/source_Vela_Jr_bw.eps}
    \fi
  % this plot came from 
  % /u/gl/lande/work/fermi/extended_catalog/2FGL/plots_for_paper/source_plots/Vela_Jr/v8/run.py
  \caption{A diffuse-emission-subtracted 10 \gev to 100 \gev counts map of
  2FGL\,J0851.7$-$4635 smoothed by a 0\fdg25 2D Gaussian
  kernel. The triangular marker (colored red in the electronic version)
  represents the 2FGL position of this source.  The plus-shaped marker
  and the circle (colored red) are the best fit position and extension of
  this source assuming a radially-symmetric uniform surface brightness profile.
  \newtext{
  The three black crosses represent background 2FGL sources
  and the three star-shaped markers (colored green) represent other 2FGL sources
  that were removed from the background model 
  The circular and square-shaped
  marker (colored blue) represents the 2FGL and relocalized position of another 2FGL source.  
  }
  This extended source is spatially
  coincident with the Vela Jr. SNR.  The contours (colored light blue)
  correspond to the \tev image of Vela Jr.
  \citep{vela_jr_hess}.
  }\label{Vela_Jr}
\end{figure}

\begin{figure}
    \ifcolorfigure
      \plotone{source_plots/source_HESS_J1614-518_color.eps}
    \else
      \plotone{source_plots/source_HESS_J1614-518_bw.eps}
    \fi
  % this plot came from 
  % /u/gl/lande/work/fermi/extended_catalog/2FGL/plots_for_paper/source_plots/1FGL_J1613.6-5100c/v8/run.py
  \caption{
    A diffuse-emission-subtracted 10 \gev to 100 \gev counts
    map of 2FGL\,J1615.0$-$5051 (upper left)
    and 2FGL\,J1615.2$-$5138 (lower right) smoothed by a 0\fdg1
    2D Gaussian kernel.  The triangular markers (colored red in the
    electronic version) represent the 2FGL positions of these sources.
    The cross-shaped markers and the
    circles (colored red) represent the best fit
    positions and extensions of these sources assuming a radially
    symmetric uniform surface brightness profile.  
    \newtext{The two black crosses represent  background 2FGL sources and
    the star-shaped
    marker (colored green) represents another 2FGL source that was removed from the background
    model.  The contours (colored light blue) correspond to the \tev
    image of HESS\,J1616$-$508 (left) and HESS\,J1614$-$518 (right)
    }
    \citep{hess_plane_survey}.   
    }\label{1FGL_J1613.6-5100c}
\end{figure}


\clearpage
\begin{figure}
    \ifcolorfigure
      \plotone{summary_plots/hess_seds_color.eps}
    \else
      \plotone{summary_plots/hess_seds_bw.eps}
    \fi
    % this plot came from /u/gl/lande/work/fermi/extended_catalog/2FGL/seds/v3/plot_seds/plot_hess_seds.py
    \caption{
    The spectral energy distribution of four extended
    sources associated with unidentified
    extended \tev sources.  The black points
    with circular markers are obtained by the LAT. The points with
    plus-shaped markers (colored red in the electronic version) are
    for the associated H.E.S.S sources.  (a) the
    LAT SED of 2FGL\,J1615.0$-$5051 together with the H.E.S.S. SED
    of HESS\,J1616$-$508. (b) 2FGL\,J1615.2$-$5138
    and HESS\,J1614$-$518. (c) 2FGL\,J1632.4$-$4753c
    and HESS\,J1632$-$478. (d) 2FGL\,J1837.3$-$0700c
    and HESS\,J1837$-$069. The H.E.S.S. data points are from
    \citep{hess_plane_survey}. Both LAT and H.E.S.S. spectral errors are
    statistical only.}
    \label{hess_seds}
  \end{figure}

\begin{figure}
    \ifcolorfigure
      \plotone{source_plots/source_HESS_J1632-478_color.eps}
    \else
      \plotone{source_plots/source_HESS_J1632-478_bw.eps}
    \fi
  % this plot came from 
  % /u/gl/lande/work/fermi/extended_catalog/2FGL/plots_for_paper/source_plots/1FGL_J1632.9-4802c/v8/run.py
  \caption{
  A diffuse-emission-subtracted 10 \gev to 100 \gev counts map of
  2FGL\,J1632.4$-$4753c (a) smoothed by a 0\fdg1 2D Gaussian
  kernel and (b) with the emission from the background sources subtracted.  
  The triangular marker (colored red in the electronic version)
  represents the 2FGL position of this source.  The plus-shaped marker
  and the circle (colored red) are the best fit position and extension of
  2FGL\,J1632.4$-$4753c assuming a radially-symmetric uniform surface
  brightness profile.  
  \newtext{
  The four black crosses represent background 2FGL
  sources subtracted in (b).  The 
  circular and square-shaped markers (colored blue) represent 
  the 2FGL and relocalized positions respectively of two
  additional background 2FGL sources subtracted in (b).
  The star-shaped marker (colored green) represent another 2FGL
  source that was removed from the background model.  
  }
  The contours (colored light blue) correspond to the \tev image of
  HESS\,J1632-478 \citep{hess_plane_survey}.
  }\label{1FGL_J1632.9-4802c}
\end{figure}

\begin{figure}
    \ifcolorfigure
      \plotone{source_plots/source_RX_J1713.7-3946_color.eps}
    \else
      \plotone{source_plots/source_RX_J1713.7-3946_bw.eps}
    \fi
    % this plot came from
  % /u/gl/lande/work/fermi/extended_catalog/2FGL/plots_for_paper/source_plots/RX_J1713.7-3946/v8/run.py
  \caption{
  \newtext{
  A diffuse-emission-subtracted 1 \gev to 100 \gev counts map
  of 2FGL\,J1712.4$-$3941 (a) smoothed by a 0\fdg15 2D
  Gaussian kernel and (b) with the emission from the background sources
  subtracted.  This source is spatially coincident with RX\,J1713.7$-$3946
  and was recently studied in \cite{rx_j1713_lat}.  The triangular marker
  (colored red in the online version) represents the 2FGL position of
  this source.  The plus-shaped marker and the circle (colored red) are
  the best fit position and extension of this source assuming a radially
  symmetric uniform surface brightness profile.  
  The two black crosses represent background 2FGL sources subtracted in (b).
  The contours (colored light blue)
  correspond to the \tev image \citep{rx_j1713_hess}.  
  }
  }\label{2FGL_J1712.4-3941}
\end{figure}


\begin{figure}
    \ifcolorfigure
      \plotone{source_plots/source_HESS_J1837-069_color.eps}
    \else
      \plotone{source_plots/source_HESS_J1837-069_bw.eps}
    \fi
  % this plot came from 
  % /u/gl/lande/work/fermi/extended_catalog/2FGL/plots_for_paper/source_plots/1FGL_J1837.5-0659c/v8/run.py
  \caption{
  A diffuse-emission-subtracted 10 \gev to 100 \gev counts map of the
  region around 2FGL\,J1837.3$-$0700c (a) smoothed by a 0\fdg15 2D Gaussian
  kernel and (b) with the emission from the background sources subtracted.
  The triangular marker (colored red in the online version) represents
  the 2FGL
  position of this source. 
  \newtext{
  The plus-shaped marker and 
  the circle (colored red) represent the best fit position and extension
  of 2FGL\,J1837.3$-$0700c assuming a radially-symmetric uniform surface
  brightness profile. The circular and square-shaped markers (colored
  blue) represent the 2FGL and the relocalized positions respectively of
  two background 2FGL sources subtracted in (b).  The star-shaped marker
  (colored green) represents a 2FGL source that was removed from the
  background model.  The contours (colored light blue) correspond to
  the \tev image of HESS\,J1837$-$069
  \citep{hess_plane_survey}.
  The diamond-shaped marker (colored orange) represents the position of PSR J1838-0655
  and the hexagonal-shaped marker (colored purple) represents the position AX J1837.3-0652
  \citep{pulsations_HESS_J1837-069}.
  }
  }\label{1FGL_J1837.5-0659c}
\end{figure}


\begin{figure}
    \ifcolorfigure
      \plotone{source_plots/source_Gamma_Cygni_color.eps}
    \else
      \plotone{source_plots/source_Gamma_Cygni_bw.eps}
    \fi
  % this plot came from 
  % /u/gl/lande/work/fermi/extended_catalog/2FGL/plots_for_paper/source_plots/Gamma_Cygni/v8/run.py
  \caption{A Diffuse-emission-subtracted 
  subtracted 10 \gev to 100 \gev counts map of the
  region around 2FGL\,J2021.5+4026 smoothed by a 0\fdg1 2D Gaussian
  kernel. The triangular marker (colored red in the online version)
  represents the 2FGL position of this source.  The plus-shaped
  marker and the circle (colored red) represent the best fit position
  and extension of 2FGL\,J2021.5+4026 assuming a radially-symmetric
  uniform surface brightness profile.  \newtext{
  The star-shaped marker (colored green)
  represents a 2FGL source that was removed from the background model.
  }
  2FGL\,J2021.5+4026
  is spatially coincident with the $\gamma$-Cygni SNR.  The contours
  (colored light blue) correspond to the 408MHz image of $\gamma$-Cygni
  observed by the Canadian Galactic Plane Survey.
  }\label{1FGL_J2020.0+4049}
\end{figure}


\clearpage
  \begin{figure}
      \ifcolorfigure
      \plotone{summary_plots/allsky_extended_sources_color.eps}
      \else
      \plotone{summary_plots/allsky_extended_sources_bw.eps}
      \fi
      % this plot came from /u/gl/lande/work/fermi/extended_catalog/2FGL/plots_for_paper/allsky/v2/run.py
      \caption{The 21
      spatially extended sources detected by the LAT
      at \gev energies 
      with two years of data.  The twelve extended sources included in
      2FGL are represented by the circular markers (colored red in the online
      version).  The nine new extended sources are represented by
      the triangular markers (colored orange).}
\label{allsky_extended_sources}
  \end{figure}


\clearpage
\begin{figure}
    \ifcolorfigure
      \plotone{summary_plots/gev_vs_tev_plot_color.eps}
    \else
      \plotone{summary_plots/gev_vs_tev_plot_bw.eps}
      \fi
    % this plot came from /u/gl/lande/work/fermi/extended_catalog/2FGL/plots_for_paper/compare_tev_gev_sizes/v4/run.py
    \caption{
    A comparison of the size of extended sources detected at
    both \gev and \tev energies.  The \tev size of W30,
    2FGL\,J1837.3$-$0700c, 2FGL\,J1632.4$-$4753c, 2FGL\,J1615.0$-$5051,
    and 2FGL\,J1615.2$-$5138 are from \cite{hess_plane_survey}.
    The \tev size of MSH\,15$-$52, HESS\,J1825$-$137,
    Vela X, Vela Jr., RX\,J1713.7$-$3946 and W28 are from
    \cite{msh_15_52_hess,hess_j1825_hess,vela_x_hess,vela_jr_hess,rx_j1713_hess,w28_hess}.
    The \tev size of IC443 is from \cite{ic443_veritas} and
    W51c is from \cite{w51c_with_magic_at_fermi_symposium}.  The \tev
    size of MSH\,15$-$52, HESS\,J1614$-$518, HESS\,J1632$-$478, and
    HESS\,J1837$-$069 have only been reported with an elliptical 2D
    Gaussian fit and so the assumed size is the average of the semi-major
    and semi-minor axis.
    The LAT extension of
    Vela X is from \cite{velax}. 
    The \tev sources were fit assuming a 2D Gaussian surface brightness profile
    so the plotted \gev and \tev extensions were first converted to
    \rsixeight (see Section~\ref{compare_source_size}).  
    Because of
    their large size, the shape of RX\,J1713.7$-$3946 and Vela Jr.
    were not directly fit at \tev energies and so are not included
    in this comparison. On the other hand, dedicated LAT publications on
    these sources showed that their morphologies are consistent
    \citep{rx_j1713_lat,vela_jr_lat}.
    The LAT
    extension errors are the statistical and systematic errors added
    in quadrature. 
}\label{gev_vs_tev_plot}
  \end{figure}

\clearpage
\begin{figure}
    \ifcolorfigure
      \plotone{summary_plots/gev_vs_tev_histogram_color.eps}
    \else
      \plotone{summary_plots/gev_vs_tev_histogram_bw.eps}
    \fi
    % this plot came from /u/gl/lande/work/fermi/extended_catalog/2FGL/plots_for_paper/extension_histogram/v5/extension_histogram.py
    \caption{
    The distribution of the size of 18 extended LAT sources
    at \gev energies
    (colored blue in the electronic version) and the size of the
    40 extended H.E.S.S. sources at \tev energies
    (colored red).  
    The H.E.S.S. sources were fit with a 2D Gaussian surface
    brightness profile so the LAT and H.E.S.S. sizes were first converted
    to \rsixeight. 
    The \gev size of Vela X is taken from \cite{velax}.  
    Because of their large size, the shape of RX\,J1713.7$-$3946 and
    Vela Jr. were not directly fit at \tev energies
    and are not included in this comparison.
    Centaurus A is not included because of its large size.
    }\label{gev_vs_tev_histogram}
  \end{figure}

\clearpage
\begin{figure}
    \ifcolorfigure
      \plotone{summary_plots/compare_index_2FGL_color.eps}
    \else
      \plotone{summary_plots/compare_index_2FGL_bw.eps}
    \fi
    % plot taken from /u/gl/lande/work/fermi/extended_catalog/2FGL/plots_for_paper/compare_index_2fgl/v3/run.py
    \caption{
    The distribution of spectral indices of the 1873 2FGL sources
    (colored red in the electronic version) and the 21 spatially extended
    sources (colored blue).  The index of Centaurus A is taken from
    \cite{second_cat} and the index of Vela X is taken from \cite{velax}.
    }\label{compare_index_2FGL}
  \end{figure}



\end{document}
